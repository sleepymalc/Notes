\lecture{10}{26 Sep.\ 11:00}{Oblivious Routing}
\begin{definition}[Routable]\label{def:routable}
	A demand matrix \(D \in \mathbb{R} _+^{n \times n}\) is \emph{routable} in a directed graph \(G = (V, E)\) with capacity \(c \colon E \to \mathbb{R} _+\) if there exists a feasible multi-commodity flow for \(D\) in \(G\).
\end{definition}

It's easy to see that \(\mathcal{D} _G \coloneqq \{ D \in \mathbb{R} _+^{n \times n} \mid D \text{ is \hyperref[def:routable]{routable} } \} \) is a convex set. Suppose we have \(w(u, v)\) for all \(u, v \in V\). We ask the following question.

\begin{problem}[Optimiznig demand polytope]\label{prb:optimize-demand-polytope}
Can we solve \(\max _{D \in \mathcal{D} _G} \sum_{u, v} w(u, v) D(u, v)\)?
\end{problem}
\begin{answer}
	We simply observe that \(D \in \mathcal{D} _G\) can be characterized by linear constraints for flows.
\end{answer}

\begin{lemma}
	\(D\) is \hyperref[def:routable]{routable} if and only if for all \(\ell \colon E \to \mathbb{R} _+\)
	\[
		\sum_{e \in E} c(e) \ell (e) \geq \sum_{u, v \in V} D(u, v) d_{\ell } (u, v)
	\]
	where \(d_{\ell } \) is the shortest path distance \(d_{\ell } \) induced by \(\ell \).
\end{lemma}


\begin{problem}[Oblivious routing]\label{def:oblivious-routing}
Given a directed graph \(G = (V, E)\) with edge capacity \(c \colon E \to \mathbb{R} _+\). The goal of \emph{oblivious routing} is to output a distribution of paths between every pair of \(u, v \in V\) such that when the demand \(D \colon V \times V \to \mathbb{R} _+\) come, the congestion is low.
\end{problem}

\begin{definition}[Oblivious routing scheme]\label{def:oblivious-routing-scheme}
	Given a graph \(G = (V, E)\), consider the following.
	\begin{definition}[Edge-based oblivious routing scheme]\label{def:edge-based-oblivious-routing-scheme}
		An \emph{edge-based oblivious routing scheme} is a collection of unit flows in \(G\), one for each \((u, v) \in V \times V\), specified via edge-based flows \(f_e^{(u, v)}\).
	\end{definition}

	\begin{definition}[Path-based oblivious routing scheme]\label{def:path-based-oblivious-routing-scheme}
		A \emph{path-based oblivious routing scheme} is similar to \hyperref[def:edge-based-oblivious-routing-scheme]{edge-based} where the unit flow is specified via a path based flow.
	\end{definition}
\end{definition}


\begin{definition}[Congestion]\label{def:congestion}
	The \emph{congestion} of an \hyperref[def:oblivious-routing-scheme]{oblivious routing scheme} is equal to
	\[
		\max _{D \in \mathcal{D} _G} \max _{e \in E} \frac{\sum_{u, v \in V} D(u, v) f_e^{(u, v)}}{c(e)}.
	\]
\end{definition}

\begin{claim}
	Every graph on \(n\) vertices has an \hyperref[def:oblivious-routing]{oblivious routing scheme} with \hyperref[def:congestion]{congestion} \(n^2\).
\end{claim}
\begin{explanation}
	Consider computing the max-flow for every pair and then scale it down to a unit flow.
\end{explanation}

\[
	\begin{aligned}
		\max ~ & \rho                                                                                                                          \\
		       & f_e^{(u, v)} \text{ defines a unit flow for pair \((u, v) \in V \times V\) } & \forall e \in E;                               \\
		       & \sum_{u, v \in V} D(u, v) f_e^{(u, v)}  \leq \rho c(e)                       & \forall e \in E, \forall D \in \mathcal{D} _G; \\
		       & f_e^{(u, v)} \geq 0                                                          & \forall e \in E, \forall u, v, \in V.
	\end{aligned}
\]

To design a separation oracle, given a solution \(f_e^{(u, v)}\), we see that checking unit flow is easy, while for the second one, given any fix edge \(e\), we simply maximize \(\sum_{u, v \in V} D(u, v) f_e^{(u, v)}\) via \autoref{prb:optimize-demand-polytope} and compare it with \(\rho c(e)\).

\begin{theorem}
	For undirected graph, \(\rho ^{\ast} = O(\log^{3 n} )\), and then it's further improved to \(\rho ^{\ast} = O(\log n)\).
\end{theorem}

We will prove a weaker result, \(O(\log n \log \log n)\), due to the notation. Given an undirected graph \(G = (V, E)\) with edge capacity \(c \colon E \to \mathbb{R} _+\), and a probability distribution over \hyperref[def:spanning-tree]{spanning tree} \(p\colon \mathcal{T} _G \to [0, 1]\) such that \(\sum_{T \in \mathcal{T} } p_T = 1\). This gives an \hyperref[def:oblivious-routing]{oblivious routing} since each tree \(T \in \mathcal{T} _G\) gives a unique path between any pair \((u, v) \in V \times V\).

\begin{notation}[Tree-based oblivious routing]
	The above scheme is what we called \emph{tree-based oblivious routing}, which is indeed \hyperref[def:path-based-oblivious-routing-scheme]{path-based}.
\end{notation}

Define the load \(L(T, e)\) on edge \(e\) to be \(c(\delta (S_e))\), where \(S_e\) is the component induced by removing \(e\) from \(T\). If \(e\) is not in \(T\), then we let \(L(T, e) = 0\). Then, the \hyperref[def:congestion]{congestion} of \(e\) is given by
\[
	\rho (e)
	= \frac{\sum_{T \in \mathcal{T} _G} p_T L(T, e)}{c(e)},
\]
with \(\rho \coloneqq \max _{e \in E} \rho (e)\).

\begin{lemma}
	Given \(p \colon \mathcal{T} \to [0, 1]\), the maximum \hyperref[def:congestion]{congestion} of the \hyperref[def:oblivious-routing-scheme]{oblivious routing} induced by \(p\) is \(\rho \).
\end{lemma}
\begin{proof}
	Fix a demand matrix \(D \in \mathcal{D} _G\). \hyperref[def:congestion]{Congestion} of \(e\) is less than \(\rho (e)\) since
	\[
		\frac{\sum_{T \in \mathcal{T} _G, T \ni e} p_T \sum_{\lvert S_e \cap \{ u, v \} \rvert =1} D(u, v)}{c(e)}
		\leq \frac{\sum_{T \in \mathcal{T} _G, T \ni e} p_T L(T, e)}{c(e)}.
	\]
	Taking maximum over \(e \in E\) gives the result.
\end{proof}

\[
	\begin{aligned}
		\min~           & \rho                                                                      \\
		                & \sum_{T \in \mathcal{T} _G} x_T = 1                                       \\
		                & \sum_{T \in\mathcal{T} _G} x_T L(T, e) \leq \rho c(e) & \forall e \in E ; \\
		\text{(P)}\quad & x_T \geq 0 ;
	\end{aligned}\quad
	\begin{aligned}
		\max~           & \beta                                                                  \\
		                & \sum_{e \in E} c(e) z_e = 1                                            \\
		                & \sum_{e \in T} L(T, e) z_e \geq \beta & \forall T \in \mathcal{T} _G ; \\
		\text{(D)}\quad & z_e \geq 0 .
	\end{aligned}
\]
We see that the dual is equivalent to
\[
	\max _{z \colon E \to \mathbb{R} _+} \min _{T \in \mathcal{T} _G} \frac{\sum_{e \in T} L(T, e) z_e}{\sum_{e \in E} c(e) z_e}
	= \max _{z \colon E \to \mathbb{R} _+} \min _{T \in \mathcal{T} _G} \frac{\sum_{uv \in E} c(uv) d_{T, z}(u, v)}{\sum_{e \in E} c(e) z_e},
\]
which is just the \hyperref[def:distortion]{distortion} of the tree, which we know that for tree, it's \(O(\log n \log \log n)\).