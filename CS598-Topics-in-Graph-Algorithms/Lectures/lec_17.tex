\chapter{Multiplicative Weight Updates}
\lecture{17}{22 Oct.\ 11:00}{Approximating Linear Programs}
Positive linear programs are a large and interesting class of linear programs that arise in combinatorial optimization. These have some useful properties that can be exploited algorithmically via iterative methods coming from first order optimization methods and this leads to fast approximation solutions.

Although these methods are old in optimization the formal analysis of these methods with provable worst-case guarantees for concrete problem can be traced to the work of faster algorithms for multi-commodity flows~\cite{shahrokhi1990maximum} which was later abstracted to general classes of linear programs~\cite{plotkin1995fast,grigoriadis1994fast}. Parallel algorithms were also derived via these methods~\cite{luby1993parallel}. There have been many developments since then.

\section{Positive Linear Programs}
Positive linear programs, as the name suggests, are linear programs where the input to the linear program consists of positive numbers. This allows one to discuss relative approximations.

\subsection{Packing, Covering and Mixture of Both}
There are three types of positive linear programs that we commonly work with.

\begin{definition}[Packing linear program]\label{def:packing-LP}
	A \emph{packing linear program} with \(n\) variables and \(m\) non-trivial constraints is of the form
	\[
		\begin{aligned}
			\max~ & c^{\top} x  \\
			      & Ax \leq b ; \\
			      & x \geq 0,
		\end{aligned}
	\]
	where \(c \in \mathbb{R} _{\geq 0}^n\), \(A \in \mathbb{R} _{\geq 0}^{m \times n}\), and \(b \in \mathbb{R} _{\geq 0}^m\).
\end{definition}

\begin{definition}[Covering linear program]\label{def:covering-LP}
	A \emph{covering linear program} with \(n\) variables and \(m\) non-trivial constraints is of the form
	\[
		\begin{aligned}
			\min~ & c^{\top} x  \\
			      & Ax \geq b ; \\
			      & x \geq 0,
		\end{aligned}
	\]
	where \(c \in \mathbb{R} _{\geq 0}^n\), \(A \in \mathbb{R} _{\geq 0}^{m \times n}\), and \(b \in \mathbb{R} _{\geq 0}^m\).
\end{definition}

\begin{definition}[Mixed packing and covering linear program]\label{def:mixed-packing-covering-LP}
	A \emph{mixed packing and covering linear program} with \(n\) variables and \((m_1 + m_2)\) non-trivial constraints is in the form of feasibility problem of inequalities\footnote{This is general enough since we can put the objective (either maximization or minimization) into the constraints.}
	\[
		\begin{aligned}
			\max~ & 0           \\
			      & Ax \leq b ; \\
			      & Cx \leq d ; \\
			      & x \geq 0,
		\end{aligned}
	\]
	where \(A \in \mathbb{R} _{\geq 0}^{m_1 \times n}\), \(b \in \mathbb{R} _{\geq 0}^{m_1}\), \(C \in \mathbb{R} _{\geq 0}^{m_2 \times n}\), and \(d \in \mathbb{R} _{\geq 0}^{m_2}\).
\end{definition}

An \emph{explicit} linear program is one in which all the data is specified in terms of the variables, constraints and the non-negative entries in the inequalities. We will assume that the representation is given in the sparse form, and we will usually let \(n\) denote the number of variables, \(m\) the number of constraints, and \(N\) the number of non-zeroes in the constraint matrices. Hence, the input size of an explicit specified positive linear program is \(\Theta (N + m + n)\).

On the other hand, an \emph{implicit} linear program is one in which we have an underlying data such as a graph or geometric object and a linear program formulation that is specified implicitly based on that data. They can have exponentially many variables or constraints and often they can be solved efficiently or approximately via Ellipsoid method. In such settings, we will not write down the explicit linear program, and we will seek iterative methods that have a running time that depends on the size of the original input that defines the linear program.

\begin{note}
	The disadvantage of the explicit linear program is that it has many variables and constraints and typical explicit linear program solvers use memory that is quadratic in the number of variables to do matrix operations while solving the linear program via the simplex or interior point method, and this makes the memory a bottleneck.
\end{note}

\subsection{Examples}
Let's see some examples of \hyperref[def:packing-LP]{packing linear programs}.

\begin{eg}[Maximum weight bipartite matching]
	Given a bipartite graph \(G = (V, E)\) with edge capacity \(w \colon E \to \mathbb{R} _{\geq 0}\), the following linear program gives an exact solution to the maximum weight bipartite matching:
	\[
		\begin{aligned}
			\max~ & \sum_{e \in E} w(e) x_e                                \\
			      & \sum_{e \in \delta (u)} x_e \leq 1 & \forall u \in V ; \\
			      & x_e \geq 0                         & \forall e \in E.
		\end{aligned}
	\]
	One can see that this is a \hyperref[def:packing-LP]{packing linear program} with \(N = 2 \lvert E \rvert \), \(n = \lvert E \rvert \), and \(m = \lvert V \rvert \).\footnote{Note that the notation of \(n, m\) is confusing here from that of typical graph usage since we are using \(n\) for the number of variables and \(m\) for the number of constraints.} Although this linear program is mainly used for bipartite matching, it is sometimes useful even for non-bipartite graphs. One can show that its integrality gap is \(2 / 3\) for general graphs.
\end{eg}

\begin{eg}[Tree packing]
	Recall the \hyperref[eq:tree-packing-LP]{tree packing linear program}, where the goal is to pack \hyperref[def:spanning-tree]{spanning trees} into the capacity of a given graph \(G = (V, E)\) with edge capacity \(w \colon E \to \mathbb{R} _{\geq 0}\):
	\[
		\begin{aligned}
			\max~ & \sum_{T \in \mathcal{T} _G} y_T                                 \\
			      & \sum_{T \ni e} y_T \leq w(e)    & \forall e \in E;              \\
			      & y_T \geq 0                      & \forall T \in \mathcal{T} _G.
		\end{aligned}
	\]
	This is an implicit \hyperref[def:packing-LP]{packing linear program} with an exponential number of variables and \(m = \lvert E \rvert \) constraints.
\end{eg}

\begin{eg}[Maximum multi-commodity flow]
	We considered the maximum multi-commodity flow problem as a relaxation for the \hyperref[prb:multi-min-cut]{multi-min-cut} problem. Given a graph \(G = (V, E)\) with non-negative edge capacities \(w \colon E \to \mathbb{Q} _{\geq 0}\) and \(k\) source-sink pairs \(\{ (s_i, t_i) \} _{i=1}^{k}\). Let \(\mathcal{P} _{s_i, t_i}\) denote the set of all \(s_i\)-\(t_i\) paths in \(G\). We have a variable \(x_P\) for each path \(P \in \bigcup_{i=1}^{k} \mathcal{P} _i\) to denote the amount of flow that is being routed on path \(P\), and we want to maximize the total flow sent between all pairs:
	\[
		\begin{aligned}
			\max~ & \sum_{i=1}^{k} \sum_{P \in \mathcal{P} _{s_i, t_i}} x_P                                                         \\
			      & \sum_{i=1}^{k} \sum_{\substack{P \in\mathcal{P} _{s_i, t_i}                                                     \\ P \ni e}} x_P \leq w(e) & \forall e \in E ;                                  \\
			      & x_P \geq 0                                                  & P \in \bigcup_{i=1}^{k} \mathcal{P} _{s_i, t_i} .
		\end{aligned}
	\]
	This is also an implicit \hyperref[def:packing-LP]{packing linear program}. We can also write it as an explicit linear program with variables \(f(e, i)\), which is the amount of flow on \(e\) for pair \(i\):
	\[
		\begin{aligned}
			\max~ & \sum_{i=1}^{k} \left[ \sum_{e \in \delta ^+ (s_i) } f(e, i) - \sum_{e \in \delta ^- (s_i) } f(e, i) \right]                                           \\
			      & \sum_{i=1}^{k} f(e, i) \leq w(e)                                                                            & \forall e \in E                         \\
			      & \sum_{e \in \delta ^-(u)} f(e, i) = \sum_{e \in \delta ^+(u)} f(e, i)                                       & u \neq s_i, t_i \text{ for } i \in [k].
		\end{aligned}
	\]
	We see that the flow conservation constraints make this linear program not positive anymore.
\end{eg}

\begin{eg}[Rectangle independent set]
	\[
		\begin{aligned}
			\max~ & \sum_{r} x_r                            \\
			      & \sum_{R \ni p} x_R \leq 1 & \forall p .
		\end{aligned}
	\]
\end{eg}

As for the \hyperref[def:covering-LP]{covering linear programs}, we have the following. First, we introduce the well-known \hyperref[prb:set-cover]{set cover} problem and its special case called \hyperref[prb:vertex-cover]{vertex cover}.

\begin{problem}[Set cover]\label{prb:set-cover}
Given a collection of \(n\) subsets \(\mathcal{S} = \{ S_i \} _{i=1}^{n}\) of \(S\) such that each \(\mathcal{S} _i \subseteq S\) is of size \(m\) and has a cost \(c_i\). The \emph{set cover} asks to find a minimum cost sub-collection of \(\mathcal{S} \) whose union is \(S\).
\end{problem}

\begin{problem}[Vertex cover]\label{prb:vertex-cover}
Given a graph \(G = (V, E)\) with edge capacity \(w \colon E \to \mathbb{R} _{\geq 0}\), the \emph{vertex cover} is a special case of \hyperref[prb:set-cover]{set cover} with \(S = E\) and \(\mathcal{S} \coloneqq \{ S_v = \delta (v)\} _{v \in V}\) with unit cost for each \(S_v\).
\end{problem}

\begin{eg}[Set cover]
	Define an \(m \times n\) matrix \(A\) where \(A_{ij} = 1\) if \(j \in S_i\), \(0\) otherwise. Then, the linear program relaxation for \hyperref[prb:set-cover]{set cover} is of the form
	\[
		\begin{aligned}
			\min~ & c^{\top} x                      \\
			      & Ax \geq 1 ;                     \\
			      & x_v \geq 0  & \forall  v \in V.
		\end{aligned}
	\]
	This is clearly a \hyperref[def:covering-LP]{covering linear program}. It is known that the integrality gap is \(O(\log m)\), and the integrality gap for \hyperref[prb:vertex-cover]{vertex cover} is \(2\).
\end{eg}

\begin{eg}[Covering integer program]
	The \emph{covering integer program} (CIP) generalizes \hyperref[prb:set-cover]{set cover}. A CIP is essentially an integer version of a \hyperref[def:covering-LP]{covering linear program}:
	\[
		\begin{aligned}
			\min~ & c^{\top} x           \\
			      & Ax \geq b ;          \\
			      & x \in \{ 0, 1 \} ^n.
		\end{aligned}
	\]
	We work with the linear program relaxation which is a \hyperref[def:covering-LP]{covering} problem. When \(A\), \(b\) are arbitrary, the integrality gap of the natural linear program for CIP can be as large as \(m\). One needs to add knapsack-cover inequalities to strengthen the linear program and then the linear program becomes much more complex but nevertheless one can solve it fast using the multiplicative weight update methods. See~\cite{chekuri2019approximating} and references to work on CIPs.
\end{eg}

Finally, we see some examples on the \hyperref[def:mixed-packing-covering-LP]{mixed packing and covering linear programs}.

\begin{eg}[Covering integer program with bounded constraints]
	One can also add bound constraint on CIPs where we can take a variable \(x_j\) only up to a bound \(d_i\). Then we get packing constraints and the linear program becomes a \hyperref[def:mixed-packing-covering-LP]{mixed packing and covering linear program}, albeit in a simple form
	\[
		\begin{aligned}
			\min~ & c^{\top} x                    \\
			      & Ax \geq b ;                   \\
			      & x \leq d ;                    \\
			      & x \in \mathbb{Z} _{\geq 0}^n.
		\end{aligned}
	\]
	Again, the natural linear program is not good in terms of integrality gap and one needs knapsack cover inequalities.
\end{eg}

\begin{eg}[Maximum concurrent flow]
	We have seen the \hyperref[eq:non-uniform-sparsest-cut-LP-dual]{maximum concurrent flow linear program} in the context of the \hyperref[prb:non-uniform-sparsest-cut]{sparsest cut problem}. If we guess \(\lambda \), then it becomes a constant, and we get a \hyperref[def:mixed-packing-covering-LP]{mixed packing and covering linear program}. To do this, we can do a binary search for \(\lambda \).
	\[
		\begin{aligned}
			\max~ & \lambda                                                                                                                 \\
			      & \sum_{P \in \mathcal{P} _{s_i, t_i}} y_P \geq \lambda D_i    & \forall i = 1, \dots , k;                                \\
			      & \sum_{i=1}^{k} \sum_{\substack{P \in \mathcal{P} _{s_i, t_i}                                                            \\ P \ni e}} y_P \leq c(e) & \forall e \in E;                                         \\
			      & y_P \geq 0                                                   & \forall P \in \bigcup_{i=1}^{k} \mathcal{P} _{s_i, t_i}; \\
			      & \lambda \geq 0,
		\end{aligned}
	\]
\end{eg}

\begin{eg}[Densest subgraph]
	Another interesting \hyperref[def:mixed-packing-covering-LP]{mixed packing and covering linear program} comes up when solving the densest subgraph problem~\cite{bahmani2014efficient}.
\end{eg}

\subsection{Known Approximation Results for Explicit Packing Linear Programs}

Finally, we list out several results known for explicit positive linear programs. We refer to~\cite{quanrud2019fast,wang2017fast} for a good overview. Here, we state a few high-level results for sequential models.
\begin{itemize}
	\item For \hyperref[def:mixed-packing-covering-LP]{mixed packing and covering linear programs}, there is a multiplicative weight update based algorithm that runs in deterministic \(O(N \log N / \epsilon ^2)\) time and outputs a \((1 - \epsilon )\)-approximate feasible solution~\cite{young2014nearly}. This implies a \((1 - \epsilon )\)-approximation algorithm for \hyperref[def:packing-LP]{packing} and \hyperref[def:covering-LP]{covering} in \(O(N \log N / \epsilon ^2)\) time.
	\item For \hyperref[def:packing-LP]{packing}, there is a randomized algorithm that yields a \((1 - \epsilon )\)-approximation feasible solution in \(O(N \log N \log (1 / \epsilon ) / \epsilon )\) time~\cite{allen2015nearly}. A similar time bound as above for \hyperref[def:covering-LP]{covering} is also known~\cite{wang2015unified}.
\end{itemize}

\section{Multiplicative Weight Updates Method}
The multiplicative weight update (MWU) method is a meta-algorithm that has many applications and arises in a number of areas, even though the central analysis is very similar. The survey of Arora, Hazan, and Kale~\cite{arora2012multiplicative} outlines the utility of seeing the general approach. Here, we are interested in applications of MWU to solving positive linear programs.Although one can derive some of these via the expert framework outlines in~\cite{arora2012multiplicative}, there are some limitations as well as the overhead of introducing the expert framework.

Even within offline optimization, the use of MWU is varied, and it is not always easy to figure out the similarities and differences even among closely related papers. We will follow the ideas and exposition from~\cite{chekuri2015multiplicative}.

\subsection{Convex Optimization, Lagrangian Relaxation, and Soft-Max Trick}
While our goal is to solve linear programs, we start by considering convex programs.

\begin{intuition}
	It makes some ideas and notations cleaner and more general, and also indicates where we use linearity and positivity when we ant refined results.
\end{intuition}

Let \(f \colon \mathbb{R} ^n \to \mathbb{R} \) be a concave function and let \(g_i \colon \mathbb{R} ^n \to \mathbb{R} \) for all \(i \in [m]\) be a set of convex functions, and let \(P\) be a convex set in \(\mathbb{R} ^n\).

\begin{note}
	We will assume that all functions are continuous and bounded in the domain of interest, and ignore some of the technical difficulties that arise when considering unbounded functions and domains.
\end{note}

Even though we typically use \(P\) as the non-negative orthant in \(\mathbb{R} ^n\), we can usually restrict the domain to be a box of sufficiently large size in the non-negative orthant. Consider the following optimization problem which generalizes the setting of \hyperref[def:packing-LP]{packing linear programs}:
\[
	\begin{aligned}
		\max~ & f(x)                                 \\
		      & g_i (x) \leq 1 & \forall i \in [m] ; \\
		      & x\in P.
	\end{aligned}
\]
This is a constrained convex program. Unconstrained convex minimization is itself a non-trivial problem and typically addresses it via gradient descent and other methods. Here, we use it as a modeling tool for explanation purposes and eventually work with linear functions. The first idea is to view the multiple constraints as a single constraint \(h(x) - 1 \leq 0\) where \(h(x) = \max _{i \in [m]} g_i(x)\):
\[
	\begin{aligned}
		\max~ & f(x)                               \\
		      & \max _{i \in [m]} g_i (x) \leq 1 ; \\
		      & x\in P.
	\end{aligned}
\]
From convex analysis, we know that \(h(x) = \max _{i \in [m]}g_i(x)\) is still convex, but not smooth anymore.

\begin{intuition}
	One can make it smooth by considering the soft-max, a smooth-out version of \(\max\).
\end{intuition}

Specifically, for a parameter \(\eta > 0\), the soft-max function is defined as
\[
	g_\eta (x)
	\coloneqq \operatorname{soft-max}_\eta (\{ g_i (x)\}_{i=1}^{m} )
	\coloneqq \frac{1}{\eta } \ln ( \sum_{i=1}^{m} \exp (\eta (g_i(x) - 1 ) )).
\]
Again, from convex analysis, \(g_\eta (x)\) is convex.

\begin{intuition}
	The soft-max function tries to blow-up the maximum value among its argument by some (large) positive number \(\eta \) and applying the exponential function to which. By doing so, if the maximum among its arguments stands out and dominates others.
\end{intuition}

It turns out that the error incurs by replacing maximum with the soft-max function is small:

\begin{claim}
	For all \(x \in \mathbb{R} ^n\), we have
	\[
		\max _{i \in [m]} g_i(x)
		\leq g_\eta (x)
		\leq \max _{i \in [m]} g_i(x) + \frac{\ln m}{\eta }.
	\]
\end{claim}

Then, if we choose \(\eta = \ln m / \epsilon \), we get a smooth proxy for \(h(x) - 1\) that has an additive error of at most \(\epsilon \). Specifically, considering
\[
	\begin{aligned}
		\max~ & f(x)                                     \\
		      & g_\eta (x) \leq 1 + \frac{\ln m}{\eta }; \\
		      & x \in P,
	\end{aligned}
\]
we see that since \(h(x) \leq g_\eta (x) \leq h(x) + \ln m / \eta = h(x) + \epsilon \), we have \(g_\eta (x) - \epsilon \leq h(x) \leq g_\eta (x)\).

\begin{remark}
	We can compress multiple constraints into one with a small loss of factor (\(1 + \epsilon \)).
\end{remark}

Another standard approach in constrained continuous (convex and non-linear) optimization is to replace a constrained problem by its Lagrangian relaxation. In the context of the above, we would replace the constrained optimization problem by the following concave optimization problem:
\[
	\max_{x \in P} f_w(x) \coloneqq \max _{x \in P} f(x) - \sum_{i=1}^{m} w_i(g_i(x) - 1),
\]
where \(w_1, w_2, \dots , w_m \geq 0\) are Lagrange multipliers.

\begin{intuition}
	We penalize the objective for violating the constraints.
\end{intuition}

The following weak duality claim is easy to check.

\begin{claim}
	For any non-negative weights \(w_1, w_2, \dots , w_m \geq 0\), \(\max _{x \in P} f_w(x) \geq \OPT\) where \(\OPT\) is the optimum value of the original optimization problem. Thus, \(\min _{w\geq 0} \max _{x \in P} f_w(x) \geq \OPT\).
\end{claim}

For convex optimization problems, the duality theorem says that under some mild conditions (e.g., \href{https://en.wikipedia.org/wiki/Slater%27s_condition}{Slater conditions}), strong duality holds, i.e., \(\min _{w \geq 0} \max _{x \in P} f_w(x) = \OPT\).

Although Lagrangian relaxation is nice, it is not obvious how to find good weights that lead to good upper and lower bounds. One can combine the idea of soft-max function and Lagrangian relaxation to reduce the problem to have a single constraint, namely \(g_\eta (x) \leq 0\). Thus, we want to solve the problem \(\max _{x \in P} f(x)\) such that \(g_\eta (x) \leq 0\). One can use some gradient descent-like approaches for solving this problem and the gradient of \(g_\eta (x)\) is relevant in this context. We will elaborate on this later.

Finally, motivated by the Lagrangian relaxation approach, we assume that we have a black-box oracle for the following problem for any given non-negative weights \(w_1, w_2, \dots , w_m \geq 0\):
\begin{equation}\label{eq:MWU-oracle}
	\begin{aligned}
		\max~ & f(x)                                                \\
		      & \sum_{i=1}^{m} w_i g_i(x) \leq \sum_{i=1}^{m} w_i ; \\
		      & x\in P.
	\end{aligned}
\end{equation}
When \(f\) and \(g_i\)'s are positive linear functions and \(P\) is the non-negative orthant, there is a simple and easy oracle for the above problem.

\begin{remark}
	If we're considering positive linear program, then the above is trivial. Consider the following positive \hyperref[def:packing-LP]{packing linear program}:
	\[
		\begin{aligned}
			\max~ & \sum_{j=1}^{n} c_j x_j                                 \\
			      & \sum_{j=1}^{n} A_{ij} x_j \leq 1 & \forall i \in [m] ; \\
			      & x\geq 0.
		\end{aligned}
	\]
	The above oracle in this case is just
	\[
		\begin{aligned}
			\max~ & \sum_{j=1}^{n} c_j x_j                                               \\
			      & \sum_{i=1}^{m} w_i \sum_{j=1}^{n} A_{ij} x_j \leq \sum_{i=1}^{m} w_i \\
			      & x\geq 0;
		\end{aligned}
		\iff \begin{aligned}
			\max~ & \sum_{j=1}^{n} c_j x_j                                                              \\
			      & \sum_{j=1}^{n} \left( \sum_{i=1}^{m} w_i A_{ij} \right) x_j \leq \sum_{i=1}^{m} w_i \\
			      & x\geq 0,
		\end{aligned}
	\]
	which is a fractional knapsack problem, hence greedy applies.
\end{remark}

\subsection{Continuous Time Multiplicative Weight Updates Algorithm}
We consider an adaptive iterative algorithm that uses a notion of time that goes from \(0\) to \(1\).

\begin{intuition}
	We're essentially solving a sequence of Lagrange relaxation and penalize the violated constraints more along the way.
\end{intuition}

From a discrete point of view, we can think of taking \(T\) steps of size \(\delta = 1 / T\) where \(\delta \) is small. Thus, after \(\ell \) steps, the time is \(t = \ell \delta \). Later, for ease of analysis, we will make this a continuous process.

The algorithm will maintain a set of non-negative weights \(w_i(t)\), one for each constraint. The non-negative weights represent the relative importance of each constraint at time \(t\). Initially, we have no information, so all weights are equally important, hence we set \(w_i(0) = 1\).

\begin{intuition}
	An alternative view is to think of the normalized weights \(w_i(t) / \sum_{k=1}^{m} w_k(t)\) as a probability distribution on the constraints.
\end{intuition}

In each step, we will use the oracle to compute a solution \(v(t)\) for the Lagrangian relaxation defined by the weights, i.e.,
\[
	\begin{aligned}
		v(t) \coloneqq \argmax~ & f(y)                                                      \\
		                        & \sum_{i=1}^{m} w_i(t) g_i(y) \leq \sum_{i=1}^{m} w_i(t) ; \\
		                        & y\in P.
	\end{aligned}
\]
Clearly, \(f(v(t)) \geq \OPT\) since we are solving a relaxation. At the end, we will take an average of convex combination of these solutions and concavity of \(f\) will guarantee that we are doing well on the objective value. We think \(x(t)\) the current solution as \(\int_{0}^{t} v(t) \,\mathrm{d}t \) as the accumulation of the solutions computed so far with initial point at \(t=0\) being \(x(0) = 0\).

The main question is to deal with the constraint violation. For this, we will maintain the invariant that \(w_i(t) = \exp (\eta \int_{0}^{t} g_i(v(t)) \,\mathrm{d}t )\), which tries to keep track of the violation of constraint \(i\).

\begin{algorithm}[H]\label{algo:MWU-discrete-uniform}
	\DontPrintSemicolon{}
	\caption{Multiplicative Weight Update Uniform Step}
	\KwData{An objective \(f\), constraints \(\{ g_i \} _{i=1}^{m}\), feasible set \(P\), step size \(\delta \), \(\eta \)}
	\KwResult{A solution \(x_{\text{out} }\)}
	\SetKwFunction{Oracle}{\hyperref[eq:MWU-oracle]{Oracle}}
	\BlankLine

	\For(){\(i = 1, \dots , m\)}{
		\(w_i(0) \gets 1\)\;
	}
	\(x(0) = 0 \)\;
	\(t = 0\)\;
	\;
	\While(){\(t < 1\)}{
	\(v(t) \gets\)\Oracle{\(f\), \(\{ g_i \} _{i=1}^{m}\), \(\{ w_i(t) \} _{i=1}^{m}\), \(P\)}\;
	\For(){\(i = 1, \dots , m\)}{
		\(w_i(t+\delta ) \gets w_i(t) \cdot e^{\eta \delta g_i(v (t))}\)\;
	}
	\(x (t+\delta ) \gets x (t) + \delta v (t)\)\;
	\(t \gets t + \delta \)\;
	}
	\Return{\(x(1)\)}\;
\end{algorithm}

Suppose we make \(\delta \to 0\), then we obtain a continuous time algorithm whose analysis can be done simply and is illuminating. For that, we set up a differential equation on the evolution of the weights. From
\[
	w_i(t+\delta )
	= w_i(t) \exp (\eta \delta g_i(v(t))),
\]
we see that
\[
	\frac{\mathrm{d}w_i(t)}{\mathrm{d}t}
	= \lim_{\delta \to 0} \frac{w_i(t+\delta ) - w_i(t)}{\delta }
	= \lim_{\delta \to 0} \frac{w_i(t) (\exp (\eta \delta g_i(v(t))) - 1)}{\delta }
	= w_i(t) \eta g_i(v(t))
\]
by using the fact that \(e^x -1 \to x\) when \(x \to 0\). The final return is then
\[
	x_{\text{out} }
	= \int_{0}^{1} v(t) \,\mathrm{d}t .
\]
This results in the following continuous version of \autoref{algo:MWU-discrete-uniform}.

\begin{algorithm}[H]\label{algo:MWU-continuous}
	\DontPrintSemicolon{}
	\caption{Multiplicative Weight Update Continuous}
	\KwData{An objective \(f\), constraints \(\{ g_i \} _{i=1}^{m}\), feasible set \(P\), step size \(\delta \), \(\eta \)}
	\KwResult{A solution \(x_{\text{out} }\)}
	\SetKwFunction{Oracle}{\hyperref[eq:MWU-oracle]{Oracle}}
	\BlankLine

	\For(){\(i = 1, \dots , m\)}{
		\(w_i(0) \gets 1\)\;
	}
	\;
	\While(){\(t = 0, \dots , 1\) continuously}{
	\(v(t) \gets\)\Oracle{\(f\), \(\{ g_i \} _{i=1}^{m}\), \(\{ w_i(t) \} _{i=1}^{m}\), \(P\)}\;
	\For(){\(i = 1, \dots , m\)}{
		Evolve \(w_i(t)\) as \(\mathrm{d} w_i(t) / \mathrm{d} t = w_i(t) \eta g_i(v(t))\)\;
	}
	}
	\Return{\(\int_{0}^{t} v(t) \,\mathrm{d}t \)}\;
\end{algorithm}

The4 following is one reason to choose the time interval to be \([0, 1]\) to naturally obtain a convex combination of solutions as the output.

\begin{lemma}\label{lma:MWU-continuous-OPT}
	In \autoref{algo:MWU-continuous}, if \(f\) is concave, then \(f(x_{\text{out} }) \geq \OPT\).
\end{lemma}
\begin{proof}
	From the concavity of \(f\), we have
	\[
		f(x_{\text{out} })
		= f \left( \int_{0}^{1} v(t) \,\mathrm{d}t \right)
		\geq \int_{0}^{1} f(v(t)) \,\mathrm{d}t
		\geq \int_{0}^{1} \mathsf{OPT} \,\mathrm{d}t
		= \mathsf{OPT},
	\]
	where \(f(v(t)) \geq \OPT\) since \(v(t)\) is an optimum solution to a Lagrangian relaxation of the original problem.
\end{proof}

We now express the constraint values in terms of the weights.

\begin{lemma}\label{lma:MWU-continuous-constraint}
	In \autoref{algo:MWU-continuous}, For each \(i \in [m]\), we have \(g_i(x_{\text{out} }) \leq \ln w_i(1) / \eta \).
\end{lemma}
\begin{proof}
	By concavity, we have
	\[
		\begin{split}
			g_i(x_{\text{out} })
			= g_i \left( \int_{0}^{1} v(t) \,\mathrm{d}t \right)
			 & \leq \int_{0}^{1} g_i(v(t)) \,\mathrm{d}t                                                           \\
			 & = \int_{0}^{1} \frac{1}{\eta } \frac{1}{w_i(t)} \frac{\mathrm{d}w_i (t)}{\mathrm{d}t} \,\mathrm{d}t
			= \frac{1}{\eta } \left( \ln w_i(1) - \ln w_i(0) \right)
			= \frac{\ln w_i(1)}{\eta },
		\end{split}
	\]
	where we know that \(\mathrm{d} w_i(t) / \mathrm{d} t = \eta w_i(t) g_i(v(t))\) in \autoref{algo:MWU-continuous}.
\end{proof}

While we do not have a direct way to bound \(w_i(1)\), but we can bound \(w_i(1)\) by \(\sum_{i=1}^{m} w_i(1)\). This is a very crude bound, but since we will see how the total sum of weights evolves, it turns out to be enough.

\begin{lemma}\label{lma:MWU-continuous-weight}
	In \autoref{algo:MWU-continuous}, for all \(t\), \(\sum_{i=1}^{m} w_i(t) \leq e^{\eta t} \sum_{i=1}^{m} w_i(0)\). Hence, \(\sum_{i=1}^{m} w_i(1) \leq m e^\eta \).
\end{lemma}
\begin{proof}

\end{proof}

Putting everything together, we have the following.

\begin{theorem}\label{thm:MWU-continuous}
	The output \(x_{\text{out} }\) of \autoref{algo:MWU-continuous} satisfies the following properties:
	\begin{enumerate}[(i)]
		\item\label{thm:MWU-continuous-i} \(f(x_{\text{out} }) \geq \OPT\);
		\item\label{thm:MWU-continuous-ii} \(g_i(x_{\text{out} }) \leq 1 + \ln m / \eta \) for all \(i \in [m]\).
	\end{enumerate}
	If there is a point \(x_0 \in P\) such that \(g_i(x_0) = 0\) for all \(i \in [m]\), then \(x_{\text{out} }^{\prime} = \theta x_{\text{out} } + (1 - \theta ) x_0\) is feasible, in that \(g_i(x_{\text{out} }^{\prime} ) \leq 1\) for all \(i \in [m]\) and \(x_{\text{out} }^{\prime} \in P\).
\end{theorem}
\begin{proof}
	\autoref{thm:MWU-continuous-i} is shown in \autoref{lma:MWU-continuous-OPT}. For \autoref{thm:MWU-continuous-ii}, we see that by combining \autoref{lma:MWU-continuous-constraint} and \autoref{lma:MWU-continuous-weight}, we have
	\[
		g_i(x_{\text{out} })
		\leq \frac{1}{\eta } \ln (w_i(1))
		\leq \frac{1}{\eta } \ln (\sum_{i=1}^{m} w_i(1))
		\leq \frac{1}{\eta } \ln (e^\eta m)
		\leq 1 + \frac{\ln m}{\eta }.
	\]
	The second part is easy to verify.
\end{proof}

To obtain a \((1 + \epsilon )\)-approximation in terms of the constraint violation, we can choose \(\eta = \ln m / \epsilon \).

\begin{remark}
	The second part of \autoref{thm:MWU-continuous} allows us to obtain a feasible solution \(x_{\text{out} }^{\prime} \) since \(x_{\text{out} }\) from \autoref{algo:MWU-continuous} might not be feasible. The impact of using \(x_{\text{out} }^{\prime} \) instead of \(x_{\text{out} }\) in terms of the objective function \(f\) is less easy to see. However, for non-negative linear objectives, we see that \(f(x_{\text{out} })^{\prime} \geq \theta f(x_{\text{out} })\), and thus we get an approximation to the objective while making the solution feasible.
\end{remark}

The analysis in \autoref{thm:MWU-continuous} does not use any property of \(f, g\) other than continuity. In particular, it does not use non-negativity of \(g\). Hence, the analysis can also be applied to have constraints of the form \(h_i(x) \geq 1\)  where \(h_i\) is concave which can be modeled as a convex constraint \(-h_i(x) + 2 \leq 1\).

Our analysis relied on keeping track of the total weight \(w(t)\) carefully. In some works, the potential function \(\phi (t) = \ln w(t) / \eta \) or proxies for it are used. Potential functions can be powerful tools that can yield strong results though they can sometimes be mysterious~\cite{bansal2017potential}.