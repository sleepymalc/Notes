\lecture{3}{3 Sep.\ 11:00}{Global Min-Cut with Tree Packing}
\section{Min-Cuts}
Consider the following famous problems about min-cuts.

\begin{problem}[\(s\)-\(t\) min-cut]\label{prb:s-t-min-cut}
Given a graph \(G = (V, E)\) with edge weight \(c \colon E \to \mathbb{R} \), the \emph{\(s\)-\(t\) min-cut} problem aims to find \(\min _{S \subseteq V \colon s \in S, t \in V\setminus S} c(\delta (S))\).
\end{problem}

\begin{problem}[Global min-cut]\label{prb:global-min-cut}
Given a graph \(G = (V, E)\) with edge weight \(c \colon E \to \mathbb{R} \), the \emph{global min-cut} problem aims to find \(\min _{\varnothing \neq S \subsetneq V} c(\delta (S))\).
\end{problem}

In what follows, we will simply use \hyperref[prb:global-min-cut]{min-cut} to refer to \autoref{prb:global-min-cut} problem. A naive way to solve it is to first fix one end \(s \in V\), and compute the \hyperref[prb:s-t-min-cut]{\(s\)-\(t\) min-cut} for all \(t \in V - s\). Fairly recent work shows how one can do it with only poly-log max-flow computations.

Over the years, several very different algorithmic approaches have been developed for these problems. One of the surprising ones is based on MA-orderings~\cite{nagamochi1992computing}, which is a combinatorial \(O(mn + n^2 \log n)\) time algorithm that does not rely on flow at all.\footnote{This approach generalizes to symmetric submodular functions.} Another approach is to combine several flow computations together via the push-relabel method~\cite{hao1994faster}, which also works for directed graphs. Karger developed elegant and powerful random contraction based algorithms for \hyperref[prb:global-min-cut]{global min-cuts}~\cite{karger1995random}, leading to many results. Two notable consequences are the following.

\begin{theorem}[\cite{karger1996new}]\label{thm:Karger-min-cut}
	There is a randomized algorithm that runs in \(O(n^2 \log n)\) time and outputs the \hyperref[prb:global-min-cut]{min-cut} with high probability.\footnote{This is a Monte-Carlo algorithm, so we cannot guarantee that the min-cut found is the correct one.}
\end{theorem}

The following is a consequence of Karger's contraction algorithm~\cite{karger1995random}.

\begin{theorem}[Approximate min-cut~\cite{karger2000minimum}]\label{thm:number-approximate-min-cut}
	The number of \hyperref[def:approximate-min-cut]{\(\alpha \)-approximate min-cuts} in a graph is at most \(O(n^{2 \alpha })\).
\end{theorem}

Karger then developed another approach via \hyperref[prb:TP]{tree packing} to obtain a randomized near-linear time algorithm for \hyperref[prb:global-min-cut]{min-cut}. He also was able to refine the bound on \hyperref[thm:number-approximate-min-cut]{approximate min-cuts} via this approach.

\begin{theorem}[\cite{karger2000minimum}]\label{thm:Karger-tree-packing-based-min-cut}
	There is a randomized algorithm that runs in time \(O(m \log ^3 n)\) and outputs the \hyperref[prb:global-min-cut]{min-cut} with high probability.
\end{theorem}

While the random contraction based algorithm is taught quite frequently due to its elegance and simplicity, the \hyperref[prb:TP]{tree packing} approach is more technical. More recently, the \hyperref[prb:TP]{tree packing} approach has led to several new results, which we now discuss.

\subsection{Tree Packing-Based Algorithm for Min-Cut}
Recall that \autoref{col:Tutte-Nash-Williams-min-cut}, which gives
\[
	\frac{\lambda (G)}{2} \frac{n}{n-1}
	\leq \tau _{\text{frac} }(G)
	\leq \lambda (G).
\]
Intuitively speaking, even if we can compute \(\tau _{\text{frac} }(G)\) exactly, the above only gives a \(2\)-approximation to \(\lambda (G)\). However, this already leads a crucial observation as follows.

\begin{intuition}
	On average, each tree can't cross the \hyperref[prb:global-min-cut]{min-cut} more than twice.
\end{intuition}

To formalize the above intuition, consider the following definition.

\begin{definition}[Respecting]\label{def:respect}
	Let \(T = (V, E_T)\) be a \hyperref[def:spanning-tree]{spanning tree} and \((S, V\setminus S)\) be a cut. The for an integer \(h \geq 1\), we say \(T\) is \emph{\(h\)-respecting} w.r.t.\ \(S\) if \(\lvert E_T \cap \delta (S) \rvert \leq h\).
\end{definition}

\begin{figure}[H]
	\centering
	\incfig{respect}
	\caption{The \hyperref[def:spanning-tree]{spanning tree} \(T\) is shown in red edges. \(T\) is \hyperref[def:respect]{\(3\)-rspecting} the cut \((S, V \setminus S)\).}
\end{figure}

We can now formalize the intuition in \autoref{lma:tree-packing-min-cut}.

\begin{lemma}\label{lma:tree-packing-min-cut}
	Suppose \(\{ y_T \} _{T \in \mathcal{T} _G}\) is a \((1 - \epsilon )\)-approximate \hyperref[prb:TP]{tree packing} of \(G\), and \(\delta (S)\) is a \hyperref[prb:global-min-cut]{min-cut} of \(G\). Let \(\ell _T \coloneqq \lvert E_T \cap \delta (S) \rvert \) be the number of edges of \(T\) that cross the cut \(S\). Furthermore, let \(p_T = y_T / \sum_{T\in \mathcal{T} _G} y_T \) and \(q \coloneqq \sum_{T \colon \ell (T) \leq 2} p_T \). Then,
	\[
		q \geq \frac{1}{2} \left( 3 - \frac{2}{1 - \epsilon } \left( 1 - \frac{1}{n} \right) \right) .
	\]
	In particular, if \(\epsilon = 0\), then \(1 \geq 1 / 2 + 1 / n\), and if \(\epsilon < 1 / 5\), then \(q > 1 / 4\).
\end{lemma}
\begin{proof}
	From the assumption \(\sum_{T \in \mathcal{T} _G} y_T \geq (1 - \epsilon ) \tau _{\text{frac} } (G)\). With \autoref{col:Tutte-Nash-Williams-min-cut}, i.e., \(\tau _{\text{frac} }(G) \geq n \lambda (G) / 2 (n-1)\), we have
	\[
		\sum_{T \in \mathcal{T} _G} y_T
		\geq (1 - \epsilon ) \frac{n}{n-1} \frac{\lambda (G)}{2}.
	\]
	Now, let \(S \subseteq V\) be a \hyperref[prb:global-min-cut]{min-cut}, we have \(1 = \sum_{T \in \mathcal{T} _G} p(T) = \sum_{T \colon \ell (T) \leq 2} p_T + \sum_{T \colon \ell (T) \geq 3} p_T\). We observe that
	\begin{itemize}
		\item each tree \(T\) with \(\ell (T) \geq 3\) uses up at least \(3\) edges from \(\delta (S)\); while
		\item each tree \(T\) with \(\ell (T) \leq 2\) uses up at least \(1\) edge from \(\delta (S)\).
	\end{itemize}
	Since the total capacity of \(\delta (S)\) is \(\lambda (G)\), and the \hyperref[prb:TP]{tree packing} solution is valid, we have
	\[
		\sum_{T \colon \ell (T) \leq 2} y_T + 3 \sum_{T \colon \ell (T) \geq 3} y_T
		\leq \lambda (G)
		\implies q + 3 (1 - q)
		\leq \frac{\lambda (G)}{\sum_{T \in \mathcal{T} _G} y_T}
		\leq \frac{2}{1 - \epsilon } \left( 1 - \frac{1}{n} \right) ,
	\]
	where the last inequality follows from the very first inequality we have derived.
\end{proof}

\begin{remark}
	\autoref{lma:tree-packing-min-cut} states that if the \hyperref[prb:TP]{tree packing} is sufficiently good, then a constant fraction of the trees in the \hyperref[prb:TP]{packing} will cross the \hyperref[prb:global-min-cut]{min-cut} at most twice.
\end{remark}

Now, we're ready to see Karger's algorithm for \hyperref[prb:global-min-cut]{min-cut}~\cite{karger2000minimum}. However, the original algorithm was more involved since at that time, there was no near-linear time approximation algorithm for \hyperref[prb:TP]{tree packing}, so he used a form of sparsification and then applied an approximation \hyperref[prb:TP]{tree packing} algorithm on the sparsified graph which is quite a feat. In our case, recall that following.

\begin{prev}
	\autoref{thm:approximate-TP} states that we can compute a \((1 - \epsilon )\)-approximate \hyperref[prb:TP]{tree packing} of \(G\), given by \(\{ y_T \} _{T \in \mathcal{T} _G}\), in \(O(m \log ^3 n / \epsilon ^2)\) time.
\end{prev}

By black-boxing this near-linear time \hyperref[prb:TP]{tree packing} algorithm, consider the following.

\begin{algorithm}[H]\label{algo:Karger-tree-packing-based-min-cut}
	\DontPrintSemicolon{}
	\caption{\hyperref[prb:TP]{Tree Packing}-Based \hyperref[prb:global-min-cut]{Min-Cut} Algorithm~\cite{karger2000minimum,chekuri2017near}}
	\KwData{A connected graph \(G = (V, E)\) with edge weight \(c \colon E \to \mathbb{R} \), \(\epsilon _0 \in < 1 / 5\)}
	\KwResult{A cut \(S\)}
	\SetKwFunction{CQ}{\hyperref[thm:approximate-TP]{Approximate-Tree-Packing}}
	\BlankLine

	\(\{ y_T \} _{T \in \mathcal{T} _G} \gets\)\CQ{\(G\), \(c\), \(\epsilon _0\)}\label{algo:Karger-tree-packing-based-min-cut-TP}\Comment*[r]{\(O(m \log ^3 n)\)}
	Sample a tree \(T\) with probability \(p_T = y_T / \sum_{T \in \mathcal{T} _G} y_T\)\label{algo:Karger-tree-packing-based-min-cut-sample}\;
	Find the cheapest cut \((S, V\setminus S)\) in \(G\) such that \(T\) is \hyperref[def:respect]{\(2\)-respecting} w.r.t.\ \(S\)\label{algo:Karger-tree-packing-based-min-cut-find-respect-cut}\;
	\Return{\(S\)}\;
\end{algorithm}

Firstly, we see that \autoref{algo:Karger-tree-packing-based-min-cut} admits the following.

\begin{lemma}
	\autoref{algo:Karger-tree-packing-based-min-cut} outputs the \hyperref[prb:global-min-cut]{min-cut} of \(G\) with probability at least \(1 / 4\).
\end{lemma}
\begin{proof}
	It's immediate from \autoref{lma:tree-packing-min-cut}.
\end{proof}

To boost the success probability, we can simply repeat the last two steps (\autoref{algo:Karger-tree-packing-based-min-cut-sample}, \autoref{algo:Karger-tree-packing-based-min-cut-find-respect-cut}) \(\Theta (\log n)\) times, which results in a success probability to at least \(1 - 1 / n^c\) for any constant \(c\). To analyze the running time, a key ingredient is \autoref{algo:Karger-tree-packing-based-min-cut-find-respect-cut}. Karger showed that one can implement \autoref{algo:Karger-tree-packing-based-min-cut-find-respect-cut} via a clever dynamic programming coupled with link-cut tree data structure:

\begin{theorem}[\cite{karger2000minimum}]\label{thm:Karger-tree-packing-based-min-cut-find-respect-cut}
	Given a graph \(G = (V, E)\) and a \hyperref[def:spanning-tree]{spanning tree} \(T = (V, E_T)\). There is a deterministic algorithm that computes a minimum cut \((S, V \setminus S)\) such that \(T\) is \hyperref[def:respect]{\(2\)-respecting} w.r.t.\ \(S\) in \(O(m \log ^2 n)\) time.
\end{theorem}

We can now prove \autoref{thm:Karger-tree-packing-based-min-cut}.

\begin{proof}[Proof of \autoref{thm:Karger-tree-packing-based-min-cut}]
	Since \autoref{algo:Karger-tree-packing-based-min-cut-TP} takes \(O(m \log ^3 n)\) for \(\epsilon _0\) being a constant, and observe that once the approximated \hyperref[prb:TP]{tree packing} \(\{ y_T \} _{T \in \mathcal{T} }\) is computed, we can reuse them and apply the repetition for \autoref{algo:Karger-tree-packing-based-min-cut-sample} and \autoref{algo:Karger-tree-packing-based-min-cut-find-respect-cut} to boost the probability of success. With \(\Theta (\log n)\) repetitions, we obtain an \(O(m \log ^3 n)\) time algorithm as desired with the running time guaranteed by \autoref{thm:Karger-tree-packing-based-min-cut-find-respect-cut}.
\end{proof}

\subsection{Bounding the Number of Approximate Min-Cuts}
As hinted in \autoref{thm:number-approximate-min-cut}, we're now interested in how many distinct \hyperref[prb:global-min-cut]{min-cuts} can an undirected graph have. The following theorem was shown a long time ago:

\begin{theorem}[\cite{dinitz1976structure}]
	The number of distinct \hyperref[prb:global-min-cut]{min-cuts} in an undirected graph is at most \(\binom{n}{2}\).
\end{theorem}

\begin{eg}[Cycle]
	The worst case example is an \(n\)-cycle \(C_n\).
\end{eg}

\begin{remark}
	All the \hyperref[prb:global-min-cut]{min-cuts} of a graph can be represented in a nice and compact data structure called the cactus (cactus representation), which was also shown in~\cite{dinitz1976structure}.
\end{remark}

In contrast, for \hyperref[prb:s-t-min-cut]{\(s\)-\(t\) min-cuts}, it can be exponentially many in \(n\).

\begin{eg}
	Consider the following multi-highway-like graph, which has exponentially many \hyperref[prb:s-t-min-cut]{\(s\)-\(t\) min-cuts} since if we choose one of the road section in each line of the road, it'll be a \hyperref[prb:s-t-min-cut]{\(s\)-\(t\) min-cut}.
	\begin{figure}[H]
		\centering
		\incfig{highway}
	\end{figure}
\end{eg}

Hence, we're interested in the number of \hyperref[def:approximate-min-cut]{\(\alpha \)-approximation min-cut}:

\begin{definition}[Approximate min-cut]\label{def:approximate-min-cut}
	For \(\alpha \geq 1\) an \emph{\(\alpha \)-approximate min-cut} is a cut \((S, V \setminus S)\) such that \(c(\delta (S)) \leq \alpha \lambda (G)\).
\end{definition}

Recall \autoref{thm:number-approximate-min-cut}, where Karger used \hyperref[prb:TP]{tree packing} to prove that the number of \hyperref[def:approximate-min-cut]{\(\alpha \)-approximation min-cuts} is at most \(O_\alpha (n^{\lfloor 2\alpha \rfloor })\). Before we prove \autoref{thm:number-approximate-min-cut}, we recall some basic facts from linear programming.

\begin{prev}
	A solution \(x^{\ast} \) to a linear program which has \(n\) non-trivial constraints means that the support size of \(x\) is at most \(n\), i.e., \(x_i > 0\) for at most \(n\) many \(i\)'s.
\end{prev}

We're now ready to prove \autoref{thm:number-approximate-min-cut}, which is based on~\cite{chekuri2020lp}.

\begin{proof}[Proof of \autoref{thm:number-approximate-min-cut}]
	Consider an optimum fraction \hyperref[prb:TP]{tree packing} solution \(\{ (T, y_T^{\ast} ) \}_{T \in \mathcal{T} _G} \). In the \hyperref[pf:col:Tutte-Nash-Williams]{proof} of \autoref{col:Tutte-Nash-Williams}, where we define the fractional \hyperref[prb:TP]{tree packing} linear program, we know that there are only \(m\) non-trivial constraints, hence there are only \(m\) many \(T\)'s such that \(y^{\ast} _T > 0\).

	Consider an \hyperref[def:approximate-min-cut]{\(\alpha \)-approximate min-cut} \(S \subseteq V\), and let \(h = \lceil 2 \alpha \rceil \). Now, let \(q_{h, \alpha }\) be the fraction of \hyperref[prb:TP]{tree packing} that \hyperref[def:respect]{\(h\)-respects} \(S \subseteq V\), i.e.,
	\[
		q_{h, \alpha }
		\coloneqq \sum_{T \colon \ell (T) \leq h} p_T.
	\]
	Using a similar analysis as the one in \autoref{lma:tree-packing-min-cut}, we can argue that
	\[
		q_{h, \alpha }
		\geq \frac{1}{h} (1 - (2\alpha - \lfloor 2 \alpha \rfloor )) \left( 1 - \frac{1}{n} \right) .
	\]
	The main intuition is the following:

	\begin{intuition}
		Say at least one tree in the \hyperref[prb:TP]{packing} \hyperref[def:respect]{\(h\)-respects} the cut (which is the case). Then, the total number of \hyperref[def:approximate-min-cut]{\(\alpha \)-approximate min-cuts} is at most \(m \cdot n^h \leq m \cdot n ^{\lfloor 2 \alpha \rfloor }\).
	\end{intuition}

	But we can do better by noticing that \(q_{h, \alpha } > 0\) is a fixed constant for any fixed \(\alpha \). Suppose \(N\) is the number of \hyperref[def:approximate-min-cut]{\(\alpha \)-approximate min-cuts}. For any fixed \hyperref[def:approximate-min-cut]{\(\alpha \)-approximate min-cut}, \(q_{h, \alpha }\) fraction of the \hyperref[prb:TP]{tree packing} is \hyperref[def:respect]{\(h\)-respecting} w.r.t.\ the cut. Consider the following question:

	\begin{problem*}
		Fix a single tree \(T\), how many distinct cuts are there such that \(T\) \hyperref[def:respect]{\(h\)-respects} w.r.t.?
	\end{problem*}
	\begin{answer}
		We can remove at most \(h\) edges from \(T\) to create at most \(h + 1\) components and combine these components into two sides of a cut, hence, each tree \(T\) correspond to at most
		\[
			2^{h+1} \binom{n-1}{h} \leq 2^{h+1} n^h
		\]
		cuts.
	\end{answer}

	Thus, the number of \hyperref[def:approximate-min-cut]{\(\alpha \)-approximate min-cuts} is at most \(2^{h+1} n^h / q_{h, \alpha }\).
\end{proof}