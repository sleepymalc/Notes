\lecture{3}{3 Sep.\ 11:00}{Min-Cut}
\section{Min-Cuts}
\begin{problem}[\(s\)-\(t\) min-cut]\label{prb:s-t-min-cut}
Given a graph \(G = (V, E)\) with edge weight \(c \colon E \to \mathbb{R} \), the \emph{\(s\)-\(t\) min-cut} problem aims to find \(\min _{S \subseteq V \colon s \in S, t \in V\setminus S} c(\delta (S))\).
\end{problem}

\begin{problem}[Global min-cut]\label{prb:global-min-cut}
Given a graph \(G = (V, E)\) with edge weight \(c \colon E \to \mathbb{R} \), the \emph{global min-cut} problem aims to find \(\min _{\varnothing \neq S \subseteq V} c(\delta (S))\).
\end{problem}

A naive way to solve the \hyperref[prb:global-min-cut]{global min-cut} problem is to first fix one end \(s \in V\), and compute the \hyperref[prb:s-t-min-cut]{\(s\)-\(t\) min-cut} for all \(t \in V - s\).

\begin{itemize}
	\item Nagemathi-Ibache [NI] \(O(mn + n^2 \log n)\).
	\item Karger: randomized contraction \(O(n^2 \log n)\).
	\item Karger: \(O(m \log ^3 n)\), tree packing.
	\item Isolating cuts.
\end{itemize}

Recall that \autoref{col:Tutte-Nash-Williams-min-cut} gives that
\[
	\frac{\lambda (G)}{2} \frac{n}{n-1}
	\leq \tau _{\text{frac} }(G)
	\leq \lambda (G).
\]
\begin{intuition}
	On average, each tree can't cross the min-cut more than twice.
\end{intuition}

\begin{definition}[Respecting]\label{def:respecting}
	Let \(T\) be a \hyperref[def:spanning-tree]{spanning tree} and \((S, V\setminus S)\) be a cut. \(T\)  is \emph{\(h\)-respecting} w.r.t.\ \(S\) if \(\lvert E_T \cap \delta (S) \rvert \leq h\).
\end{definition}

Assuming that we have access of the algorithm in \autoref{thm:approximate-TP}.

\begin{notation}
	Fix some min-cut \((S, V-S)\) for the tree \(T\), let \(\ell _T \coloneqq \lvert E_T \cap \delta (S) \rvert \).
\end{notation}

\begin{lemma}\label{lma:}
	Suppose \(y_T\), \(T \in \mathcal{T} _G\) is a \((1 - \epsilon )\)-approximate \hyperref[prb:TP]{tree packing}. Let \(p_T = y_T / \sum_{T\in \mathcal{T} _G} y_T \).  Let \(q \coloneqq \sum_{T \colon \ell (T) \leq 2} p_T \). Then,
	\[
		q \geq \frac{1}{2} \left( 3 - \frac{2}{1 - \epsilon } \left( 1 - \frac{1}{n} \right) \right) .
	\]
	In particular, if \(\epsilon = 0\), \(1 \geq 1 / 2 + 1 / n\), and if \(\epsilon < 1 / 5\), then \(q > 1 / 4\).
\end{lemma}
\begin{proof}
	Consider
	\[
		\sum_{T \colon \ell (T) \leq 2} y_T + 3 \sum_{T \colon \ell (T) \geq 3} y_T
		\leq \lambda (G)
		\implies q + 3 (1 - q)
		\leq \frac{\lambda (G)}{\sum_{T \in \mathcal{T} _G} y_T}
		\leq \frac{2}{1 - \epsilon }
	\]
	where \(\sum_{T \in \mathcal{T} _G} y_T \geq (1 - \epsilon ) \tau _{\text{frac} }(G) \geq (1 - \epsilon ) \lambda (G) / 2 \).
\end{proof}

\begin{theorem}
	Given a graph \(G = (V, E)\) and a \hyperref[def:spanning-tree]{spanning tree} \(T = (V, E^{\prime} )\). The min-cut \((S, V - S)\) in \(G\) that crosses \(T\) less than \(2\) times can be found in \(O(m \log ^2 n)\) time.
\end{theorem}

\begin{enumerate}
	\item Compute a \((1 - \epsilon _0)\)-approximation \hyperref[prb:TP]{tree packing} for \(\epsilon _0 < 1 / 5\) and get \(\{ y_T \} _{T \in \mathcal{T} _G}\).
	\item Pick a tree at random where \(pvT = y_T / \sum_{T \in \mathcal{T} _G} y_T \).
	\item Use DP+ data structure to find the cheapest-cut that \hyperref[def:respecting]{\(2\)-respecting} \(T\).
\end{enumerate}

We see that since the first step takes \(O(m \log ^3 n)\), and assuming that it's possible to sample (step 2), then the algorithm gives the min-cut with probability at least \(1 / 4\). By repeating step 2 and 3 \(O(\log n)\) times, the whole algorithm takes \(O(m \log ^3 n)\) and will success with high probability.

Given a graph \(G = (V, E)\), how many different min-cuts are there? For \hyperref[prb:s-t-min-cut]{\(s\)-\(t\) min-cuts}, it can be exponential. On the other hand, for \hyperref[prb:global-min-cut]{global min-cut}, it's at most \(\binom{n}{2}\).

We're also interested in \(\alpha \)-approximation min-cut for \(\alpha \geq 1\), i.e., a cut \((S, V \setminus S)\) such that \(c(\delta (S)) \leq \alpha \lambda (G)\).

\begin{theorem}
	The number of \(\alpha \)-approximation min-cuts is at most \(O_\alpha (n^{\lfloor 2\alpha \rfloor })\).
\end{theorem}
\begin{proof}
	If \(q\) is the fraction of \hyperref[def:respecting]{\(2\)-respecting} trees w.r.t.\ \(S\), then \(q \geq 1 / 2 + 1 / n\). Let \(h \coloneqq \lfloor 2 \alpha \rfloor \), and
	\[
		q_{h, \alpha }
		\coloneqq \sum_{T \colon \ell (T) \leq h} p_T,
	\]
	and after the same calculation as before, we have
	\[
		q_{h, \alpha }
		\geq \frac{1}{h} (1 - (2\alpha - \lfloor 2 \alpha \rfloor )) \left( 1 - \frac{1}{n} \right) .
	\]
	Consider an optimal solution \(y^{\ast} _T\), which has the support with size smaller than \(m\). Fix \(T\) such that \(y^{\ast} _T > 0\). It's only possible to create cuts by first removing \(h\) edges, i.e., \(\binom{n-1}{h} 2^{h+1} \times m \leq n^{\lfloor 2\alpha \rfloor } 2^{\lfloor 2\alpha \rfloor }m\).
\end{proof}