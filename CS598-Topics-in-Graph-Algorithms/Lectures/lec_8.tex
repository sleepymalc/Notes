\lecture{8}{19 Sep.\ 11:00}{Randomized Rounding for Sparsest Cut and Expanders}
\subsubsection{Randomized Rounding Algorithm with Metric Embeddings}
Now, we see how to utilize \autoref{thm:Borugain*} to design a randomized rounding algorithm for the \hyperref[prb:non-uniform-sparsest-cut]{non-uniform sparsest cut problem}. The main theorem is the following.

\begin{theorem}\label{thm:Borugain-graph}
	Let \(G = (V, E)\) be a graph. Suppose any finite metric induced by edge lengths on \(E\) can be embedded into \hyperref[def:l1-metric]{\(\ell _1\)} with \hyperref[def:distortion]{distortion} \(\alpha \), then the integrality gap of the \hyperref[eq:non-uniform-sparsest-cut-LP-primal]{linear program} for the \hyperref[prb:non-uniform-sparsest-cut]{non-uniform sparsest cut} is at most \(\alpha \) for any instance on \(G\).
\end{theorem}
\begin{proof}
	Let \((x, y)\) be a feasible fraction solution of the \hyperref[eq:non-uniform-sparsest-cut-LP-primal]{linear program relaxation}, and let \(d\) be the metric induced by edge lengths given by \(x\). Let \(\lambda \) be the value of the solution, i.e.,
	\[
		\lambda = \frac{\sum_{uv \in E} c(uv) d(u, v)}{\sum_{i=1}^{k} D_i d(s_i, t_i)}.
	\]
	Since \(d\) can be embedded into \hyperref[def:l1-metric]{\(\ell _1\)} with \hyperref[def:distortion]{distortion} at most \(\alpha \), and any \hyperref[def:l1-metric]{\(\ell _1\) metric} is in the \hyperref[def:cut-cone]{cut cone} from \autoref{lma:l1-metric-iff-cut-cone}, it implies that there are scalars \(z_S\), \(S \subseteq V\), such that for all \(u, v \in V\),
	\[
		\frac{1}{\alpha } \sum_{S \subseteq V}  z_S d_S(u, v)
		\leq d(u, v)
		\leq \sum_{S \subseteq V} z_S d_S(u, v).
	\]
	Without loss of generality, we assume that the embedding is a \hyperref[def:contraction]{contraction}. Then, we have
	\[
		\begin{split}
			\lambda
			= \frac{\sum_{uv \in E} c(uv) d(u, v)}{\sum_{i=1}^{k} D_i d(s_i, t_i)}
			 & \geq \frac{1}{\alpha } \frac{\sum_{uv \in E} c(uv) \sum_{S \subseteq V} z_S d_S(u, v)}{\sum_{i=1}^{k} D_i \sum_{S \subseteq V} d_S(s_i, t_i)} \\
			 & = \frac{1}{\alpha } \frac{\sum_{S \subseteq V} z_S c(\delta _G(S))}{\sum_{S \subseteq V} z_S D(\delta _H(S))  }
			\geq \frac{1}{\alpha } \min _{S \subseteq V} \frac{c(\delta _G(S))}{D(\delta _H(S)) }.
		\end{split}
	\]
	Hence, there is a cut whose \hyperref[def:sparsity]{sparsity} is at most \(\alpha \lambda \).
\end{proof}

\autoref{thm:Borugain-graph} shows that one of the cuts with \(z_S > 0\) has \hyperref[def:sparsity]{sparsity} at most \(\alpha \lambda \). Now, suppose we have an \hyperref[def:l1-metric]{\(\ell _1\)} embedding into \(h\)-dimensions, i.e., \(\mathbb{R} ^h\). Observe the following.

\begin{intuition}
	First, each dimension in \(\mathbb{R} ^h\) corresponds to a \hyperref[def:line-metric]{line} embedding, and each \hyperref[def:line-metric]{line} embedding is in the \hyperref[def:cut-cone]{cut cone} with only \(n-1\) cuts used to express it (recall \autoref{lma:l1-metric-iff-cut-cone}). Thus, given an \hyperref[def:l1-metric]{\(\ell _1\)} embedding into \(\mathbb{R} ^h\) with \hyperref[def:distortion]{distortion} \(\alpha \), we only need to try \(d(n-1)\) cuts and one of them will be guaranteed to have \hyperref[def:sparsity]{sparsity} at most \(\alpha \lambda \).
\end{intuition}

\autoref{algo:sparsest-cut-embedding} exploits this intuition.

\begin{algorithm}[H]\label{algo:sparsest-cut-embedding}
	\DontPrintSemicolon{}
	\caption{\hyperref[prb:non-uniform-sparsest-cut]{Non-Uniform Sparsest Cut} via Embedding}
	\KwData{A supply graph \(G = (V, E)\) with edge capacity \(c \colon E \to \mathbb{R} _+\), demand graph \(H = (V, F)\) with demand capacity \(D\colon F \to \mathbb{R} _+\)}
	\KwResult{The \hyperref[prb:non-uniform-sparsest-cut]{sparsest cut} \(S \subseteq V\)}
	\SetKwFunction{LPSolve}{LP-Solve}
	\SetKwFunction{SparsestCutLP}{\hyperref[eq:non-uniform-sparsest-cut-LP-primal]{Non-Uniform-Sparsest-Cut-LP}}
	\SetKwFunction{BourgainEmbed}{\hyperref[thm:Borugain*]{Bourgain-Embedding}}

	\BlankLine
	\((\{ x_e \} _{e\in E} , \{ y_i \}_{i=1}^{k} )\gets\)\LPSolve{\SparsestCutLP{\(G\), \(c\), \(H\), \(D\)}}\;
	\(f \gets\)\BourgainEmbed{\((V, d_x)\)}\Comment*[r]{\(f \colon V \to\mathbb{R} ^h\), \(h = O(\log ^2 n)\)}
	\For(){\(\ell = 1, \dots , h\)}{
	Sort \(\ell ^{\text{th} }\) coordinate of \(\{f(v)\}_{v \in V}\) as \(x_1^{(\ell )} \leq x_2^{(\ell )} \leq \dots \leq x_n^{(\ell )}\)\Comment*[r]{\(f(v)_{\ell } = x_i^{(\ell )}\) for some \(i\)}
	\For(){\(j = 1, \dots , n-1\)}{
	\(S_j^{(\ell )} \gets \{ v \in V \mid f(v)_{\ell } \leq x_j^{(\ell )} \} \)\;
	}
	}
	\(S \gets \argmin_{\ell \in [h], j \in [n-1]} c(\delta _G(S_j^{(\ell )})) / D(\delta _H(S_j^{(\ell )}))\)\;
	\Return{\(S\)}\;
\end{algorithm}

The guarantee of \autoref{algo:sparsest-cut-embedding} can be derived from \autoref{thm:Borugain-graph}.

\begin{theorem}\label{thm:sparsest-cut-embedding}
	\autoref{algo:sparsest-cut-embedding} outputs cuts of \hyperref[def:sparsity]{sparsity} at most \(\alpha \lambda ^{\ast} \),\footnote{\(\lambda ^{\ast} \) is the optimal solution of the \hyperref[eq:non-uniform-sparsest-cut-LP-dual]{dual linear program}.} in particular, it's a randomized \(O(\log k)\)-approximation algorithm for \hyperref[prb:non-uniform-sparsest-cut]{non-uniform sparsest cut}.
\end{theorem}
\begin{proof}
	From \autoref{thm:Borugain*}, \(f\) is a \hyperref[def:contraction]{contraction} and with \hyperref[def:distortion]{distortion} \(\alpha = O(\log k)\). From a similar argument as in \autoref{thm:Borugain-graph}, we see that for some \(z_S\), \(S \subseteq V\), we have
	\[
		\begin{split}
			\lambda ^{\ast}
			 & = \frac{\sum_{e \in E} c(e) x_e}{\sum_{i=1}^{k} D_i d_x(s_i, t_i)}                                                                                                                                                                                            \\
			 & \geq \frac{\sum_{uv \in E} c(uv) \lVert f(u) - f(v) \rVert _1}{\alpha \sum_{i=1}^{k} D_i \lVert f(s_i) - f(t_i) \rVert _1}                                                                                                                                    \\
			 & = \frac{1}{\alpha } \frac{\sum_{uv \in E} c(uv) \sum_{\ell =1}^{h} \lvert f(u)_{\ell } - f(v) _\ell  \rvert }{\sum_{i=1}^{k} D_i \sum_{\ell =1}^{h} \lvert f(s_i)_{\ell } - f(t_i)_{\ell }  \rvert }                                                          \\
			 & = \frac{1}{\alpha } \frac{\sum_{\ell =1}^{h} \sum_{j=1}^{n-1} \lvert x_{j+1}^{(\ell )} - x_{j}^{(\ell )}\rvert c(\delta _G(S_j^{(\ell )}))}{\sum_{\ell =1}^{h} \sum_{j=1}^{n-1} \lvert x_{j+1}^{(\ell )} - x_{j}^{(\ell )}\rvert D(\delta _H(S_j^{(\ell )}))}
			= \frac{1}{\alpha } \frac{\sum_{S \subseteq V} z_S c(\delta _G(S))}{\sum_{S \subseteq V} z_S D(\delta _H(S))}
			\geq \frac{1}{\alpha } \min _{S \subseteq V} \frac{c(\delta _G(S))}{D(\delta _H(S))},
		\end{split}
	\]
	where the second last equality follows from the fact that \(s_i, t_i \in V\) as well.
\end{proof}

\subsection{Line, \(\ell _1\), and Tree Embeddings, and State-of-the-Art}
\autoref{thm:Borugain*} shows that any finite metric space on \(n\) points embeds into \hyperref[def:l1-metric]{\(\ell _1\)} with \hyperref[def:distortion]{distortion} \(O(\log n)\). Here, we hint on the underlying algorithm of the construction: from \autoref{lma:l1-metric-iff-non-negative-combination-of-line-metric}, \hyperref[def:l1-metric]{\(\ell _1\)} embeddings are a non-negative combination of \hyperref[def:line-metric]{line} embeddings. A particular type of \hyperref[def:line-metric]{line} embedding is the following.

\begin{definition}[Fréchet embedding]\label{def:Fréchet-embedding}
	Let \((V, d)\) be a metric space and let \(S \subseteq V\). The \emph{Fréchet embedding} is a \hyperref[def:contraction]{contraction} \(f\colon V \to \mathbb{R} \) such that \(f(v) = d(S, v)\).
\end{definition}

Many results in embeddings into \(\ell _p\) spaces are based on using \hyperref[def:Fréchet-embedding]{Fréchet embeddings} in various clever and often highly non-trivial ways. In particular, \autoref{thm:Borugain*} is based on picking many random sets and combining the resulting \hyperref[def:Fréchet-embedding]{Fréchet embeddings}.

Now, we note that \autoref{thm:Borugain*} can also be derived via \hyperref[def:probabilistic-approximation]{probabilistic} tree embeddings because every \hyperref[def:dominating-tree-metric]{tree metric} embeds into \hyperref[def:l1-metric]{\(\ell _1\)} \hyperref[def:isometric-embedding]{isometrically}. For general metrics, tree embeddings provide a more constrained space while yielding the same worst-case \hyperref[def:distortion]{distortion}. However, one can ask if \hyperref[def:l1-metric]{\(\ell _1\)} embeddings provide better \hyperref[def:distortion]{distortion} for concrete graph classes. This is indeed the case.

\begin{eg}[Ring]
	Consider a ring graph (a cycle with capacities). One can prove that tree embeddings require a \hyperref[def:distortion]{distortion} \(2\), while the ring metric can be \hyperref[def:isometric-embedding]{isometrically embedded} into \hyperref[def:l1-metric]{\(\ell _1\)}. Thus, the flow-cut gap on ring is \(1\) which is not obvious.
\end{eg}

Rather than looking at \hyperref[def:distortion]{distortion}, one can ask about the flow-cut gap obtained via different embeddings for a particular graph class. First, recall the followings for the \hyperref[prb:non-uniform-sparsest-cut]{non-uniform sparsest cut}.

\begin{prev}[\autoref{thm:sparsest-cut-embedding}]
	The flow-cut gap in general undirected graphs is \(O(\log k)\).
\end{prev}

Additionally, for general graph, a lower bound is also known.

\begin{remark}[Lower bound]
	\hyperref[def:expander]{Expanders} give a lower bound on the flow-cut gap to be \(\Omega (\log k)\) even for \hyperref[prb:sparsest-cut]{uniform sparsest cut}~\cite{leighton1999multicommodity}.
\end{remark}

With these general bounds in mind, we now consider the flow-cut gap for planar graphs in particular.

\begin{eg}[Planar graph]
	There is a famous conjecture that the flow-cut gap in planar graphs is \(O(1)\)~\cite{gupta2004cuts}. Interestingly, for tree embeddings, there is a lower bound of \(\Omega (\log n)\) even on the special case of planar graphs called series parallel graphs. Hence, tree embeddings are not powerful enough to prove the conjecture.

	The best flow-cut gap so far is \(O(\sqrt{\log n} )\) via \hyperref[def:l1-metric]{\(\ell _1\)} embeddings, thus separating the general graph case from the planar graph case. For series parallel graphs, we know that the flow-cut gap is a tight bound of \(2\) and establishing this tight bound took a fair amount of work.

	For \hyperref[prb:sparsest-cut]{uniform sparsest cut}, the flow-cut gap in planar graphs is \(O(1)\)~\cite{klein1993excluded}. One can show a tight connection between embeddability into \hyperref[def:l1-metric]{\(\ell _1\)} and flow-cut gap~\cite{gupta2004cuts}.
\end{eg}

On the other hand, we can ask for a better approximation guarantee. First, note the following.

\begin{note}
	Approximating the \hyperref[prb:non-uniform-sparsest-cut]{non-uniform sparsest cut problem} is not the same as establishing the flow-cut gap as the flow-cut gap relies on the \hyperref[eq:non-uniform-sparsest-cut-LP-primal]{linear program relaxation}.
\end{note}

\begin{problem*}\label{prb:SDP-sparsest-cut}
	Can we obtain a better approximation than \(O(\log k)\) for \hyperref[prb:non-uniform-sparsest-cut]{non-uniform sparsest cut}?
\end{problem*}
\begin{answer}
	Yes! By using semi-definite programming based relaxation, one can obtain an \(O(\sqrt{\log n} )\)-approximation for \hyperref[prb:sparsest-cut]{uniform sparsest cut}~\cite{arora2009expander} and also the \hyperref[prb:product-instance-of-sparsest-cut]{product instances}. Based on this, an \(O(\sqrt{\log n} \log \log n )\)-approximation for \hyperref[prb:non-uniform-sparsest-cut]{non-uniform sparsest cut} is achieved~\cite{arora2005euclidean,arora2007frechet}.\footnote{There was a conjecture that the SDP based relaxation would yield an \(O(1)\)-approximation, but it was shown that the integrality gap is essentially close to \(\Omega (\sqrt{\log n} )\).}
\end{answer}

\subsection{Node Capacities\footnote{We refer to the \href{https://courses.grainger.illinois.edu/cs598csc/fa2024/Notes/lec-sparsest-cut.pdf}{note} for further information on further generalizations to node capacitated graph and directed graph.}}
A slight generalization of the \hyperref[prb:sparsest-cut]{uniform sparsest cut problem} is obtained by considering demands induced by weights on vertices, which we refer to \hyperref[prb:product-instance-of-sparsest-cut]{product instance}.

\begin{problem}[Product instance of sparsest cut]\label{prb:product-instance-of-sparsest-cut}
Given a graph \(G = (V, E)\) with edge capacity \(c \colon E \to \mathbb{R} _+\) and vertex weight \(\pi \colon V \to \mathbb{R} _+\). The \emph{product instance of sparsest cut problem} is the \hyperref[prb:non-uniform-sparsest-cut]{non-uniform sparsest cut problem} with demand \(D(u, v)\) for edge \(uv \in E\) setting to be \(\pi (u) \pi (v)\).
\end{problem}

\begin{notation}
	In this case, the dual flow instances are called \emph{product multi-commodity flow}.
\end{notation}

We see that the \hyperref[prb:product-instance-of-sparsest-cut]{product instances} indeed generalizes \hyperref[prb:sparsest-cut]{uniform sparsest cut}: If \(\pi (u) = 1\) for all \(u\), then this reduces to the \hyperref[prb:sparsest-cut]{uniform sparsest cut problem}.

\begin{remark}[Subset sparsity]
	If \(\pi (u) \in \{ 0, 1 \} \) for all \(u\), then we are focusing our attention on the \hyperref[def:sparsity]{sparsity} w.r.t.\ the set \(V^{\prime} = \{ v \in V \mid \pi (v) = 1 \} \). Since vertices with \(\pi (u) = 0\) play no role.
\end{remark}

\section{Expander and Well-Linked Set}
We now introduce \hyperref[def:expander]{expanders}~\cite{hoory2006expander}, which relate to the \hyperref[prb:non-uniform-sparsest-cut]{non-uniform sparsest cut} in an intricate way.

\begin{definition}[Expander]\label{def:expander}
	An \emph{expander} with parameter \(\alpha \) is a graph \(G = (V, E)\) such that for all \(S \subseteq V\) with \(\lvert S \rvert \leq \lvert V \rvert / 2\) such that \(\lvert \delta (S) \rvert \geq \alpha \lvert S \rvert\).
\end{definition}

\begin{definition}[Expansion]\label{def:expansion}
	The \emph{expansion} of a graph \(G = (V, E)\) is \(\min _{S\colon \lvert S \rvert \leq \lvert V \rvert / 2} \lvert \delta (S) \rvert / \lvert S \rvert \).
\end{definition}

Another related notion called \hyperref[def:conductance]{conductance} also has a nice connection to \hyperref[prb:sparsest-cut]{uniform sparsest cut}.

\begin{definition}[Conductance]\label{def:conductance}
	Given a graph \(G = (V, E)\) and a cut \(S \subseteq V\), the \emph{conductance} \(\phi (G)\) of \(G\) is defined as \(\lvert \delta (S) \rvert / \operatorname{vol}(S) \) where \(\operatorname{vol}(S) = \sum_{v \in S} \deg(v)\).
\end{definition}

It's clear that \(G\) is an \hyperref[def:expander]{\(\alpha \)-expander} if the \hyperref[def:expansion]{expansion} of \(G\) is at least \(\alpha \). Initially, \hyperref[def:expansion]{expansion} arose from the \hyperref[prb:graph-bisection]{graph bisection problem} as we have seen before. However, as we will soon see, \hyperref[def:expander]{expander} itself is quite interesting. Firstly, we note that \hyperref[def:expander]{expanders} do exist, and they are quite common.

\begin{lemma}\label{lma:expander-existence}
	There exists \hyperref[def:expander]{expanders} with degree \(3\) with \hyperref[def:expansion]{expansion} \(\alpha = \Omega (1)\). More specifically, a random \(3\)-regular graph is an \hyperref[def:expander]{expander} with high probability.
\end{lemma}

Hence, with \autoref{lma:expander-existence}, a natural way to generate an \hyperref[def:expander]{expander} is to first sample a random regular graph, then check its \hyperref[def:expansion]{expansion}. However, computing the \hyperref[def:expansion]{expansion} is \(\coNP\)-hard.

\subsection{Expansion and Conductance via Sparsest Cut}
The first connection of \hyperref[def:expander]{expander} to the \hyperref[prb:sparsest-cut]{uniform sparsest cut} is that the latter can be used to find the \hyperref[def:expansion]{expansion} of a graph. In particular, we see that when \(\lvert S \rvert \leq \lvert V \rvert / 2\), we have
\[
	\frac{1}{\lvert V \rvert } \frac{\lvert \delta (S) \rvert }{\lvert S \rvert }
	\leq \frac{\lvert \delta (S) \rvert }{\lvert S \rvert \lvert V \setminus S \rvert }
	\leq \frac{2}{\lvert V \rvert } \frac{\lvert \delta (S) \rvert }{\lvert S \rvert }.
\]

\begin{remark}\label{rmk:expansion-sparsity}
	The \hyperref[def:expansion]{expansion} and the \hyperref[prb:sparsest-cut]{uniform sparsest cut}'s \hyperref[def:sparsity]{sparsity} are within a factor of \(2\) of each other. Hence, while determining the \hyperref[def:expansion]{expansion} exactly is \(\coNP\)-hard, we can use \hyperref[prb:sparsest-cut]{uniform sparsest cut} to certify the \hyperref[def:expansion]{expansion} of a graph within a factor of \(2\).
\end{remark}

Sometimes it is useful to consider \hyperref[def:expansion]{expansion} with vertex weights \(w \colon V \to \mathbb{R} _+\) as well. In this case, the expansion is defined as \(\min _{S \colon w(S) \leq w(V) / 2} \lvert \delta (S) \rvert / w(S)\).

\begin{note}
	It's dual corresponds to the \emph{product multi-commodity flow instances} where \(\pi (v) = w(v)\).
\end{note}

As for \hyperref[def:conductance]{conductance}, it's easy to see that one can capture it by \hyperref[def:expansion]{expansion} via setting weights on vertices with \(w(v) = \deg (v)\), which further reduces to the \hyperref[prb:product-instance-of-sparsest-cut]{product instance of sparsest cut}.

\begin{claim}
	For regular graphs, \hyperref[def:expansion]{expansion} and \hyperref[def:conductance]{conductance} are the same.
\end{claim}

\subsection{Spectral Relaxation for Conductance}
In several applications it is important to obtain constant-degree \hyperref[def:expander]{expanders} with constant \hyperref[def:expansion]{expansion}.

\begin{intuition}
	The \(O(\sqrt{\log k} )\)-approximation \hyperref[algo:sparsest-cut-embedding]{algorithms} we saw are not useful in this regime.
\end{intuition}

It turns out that there is a very different method based on spectral graph theory that helps in this regime. For an undirected graph on \(n\) vertices, consider the Laplacian \(\mathcal{L} _G \coloneqq D - A\), where \(D\) is the diagonal degree matrix and \(A\) is the adjacent matrix. In particular, we have
\[
	(\mathcal{L} _G )_{ij}
	\coloneqq \begin{dcases}
		\deg (v_i), & \text{ if } i = j ;                           \\
		-1 ,        & \text{ if } i \neq j \text{ and } A_{ij} = 1; \\
		0,          & \text{ otherwise} .
	\end{dcases}
\]
Since \(\mathcal{L} _G\) is symmetric, all its eigenvalues are real. Moreover, this matrix is also positive semi-definite, hence all its eigenvalues are actually non-negative. Let \(0 = \lambda _1 \leq \lambda _2 \leq \dots \leq \lambda _n\) be its eigenvalues, then a well-known and famous result in spectral graph theory is the following.

\begin{theorem}[Cheeger's inequality]\label{thm:Cheeger-inequality}
	Given a graph \(G\) with \hyperref[def:conductance]{conductance} \(\phi (G)\),
	\[
		\frac{\lambda _2}{2}
		\leq \phi (G)
		\leq \sqrt{2 \lambda _2}.
	\]
\end{theorem}

\begin{remark}
	\(\lambda _2\) provides a constant factor approximation for the \hyperref[def:conductance]{conductance} when it is a constant!
\end{remark}

Since the \hyperref[def:expansion]{expansion} and \hyperref[def:conductance]{conductance} are related by the maximum degree, when the degree is a small constant, one can use \(\lambda _2\) to certify \hyperref[def:expansion]{expansion}. Due to its importance for certifying \hyperref[def:expansion]{expansion}/\hyperref[def:conductance]{conductance}, some use \(\lambda _2\) as the definition of \hyperref[def:expansion]{expansion} since it is computable and also helps in construction of \hyperref[def:expander]{expanders}.

\subsection{Expander Decomposition}
Even if a graph is not an \hyperref[def:expander]{expander} at first, it is possible to decompose a graph into smaller subgraphs such that each of them has good \hyperref[def:expansion]{expansion}/\hyperref[def:conductance]{conductance}. More explicitly, the goal is to remove as few edges as possible such that the graph decomposes into \hyperref[def:expander]{expanders}. This is useful since \hyperref[def:expander]{expander} leads to good algorithms. The question is the trade-off between the number of edges that we remove and the \hyperref[def:expansion]{expansion} that we can guarantee for the pieces.

\begin{notation}
	Technically the process works with \hyperref[def:conductance]{conductance}, but we use the terminology of \hyperref[def:expander-decomposition]{expander decomposition} for historical reasons.
\end{notation}

Since one is often interested in finding fast algorithms for \hyperref[def:expander-decomposition]{expander decomposition}, it is common to explore trade-offs between the quality of the \hyperref[def:conductance]{conductance} of the pieces and the number of edges.

\begin{definition}[Expander decomposition]\label{def:expander-decomposition}
	A \emph{\((\phi, \epsilon )\)-expander decomposition} for some \(\epsilon \in (0, 1)\) is a partition of the (connected) graph \(G = (V, E)\) into vertex induced subgraphs \(\{ G_i = G[V_i] \}_{i=1}^{h} \) such that each \(G_i\) has \hyperref[def:conductance]{conductance} at least \(\phi \), and the number of inter-cluster edges is at most \(\epsilon m\), i.e.,
	\[
		\frac{1}{2} \sum_{i=1}^{h} \lvert \delta (V_i) \rvert
		\leq \epsilon m.
	\]
\end{definition}