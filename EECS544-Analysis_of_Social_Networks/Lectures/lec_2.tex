\lecture{2}{1 Sep. 12:30}{Graph Theory}
\begin{definition}[Sub-graph]
	A \emph{sub-graph} \(\mathcal{G}^\prime\) is defined by a tuple \((\mathcal{V}^\prime, \mathcal{E}^\prime)\) such that
	\[
		\mathcal{V}^\prime \subseteq \mathcal{V}, \qquad \mathcal{E}^\prime\subseteq\mathcal{E}.
	\]
\end{definition}
\begin{note}
	Notice that the edges in \(\mathcal{E}^\prime\) need to be well-defined. Namely, for every
	\(e = (a,b) \in\mathcal{E}^\prime\), \(a, b\) need to also be in \(\mathcal{V}^\prime\).
\end{note}

\begin{definition}[Giant component]
	A sub-graph is called a \emph{giant component} of a graph if it is a connected component and with a significant number of nodes of the original graph.
\end{definition}

\section{Matrices Associated with Graphs}
\subsection{Adjacency Matrix}
\begin{definition}[Adjacency matrix of an undirected graph]
	The \emph{adjacency matrix of an undirected graph} is a matrix such that
	\[
		A_{ij} =\begin{dcases}
			1, & \text{ if }(i, j)\in\mathcal{E} \\
			0, & \text{ otherwise.}
		\end{dcases}
	\]
\end{definition}
\begin{remark}
	The adjacency matrix of an undirected graph must be symmetric by definition.
\end{remark}

\begin{definition}[Adjacency matrix of an directed graph]
	The \emph{adjacency matrix of a directed} graph is a matrix such that
	\[
		A_{ij} = \begin{dcases}
			1, & \text{ if }i\to j\in\mathcal{E} \\
			0, & \text{ otherwise.}
		\end{dcases}
	\]
\end{definition}
