\lecture{9}{29 Sep. 12:30}{Analyze HITS}
\begin{notation}
	We denote \texttt{auth(i)} as \texttt{a(i)}, and \texttt{hub(i)} as \texttt{h(i)}.
\end{notation}

\begin{itemize}
	\item Authority update:\[
		      \begin{split}
			      \mathtt{a_k(i)} & = \sum\limits_{j\in \mathcal{V}: j\to i\in \mathcal{E}}\mathtt{h_{k-1}(j)}\\
			      &=\sum\limits_{j\in\mathcal{V}\colon A_{ji} = 1}\mathtt{h_{k-1}(j)}\\
			      &=\sum\limits_{j\in\mathcal{V}}A_{ji}\mathtt{h_{k-1}(j)}\\
			      &= \left(A^{\top} \mathtt{h_{k-1}}\right)_i.
		      \end{split}
	      \]
	      We see that
	      \[
		      \mathtt{a_k} = A^{\top}\mathtt{h_{k-1}}.
	      \]
	\item Hub update:\[
		      \begin{split}
			      \mathtt{h_k(j)} &= \sum\limits_{j\in\mathcal{V}: i\to j\in\mathcal{E}} \mathtt{a_k(j)}\\
			      &= \sum\limits_{j\in\mathcal{V}} \mathtt{a_k(j)}\\
			      &= \sum\limits_{j\in\mathcal{V} }A_{ij}\mathtt{a_k(j)}\\
			      &= \left(A\mathtt{a_k}\right)_i.
		      \end{split}
	      \]
	      We see that
	      \[
		      \mathtt{h_{k}} = A \mathtt{a_{k}}.
	      \]
	      Furthermore,
	      \[
		      \mathtt{h_k} = A\mathtt{a_k}\implies \mathtt{h_k}= A\mathtt{a_k} = A(A^{\top}\mathtt{h_{k-1}}) = A A^{\top} \mathtt{h_{k-1}}.
	      \]
\end{itemize}
We ignore the normalization now, then
\[
	\mathtt{h_k} = A A^{\top}\mathtt{h_{k-1}} = (A A^{\top})A A^{\top} h_{k-2} = \cdots = (A A^{\top})^k \mathtt{h_0}.
\]
Hence, we need to study \(A A^{\top}\).
\begin{itemize}
	\item PSD
	\item Symmetric
\end{itemize}
\[
	0 \leq \lambda_1 \leq \lambda_2 \leq \cdots \leq \lambda_V.
\]

From the Spectral Theorem, let the eigenvector be \(u_1, \ldots , u_V\), then
\[
	\mathtt{h_0} = \sum\limits_{i=1}^{V} \alpha_i u_{i}
\]
where \(\alpha_{i}\) are unique and
\[
	\alpha_{i} = u_{i}^{\top}\mathtt{h_0} = \sum\limits_{j=1}^{V} \alpha_{j}u_{i}^{\top}u_{j} = \alpha_{i}
\]
where \(u_{i}\) are the eigenvectors of \(A A^{\top}\).

Moreover,
\[
	\begin{split}
		\mathtt{h_1} &= (A A^{\top})\mathtt{h_0} = \sum\limits_{i=1}^{V} \alpha_{i} \lambda_{i}u_{i}\\
		\vdots\\
		\mathtt{h_k} &= (A A^{\top})^k\mathtt{h_0} = \sum\limits_{i=1}^{V} \alpha_{i} \lambda_{i}^k u_{i}\\
	\end{split}
\]

With the above formula, we now consider
\[
	0 \leq \lambda_1 \leq \lambda_2 \leq \cdots \leq \underbrace{\lambda_V}_{\text{largest}}.
\]
\begin{enumerate}
	\item[Case i.] Unique largest eigenvalue: \(\lambda_V > \lambda_{V-1} \geq \cdots \geq \lambda_{1}\geq 0\), then the term \(\lambda^k\)
		will dominate, namely
		\[
			\mathtt{h_k} = \sum\limits_{i=1}^{V} \alpha_{i}\lambda_{i}^k u_{i}.
		\]
		Dividing both sides by \(\lambda_{V}^k\), we have
		\[
			\frac{\mathtt{h_{k}}}{\lambda_{V}^k} = \sum\limits_{i=1}^{V - 1}\alpha_{i}\left(\frac{\lambda_{i}}{\lambda_{V}}\right)^k u_{i} + \alpha_{V}u_{V}.
		\]
		Then since \(\lambda_{i}/\lambda_{V}< 1\) for \(1\leq i\leq V-1\), hence when \(k\to \infty \), all terms in the sum will converge to \(0\), leaving us with
		\[
			\lim_{k \to \infty} \frac{\mathtt{h_k}}{\lambda_V^k}= \alpha_V u_V.
		\]

		Since \(u_V\) is a non-zero vector, we consider the following two cases:
		\begin{enumerate}
			\item All terms are non-negative with one positive.
			\item At least one term is positive.
		\end{enumerate}
		Firstly, suppose \(\mathtt{h_0}= x>0\) such that \(u_{V}^{\top}x = \alpha_{V} >0\). Then we have
		\[
			\lim\limits_{h \to \infty} \frac{\left(A A^{\top}\right)^k x}{\lambda_{V}^k} = \alpha_{V}u_{V}.
		\]
		We see that the left-hand side is greater or equal to \(0\), and stays in positive for every \(k\implies \) the limit is also positive. Now, since \(\alpha_{V}>0\),
		we can conclude that \(u_{V}\geq 0\). In other words, \(\mathtt{h_0}\) can be any vector with all coefficients positive and then \(\alpha_{V} = u^{\top}_{V}\mathtt{h_0}>0\).

		At least one term is positive. \(\mathtt{h_0} = x > 0\) such that \(u_{v}^{\top}x>0\).

		\begin{remark}
			To rank nodes, all we need is the vector \(u_{V}\). And the limiting value of \(u_{V}\) is called the \textbf{hub score}.
		\end{remark}
	\item[Case ii.] Non-unique largest eigenvalues: For discussion, we assume that the largest eigenvalue has algebraic multiplicity of \(2\). Namely
		\[
			\lambda_V = \lambda_{V-1} \geq \cdots \geq \lambda_{1}\geq 0.
		\]
		Again, from the same discussion, we see that
		\[
			\mathtt{h_{0}} = \sum\limits_{i=1}^{V} \alpha_{i}u_{i}.
		\]
		Furthermore, we see that
		\[
			\frac{\left(A A^{\top}\right)^k h_0}{\lambda_{V}^k} = \sum\limits_{i=1}^{V - 2} \alpha_{i}\left(\frac{\lambda_{i}}{\lambda_{V}}\right)^k u_{i} + \alpha_{V-1}u_{V-1}+\alpha_{V} u_{V}.
		\]
		Since \(\lambda_{i}/\lambda_{V}< 1\) for \(1\leq i\leq V-2\), hence when \(k\to \infty \), all terms in the sum will converge to \(0\), leaving us with
		\[
			\lim_{k \to \infty} \frac{(A A^{\top})^{k}\mathtt{h_0}}{\lambda_V^k} = \alpha_{V-1}u_{V-1} + \alpha_V u_V.
		\]
		\begin{remark}
			Convergence still holds but it may not be \(u_{V}\).
		\end{remark}
\end{enumerate}



