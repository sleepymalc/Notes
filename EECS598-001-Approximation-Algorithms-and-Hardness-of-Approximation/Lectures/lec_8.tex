\lecture{8}{26 Sep. 10:30}{Local Search for \(k\)-Median}
We'll now see a completely different algorithm which solve \autoref{prb:k-median} with \((3 + \epsilon )\)-approximation ratio by local search.
\subsection{Local Search}
The idea is to iteratively improve the current solution. We first see the algorithm.

\begin{algorithm}[H]\label{algo:k-median-local-search}
	\DontPrintSemicolon
	\caption{\hyperref[prb:k-median]{\(k\)-Median} -- Local Search}
	\KwData{A set of clients \(P\subseteq X\), a set of (possible) facilities \(Q\subseteq X\), \(k\in \mathbb{\MakeUppercase{n}} \), width \(w\)}
	\KwResult{A set of opened facilities \(Q^\prime \subseteq Q\) with \(\left\vert Q ^\prime \right\vert = k \)}
	\BlankLine

	\(Q^\prime \gets \) arbitrary \(k\) centers in \(Q\)\;
	\While(\label{algo:k-median-local-search-while}){\(\exists Q^{\prime\prime}\) s.t. \(\left\vert Q^{\prime\prime} \right\vert = k \) and \(\mathop{\mathrm{cost}}(Q^{\prime\prime}) < \mathop{\mathrm{cost}}(Q^\prime )\) and \(\left\vert Q^\prime \triangle Q^{\prime\prime} \right\vert \leq w\)\footnote{The symmetric difference \(A \triangle B\) is defined as \(A \triangle B \coloneqq (A \setminus B) \cup (B\setminus A)\).}}{
		\(Q^\prime \gets Q^{\prime\prime}\)\;
	}
	\Return{\(Q^\prime \)}\;
\end{algorithm}

\begin{remark}[Runtime]
	In \autoref{algo:k-median-local-search-while}, each iteration in \autoref{algo:k-median-local-search} takes \((n+m)^{O(w)}\) time for \(n \coloneqq \left\vert P \right\vert \) and \(m \coloneqq \left\vert Q \right\vert \). But we have no control of how many iterations \autoref{algo:k-median-local-search} might take since we might decrease the cost by a little each time hence we might fall into exponentially many updates. To solve this, we can ask for
	\[
		\mathop{\mathrm{cost}}(Q^{\prime\prime}) < \mathcolor{red}{(1 - \epsilon )}\mathop{\mathrm{cost}}(Q^\prime )
	\]
	instead to make sure we decrease a reasonable amount each time, which guarantees that we can bound the number of iterations by
	\[
		\log_\frac{1}{1-\epsilon } \left( \frac{\mathop{\mathrm{cost}}(\text{starting } Q^\prime )}{\OPT} \right).
	\]
\end{remark}

\subsubsection{Analysis}
Firstly, note that for any solution \(Q^\prime \) output from \autoref{algo:k-median-local-search}, we have that there exists no \(Q^{\prime\prime}\) such that \(\left\vert Q^\prime \triangle Q^{\prime\prime} \right\vert \leq w\), \(\left\vert Q^{\prime\prime} \right\vert = k\) and \(\mathop{\mathrm{cost}}(Q^{\prime\prime}) < \mathop{\mathrm{cost}}(Q^\prime )\).

\begin{note}[Local optimum]
	We say this \(Q^\prime \) is a \emph{local optimum}.
\end{note}

Let \(Q^{\ast} \subseteq Q\) be the optimal solution, and without loss of generality (by duplicating facilities), assume \(Q^\prime \cap Q^{\ast} = \varnothing \). We define something called \hyperref[not:swap]{swap}.

\begin{notation}[Swap]\label{not:swap}
	A \emph{swap} \(S \subseteq Q^\prime \cup Q^{\ast} \) satisfies \(\left\vert S \cap Q^\prime  \right\vert = \left\vert S \cap Q^{\ast}  \right\vert \leq w / 2\).
\end{notation}

\begin{note}
	From local optimality of \(Q^\prime \), for any \hyperref[not:swap]{swap} \(S\), \(\mathop{\mathrm{cost}}(Q^\prime ) \leq \mathop{\mathrm{cost}}(Q^\prime \triangle S)\).
\end{note}

Now, consider constructing \hyperref[not:swap]{swaps} \(S_1, \ldots, S_t\) with weights \(p_1, \ldots , p_t \in \mathbb{\MakeUppercase{r}} ^+\) such that \(\mathop{\mathrm{cost}}(Q^\prime ) \leq \mathop{\mathrm{cost}}(Q^\prime \triangle S_i)\) for all \(i\), we have
\begin{equation}\label{eq:lec8}
	\sum_{i=1} ^t p_i \cdot (\mathop{\mathrm{cost}}(Q^\prime ) - \mathop{\mathrm{cost}}(Q^\prime \triangle S_i)) \leq 0.
\end{equation}
Our goal is to show that \autoref{eq:lec8} implies \(\mathop{\mathrm{cost}}(Q^\prime ) \leq \alpha\cdot \mathop{\mathrm{cost}}(Q^{\ast} )\) for some \(\alpha \in \mathbb{\MakeUppercase{r}} ^+\). To do this, we require the set of \hyperref[not:swap]{swaps} to have the following properties.

\begin{enumerate}[(a)]
	\item For all \(j\in Q^{\ast} \), \(\sum_{S_i \ni j}p_i = 1 \), and let \(p^\prime \coloneqq \max _{j\in Q^\prime }\sum_{S_i\ni j}p_i\).
	\item For all \(j\in Q^{\ast} \), let \(\pi (j)\in Q^\prime \) be the facility closest to \(j\). Then if \(S_i\) contains \(j\in Q^\prime \), \(\pi ^{-1} (j)\subseteq S_i\).\footnote{In particular, if \(\left\vert \pi ^{-1} (j) \right\vert > w / 2 \), then \(j\) is not contained in any \hyperref[not:swap]{swap}.}
\end{enumerate}

The existence of such \hyperref[not:swap]{swaps} family is ensured by the following.
\begin{lemma}\label{lma:lec8}
	There exists \(S_1, \ldots, S_t\) with \(p_1, \ldots  , p_t\) such that \(\forall j\in Q^{\ast} \), \(\sum_{S_i \ni j}p_i = 1\) and if \(j\in S_i\), \(\pi ^{-1} (j)\subseteq S_i\) with \(p^\prime = \max _{j\in Q^\prime }\sum_{S_i\ni j}p_i = 1 + 2 / w\).
\end{lemma}
\begin{proof}

\end{proof}

With \autoref{lma:lec8}, we're ready to prove the following.

\begin{theorem}\label{thm:lec8}
	\autoref{algo:k-median-local-search} is a \((3 + \epsilon)\)-approximation algorithm for arbitrary small \(\epsilon > 0\).
\end{theorem}
\begin{proof}
	Fix \(i\in P\), we analyze how it contributes to the left-hand side of \autoref{eq:lec8}. Let \(j^\prime \in Q^\prime \) and \(j^{\ast} \in Q^{\ast} \) be facilities closest to \(i\), and \(d_i^\prime \coloneqq d(i, j^\prime )\), \(d_i^{\ast} = d(i, j^{\ast} )\), then for every \(S_{\ell}\), we have
	\begin{enumerate}[(a)]
		\item \(S_{\ell }\ni j^{\ast} \). Then contribution of \(i\) to \(\mathop{\mathrm{cost}}(Q^\prime ) - \mathop{\mathrm{cost}}(S_{\ell }\triangle Q^\prime)\) is at least \(d_i^\prime - d_i^{\ast}\) (\(i\) can go to \(j^{\ast} \))
		\item \(S_{\ell }\ni j^\prime \). By the second property, either \(\pi (j^{\ast} )\in S_{\ell } \) or \(S_{\ell }\ni j^{\ast}\). The latter falls back to the first case, and if \(\pi (j^{\ast} )\in S_{\ell } \), contribution of \(i\) to \(\mathop{\mathrm{cost}}(Q^\prime ) - \mathop{\mathrm{cost}}(S_{\ell }\triangle Q^\prime)\) is at least \(d_i^\prime - (2d_i^{\ast} + d_i^\prime ) = -2d_i^{\ast}\). (\(i\) can go to \(\pi (j^{\ast} )\))
		\item Otherwise, contribution of \(i\) to \(\mathop{\mathrm{cost}}(Q^\prime ) - \mathop{\mathrm{cost}}(S_{\ell }\triangle Q^\prime)\) is at least \(0\). (\(i\) can stay with \(j^\prime \))
	\end{enumerate}

	In all, we see that the first case has total weight \(1\), while (b) \(-\) (c) has total weight \(\leq p^\prime \), hence \autoref{eq:lec8} implies
	\[
		\sum_{i\in P} \left[ (d_i^\prime - d_i^{\ast} )\cdot 1 - (2d_i^{\ast} )\cdot p^\prime \right] \leq \sum_{i=1} ^t p_i(\mathop{\mathrm{cost}}(Q^\prime ) - \mathop{\mathrm{cost}}(S_{\ell }\triangle Q^\prime)) \leq 0,
	\]
	which is equivalent to say
	\[
		\mathop{\mathrm{cost}}(Q^\prime ) - (1 - 2p^\prime )\OPT \leq 0,
	\]
	so we get a \((1 + 2p^\prime )\)-approximation ratio. Furthermore, From \autoref{lma:lec8}, we have \(p^\prime = 1 + 2 / w\), hence we can achieve \(1 + 2(1 + 2 / w) = 3 + 4 / w\)-approximation ratio. Given \(\epsilon > 0\), by setting \(w \coloneqq 4/\epsilon \), we're done.
\end{proof}