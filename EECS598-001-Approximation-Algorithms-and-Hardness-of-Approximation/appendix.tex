\chapter{Review}
\section{Boolean Satisfaction Problem}
Here, we give a quick review toward the \hyperref[prb:max-3SAT]{MAX-3SAT} problem.

\begin{definition}[Conjunctive normal form]\label{def:CNF}
	A \emph{conjunctive normal form} (CNF) formula is a conjunction \(\varphi\) of one or more boolean clauses on \(x_1, x_2, \ldots  , x_n\) with boolean valued \(\{0, 1\}\). Explicitly, \(\varphi\) is in CNF if
	\[
		\varphi(x_1, x_2, \ldots, x_n) = \text{clause}_1 \wedge \text{clause}_2 \wedge \text{clause}_3 \wedge \ldots  \wedge \text{clause}_k
	\]
	where each clause is an or of literals, with a literal being some \(x_i\) or its negation \(\neg x_i\).
\end{definition}

\begin{note}[Disjunctive normal form]
	For every \hyperref[def:CNF]{conjunctive normal form}, there is an equivalent way to write it in the so-called \emph{disjunctive normal form}.
\end{note}

\begin{definition}[\(k\)-CNF]\label{def:k-CNF}
	A \emph{\(k\)-CNF formula} is a \hyperref[def:CNF]{CNF} formula in which each clause has exactly \(k\) literals from distinct variables.
\end{definition}

\begin{eg}[3-CNF]
	A \hyperref[def:k-CNF]{\(3\)-CNF formula} can be like
	\[
		\varphi = (x_1 \vee \neg x_2 \vee x_4) \wedge (\neg x_3 \vee x_4 \vee x_5) \wedge (\neg x_1 \vee \neg x_5 \vee \neg x_6).
	\]
\end{eg}

Now, the boolean satisfability problem asks the following question: given a \hyperref[def:k-CNF]{\(k\)-CNF formula} \(\varphi \), does an assignment exist such that \(\varphi \) is evaluated as true? Formally, we have \autoref{prb:k-SAT}.

\begin{problem}[\(k\)-SAT]\label{prb:k-SAT}
Given a \hyperref[def:k-CNF]{\(k\)-CNF formula} \(\varphi\), the \emph{\(k\)-SAT} problem asks whether \(\varphi\) is satisfiable.
\end{problem}

Instead of looking at a general \(k\), we consider a simple but also hard enough case when \(k = 3\). Specifically, we ask the following question.
\begin{problem}[MAX-3SAT]\label{prb:max-3SAT}
Given a \hyperref[def:k-CNF]{\(3\)-CNF formula} \(\varphi\) and \(\ell\), the \emph{MAX-3SAT} problem asks is there an assignment of variables such that it satisfies at least \(\ell \) clauses?
\end{problem}

\begin{remark}
	We often call \hyperref[prb:max-3SAT]{MAX-3SAT} as 3SAT for brevity.
\end{remark}