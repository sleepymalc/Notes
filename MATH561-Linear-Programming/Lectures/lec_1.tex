\chapter{Introduction to Linear Programming}
\lecture{1}{30 Aug.\ 8:00}{Introduction}
\section{General Linear Programming Problem}
Let's start with the definition of a \hyperref[def:general-linear-programming-problem]{linear programming problem}.
\begin{definition}[General linear programming problem]\label{def:general-linear-programming-problem}
	A \emph{general linear programming problem} is to either minimize or maximize an \hyperref[def:objective-function]{objective function} with the \hyperref[def:constraint]{constraints} defined as follows.

	\begin{definition}[Objective function]\label{def:objective-function}
		An \emph{objective function} is in the form of
		\[
			c_1 x_1 + c_2 x_2 + \dots +c_n x_n,
		\]
		where \(c_i, x_i \in \mathbb{R}\), and \(x_{i}\) are variables for \(i = 1, \dots, n\).
	\end{definition}

	\begin{definition}[Constraint]\label{def:constraint}
		The \emph{constraints} are the combination of \hyperref[def:structured-constraint]{structured constraints} and also the \hyperref[def:signed-constraint]{signed constraints}.
		\begin{definition}[Structured constraint]\label{def:structured-constraint}
			The \emph{structured constraints} are in the form of
			\[
				a_{j1} x_1 + \dots + a_{jn}x_n \gtreqqless b_j
			\]
			where \(a_{ji}\in \mathbb{R} \) for all \(i=1, \dots , n\) and \(j = 1, \dots, m\).
		\end{definition}
		\begin{definition}[Signed constraint]\label{def:signed-constraint}
			The \emph{signed constraints} are in the form of
			\[
				x_i \gtreqqless 0
			\]
			for some \(i=1, \dots , n\).
		\end{definition}
	\end{definition}
\end{definition}

\begin{notation}
	We often referred \(\gtreqqless \) to either \(\geq , \leq\) or \(=\).
\end{notation}

\begin{remark}
	For a \hyperref[def:general-linear-programming-problem]{linear programming problem}, we often assume there are \(n\) variables indexed by \(i\), and \(m\) \hyperref[def:structured-constraint]{structured constraints} indexed by \(j\), and also \hyperref[def:signed-constraint]{signed constraints} for some \(x_i\).
\end{remark}

Given a \hyperref[def:general-linear-programming-problem]{general linear programming problem}, we have the following definitions.
\begin{definition}[Solution]\label{def:solution}
	We called an assignment of values to variable \(x\) as a \emph{solution}.

	\begin{definition}[Feasible solution]\label{def:feasible-solution}
		If this \hyperref[def:solution]{solution} satisfies the \hyperref[def:constraint]{linear constraints}, we say that this \hyperref[def:solution]{solution} is a \emph{feasible solution}.
	\end{definition}

	\begin{definition}[Feasible region]\label{def:feasible-region}
		The set of \hyperref[def:feasible-solution]{feasible solutions} is called the \emph{feasible region}.
	\end{definition}

	\begin{definition}[Optimal solution]\label{def:optimal-solution}
		A \hyperref[def:solution]{solution} is \emph{optimal} if there is no \hyperref[def:feasible-solution]{feasible solution} with better objective value.
	\end{definition}
\end{definition}

\begin{remark}
	A \hyperref[def:feasible-region]{feasible region} is a polyhedron for a \hyperref[def:general-linear-programming-problem]{general linear programming problem}.
\end{remark}

\subsection{Standard Form}
It's convenient to group the coefficients together as
\[
	c= \begin{pmatrix}
		c_1    \\
		\vdots \\
		c_n    \\
	\end{pmatrix},\quad
	x = \begin{pmatrix}
		x_1    \\
		\vdots \\
		x_n    \\
	\end{pmatrix}, \quad
	A = \begin{pmatrix}
		a_{11} & \dots  & a_{1n} \\
		\vdots & \ddots & \vdots \\
		a_{m1} & \dots  & a_{mn} \\
	\end{pmatrix},\quad
	b = \begin{pmatrix}
		b_1    \\
		\vdots \\
		b_m    \\
	\end{pmatrix},
\]
since in this way, we can define the \hyperref[def:standard-form]{standard form} of a \hyperref[def:general-linear-programming-problem]{linear programming}.

\begin{definition}[Standard form]\label{def:standard-form}
	The \emph{standard form} \hyperref[def:general-linear-programming-problem]{linear programming} has the form of
	\[
		\begin{aligned}
			\min~ & c^{\top}x \\
			      & Ax = b    \\
			      & x\geq 0
		\end{aligned}
	\]
	with the condition that rows of \(A\) are linear independent.
\end{definition}

\begin{remark}[Compactness of the solution space]
	We only consider finitely many of \hyperref[def:constraint]{constrains} since with finite dimensional solution space will be compact, hence \hyperref[def:objective-function]{objective function} can attain its extremum.
\end{remark}

\begin{remark}[Convert to standard form]
	Every \hyperref[def:general-linear-programming-problem]{general linear programming problem} can be converted to \hyperref[def:standard-form]{standard form}.
\end{remark}
\begin{explanation}
	Given a \hyperref[def:general-linear-programming-problem]{general linear programming problem} with our notation, we see that
	\begin{itemize}
		\item Sign:
		      \begin{itemize}
			      \item If \(x_i\leq 0 \implies x_i \to -x_i^-\), where \(x_i^- \geq 0\).
			      \item If \(x_i\) is unrestricted \(\implies x_i \to x_i^+ - x_i^-\), where \(x_i^{\pm} \geq 0\).
		      \end{itemize}
		\item Constraints:
		      \begin{itemize}
			      \item \(\sum\limits_{i=1}^{n} a_{ji} x_i \leq b_j \implies \sum\limits_{i=1}^{n} a_{ji} x_i + s_j = b_j\), where \(s_j \geq 0\).

			      \item \(\sum\limits_{i=1}^{n} a_{ji} x_i \geq b_j \implies \sum\limits_{i=1}^{n} a_{ji} x_i - s_j = b_j\), where \(s_j \geq 0\).

		      \end{itemize}
		\item Maximize: \(\max \sum c_{j}x_{j} \implies -\min -\sum c_{j}x_{j}\).
	\end{itemize}
\end{explanation}

We see that in order to convert to \hyperref[def:standard-form]{standard form}, we sometimes need to introduce variable \(s_j\) for \hyperref[def:structured-constraint]{constraint} \(j\). This leads to the following.

\begin{definition*}
	Consider the new variable \(s_j\) introduced for \(j^{th} \) non-equal \hyperref[def:structured-constraint]{constraint} when converting to the \hyperref[def:standard-form]{standard form}.

	\begin{definition}[Slack variable]\label{def:slack-variable}
		The \emph{slack variable} \(s_j\) is introduced for \(\leq \) \hyperref[def:structured-constraint]{constraint}.
	\end{definition}
	\begin{definition}[Surplus variable]\label{def:surplus-variable}
		The \emph{surplus variable} \(s_j\) is introduced for \(\geq \) \hyperref[def:structured-constraint]{constraint}.
	\end{definition}
\end{definition*}