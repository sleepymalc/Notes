\lecture{3}{8 Sep. 08:00}{Production Problem and Norm Minimization}
\section{Modeling}
We now see two main examples illustrating why \hyperref[def:general-linear-programming-problem]{linear programming} is interesting to study.

\subsection{Production Problem}\label{subsec:production-problem}
The \emph{production problem} can be formulated as follows.
\[
	\begin{aligned}
		\max~ & c^{\top}x      \\
		      & Ax \leq b      \\
		      & x\geq \vec{0},
	\end{aligned}
\]
where
\begin{itemize}
	\item \(n\) products activities.
	\item \(c_{j}=\) per-unit revenue for activity \(j = 1\ldots n\).
	\item \(b_{i}=\) resource endowment for resource \(i = 1\ldots m\).
	\item \(a_{ij}=\) amount of resource \(i\) consumed by activity \(j\).
\end{itemize}

\begin{note}
	This is a very important and practical problems considered all across the industry!
\end{note}

\subsection{Norm Minimization}
It's also useful to minimizing the norm of the variable \(x\). And interestingly, this can be done by \hyperref[def:general-linear-programming-problem]{linear programming problems} as well. Let's first consider minimizing a \(\lVert \cdot \rVert _\infty \).

\begin{definition}[Maximum norm]\label{def:maximum-norm}
	The \emph{maximum norm} of \(x\in \mathbb{R} ^n\) is defined as
	\[
		\left\lVert x\right\rVert _{\infty } \coloneqq \max_{1\leq i \leq n} \vert x_i \vert .
	\]
\end{definition}

Consider
\[
	\begin{aligned}
		\min~ & \left\lVert x\right\rVert _{\infty } \\
		      & Ax = b,
	\end{aligned}
\]
we set up
\[
	\begin{aligned}
		\min~ & t                                      \\
		      & t\geq x_i,\text{ for }i = 1, \ldots ,n \\
		      & t\leq x_i,\text{ for }i = 1, \ldots ,n \\
		      & Ax = b.
	\end{aligned}
\]

We see that this optimization \textbf{pressure} will force the maximum of \(\left\vert x_i \right\vert \) being small, hence we'll get the minimum among \(\left\vert x_i \right\vert \). Similarly, we can consider the following minimizing \(\lVert \cdot \rVert _1\).

\begin{definition}[\(1\)-norm]\label{def:1-norm}
	The \emph{\(1\)-norm} of \(x\in \mathbb{R} ^n\) is defined as
	\[
		\left\lVert x\right\rVert _{1} \coloneqq \sum\limits_{i=1}^{n} \left\vert x_i \right\vert.
	\]
\end{definition}

\begin{note}[\(p\)-norm]
	More generally, the \emph{\(p\)-norm} of \(x\in \mathbb{R} ^n\) is defined as
	\[
		\left\lVert x\right\rVert _{p} \coloneqq \left( \sum\limits_{i=1}^{n} \left\vert x_i \right\vert^p \right)^{1 / p},
	\]
	and \(\lVert \cdot \rVert_1 \), and even \(\lVert \cdot \rVert _{\infty }\) are both special cases when \(p = 1\) and \(p = \infty \), respectively.
\end{note}

Consider
\[
	\begin{aligned}
		\min~ & \left\lVert x\right\rVert _1 \\
		      & Ax = b,
	\end{aligned}
\]
we set up
\[
	\begin{alignedat}{3}
		\min~ & \sum\limits_{i=1}^{n} t_i                \\
		& t_i\geq x_i, && \text{ for }i = 1, \ldots ,n  \\
		& t_i\leq -x_i, && \text{ for }i = 1, \ldots ,n \\
		& Ax = b.
	\end{alignedat}
\]

Again, we see that the optimization pressure will force \(t_i\) goes to \(\left\vert x_i \right\vert \), resulting \(\sum_{i=1}^{n} t_i\) being \(\left\lVert x\right\rVert _1\).

\begin{remark}
	Minimize \(\left\lVert x\right\rVert _1\) tends to make \(x\) \textbf{sparse} (lots of zeros).
	\begin{figure}[H]
		\centering
		\incfig{1-norm}
		\caption{The best approximated convex function of \(I_{x = 0}\) }
		\label{fig:1-norm}
	\end{figure}
\end{remark}