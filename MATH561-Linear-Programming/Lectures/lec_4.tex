\chapter{Algebra and Geometry}
\lecture{4}{13 Sep.\ 8:00}{Basis Partition}
In this section, we're going to study various of fundamental building blocks for designing a mathematical complete algorithm for solving a \hyperref[def:standard-form]{standard form} \hyperref[def:general-linear-programming-problem]{linear programming problem}. This requires both algebraic and geometric understanding of the problem. Let's start with algebraic tools we need.

\section{Elementary Row Operations}
There are some simple operations we can apply to a matrix called \hyperref[def:elementary-row-operations]{elementary row operations}.

\begin{definition}[Elementary row operations]\label{def:elementary-row-operations}
	The following are called \emph{elementary row operations}.
	\begin{enumerate}[(a)]
		\item Permute rows.
		\item Multiply a row by a non-zero factor.
		\item Add a multiple of a row to another row.
		\item Permutation in columns.
	\end{enumerate}
\end{definition}

\begin{note}[Permutation in columns]
	Consider
	\[
		A = \left[ A_1, \dots , A_n \right]_{m\times n} .
	\]
	A permutation is a function \((1, \dots , n) \mapsto (\sigma(1), \dots , \sigma(n))\), inducing the permuted matrix \(A_{\sigma}\)
	\[
		A_{\sigma} = \left[ A_{\sigma(1)}, \dots , A_{\sigma(n)} \right].
	\]
	With the same permutation for \(x\), we have
	\[
		x_{\sigma} = \begin{pmatrix}
			x_{\sigma(1)} \\
			\vdots        \\
			x_{\sigma(n)} \\
		\end{pmatrix}.
	\]
	We then easily see that
	\[
		Ax = \sum\limits_{j=1}^{n} A_i x_i = \sum\limits_{j=1}^{n} A_{\sigma(j)}x_{\sigma(j)}.
	\]
	Hence,
	\[
		Ax = b \iff A_{\sigma} x_{\sigma} = b.
	\]
\end{note}

\section{Basic Feasible Solutions and Extreme Points}

\subsection{Basic Partition}
Let's first define the so-called \hyperref[def:partition]{partition}, where the intention will become clear soon.

\begin{definition}[Partition]\label{def:partition}
	A \emph{partition} \((\beta , \eta )\) of \(\{1, \dots , n\}\) is defined as
	\[
		\beta \coloneqq (\beta_1, \dots , \beta_m),\quad \eta \coloneqq (\eta_1, \dots , \eta_{n-m}),
	\]

	\begin{definition}[Basis]\label{def:basis}
		\(\beta\) is called \emph{basis}.
	\end{definition}
	\begin{definition}[Non-basis]\label{def:non-basis}
		\(\eta\) is called \emph{non-basis}.
	\end{definition}
\end{definition}

The main idea of introducing \hyperref[def:partition]{partition} is because we want to characterize what subset of the \hyperref[def:structured-constraint]{structured constraints} is solvable, i.e., some submatrix \(A^\prime \) of \(A\) is invertible. \autoref{def:partition} induces the following.

\begin{definition}[Basic partition]\label{def:basic-partition}
	A \hyperref[def:partition]{partition} is a \emph{basic partition} if
	\[
		A_{\beta} = \left[ A_{\beta_1}, \dots , A_{\beta_m} \right]_{m\times m}
	\]
	is invertible.
\end{definition}

\subsection{Basic Feasible Solutions}
With the notion of \hyperref[def:basic-partition]{basic partition}, we define the following.

\begin{definition}[Basic solution]\label{def:basic-solution}
	The \emph{basic solution} \(\overline{x}\) for a \hyperref[def:basic-partition]{basic partition} is defined as
	\[
		\overline{x}_{\eta} = \begin{pmatrix}
			\overline{x}_{\eta_1}     \\
			\vdots                    \\
			\overline{x}_{\eta_{n-m}} \\
		\end{pmatrix}\coloneqq \begin{pmatrix}
			0      \\
			\vdots \\
			0      \\
		\end{pmatrix},\qquad \overline{x}_{\beta} = \begin{pmatrix}
			\overline{x}_{\beta_1} \\
			\vdots                 \\
			\overline{x}_{\beta_m} \\
		\end{pmatrix}\coloneqq A^{-1}_{\beta}b.
	\]
\end{definition}

\begin{intuition}
	This of course makes sense, since we know that if this is a \hyperref[def:feasible-solution]{feasible solution} for a \hyperref[def:standard-form]{standard form} problem,
	then \(A\overline{x} = b\), which means
	\[
		\left[ A_{\beta}, A_{\eta} \right] \begin{pmatrix}
			\overline{x}_{\beta} \\
			\overline{x}_{\eta}  \\
		\end{pmatrix} = b \implies A_{\beta}\overline{x}_{\beta} + A_{\eta}\underbrace{\overline{x}_{\eta}}_{=0} = b\implies \overline{x}_{\beta} = \underbrace{A^{-1}_{\beta}}_{\text{invertible}}b
	\]
\end{intuition}

\begin{remark}
	After choosing \(\eta\), we see that \(\overline{x}_{\beta}\) is determined.
\end{remark}