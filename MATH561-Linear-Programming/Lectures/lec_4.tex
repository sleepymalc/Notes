\chapter{Algebra v.s. Geometry}
\lecture{4}{13 Sep. 08:00}{Basis Partition}
\section{Elementary Row Operations}
There are some simple operations we can apply to a matrix called \hyperref[def:elementary-row-operations]{elementary row operations}.
\begin{definition}[Elementary row operations]\label{def:elementary-row-operations}
	The following operations are called \emph{elementary row operations}.
	\begin{enumerate}[(a)]
		\item Permute rows.
		\item Multiply a row by a non-zero factor.
		\item Add a multiple of a row to another row.
		\item Permutation in columns.
	\end{enumerate}
\end{definition}

\begin{note}[Permutation in columns]
	Consider
	\[
		A = \left[ A_1, \ldots , A_n \right]_{m\times n} .
	\]
	A permutation is a function like
	\[
		(1, \ldots , n) \to (\sigma(1), \ldots , \sigma(n)).
	\]
	Then the permuted matrix \(A_{\sigma}\) i
	\[
		A_{\sigma} = \left[ A_{\sigma(1)}, \ldots , A_{\sigma(n)} \right].
	\]
	With the same permutation for \(x\), we have
	\[
		x_{\sigma} = \begin{pmatrix}
			x_{\sigma(1)} \\
			\vdots        \\
			x_{\sigma(n)} \\
		\end{pmatrix}.
	\]
	We then easily see that
	\[
		Ax = \sum\limits_{j=1}^{n} A_i x_i = \sum\limits_{j=1}^{n} A_{\sigma(j)}x_{\sigma(j)}.
	\]
	Hence,
	\[
		Ax = b \iff A_{\sigma} x_{\sigma} = b.
	\]
\end{note}

\section{Basic Partition}
\begin{definition}[Partition]\label{def:partition}
	We denote a \emph{partition} by
	\[
		\beta \coloneqq (\beta_1, \ldots , \beta_m),\quad \eta \coloneqq (\eta_1, \ldots , \eta_{n-m}),
	\]
	which is a partition of \(\{1, \ldots , n\}\).

	\begin{definition}[Basic]\label{def:basic}
		\(\beta\) is called \emph{basic}
	\end{definition}
	\begin{definition}[Non-basic]\label{def:non-basic}
		\(\eta\) is called \emph{non-basic}.
	\end{definition}
\end{definition}

Given the above definitions, we have the following naturally induced notion.

\begin{definition}[Basic partition]\label{def:basic-partition}
	A \hyperref[def:partition]{partition} is a \emph{basic partition} if
	\[
		A_{\beta} = \left[ A_{\beta_1}, \ldots , A_{\beta_m} \right]_{m\times m}
	\]
	is invertible.
\end{definition}

\begin{definition}[Basic solution]\label{def:basic-solution}
	Associate a \hyperref[def:basic-partition]{basic partition} with a \emph{basic solution} \(\overline{x}\), which is defined as
	\[
		\overline{x}_{\eta} = \begin{pmatrix}
			\overline{x}_{\eta_1}     \\
			\vdots                    \\
			\overline{x}_{\eta_{n-m}} \\
		\end{pmatrix}\coloneqq \begin{pmatrix}
			0      \\
			\vdots \\
			0      \\
		\end{pmatrix},\qquad \overline{x}_{\beta} = \begin{pmatrix}
			\overline{x}_{\beta_1} \\
			\vdots                 \\
			\overline{x}_{\beta_m} \\
		\end{pmatrix}\coloneqq A^{-1}_{\beta}b.
	\]
\end{definition}

\begin{intuition}
	This of course makes sense, since we know that if this is a \hyperref[def:feasible-solution]{feasible solution} for a \hyperref[def:standard-form]{standard form} problem,
	then \(\overline{Ax} = b\), which means
	\[
		\left[ A_{\beta}, A_{\eta} \right] \begin{pmatrix}
			\overline{x}_{\beta} \\
			\overline{x}_{\eta}  \\
		\end{pmatrix} = b \implies A_{\beta}\overline{x}_{\beta} + A_{\eta}\underbrace{\overline{x}_{\eta}}_{=0} = b\implies \overline{x}_{\beta} = \underbrace{A^{-1}_{\beta}}_{\text{invertible}}b
	\]
\end{intuition}

\begin{remark}
	After choosing \(\eta\), we see that \(\overline{x}_{\beta}\) is determined.
\end{remark}