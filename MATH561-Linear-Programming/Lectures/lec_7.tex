\chapter{Simplex Algorithm}
\lecture{7}{22 Sep.\ 8:00}{Worry-Free Simplex Algorithm}
\section{A Sufficient Optimality Criterion}
Turns out that in order to design an algorithm which solves the \hyperref[def:primal]{primal}, looking at \hyperref[def:dual]{dual} is always helpful. This suggests we should first study the \hyperref[def:solution]{solution} of the \hyperref[def:dual]{dual}.

\subsection{Dual Basic Solution}
We start by considering the \hyperref[def:standard-form]{standard form} problem
\[
	\begin{aligned}
		\min~          & c^{\top}x \\
		               & Ax = b    \\
		(\Primal)\quad & x\geq  0,
	\end{aligned}\quad \begin{aligned}
		\max ~       & y^{\top}b               \\
		             & y^{\top}A\leq c^{\top}. \\
		(\Dual)\quad &
	\end{aligned}
\]

\begin{definition}[Dual basic solution]\label{def:dual-basic-solution}
	The \emph{dual basic solution} \(\overline{y}\in \mathbb{R}^m\) is defined as
	\[
		\overline{y}^{\top} = c_{\beta}^{\top} A^{-1}_{\beta}.
	\]
\end{definition}

\begin{lemma}\label{lma:lec7-1}
	If \(\beta, \eta\) is a \hyperref[def:basic-partition]{basic partition}, and \(\overline{x}\) is the associated \hyperref[def:basic-solution]{primal basic solution}
	and \(\overline{y}\) is the associated \hyperref[def:dual-basic-solution]{dual basic solution}, then
	\[
		c^{\top}\overline{x} = \overline{y}^{\top} b.
	\]
\end{lemma}
\begin{proof}
	\[
		c^{\top}\overline{x} = \begin{pmatrix}
			c^{\top}_{\beta} & c^{\top}_{\eta} \\
		\end{pmatrix}\begin{pmatrix}
			\overline{x}_{\beta} \\
			\overline{x}_{\eta}  \\
		\end{pmatrix}= c_{\beta}^{\top} \overline{x}_{\beta}+ c_{\eta}^{\top} \overline{c}_{\eta} = c_{\beta} A^{-1}_{\beta}b = \overline{y}b.
	\]
\end{proof}

Recall that
\begin{prev}
	\[
		\begin{aligned}
			\min~ & c^{\top}_{\beta}x_{\beta} + c^{\top}_{\eta}x_{\eta} \\
			      & A_{\beta}x_{\beta} + A_{\eta}x_{\eta} = b           \\
			      & x_{\beta}\geq 0, x_{\eta}\geq 0,
		\end{aligned}
	\]
	and hence
	\[
		\begin{aligned}
			c^{\top}_{\beta}A^{-1}_{\beta}b + \min~ & (c_{\eta}^{\top} - c_{\beta}^{\top}A^{-1}_{\beta}A_{\eta})x_{\eta} \\
			                                        & A^{-1}_{\beta}A_{\eta}x_{\eta} \leq A^{-1}_{\beta}b                \\
			                                        & x_{\beta}\geq 0, x_{\eta}\geq 0.
		\end{aligned}
	\]
\end{prev}

We now formalize the concept of \hyperref[note:reduced-cost]{reduced cost}.

\begin{definition}[Reduced cost]\label{def:reduced-cost}
	The \emph{reduced cost} \(\overline{c}_{\eta}\) for \hyperref[def:non-basis]{non-basis} variables is defined as
	\[
		\overline{c}_{\eta}^{\top} \coloneqq c_{\eta}^{\top} - c_{\beta}^{\top} A^{-1}_{\beta}A_{\eta} = c_{\eta}^{\top} - \overline{y}^{\top}A_{\eta}.
	\]
\end{definition}

\subsection{Dual Feasibility}
We have the following characterization of the \hyperref[def:dual-basic-solution]{dual basic solution} \(\overline{y} \).

\begin{lemma}\label{lma:lec7-2}
	\(\overline{y}\) is \hyperref[def:feasible-solution]{feasible} for \((\Dual)\) if and only if \(\overline{c}_{\eta}\geq 0\).
\end{lemma}
\begin{proof}
	Observe that
	\[
		y^{\top}A\leq c^{\top} \iff y^{\top}\begin{bmatrix}
			A_{\beta} & A_{\eta} \\
		\end{bmatrix}\leq \begin{pmatrix}
			c_{\beta}^{\top} & c_{\eta}^{\top} \\
		\end{pmatrix}
	\]
	since
	\[
		\begin{dcases}
			y^{\top} A_{\beta} \leq c_{\beta}^{\top} \\
			y^{\top} A_{\eta}  \leq c_{\eta}^{\top} \implies c_{\eta}^{\top} - y^{\top}A_{\eta}\geq 0.
		\end{dcases}
	\]
\end{proof}

\begin{corollary}\label{col:lec7-1}
	Suppose \(\hat{x}\) is \hyperref[def:feasible-solution]{feasible} for \((\Primal)\) and \(\hat{y}\) is \hyperref[def:feasible-solution]{feasible} for \((\Dual)\). If
	\(c^{\top}\hat{x} = \hat{y}^{\top} b\), then \(\hat{x}\) and \(\hat{y}\) are \hyperref[def:optimal-solution]{optimal}.
\end{corollary}

\begin{theorem}[Weak optimal basis theorem]\label{thm:weak-optimal-basis-theorem}
	Let \(\overline{x}\) and \(\overline{y}\) are \hyperref[def:basic-solution]{basic primal} and \hyperref[def:dual-basic-solution]{dual solutions} for \((\Primal)\) and \((\Dual)\).
	Then if \(\beta\) is a feasible \hyperref[def:basis]{basis} and \(\overline{c}_{\eta}\geq 0\), \(\overline{x}\) and \(\overline{y}\) are \hyperref[def:optimal-solution]{optimal}.
\end{theorem}
\begin{proof}
	Obvious from the \hyperref[def:standard-form]{standard problem} in the form of
	\[
		\begin{aligned}
			c^{\top}_{\beta}A^{-1}_{\beta}b + \min~ & (c_{\eta}^{\top} - c_{\beta}^{\top}A^{-1}_{\beta}A_{\eta})x_{\eta} \\
			                                        & A^{-1}_{\beta}A_{\eta}x_{\eta} \leq A^{-1}_{\beta}b                \\
			                                        & x_{\beta}\geq 0, x_{\eta}\geq 0.
		\end{aligned}
	\]
\end{proof}

\begin{note}
	The order of the arguments in text book for \autoref{thm:weak-optimal-basis-theorem} is slightly different.
\end{note}

\section{Worry-Free Simplex Algorithm}
From the above discussion, we can come up with the following algorithmic approach to find the \hyperref[def:optimal-solution]{optimal solution} \(\overline{x} \) and \(\overline{y} \) given a \hyperref[def:standard-form]{standard form}.

\begin{enumerate}[(a)]
	\item Start with a \hyperref[def:basic-partition]{basis partition} \(\beta, \eta\) with \(\overline{x}_{\beta}\geq 0\).
	\item If \(\overline{c}_{\eta}\geq 0\), then \(\overline{x}\) and \(\overline{y}\) are \hyperref[def:optimal-solution]{optimal}, so we can stop.
	\item Otherwise, choose \(\eta_j\) with \(\overline{c}_{\eta_j}<0\). Consider the associated \hyperref[def:basic-direction]{basis direction} \(\overline{z}\).\footnote{Idea is that \(\overline{x}\to \overline{x}+\lambda \overline{z}\) with \(\lambda >0\).} Then
	      \[
		      c^{\top}(\overline{x} + \lambda \overline{z}) = c^{\top} \overline{x} + \lambda c^{\top} \overline{z} = c^{\top} \overline{x} + \lambda \overline{c}_{\eta_j},
	      \]
	      where \begin{itemize}
		      \item \(c^{\top} \overline{x}\) is the current objective value
		      \item \(c^{\top}\overline{z}\) is
		            \[
			            \begin{split}
				            c^{\top}\overline{z} & = c_{\eta}^{\top} \overline{z}_{\eta}+c_{\beta}^{\top} \overline{z}_{\beta} \\
				                                 & =c_{\eta}^{\top}e_{j} - c_{\beta}^{\top}(A^{-1}_{\beta}A_{\eta_j})          \\
				                                 & =c_{\eta_{j}} - c_{\beta}^{\top} A^{-1}_{\beta}A_{\eta_j}                   \\
				                                 & = \overline{c}_{\eta_{j}}
			            \end{split}
		            \]
		      \item \(\lambda \overline{c}_{\eta_{j}}\) is the \emph{rate} of change of objective value as we move in direction \(\overline{z}\).
	      \end{itemize}

	      Then we move from \(\overline{x}\) to \(\overline{x}+\overline{\lambda} \overline{z}\), where we let \(\overline{\lambda}\) as large as possible.
	      Operationally, since we need
	      \[
		      \overline{x}_{\beta} + \lambda \overline{z}_{\beta} \geq 0,
	      \]
	      where \(\overline{z}_{\eta} = e_{j},\ \overline{z}_{\beta} = -A^{-1}_{\beta}A_{\eta_{j}}\). We then have
	      \[
		      \begin{alignedat}{3}
			       & \overline{x}_{\beta_{i}} - \lambda \overline{a}_{i, \eta_{j}}\geq 0,      &  & \text{ for }i = 1, \dots , m                                \\
			       & \lambda \leq \frac{\overline{x}_{\beta_{i}}}{\overline{a}_{i, \eta_{j}}}, &  & \text{ for }i \text{ such that }\overline{a}_{i, \eta_j}>0.
		      \end{alignedat}
	      \]
	      Hence,
	      \[
		      \overline{\lambda} \coloneqq \min_{i: \overline{a}_{i, \eta_{j}}>0} \left\{ \frac{\overline{x}_{\beta_{i}}}{\overline{a}_{i, \eta_{j}}} \right\} \geq 0.
	      \]
	      \begin{remark}
		      If \(\overline{a}_{i, \eta_{j}}\leq 0\) for all \(i = 1, \dots , m\), namely
		      \[
			      \overline{A}_{i, \eta_{j}}\leq 0 \iff -A^{-1}_{\beta}A_{\eta_{j}}\geq 0 \iff \overline{z}\geq 0,
		      \]
		      then \(\overline{z}\) is a ray. This means \((\Primal)\) is unbounded below, hence we terminate.
	      \end{remark}
\end{enumerate}

By formalizing the above procedure, we get the very first algorithmic approach to solve a \hyperref[def:general-linear-programming-problem]{linear program} called the \hyperref[algo:worry-free-simplex-algorithm]{worry-free simplex algorithm}. Specifically, consider the \hyperref[def:standard-form]{standard form} problem
\[
	\begin{aligned}
		\min~          & c^{\top}x \\
		               & Ax = b    \\
		(\Primal)\quad & x\geq  0.
	\end{aligned}
\]

\begin{algorithm}[H]\label{algo:worry-free-simplex-algorithm}
	\DontPrintSemicolon
	\caption{Worry-Free Simplex Algorithm}
	\KwData{A \hyperref[def:standard-form]{standard form} \((\Primal)\), \hyperref[def:basic-partition]{basic partition} \(\beta , \eta\) with \(x_{\beta}\geq 0\)}
	\KwResult{\hyperref[def:optimal-solution]{Optimal solutions} \(\overline{x}, \overline{y}\), or report \((\Primal)\) is unbounded}
	\BlankLine

	\While{\(\True\)}{
	\(\overline{x}_{\beta} \gets A^{-1}_{\beta} b(\geq 0)\)\;
	\(\overline{c}_{\eta}^{\top} \gets c_{\eta}^{\top} - c_{\beta}^{\top} A^{-1}_{\beta} A_{\eta}\)\;
	\uIf(\Comment*[f]{\(\overline{x}\) is \hyperref[def:optimal-solution]{optimal} for \((\Primal)\)}){\(\overline{c}_{\eta} \geq 0\)}{
	\(\overline{y} \gets c_{\beta}^{\top}A^{-1}_{\beta}\) \Comment*[r]{From \autoref{thm:weak-optimal-basis-theorem}}
	\Return{\(\overline{x}, \overline{y} \)}
	}
	\Else(\Comment*[f]{\hyperref[def:basic-direction]{Basic direction} \(\overline{z}\), then \(c^{\top}\overline{z} = \overline{c}_{\eta_{j}}<0\)}){
	choose \(j\) where \(1\leq j\leq n-m\) such that \(\overline{c}_{\eta_j} < 0\)\;
	\If(\Comment*[f]{\(\overline{A}_{\eta_{j}} \leq 0\implies (\Primal) \text{ is unbounded}\)}){\(\overline{A}_{\eta_{j}} \leq 0\)}{
	\Return{\((\Primal)\) is unbounded}
	}
	\(\lambda \gets \min_{i\colon\overline{a}_{i \eta_{j}}>0}\left\{ \frac{\overline{x}_{\beta_{i}}}{\overline{a}_{i, \eta_{j}}} \right\}\)\Comment*[r]{Largest choice so that \(\overline{x} + \lambda \overline{z} \geq 0\)}
	\(\overline{x} \gets \overline{x} + \lambda \overline{z} \)\;
	Redetermine \(\beta , \eta \)\;
	}
	}
\end{algorithm}

\begin{remark}
	Note that \(x\) is assumed to be a \hyperref[def:basic-solution]{basic} \hyperref[def:feasible-solution]{feasible solution}.
\end{remark}
\begin{problem*}
	The problem is that is \(\overline{x}+\lambda \overline{z}\) still a \hyperref[def:basic-solution]{basic solution}? And if it is, what is the \hyperref[def:basic-partition]{basic partition}
	that goes with it?
\end{problem*}
\begin{answer}
	We see that after one iteration, one of the \hyperref[def:basis]{basis} index \(i^{\ast}\) will become \hyperref[def:non-basis]{non-basis}, namely
	\[
		(\overline{x} + \lambda \overline{z})_{\beta_{i^{\ast}}} = 0;
	\]
	while one of the \hyperref[def:non-basis]{non-basis} index will need to become \hyperref[def:basis]{basis}, since
	\[
		(\overline{x} + \lambda \overline{z})_{\beta_{i^{\ast}}} = \lambda \overline{e}_j.
	\]
	Namely,
	\begin{table}[H]
		\centering
		\begin{tabular}{cc|c|c|cc}
			\toprule
			                          &           & \(\overline{x}\)            & \(\overline{z}\)                & \(\overline{x} + \lambda \overline{z}\) &                                                                         \\
			\midrule
			\(\beta_{i^{\ast}} \to \) & \(\beta\) & \(\overline{x}_{\beta}\)    & \(\overline{z}_{\beta}\)        & \(\to 0\)                               & \emph{\(\beta_{i^{\ast}}\) becomes \hyperref[def:non-basis]{non-basis}} \\\hline
			                          & \(\eta\)  & \(\overline{x}_{\eta} = 0\) & \(\overline{x}_{\eta} = e_{j}\) & \(\lambda \overline{e}_j\)              & \emph{\(\eta_{j}\) becomes \hyperref[def:basis]{basis}}                 \\
			\bottomrule
		\end{tabular}
	\end{table}
\end{answer}

Now, suppose \(i^{\ast}\) is that chosen index with \(\overline{a}_{i^{\ast} \eta_{j}}>0\) and \(\frac{\overline{x}_{\beta_{i^{\ast}}}}{\overline{a}_{i^{\ast} \eta_{j}}} = \overline{\lambda}\). Then we have \(\beta_{i^{\ast}}\) such that
\[
	\overline{x} + \lambda \overline{z} \implies \overline{x}_{\beta_{i^{\ast}}} + \overline{\lambda} \overline{z}_{\beta_{i^{\ast}}} = \overline{x}_{\beta_{i^{\ast}}}+\frac{\overline{x}_{\beta_{i^{\ast}}}}{\overline{a}_{i^{\ast} \eta_{j}}}\left( -\overline{a}_{i^{\ast}, \eta_{j}} \right) = 0.
\]

So we reasonably suspect that there is a new \hyperref[def:basic-partition]{basic partition} such that
\[
	\begin{alignedat}{5}
		\widetilde{\beta} & \coloneqq ( \beta_1, \beta_2, \dots , \beta_{i^{\ast}-1}, &  & \eta_{j}, \beta_{i^{\ast}+1}, \dots , \beta_m )     \\
		                  &                                                           &  & \updownarrow                                        \\
		\widetilde{\eta}  & \coloneqq ( \eta_1, \eta_2, \dots , \eta_{j - 1},         &  & \beta_{i^{\ast}}, \eta_{j+1}, \dots , \eta_{n-m} ).
	\end{alignedat}
\]

The remaining question is that, is \(A_{\widetilde{\beta}}\) still invertible? Namely, is \(\det A_{\widetilde{\beta}} \neq 0\)?
\begin{lemma}\label{lma:lec7-3}
	After one iteration of \hyperref[algo:worry-free-simplex-algorithm]{worry-free simplex algorithm}, \(A_{\widetilde{\beta}}\) is still invertible.
\end{lemma}
\begin{proof}
	We see that \(A_{\widetilde{\beta}}\) is invertible if and only if \(A^{-1}_{\beta}A_{\widetilde{\beta}}\) is invertible. And since
	\[
		A^{-1}_{\beta}A_{\widetilde{\beta}} = \begin{bmatrix}
			e_1 & e_2 & \dots & e_{i^{\ast}-1} & \overline{A}_{\eta_{j}} & e_{i^{\ast}+1} & \dots & e_m \\
		\end{bmatrix},
	\]
	and since \(\det(A^{-1}_\beta A_{\widetilde{\beta}}) = \overline{a}_{i^{\ast} \eta_{j}}\), if \(\overline{a}_{i^{\ast} \eta_{j}}\neq 0\), it is indeed invertible. But this is an obvious fact by our choice of \(i^{\ast}\).
\end{proof}

\begin{figure}[H]
	\centering
	\incfig{pivot-swap}
	\caption{Pivot swap in terms of \hyperref[def:feasible-region]{feasible region}.}
	\label{fig:pivot-swap}
\end{figure}

Finally, we check that the unique \hyperref[def:basic-solution]{basic solution} for this \hyperref[def:basic-partition]{basic partition}
\(\widetilde{\beta},\ \widetilde{\eta}\) are exactly \(\overline{x} + \overline{\lambda}\overline{z}\).
\begin{lemma}\label{lma:lec7-4}
	The unique solution of \(Ax = b\) having \(x_{\widetilde{\eta}} = 0\) is \(\overline{x} + \overline{\lambda}\overline{z}\).
\end{lemma}
\begin{proof}
	Firstly, \((\overline{x} + \overline{\lambda}\overline{z})_j = 0\) for \(j\in \widetilde{\eta}\). Moreover, \(\overline{x} + \overline{\lambda}\overline{z}\) is
	the unique solution to \(Ax = b\) having \(x_{\widetilde{\eta}} = 0\) because \(A_{\widetilde{\beta}}\) is invertible, namely
	\[
		Ax = b \implies \underbrace{\cancel{A_{\widetilde{\eta}}x_{\widetilde{\eta}}}}_{=0} + A_{\widetilde{\beta}}x_{\widetilde{\beta}} = b \implies x_{\widetilde{\beta}} = A^{-1}_{\widetilde{\beta}}b.
	\]
\end{proof}