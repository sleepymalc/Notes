\lecture{9}{29 Sep.\ 8:00}{Practical Simplex Algorithm}
\begin{note}[\(A_{\beta}^{-1}\) in reality]
	In reality, we don't really calculate \(A^{-1}_{\beta}\), since when calculating
	\[
		A_{\beta}x_{\beta} = b,
	\]
	we do not use \(\overline{x}_{\beta} = A^{-1}_{\beta}b\), instead, we use \href{https://en.wikipedia.org/wiki/LU_decomposition}{LU-Factorization}. And since after applying pivot change, there is only a column change in \(A^{-1}_{\beta}\), we can use the previous result to calculate the new \(\overline{x}_{\beta}\) much faster.
\end{note}

\section{The Word \emph{Simplex}}
For a \hyperref[def:standard-form]{standard form} problem
\[
	\begin{aligned}
		\min~          & c^{\top}x \\
		               & Ax = b    \\
		(\Primal)\quad & x\geq 0,
	\end{aligned}
\]
we can instead consider an equivalent problem formulated as
\[
	\begin{aligned}
		\min~ & z                                        \\
		      & z - c^{\top} x = 0 \iff (c^{\top} x = z) \\
		      & Ax = b                                   \\
		      & x\geq 0.
	\end{aligned}
\]

\begin{prev}
	Our picture is in \(\mathbb{R}^{n-m}\), but we consider \emph{Dantzig picture}, which is in \(\mathbb{R}^{m+1}\)
\end{prev}

\subsection{Column geometry}
Plot columns:
\[
	\underbrace{
		\begin{pmatrix}
			c_1 \\
			A_1 \\
		\end{pmatrix}
		\begin{pmatrix}
			c_2 \\
			A_2 \\
		\end{pmatrix}
		\cdots
		\begin{pmatrix}
			c_n \\
			A_n \\
		\end{pmatrix}}_{n \text{ points in }\mathbb{R}^{m+1}}
\]

The requirement line is
\[
	\begin{pmatrix}
		z \\
		b \\
	\end{pmatrix}.
\]

\begin{figure}[H]
	\centering
	\incfig{column-geometry}
	\caption{Column Geometry.}
	\label{fig:column-geometry}
\end{figure}

\subsection{Simplices (Plural of Simplex)}
\begin{eg}[Simplex]
	An \(n-1\) dimensional simplex in \(\mathbb{R}^n\) with \(n\) standard unit vectors are the corner can be described as
	\[
		\left\{x\in \mathbb{R}^n \colon \sum\limits_{i=1}^{n} x_i = 1, x_i \geq 0\right\}.
	\]
	\begin{figure}[H]
		\centering
		\incfig{simplex}
		\caption{Simplex.}
		\label{fig:simplex}
	\end{figure}
\end{eg}

\begin{note}
	\(m+1\) points of a simplex of dimension.
\end{note}

A simplicial cone is rather simple, the graph below is informative enough.
\begin{figure}[H]
	\centering
	\incfig{simplicial-cones}
	\caption{Simplicial Cones}
	\label{fig:simplicial-cones}
\end{figure}

\section{The Simplex Algorithm}
Now, given a \hyperref[def:standard-form]{standard form} problem \((\Primal)\):
\[
	\begin{aligned}
		\min~          & c^{\top}x \\
		               & Ax = b    \\
		(\Primal)\quad & x\geq 0,
	\end{aligned}
\]
we can then solve \((\Primal)\) in a mathematical rigorous and complete way.

\begin{algorithm}[H]\label{algo:simplex-algorithm}
	\DontPrintSemicolon
	\caption{Simplex Algorithm}
	\KwData{\hyperref[def:standard-form]{standard form} LP \((\Primal)\)}
	\KwResult{\hyperref[def:optimal-solution]{optimal solutions} \(\overline{x}, \overline{y}\), or report \((\Primal)\) is unbounded/infeasible}
	\SetKwData{result}{result}
	\BlankLine
	\SetKwFunction{WFSimplex}{\hyperref[algo:worry-free-simplex-algorithm]{WorryFreeSimplexAlgorithm}}

	\((\Phi_{\epsilon})\gets\)\hyperref[def:perturbed-problem]{algebraic perturbation} to the \hyperref[def:phase-one-problem]{phase one problem} \((\Phi)\)\;
	\(\beta \gets \) \hyperref[prob:phase-one-problem]{basic feasible solution for \((\Phi_{\epsilon })\)}\;
	\result \(\gets\) \WFSimplex{\((\Phi _\epsilon )\), \(\beta\)}\footnote{Adapted to \hyperref[def:perturbed-problem]{algebraically perturbed problems}, and always giving preference to \(x_{n+1}\) for leaving the \hyperref[def:basis]{basis} whenever it's eligible to leave for the unperturbed problem.}\;
	\uIf{\result is unbounded}{
		\Return{\((\Primal)\) has no solution}
	}
	\Else{
		\(\beta\gets\result\)\Comment*[f]{Retrieve the feasible basis for \((\Primal)\)}
	}
	\;
	\((P_{\epsilon} )\gets\) \hyperref[def:perturbed-problem]{algebraic perturbation} to \((\Primal)\)\;
	\Return{\WFSimplex{\((P_{\epsilon} )\), \(\beta\)}\footnote{Again, adapted to \hyperref[def:perturbed-problem]{algebraically perturbed problems}.}}
\end{algorithm}

\chapter{Duality}
Consider the \hyperref[def:standard-form]{standard problem} and its \hyperref[def:dual]{dual}
\[
	\begin{aligned}
		\min~          & c^{\top}x \\
		               & Ax = b    \\
		(\Primal)\quad & x\geq  0,
	\end{aligned}\quad \begin{aligned}
		\max ~       & y^{\top}b               \\
		             & y^{\top}A\leq c^{\top}. \\
		(\Dual)\quad &
	\end{aligned}
\]

\section{The Strong Duality Theorem}
\begin{prev}
	The \hyperref[thm:weak-duality]{weak duality theorem} states that if \(\hat{x}\) is \hyperref[def:feasible-solution]{feasible} for \((\Primal)\), and \(\hat{y}\) is \hyperref[def:feasible-solution]{feasible} for \((\Dual)\), then
	\[
		c^{\top} \hat{x} \geq  \hat{y}^{\top} b.
	\]
	Moreover, the equality holds if and only if \(\hat{x}\) and \(\hat{y}\) are \hyperref[def:optimal-solution]{optimal}.
\end{prev}

\begin{prev}
	The \hyperref[thm:weak-optimal-basis-theorem]{weak optimal basis theorem} states that if we have a \hyperref[def:basic-partition]{basic partition} \(\beta, \eta\), and we also have \(\overline{x}_{\beta}\geq \vec{0}\)(\(\overline{x}\) is \hyperref[def:feasible-solution]{feasible} for \((\Primal)\)) and \(\overline{c}_{\eta} \geq \vec{0}\)(\(\overline{y}\) is \hyperref[def:feasible-solution]{feasible} for \((\Dual)\)), then \(\overline{x}\) and \(\overline{y}\) are both \hyperref[def:optimal-solution]{optimal}.
\end{prev}

Now, we have so-called \hyperref[thm:strong-optimal-basis]{strong optimal basis theorem}.

\begin{theorem}[Strong optimal basis theorem]\label{thm:strong-optimal-basis}
	If \((\Primal)\) has a \hyperref[def:feasible-solution]{feasible solution}, and if \((\Primal)\) is not unbounded, then there exist a \hyperref[def:basic-partition]{basic partition} \(\beta, \eta\) such that \(\overline{x}\) and \(\overline{y}\) are \hyperref[def:optimal-solution]{optimal}, and
	\[
		c^{\top} \overline{x} = \overline{y}^{\top} b.
	\]
\end{theorem}
\begin{proof}
	Since if \((\Primal)\) has a \hyperref[def:feasible-solution]{feasible solution} and is not unbounded, we can just run the \hyperref[algo:simplex-algorithm]{simplex algorithm}, which will terminate with a basis \(\beta\) such that the associated \hyperref[def:basic-solution]{basic solution} \(\overline{x}\) and the associated \hyperref[def:dual-basic-solution]{dual solution} \(\overline{y}\) are \hyperref[def:optimal-solution]{optimal}.
\end{proof}

We see that this leads to another similar result.

\begin{theorem}[Strong duality theorem]\label{thm:strong-duality}
	If \((\Primal)\) has a \hyperref[def:feasible-solution]{feasible solution} and \((\Primal)\) is not unbounded, then there exist \hyperref[def:optimal-solution]{optimal solutions} \(\hat{x}\) and \(\hat{y}\) with
	\[
		c^{\top} \hat{x} = \hat{y}^{\top} b.
	\]
\end{theorem}

\begin{note}
	The proof of these two theorems are by directly using the \emph{mathematical complete} version of \hyperref[algo:simplex-algorithm]{simplex algorithm}, hence the completeness of \hyperref[algo:simplex-algorithm]{simplex algorithm} (namely the \hyperref[def:phase-one-problem]{phase one problem} and the \hyperref[def:perturbed-problem]{perturbation}) is important.
\end{note}

\begin{table}[H]
	\centering
	\begin{tabular}{c|c|c|c|c}
		\toprule
		\hyperref[algo:simplex-algorithm]{\emph{Simplex Algorithm}}                                                         & \((\Primal) \backslash (\Dual)\)                                                                                         & \begin{tabular}{@{}c@{}}\hyperref[def:optimal-solution]{optimal}\\\hyperref[def:optimal-solution]{solution}\end{tabular} & infeasible  & unbounded   \\
		\midrule
		\(\overline{c}_{\eta}\geq \vec{0}\implies\) Stop                                                                    & \begin{tabular}{@{}c@{}}\hyperref[def:optimal-solution]{optimal}\\\hyperref[def:optimal-solution]{solution}\end{tabular} & \(\surd \)                                                                                                               & \(\times \) & \(\times \) \\
		\midrule
		\begin{tabular}{@{}c@{}}\hyperref[def:optimal-solution]{optimal} \(x_{n+1}\) in \(\Phi\) \\is positive\end{tabular} & infeasible                                                                                                               & \(\times \)                                                                                                              & \(\surd\)   & \(\surd \)  \\
		\midrule
		\(\overline{A}_{\eta_{j}}\leq \vec{0}\implies\) Stop                                                                & unbounded                                                                                                                & \(\times \)                                                                                                              & \(\surd \)  & \(\times \) \\
		\bottomrule
	\end{tabular}
	\caption{Comparison between \((\Primal)\) and \((\Dual)\)}
	\label{tab:label}
\end{table}