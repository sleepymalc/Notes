\lecture{24}{06 Dec. 08:00}{Cutting Planes Algorithm}
\subsection{Cutting Planes Algorithm}
\begin{prev}
	We focus on the problem in the form of
	\begin{align*}
		\max~                    & y^{\top}b                                   \\
		                         & y^{\top}A\leq c^{\top}                      \\
		(D_{\mathfrak{X} })\quad & y_{i}\text{ integer for }i = 1, \ldots , m.
	\end{align*}
	\begin{note}
		Compare to what we have seen, now we require all \(y_{i}\) be integer. Further, as before, we also let \((P)\) be
		\begin{align*}
			\min~    & c^Tx     \\
			         & Ax = b   \\
			(P)\quad & x\geq 0.
		\end{align*}
	\end{note}
\end{prev}
\begin{remark}
	We assume that the data is all integer.
\end{remark}

Now, we choose \(w\in \mathbb{\MakeUppercase{R}}^n\), \(w\geq 0\). Then the constraint becomes
\[
	y^{\top}(Aw) \leq c^{\top}w.
\]
\begin{remark}
	This valid for all \(y\) such that
	\[
		y^{\top}A\leq c^{\top},
	\]
	no matter it's integer or not.
\end{remark}

Suppose \(Aw\in\mathbb{\MakeUppercase{Z}}^m\). With the fact that \(y\in \mathbb{\MakeUppercase{Z}}^m\), then for
\[
	y^{\top}(Aw)\leq c^{\top}w,
\]
we can actually get
\[
	y^{\top}(Aw)\leq \left\lfloor c^{\top}w \right\rfloor.
\]
\begin{remark}
	This is valid for all \(y\) that satisfies
	\[
		y^{\top}(Aw)\leq c^{\top}w
	\]
	and \(y\in \mathbb{\MakeUppercase{Z}}^m\).
\end{remark}
\begin{figure}[H]
	\centering
	\incfig{Cutting-planes-tighter-bound}
	\label{fig:Cutting-planes-tighter-bound}
\end{figure}

We now solve \((P)\) and get an optimal basis \(\beta\). Consider
\[
	\overline{y}^{\top}\coloneqq c_{\beta}^{\top}A^{-1}_{\beta}.
\]
Notice that if \(\overline{y}\in\mathbb{\MakeUppercase{Z}}^m\), then \(\overline{y}\) solves \(D_{\mathfrak{X}}\). Otherwise, suppose
\(\overline{y}_{i}\notin \mathbb{\MakeUppercase{Z}}\), then let
\[
	\widetilde{b}\coloneqq e_{i} + A_{\beta}r\in\mathbb{\MakeUppercase{Z}}^m,
\]
where \(r\in\mathbb{\MakeUppercase{Z}}^m\). We then see a theorem.
\begin{theorem}
	If \(\overline{y}^{\top}\widetilde{b}\notin \mathbb{\MakeUppercase{Z}}\), then
	\[
		y^{\top}\widetilde{b}\leq \left\lfloor \overline{y}^{\top}\widetilde{b} \right\rfloor
	\]
	cuts off \(\overline{y}\).
\end{theorem}
\begin{proof}
	Since
	\[
		\overline{y}^{\top}\widetilde{b} = \overline{y}^{\top}(e_{i}+A_{\beta}r) = \overline{y}_{i}^{\top} + \overline{y}^{\top}A_{\beta}r = \overline{y}_{i}^{\top} +c_{\beta}^{\top}A^{-1}_{\beta}A_{\beta}r = \overline{y}_{i}^{\top} +c^{\top}_{\beta}r
	\]
	We see that \(\overline{y}_{i}\notin\mathbb{\MakeUppercase{Z}}\), \(c_{\beta}^{\top}, r\in \mathbb{\MakeUppercase{Z}}\), hence we have
	\[
		\overline{y}^{\top}\widetilde{b} = \overline{y}_{i}+c^{\top}_{\beta}r\notin \mathbb{\MakeUppercase{Z}}.
	\]
	Now, we need to check that \(y^{\top}\widetilde{b}\geq \left\lfloor \overline{y}^{\top}\widetilde{b} \right\rfloor\) is satisfied by \(\overline{y}\).

	\begin{intuition}
		Consider if the inequality is
		\[
			\vec{0}^{\top}y\leq -1,
		\]
		then it makes no sense.
	\end{intuition}

	Let \(H\coloneqq A^{-1}_{\beta}\), then \(H_{\cdot i} = A^{-1}_{\beta}e_{i}\). Further, we let \(w\coloneqq H_{\cdot i}+r\). Since we need
	\(w\geq \vec{0}\), we can always choose \(r\in \mathbb{\MakeUppercase{Z}}^m\) so that \(w\geq \vec{0}\). Specifically, we choose
	\[
		r_K \geq -\left\lfloor h_{Ki} \right\rfloor
	\] for \(K = 1, \ldots , m\).

	Instead of considering \(y^{\top}A\leq c^{\top}\), we consider \(y^{\top}A_{\beta}\leq c_{\beta}^{\top}\). Then we have
	\[
		\left(y^{\top}A_{\beta}\right)\left(H_{\cdot i}+r\right) \leq c_{\beta}^{\top}\left(H_{\cdot i}+r\right).
	\]
	This is equivalence to
	\[
		\left(y^{\top}A_{\beta}\right)\left(A^{-1}_{\beta}e_{i}+r\right) \leq c_{\beta}^{\top}\left(A^{-1}_{\beta}e_{i}+r\right).
	\]
	After expanding, we have
	\[
		y_{i}+y^{\top}A_{\beta}r\leq \overline{y}_{i}+c_{\beta}^{\top}r,
	\]
	which can be written as
	\[
		y^{\top}\left(e_{i}+A_{\beta}r\right) \leq \overline{y}^{\top}\left(e_{i}+A_{\beta}r\right)
	\]
	since \(\overline{y}^{\top} = c_{\beta}^{\top}A^{-1}_{\beta}\). Then we see
	\[
		y^{\top}\widetilde{b}\leq \left\lfloor \overline{y}^{\top}\widetilde{b} \right\rfloor.
	\]

	Lastly, we need \(Aw\) are all integers. This is true since
	\[
		A_{\beta}w = A_{\beta}\left(A^{-1}_{\beta}e_{i}+r\right) = e_{i}+A_{\beta}r \in \mathbb{\MakeUppercase{Z}}^m.
	\]
\end{proof}

Revisiting the \hyperref[eg:branch-and-bound]{example}. Now we see that the cutting plane algorithm will need at least \(2k\) steps
for such a triangle with height \(k\), since it can only cut off one point at a time.

\begin{eg}
	Now we see some bad examples for Branch and Bound. Consider the following integer programming problem.
	\begin{align*}
		\min~ & y_{n+1}                                                              \\
		      & 2y_{1} + 2y_2 + \ldots +2y_n + y_{n+1} = \underbrace{n}_{\text{odd}} \\
		      & 0\leq y_{i}\leq 1 \text{ for }i = 1, \ldots , n+1, \text{ integer}.
	\end{align*}
	We see that the optimum has \(y_{n+1} = 1\).

	If \(n = 17\). Then we can let
	\[
		y_{18} = 0,\quad y_1 = y_2 = \ldots = y_8 = 1,\quad y_9 = \frac{1}{2},\quad y_{10} = \ldots = y_{17} = 0.
	\]
	We immediately see there are lots of solutions like this, namely there are lots of symmetric groups going on such that
	half of the variables are \(1\), and another half of the variables are \(0\). This is pretty bad
	for the branch and bound algorithm since it will look at all of them. Analytically, we see that this will go into \(\frac{n}{2}\)
	depth in the recursion tree, hence it's clearly exponential.
\end{eg}