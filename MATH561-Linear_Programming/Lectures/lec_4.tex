\chapter{Algebra Versus Geometry}
\lecture{4}{13 Sep. 08:00}{Basis Partition}
\section{Elementary Row Operations}
We can
\begin{enumerate}
	\item permute rows
	\item multiply a row by a non-zero factor
	\item add a multiple of a row to another row
	\item \emph{permutation in columns}
\end{enumerate}
What does 4. actually means?
Consider
\[
	A = \left[ A_1, \ldots , A_n \right]_{m\times n} .
\]
A permutation is a function like
\[
	(1, \ldots , n) \to (\sigma(1), \ldots , \sigma(n)).
\]
Then the permuted matrix \(A_{\sigma}\) i
\[
	A_{\sigma} = \left[ A_{\sigma(1)}, \ldots , A_{\sigma(n)} \right].
\]
With the same permutation for \(x\), we have
\[
	x_{\sigma} = \begin{pmatrix}
		x_{\sigma(1)} \\
		\vdots        \\
		x_{\sigma(n)} \\
	\end{pmatrix}.
\]
We then easily see that
\[
	Ax = \sum\limits_{j=1}^{n} A_i x_i = \sum\limits_{j=1}^{n} A_{\sigma(j)}x_{\sigma(j)}.
\]
Hence,
\[
	Ax = b \iff A_{\sigma} x_{\sigma} = b.
\]

\section{Basic Partition}
We denote a partition by
\[
	\beta \coloneqq (\beta_1, \ldots , \beta_m),\quad \eta \coloneqq (\eta_1, \ldots , \eta_{n-m}),
\]
where \(\beta\) is called \emph{basic}, while \(\eta\) is called \emph{non-basic}. This is a partition of \(\{1, \ldots , n\}\).
The only condition we require for a basic partition is that
\[
	A_{\beta} = \left[ A_{\beta_1}, \ldots , A_{\beta_m} \right]_{m\times m}
\]
is invertible.

Associate a basic partition with a \emph{basic solution} \(\overline{x}\), which is \textbf{defined} as
\[
	\overline{x}_{\eta} = \begin{pmatrix}
		\overline{x}_{\eta_1}     \\
		\vdots                    \\
		\overline{x}_{\eta_{n-m}} \\
	\end{pmatrix}\coloneqq \begin{pmatrix}
		0      \\
		\vdots \\
		0      \\
	\end{pmatrix},\qquad \overline{x}_{\beta} = \begin{pmatrix}
		\overline{x}_{\beta_1} \\
		\vdots                 \\
		\overline{x}_{\beta_m} \\
	\end{pmatrix}\coloneqq A^{-1}_{\beta}b
\]

\begin{intuition}
	This of course makes sense, since we know that if this is a feasible solution for a standard form problem, then \(\overline{Ax} = b\), which means
	\[
		\left[ A_{\beta}, A_{\eta} \right] \begin{pmatrix}
			\overline{x}_{\beta} \\
			\overline{x}_{\eta}  \\
		\end{pmatrix} = b \implies A_{\beta}\overline{x}_{\beta} + A_{\eta}\underbrace{\overline{x}_{\eta}}_{=0} = b\implies \overline{x}_{\beta} = \underbrace{A^{-1}_{\beta}}_{\text{invertible}}b
	\]
\end{intuition}

\begin{remark}
	After choosing \(\eta\), we see that \(\overline{x}_{\beta}\) is determined.
\end{remark}

