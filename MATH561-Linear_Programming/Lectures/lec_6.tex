\lecture{6}{20 Sep. 08:00}{Feasible Direction and Ray}
\section{Feasible Directions}
We'll talk about an important concept, but before this, we first play around with the \hyperref[def:standard-form]{standard form} problem a little. Consider
\begin{align*}
	\min~ & c^Tx     \\
	      & Ax = b   \\
	      & x\geq 0.
\end{align*}
It's obvious that it's equivalent to
\begin{align*}
	\min~ & c^{\top}_{\beta}x_{\beta} + c^{\top}_{\eta}x_{\eta} \\
	      & A_{\beta}x_{\beta} + A_{\eta}x_{\eta} = b           \\
	      & x_{\beta}\geq 0, x_{\eta}\geq 0
\end{align*}
Further, we have
\begin{align*}
	\min~ & c^{\top}_{\beta}(A^{-1}_{\beta}b - A^{-1}_{\beta}A_{\eta}x_{\eta} ) + c_{\eta}^{\top}x_{\eta} \\
	      & x_{\beta} + A^{-1}_{\beta}A_{\eta}x_{\eta} = A^{-1}_{\beta}b                                  \\
	      & x_{\beta}\geq 0, x_{\eta}\geq 0
\end{align*}
since from the \hyperref[def:constraints]{constraints}, we have \(x_{\beta} = A^{-1}_{\beta}b - A^{-1}_{\beta}A_{\eta}x_{\eta}\). Finally, we see that
the \hyperref[def:objective-function]{objective function} now only depends on \(x_{\eta}\), hence,
\begin{align*}
	c^{\top}_{\beta}A^{-1}_{\beta}b + \min~ & (c_{\eta}^{\top} - c_{\beta}^{\top}A^{-1}_{\beta}A_{\eta})x_{\eta} \\
	                                        & A^{-1}_{\beta}A_{\eta}x_{\eta} \leq A^{-1}_{\beta}b                \\
	                                        & x_{\beta}\geq 0, x_{\eta}\geq 0.
\end{align*}

\begin{note}[Reduced cost]\label{note:reduced-cost}
	\(c_{\eta}^{\top} - c_{\beta}^{\top}A^{-1}_{\beta}A_{\eta}\) is what we called \emph{reduced costs}.
	We'll see that we want \hyperref[def:reduced-cost]{reduced costs} to be zero.
\end{note}

Now, with this intuition, we have the following definition.
\begin{definition}[Feasible direction]\label{def:feasible-direction}
	Suppose \(\hat{x}\in \mathcal{S}\), where \(\mathcal{S}\) is a \hyperref[def:convex-set]{convex set}.
	\(\hat{z}\) is a \emph{feasible direction} relative to \(\hat{x}\) if there exists some \(\epsilon>0\) such that
	\[
		\hat{x}+\epsilon \hat{z} \in \mathcal{S}.
	\]
	\begin{figure}[H]
		\centering
		\incfig{feasible-direction}
		\label{fig:feasible-direction}
	\end{figure}
\end{definition}

\begin{remark}
	For a \hyperref[def:primal]{primal} \((P)\), we must have \(A \hat{z} = 0\) if \(\hat{z}\) is a \hyperref[def:feasible-direction]{feasible direction}.
\end{remark}
\begin{explanation}
	We see that in order to let \(\hat{z}\) to be a \hyperref[def:feasible-direction]{feasible direction}, we need to have
	\[
		A(\hat{x} + \epsilon \hat{z}) = \underbrace{A \hat{x}}_{=b} + \epsilon A \hat{z} = b \iff A \hat{z} = 0
	\]
\end{explanation}

Let the \hyperref[def:basic-partition]{basic partition} \(\beta, \eta\) being
\[
	\beta = (\beta_1, \ldots , \beta_m), \qquad \eta = (\eta_1, \underset{\substack{\uparrow \\ \eta_j}}{\ldots} , \eta_{n-m}),
\]
where we choose \(j\) from \(1\leq j \leq n-m\), which means we choose an \(\eta_j\) from \(\eta\). Then, we see that there is a \hyperref[def:basic-direction]{basic direction}
\(\overline{z}\) associated with this particular \hyperref[def:basic]{basic} and this \(j\) defined as follows.

\begin{definition}[Basic direction]\label{def:basic-direction}
	Given a \hyperref[def:basic-partition]{basic partition} \(\beta, \eta\), we say that \(\overline{z}\) is a \emph{basic direction} associated with this \hyperref[def:basic]{basic} \(\beta\)
	and a \(j\) such that \(1 \leq j \leq n-m\) if
	\[
		\overline{z}_{\eta_j} = 1 \implies \begin{dcases}
			\overline{z}_{\eta}  & \coloneqq e_j = \begin{pmatrix}
				                                       0      \\
				                                       \vdots \\
				                                       1      \\
				                                       \vdots \\
				                                       0      \\
			                                       \end{pmatrix}\leftarrow j \\
			\overline{z}_{\beta} & \coloneqq -A^{-1}_{\beta}A_{\eta_j}.
		\end{dcases}
	\]
\end{definition}

\begin{lemma}\label{lma:lec6-1}
	Given a \hyperref[def:basic-direction]{basic direction} \(\overline{z}\), \(\overline{z}\) is \hyperref[def:feasible-direction]{feasible} from \(\overline{x}\) if
	\[
		0< \min\left\{\frac{\overline{b}_i}{\overline{a}_{i \eta_{j}}}\geq 0 \text{ for }i \text{ such that }\overline{a}_{i \eta_{j}}>0\right\}.
	\]
\end{lemma}
\begin{proof}
	For a \hyperref[def:basic-direction]{basic direction} \(\overline{z}\) being \hyperref[def:feasible-direction]{feasible}, we need to check
	\begin{enumerate}
		\item \(A(\overline{x} + \epsilon \overline{z}) = b\):
		      Since \[
			      A \overline{z} = 0 \iff A_{\beta}\overline{z}_{\beta} + A_{\eta}\overline{z}_{\eta} = 0 \iff A_{\beta}\overline{z}_{\beta}+A_{\eta}e_{j} = A_{\beta}\overline{z}_{\beta}+A_{\eta_j} = 0,
		      \]
		      hence \(A \overline{z} = 0\) from the fact that \(\overline{z} _\beta = - A_\beta ^{-1} A_{\eta _{j} }\), which implies
		      \[
			      A_\beta \overline{z} _\beta + A_{\eta _{j} } = A_\beta (- A_\beta ^{-1} A_{\eta _{j} }) + A_{\eta _{j} } = -A_{\eta _{j} } + A_{\eta _{j} } = 0,
		      \]
		      hence \(A \overline{z} = 0\), which means \(A(\overline{x} +\epsilon \overline{z} ) = A \overline{x} = b\). \(\surd\)
		\item \(\overline{x} + \epsilon \overline{z} \geq 0\):
		      Since
		      \[
			      \begin{split}
				      &\overline{x}_{\eta} + \epsilon \overline{z}_{\eta} = 0 + \epsilon e_j \geq  0\\
				      &\overline{x}_{\beta}+ \epsilon \overline{z}_{\beta} = \underbrace{A^{-1}_{\beta}b}_{\geq 0} - \underbrace{\vphantom{A^{-1}_{\beta}}\epsilon}_{>0} A^{-1}_{\beta}A_{\eta_j}\overset{?}{\geq} 0,
			      \end{split}
		      \]
		      hence we just need to make sure \(\overline{x} _\beta + \epsilon \overline{z} _\beta \geq 0\). Denote
		      \(\overline{b} \coloneqq A^{-1}_{\beta} b,\ \overline{A}_{\eta_j} \coloneqq A^{-1}_{\beta}A_{\eta_j}\), then the requirement becomes
		      \[
			      \begin{split}
				      \overline{b} - \epsilon \overline{A}_{\eta_{j}}\geq 0 &\iff \overline{b}_i - \epsilon \overline{a}_{i \eta_{j}}\geq 0, \text{ for }i = 1, \ldots , m\\
				      &\iff \underbrace{\overline{b}_i}_{\geq 0} \geq \epsilon \overline{a}_{i \eta_{j}}, \text{ for }i - 1, \ldots , m.
			      \end{split}
		      \]
		      We finally have
		      \[
			      \epsilon \leq \frac{\overline{b}_i}{\overline{a}_{i \eta_j}},\qquad \underset{1\leq i\leq m}{\forall}\  \overline{a}_{i \eta_{j}}>0.
		      \]
		      Notice that if \(\overline{a}_{i \eta_{j}}\leq 0\), there is no restriction on \(\epsilon\) being \(\geq 0\), so the result follows.
	\end{enumerate}
\end{proof}

\begin{note}
	Notice that we can denote \(A\) by
	\[
		A = \begin{bmatrix}
			A_{\eta} & A_{\beta} \\
		\end{bmatrix}.
	\]
	Then since \(A_{\beta}\) is invertible, so
	\[
		A^{-1}_{\beta}\begin{bmatrix}
			A_{\eta} & A_{\beta} \\
		\end{bmatrix} = \begin{bmatrix}
			A^{-1}_{\beta}A_{\eta} & I \\
		\end{bmatrix}_{m\times n}.
	\]

	Considering
	\[
		\begin{bmatrix}
			I                       \\
			-A^{-1}_{\beta}A_{\eta} \\
		\end{bmatrix},
	\]
	we have
	\[
		\underbrace{
			\begin{bmatrix}
				I                       \\
				-A^{-1}_{\beta}A_{\eta} \\
			\end{bmatrix}}_{\dim(CS) = n-m}\underbrace{\vphantom{\begin{bmatrix}
					\\ \\
				\end{bmatrix}}\begin{bmatrix}
				A^{-1}_{\beta}A_{\eta} & I \\
			\end{bmatrix}}_{\dim(RS)=m} = 0.
	\]
	And since the dimension for the first matrix is \(n\times (m-n)\), we see that the columns of the first matrix form a \textbf{basis} for the null space of \(\begin{bmatrix}
		A_{\eta} & A_{\beta} \\
	\end{bmatrix}\), namely \(A\). Furthermore, one can see that \(\overline{z}\) is the \(j^{th}\) columns of \(\begin{bmatrix}
		I                       \\
		-A^{-1}_{\beta}A_{\eta} \\
	\end{bmatrix}\) for a choice of \(j\).
\end{note}

\section{Feasible Rays}
\begin{definition}[Ray]\label{def:ray}
	\(\hat{z}\) is called a \emph{ray} of a \hyperref[def:convex-set]{convex set} \(\mathcal{C}\) of \(\hat{x}\in \mathcal{C}\) if
	\[
		\forall \lambda>0\quad \hat{x} + \lambda \hat{z} \in \mathcal{C}.
	\]
	\begin{figure}[H]
		\centering
		\incfig{ray}
		\label{fig:ray}
	\end{figure}
\end{definition}

Suppose \(\hat{x}\in\mathcal{C}\), where \(\mathcal{C}\) is the \hyperref[def:feasible-region]{feasible region} of
\begin{align*}
	Ax & \geq b  \\
	x  & \geq 0,
\end{align*}
then we see that in order to let \(\lambda\) arbitrarily large, we need
\[
	A(\hat{x} + \lambda \hat{z}) = \underbrace{A \hat{x}}_{=b} + \lambda \underbrace{A \hat{z}}_{=0} = b\implies \hat{z} \in n.s.(A).
\]

\begin{problem}
\[
	\underbrace{\hat{x}}_{\geq 0} + \underbrace{\lambda}_{>0} \hat{z} \overset{?}{\geq} 0 \implies \hat{z} \geq 0.
\]
\end{problem}

This means that starts from the idea of \hyperref[def:basic-direction]{basic direction}, \(\hat{z}\) is a \hyperref[def:ray]{ray} if
\[
	\hat{z} \geq 0 \iff A^{-1}_{\beta}A_{\eta_j} \leq 0.
\]

We have another concept about \hyperref[def:ray]{ray}.
\begin{definition}[Extreme ray]\label{def:extreme-ray}
	\(\hat{z}\) is an \emph{extreme ray} of a \hyperref[def:convex-set]{convex set} \(\mathcal{S}\) if we \textbf{cannot} write
	\[
		\hat{z} = z^1 + z^2 \text{ with }z^1 \neq \mu z^2,
	\]
	where \(z^1, z^2\) being \hyperref[def:ray]{rays} of \(\mathcal{S}\) and \(\mu\neq 0\).
	\begin{figure}[H]
		\centering
		\incfig{extreme-ray}
		\caption{All three arrows are \hyperref[def:ray]{rays}, but only the red ones are \hyperref[def:extreme-ray]{extreme}.}
		\label{fig:extreme-ray}
	\end{figure}
\end{definition}

\begin{remark}
	We can compare the \emph{non-negative \hyperref[def:basic-direction]{basic direction}} with \emph{\hyperref[def:extreme-ray]{extreme ray}}.
	\[
		\begin{alignedat}{4}
			\text{\hyperref[def:basic-solution]{Basic solution} }\overline{x}=\begin{cases}
				\overline{x}_{\beta}\coloneqq A^{-1}_{\beta}b\geq 0 \\
				\overline{x}_{\eta} \coloneqq 0
			\end{cases} &&\qquad\iff\qquad&\text{Extreme points of the \hyperref[def:feasible-region]{feasible region}}\\
			\text{\textbf{\hyperref[def:basic-direction]{basic} \hyperref[def:feasible-direction]{feasible} direction}} && \qquad \text{v.s.}\qquad&\text{\textbf{Geometry}}\\
			\text{(\hyperref[def:basic-direction]{Basic direction} that are non-negative)}&& &\text{(\hyperref[def:extreme-ray]{Extreme Ray})}
		\end{alignedat}
	\]
\end{remark}