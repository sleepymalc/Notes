\lecture{3}{10 Jan.\ 10:00}{Deformation Retraction}
\begin{prev}
	A \hyperref[def:deformation-retraction]{\emph{deformation retraction}} is a \hyperref[def:homotopy]{homotopy}
	of maps \(\mathrm{rel} B\) \(X\to X\) from \(\identity_X\) to a \hyperref[def:retraction]{retraction} from \(X\) to \(B\). Then \(B\) is a
	\hyperref[def:deformation-retraction]{\emph{deformation retract}}.
\end{prev}

\begin{eg}
	\(S^1\) is a \hyperref[def:deformation-retraction]{deformation retraction} of \(D^{2}\setminus \{0\}\).
\end{eg}
\begin{explanation}
	Indeed, since
	\[
		F_t(x) = t\cdot \frac{x}{\left\lVert x\right\rVert } + (1 - t)x.
	\]
	\begin{figure}[H]
		\centering
		\incfig{eg:punched-circle}
		\caption{The \hyperref[def:deformation-retraction]{deformation retraction} of \(D^{2}\setminus \{0\}\) is just to
			\emph{enlarge} that hole and push all the interior of \(D^2\) to the boundary, which is \(S^1\).}
		\label{fig:eg:punched-circle}
	\end{figure}
\end{explanation}
\begin{eg}
	\(\mathbb{R} ^n\) \hyperref[def:deformation-retraction]{deformation retracts} to \(0\).
\end{eg}
\begin{explanation}
	Indeed, since
	\[
		F_t(x). = (1 - t)x.
	\]
	\begin{remark}
		This implies that \(\mathbb{R} ^n\simeq *\), hence we see that
		\begin{itemize}
			\item dimension
			\item compactness
			\item etc.
		\end{itemize}
		are \textbf{not} \hyperref[def:homotopy]{homotopy} invariants.
	\end{remark}
\end{explanation}
\begin{eg}
	\(S^1\) is a \hyperref[def:deformation-retraction]{deformation retract} of a cylinder and a Möbius band.
\end{eg}
\begin{explanation}
	For a cylinder, consider \(X\times I \to X\). Define \hyperref[def:homotopy]{homotopy} on a closed rectangle, then
	verify it induces map on quotient.

	For a Möbius band, we define a \hyperref[def:homotopy]{homotopy} on a closed rectangle, then verify
	that it respect the equivalence relation.

	Finally, we use the universal property of quotient topology to argue that we get a
	\hyperref[def:homotopy]{homotopy} on Möbius band.

	\begin{figure}[H]
		\centering
		\incfig{eg:cylinder-mobiusband}
		\caption{The \hyperref[def:deformation-retraction]{deformation retraction} for Cylinder and Möbius band}
		\label{fig:eg:cylinder-mobiusband}
	\end{figure}

	\begin{remark}
		We see that \(\text{Möbius band } \simeq S^1 \simeq \text{ cylinder} \), hence the orientability
		is \textbf{not} \hyperref[def:homotopy]{homotopy} invariant.
	\end{remark}
\end{explanation}