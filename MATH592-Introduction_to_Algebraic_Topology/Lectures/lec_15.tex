\lecture{15}{9 Feb. 10:00}{}
Before proving \autoref{prop:lifting-criterion}, we first see an application.
\begin{eg}
	Prove that every continuous map \(f\colon \mathbb{\MakeUppercase{r}} P^{2}\to S^1\) is \hyperref[def:nullhomotopic]{nullhomotopic}.
	\begin{proof}
		If we can show that there is a \hyperref[prop:homotopy-lifting-property]{lift}\(\widetilde{f} \colon \mathbb{\MakeUppercase{r}} P^{2}\to \mathbb{\MakeUppercase{r}}\) of \(f\),
		then we're done since we can apply the \hyperref[eg:lec1:straight-line-homotopy]{straight line} \hyperref[def:nullhomotopic]{nullhomotopy}
		on \(\mathbb{\MakeUppercase{r}} \) since
		\[\begin{tikzcd}
				& {\mathbb{R}} \\
				{\mathbb{R}P^2} & {S^1}
				\arrow["f"', from=2-1, to=2-2]
				\arrow["{\widetilde{f}}", from=2-1, to=1-2]
				\arrow["p", from=1-2, to=2-2]
			\end{tikzcd}\]
		and consider \(f = p \circ \widetilde{f} \) compose \hyperref[def:nullhomotopic]{nullhomotopy} with \(p\), so \(f\simeq \text{constant map} \).
		Specifically, since \(\pi _1(\mathbb{\MakeUppercase{r}} P^{2}) = \quotient{\mathbb{\MakeUppercase{z}}}{2 \mathbb{\MakeUppercase{z}}}\) and
		\(\pi _1(S^1) = \mathbb{\MakeUppercase{z}} \), hence
		\[
			f_\ast (\pi _1(\mathbb{\MakeUppercase{r}} P^{2} )) = 0
		\]
		since \(\mathbb{\MakeUppercase{z}} \) has no (nonzero) torsion. So it \hyperref[prop:homotopy-lifting-property]{lifts} by
		\autoref{prop:lifting-criterion}.
	\end{proof}
\end{eg}

Now we can proof \autoref{prop:lifting-criterion}.
\begin{proof}
	We prove two directions as follows.
	\begin{itemize}
		\item \textbf{Necessary}. We see that
		      \[
			      p_\ast \circ \widetilde{f} _\ast = f_\ast
		      \]
		      follows from the functoriality of \(\pi _1\).
		\item \textbf{Sufficient}. Let \(x\in X\). Choose a path \(\gamma\) from \(x_0\) to \(x\) by the assumption that \(X\) is \hyperref[def:path]{path}-connected.
		      Then, \(f \gamma \) has a unique \hyperref[prop:homotopy-lifting-property]{lift} starting at \(\widetilde{y} _0\), denote by \(\widetilde{f\gamma}\).
		      Now, define
		      \[
			      \widetilde{f} (x) = \widetilde{f \gamma } (1).
		      \]
		      Then, we need to check
		      \begin{enumerate}
			      \item \(\widetilde{f} \) is well-defined. Suppose \(\gamma , \gamma ^\prime \) are \hyperref[def:path]{paths} in \(X\) from \(x_0\)
			            to \(x\). \textbf{WTS}
			            \[
				            \widetilde{f \gamma } ^\prime (1) = \widetilde{f \gamma } (1)
			            \]
			            since \([(f \gamma ^\prime)\cdot (f \gamma ) ]\) is a class of loops in \(\mathrm{Im} (\pi _1(X))\). By hopothesis, this class in \(\mathrm{Im} (p_*)\).
			            It lifts to a loop based at \(\widetilde{y} _0\). By uniqueness of lifts, this loop is
			            \[
				            (f \gamma ^\prime )\cdot (f \gamma ).
			            \]
			            Thus the endpoints agree.
			            \begin{figure}[H]
				            \centering
				            \incfig{pf:prop:lifting-criterion}
				            \label{fig:pf:prop:lifting-criterion}
			            \end{figure}
			      \item \(\widetilde{f} \) is continuous.
		      \end{enumerate}

	\end{itemize}
\end{proof}