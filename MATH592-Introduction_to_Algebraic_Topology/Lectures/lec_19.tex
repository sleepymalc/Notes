\lecture{19}{18 Feb. 10:00}{Homology}
\section{Homology}
\subsection{Motivation for Homology}
Informally, the higher \hyperref[def:homotopy]{homotopy} groups is defined as
\[
	\pi _{n} (X, x_0)\colon I^n_\ast \to (X, x_0),\quad \partial I^n \mapsto x_0
\]

\begin{figure}[H]
	\centering
	\incfig{higher-homotopy-group-dim2}
	\label{fig:higher-homotopy-group-dim2}
\end{figure}
\begin{figure}[H]
	\centering
	\incfig{higher-homotopy-group-dim2-2}
	\label{fig:higher-homotopy-group-dim2-2}
\end{figure}


We see that it's extremely hard to compute higher \hyperref[def:fundamental-group]{fundamental group}. Hence instead,
we will study higher dimension structure of \(X\) via \emph{homology}.

\begin{itemize}
	\item \textbf{Cons.}
	      \begin{itemize}
		      \item The definition is more opaque at first encounter.
	      \end{itemize}
	\item \textbf{Pros.}
	      \begin{itemize}
		      \item Lots of computational tools
		      \item Functional
		      \item Abelian Groups
		            \begin{remark}
			            More like \(\pi _n\) for \(n>1\).
		            \end{remark}
		      \item No basepoints
		      \item Can compute using \hyperref[def:CW-subcomplex]{CW} structure.
		      \item Good properties. For example, \(H_{n} = 0\) if \(n > \mathrm{dim} X\)
	      \end{itemize}
\end{itemize}

\subsection{Simplical Homology}
\begin{figure}[H]
	\centering
	\incfig{simplical-homology-venn-diagram}
	\label{fig:simplical-homology-venn-diagram}
\end{figure}
\subsubsection{Simplex}
We see that
\begin{enumerate}
	\item \(0\)-simplex. A point.
	\item \(1\)-simplex. Interval.
	\item \(2\)-simplex. Triangle.
	      \par The top of which is the \(2\)-disk and remember cell structure (edges and vertices) and remember orientation (ordering on vertices).
	\item \(3\)-simplex. Tetrahedron.
	      \par The top of which is the \(3\)-disk and cells and the orientation.
	\item \(n\)-simplex. The convex hull of \((n+1)\)-points in general position in \(\mathbb{\MakeUppercase{r}} ^n\).
	      \par For example, \(0+n\) standard basis elements.
\end{enumerate}
\begin{figure}[H]
	\centering
	\incfig{simplex}
	\label{fig:simplex}
\end{figure}

An alternative definition can be done.
\begin{definition}[Standard simplex]\label{def:standard-simplex}
	We say that an \(n\)-dimensional \emph{standard simplex}, denoted by \(\triangle^n\) is
	\[
		\triangle^n = \left\{(t_0, \ldots , t_{n}  \in \mathbb{\MakeUppercase{r}} ^n\mid t_i \geq 0, \sum\limits_{i}t_{i}  = 1 )\right\}
	\]
	\begin{figure}[H]
		\centering
		\incfig{def:standard-simplex}
		\label{fig:def:standard-simplex}
	\end{figure}
\end{definition}