\lecture{27}{16 Mar. 10:00}{}
\begin{theorem}[Excision]\label{thm:excision}
	Suppose we have subspace $Z \subseteq A \subseteq X$ such that $\bar{Z} \subseteq \operatorname{Int} (A)$. Then the inclusion:
	\[
		(X - Z, A - Z) \hookrightarrow (X, A)
	\]
	induces isomorphisms:
	\[
		H_n(X - Z, A - Z) \xrightarrow{\cong} H_n(X, A)
	\]
\end{theorem}

\begin{exercise}
	Equivalently for subspaces $A, B \subseteq X$ whose interiors cover $X$, the inclusion:
	\[
		(B, A \cap B) \hookrightarrow (X, A)
	\]
	induces an isomorphism:
	\[
		H_n(B, A \cap B) \xrightarrow{\cong} H_n(X, A)
	\]
	Hint: $B = Z \setminus Z$, $Z = X \setminus B$.
\end{exercise}
Picture!
\begin{figure}[H]
	\centering
	\incfig{eg:excision-1}
	\label{fig:eg:excision-1}
\end{figure}

\begin{proof}[Proof Sketch]
	We sketch the proof here, which is notorious for being hairy.
	\begin{itemize}
		\item Given a relative cycle $x$ in $(X, A)$, subdivide the simplices to make $x$ a linear combination of chains on
		      ``smaller simplices,'' each contained in $\operatorname{Int}(A)$ or $X \setminus Z$.
		      \begin{figure}[H]
			      \centering
			      \incfig{pf:excision}
			      \caption{\(\delta ^n\to X\) subdivide into \hyperref[def:subsimplex]{subsimplices} with images in. }
			      \label{fig:pf:excision}
		      \end{figure}
		      This means $x$ is homologous to sum of subsimplices with images in $\operatorname{Int}(A)$ or $X \setminus Z$. One of the things we
		      use is that simplices are compact, so this process takes finite time.

		      Key: \underline{Subdivision operator} is chain homotopic to the identity.
		\item Since we are working relative to $A$, the chains with image in $A$ are zero. Thus we have a relative cycle homologous to $x$
		      with all simplices contained in $X \setminus Z$.
	\end{itemize}
\end{proof}

\begin{exercise}
	$H_\ast(Y, y_0) \cong \widetilde{H}(Y)$.
\end{exercise}

\begin{theorem}\label{thm:good-pairs-relative-homology}
	Let $(X, A)$ be a good pair. Then the quotient map $q : (X, A) \to (\quotient{X}{A} , \quotient{A}{A} )$ induces an isomorphism:
	\begin{align*}
		H_n(X, A) \xrightarrow{\cong} H_n(\quotient{X}{A}, \quotient{A}{A}) \cong \widetilde{H}_n(\quotient{X}{A})
	\end{align*}
	where the last equality is from the exercise.
\end{theorem}

\begin{proof}[Proof Outline]
	Let $A \subseteq V \subseteq X$ where $V$ is a neighborhood of $A$ that deformation retracts onto $A$. Using excision, we obtain a commutative diagram as follows.
	\par
	\adjustbox{scale=0.9,center}{%
		\begin{tikzcd}
			{H_n(X, A)} & {H_n(X, V)} & {H_n(X-A, V-A)} \\
			{H_n(\quotient{X}{A}-\quotient{A}{A})} & {H_n(\quotient{X}{A}-\quotient{V}{A})} & {H_n(\quotient{X}{A}-\quotient{A}{A}, \quotient{V}{A}-\quotient{A}{A})}
			\arrow["{q_\ast}", from=1-1, to=2-1]
			\arrow["{\color{green}{\cong}}", from=2-1, to=2-2]
			\arrow["{\color{green}{\cong}}", from=1-1, to=1-2]
			\arrow["{\color{red}{\cong}}"', from=1-3, to=1-2]
			\arrow["{\color{red}{\cong}}"', from=2-3, to=2-2]
			\arrow["{q_\ast}", from=1-3, to=2-3]
			\arrow["{\color{pink}{\cong}}"', draw=none, from=1-3, to=2-3]
		\end{tikzcd}
	}

	\par Done if we can prove all the colored isomorphisms.
	\begin{itemize}
		\item $\color{red}{\cong}$ is an isomorphism by excision
		\item $\color{pink}{\cong}$ is an isomorphism by direct calculation (since $q$ is a homeomorphism on the complement of $A$)
		\item $\color{green}{\cong}$ on Homework, since $V$ deformation retracts to $A$.
	\end{itemize}
\end{proof}