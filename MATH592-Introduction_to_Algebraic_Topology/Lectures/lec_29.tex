\lecture{29}{21 Mar. 10:00}{}
\begin{prev}
	Last time we defined \hyperref[def:degree]{degree}. Now we list some of its properties.
\end{prev}

\begin{remark}[Properties of Degree]
	\begin{enumerate}
		\item \(\deg(\identity_{S^n}) = 1\) since \((\identity_{S_n})_\ast = \identity_{\mathbb{Z}}\).
		\item If \(f \colon S^n \to S^n\) is not surjective, then \(\deg(f) = 0\). To see this, we know that \(f_\ast\) factors as
		      \[
			      \begin{tikzcd}
				      {H_n(S^n)} & {H_n(S^n-\{\ast\})=0} & {H_n(S^n)}
				      \arrow[from=1-1, to=1-2]
				      \arrow[from=1-2, to=1-3]
			      \end{tikzcd}
		      \]
		      And since the middle group is zero, $f_\ast = 0$.
		\item If \(f \simeq g\), then \(f_\ast = g_\ast\), so \(\deg(f) = \deg(g)\).
		      \begin{note}
			      The converse is true! We'll see this later.
		      \end{note}
		\item \((f \circ g)_\ast = f_\ast \circ g_\ast\), and so \(\deg(f \circ g) = \deg(f)\deg(g)\).

		      \par Consequently, if \(f\) is a \hyperref[def:homotopy-equivalence]{homotopy equivalence} then \(\deg f = \pm 1\).

		      \begin{exercise}
			      It is possible to put a \hyperref[def:delta-complex]{\(\Delta\)-complex} structure with \(2\) \(n\)-cells, \(\Delta_1\) and \(\Delta_2\) glued
			      together along their \hyperref[def:boundary]{boundary} \((\cong S^{n-1})\), and \(H_n(S^n) = \langle \Delta_1, \Delta_2 \rangle \).
		      \end{exercise}
		\item Consequences: If \(f\) is a reflection fixing the equator, and swapping the \(2\)-cells, then \(\deg f = -1\).
		      \begin{figure}[H]
			      \centering
			      \incfig{reflection-about-equator}
			      \label{fig:reflection-about-equator}
		      \end{figure}
		\item We now have the following linear algebra exercise.
		      \begin{exercise}
			      The map \(\mathbb{R}^{n + 1} \to \mathbb{R}^{n + 1}\) given by \(x \mapsto -x\) is the composite of \((n + 1)\) reflections.
		      \end{exercise}
		      So the antipodal map \(S^n \to S^n\) given by \(x \mapsto -x\) has degree which is the product of \(n + 1\) copies of \((-1)\), and so it has
		      \hyperref[def:degree]{degree} \((-1)^{n + 1}\).
		\item We again start with an exercise
		      \begin{exercise}
			      If \(f\) has no fixed points, then we can homotope \(f\) to the antipodal map via:
			      \[
				      f_t(x) = \frac{(1 - t)f(x) - tx}{\left\lVert (1 - t)f(x) - tx\right\rVert}.
			      \]
		      \end{exercise}

		      Therefore, \(\deg f = (-1)^{n + 1}\).
	\end{enumerate}
\end{remark}

\begin{theorem}[Hairy ball Theorem]\label{thm:hairy-ball-theorem}
	See the homework. This essentially says that there is no nonvanishing continuous tangent vector field on even-dimensional spheres.
\end{theorem}

\begin{theorem}[Groups acting on $S^{2n}$]\label{thm-actions-on-spheres}
	If \(G\) acts on \(S^{2n}\) \hyperref[def:free-group]{freely}, then
	\[
		G = \quotient{\mathbb{Z}}{2\mathbb{Z}}
	\]
	or \(G = 1\).
\end{theorem}

\begin{corollary}
	\(S^{2n}\) is only the trivial cover \(S^{2n} \to S^{2n}\) or \hyperref[def:degree]{degree} \(2\) cover (for example, \(S^{2n} \to \mathbb{R}P^{2n}\)).
	This follows since any covering space action acts \hyperref[def:free-group]{freely}.
\end{corollary}

\begin{proof}
	There exists a homomorphism given by:
	\[
		\begin{split}
			G & \to \{\pm 1\}        \\
			g & \mapsto \deg(\tau_g)
		\end{split}
	\]
	Where \(\tau_g\) is the action of \(g \in G\) on \(S^{2n}\) as a map \(S^{2n} \to S^{2n}\). We know this map is well-defined since \(\tau_g\) is invertible
	(simply take \(\tau_{g^{-1}}\)) for each \(g \in G\). Our note on composites shows this is a homomorphism.

	We want to show that the kernel is trivial, since then by the first isomorphism theorem \(G \cong \im\), and the image is either trivial or
	\(\quotient{\mathbb{Z}}{2\mathbb{Z}}\). Suppose that \(g\) is a nontrivial element of \(g\), then since \(G\) acts \hyperref[def:free-group]{freely} we know that
	\(\tau_g\) has no fixed points. With this in mind we have
	\[
		\deg \tau_g = (-1)^{2n + 1} = - 1.
	\]
	Thus, \(g \not\in \ker\), hence the kernel is trivial as desired.
\end{proof}

\begin{definition}[Local degree]\label{def:local-degree}
	Let \(f \colon S^n \to S^n\) (\(n > 0\)). Suppose there exists \(y \in S^n\) such that \(f^{-1}(y)\) is finite, say, \(\{x_1, \ldots, x_m\}\).
	Then let \(U_1, \ldots, U_m\) be disjoint neighborhoods of \(x_1, \ldots, x_m\) that are mapped by \(f\) to some neighborhood \(V\) of \(y\).
	\begin{figure}[H]
		\centering
		\incfig{def:local-degree}
		\label{fig:def:local-degree}
	\end{figure}

	The \emph{local degree} of \(f\) at \(x_i\), denote as \(\at{\deg f}{x_i}{}\), is the \hyperref[def:degree]{degree} of the map
	\[
		f_\ast \colon \mathbb{Z} \cong H_n(U_i, U_i - \{x_i\}) \to H_n(V, V - \{y\}) \cong \mathbb{Z}.
	\]
\end{definition}

\begin{theorem}\label{thm-calculation-with-local-degree}
	Let \(f \colon  S^n \to S^n\) with \(f^{-1}(y) = \{x_1, \ldots, x_m\}\) as above, then,
	\[
		\deg f = \sum_{i = 1}^m \at{\deg f}{x_i}{}.
	\]
\end{theorem}
\begin{remark}
	Thus, we can compute the \hyperref[def:degree]{degree} of $f$ by computing these \hyperref[def:local-degree]{local degrees}.
\end{remark}