\lecture{33}{30 Mar. 10:00}{Cellular Homology Examples}
\begin{eg}[Cellular homology group of torus]
	Calculate the \hyperref[def:cellular-homology-group]{cellular homology group} of a torus.
\end{eg}
\begin{explanation}
	Let the torus equips with the following \hyperref[def:CW-Complex]{CW complex} structure.
	\begin{figure}[H]
		\centering
		\incfig{eg:lec31:CW-complex-torus}
		\label{fig:eg:lec31:CW-complex-torus}
	\end{figure}

	The \hyperref[def:cellular-chain-complex]{cellular chain complex} looks like
	\[
		\begin{tikzcd}
			0 & {\left<D\right>} & {\left<a, b\right>} & {\left<x\right>} & 0
			\arrow[from=1-1, to=1-2]
			\arrow[from=1-2, to=1-3]
			\arrow[from=1-3, to=1-4]
			\arrow[from=1-4, to=1-5]
		\end{tikzcd}
	\]
	where we choose \(x\) as a base point (i.e. the \hyperref[def:cell]{\(0\)-cell}).

	For \(\partial _1\), since this is defined as the same as the usual \hyperref[def:boundary-homomorphism]{simplicial boundary map},
	hence by \(a \mapsto x - x = 0\) and \(b \mapsto x - x = 0\), we have \(\partial_1 = 0\).

	Now for \(\partial _2\), since \(D\) is glued along \(aba^{-1}b^{-1}\), so we look at the composed up maps
	\begin{figure}[H]
		\centering
		\incfig{eg:cellular-homology-calc-torus}
		\label{fig:eg:cellular-homology-calc-torus}
	\end{figure}
	We wind forwards then backwards around \(a\),\footnote{Intuitively, since we \hyperref[CW-complex-quotient]{quotient} out \(b\),
		hence the gluing map is \hyperref[def:homotopy]{homotopy} to constant maps.} so the \hyperref[def:degree]{degree} is zero.
	The same thing happens for \(b\), so
	\[
		\partial_2 D = \underbrace{0 \cdot a}_{\partial_{\alpha \beta_a }a} + \underbrace{0 \cdot b}_{\partial _{\alpha \beta_b } b} = 0,
	\]
	where we assume that \(\alpha\) is the index of \(D\), and \(\beta _a\) is the index of \(a\) and same for \(b\).

	This gives a nice \textbf{principle}, namely if a \hyperref[def:cell]{\(2\)-cells}l \(D\) is glued down via some \hyperref[def:word]{words}
	\(w\) (this only makes sense for \hyperref[def:cell]{\(2\)-cells}), then the
	coefficient\footnote{i.e. \(\partial_{\alpha \beta } (a)\) where \(\alpha\) is the index of \(a\).} to a letter \(a\) in \(\partial_2 D\)
	is the sum of the exponents of \(a\) in \(w\).
	In this case, for both \(a\) and \(b\), the coefficients for are both \(1 + (-1) = 0\).

	Now we just have that the \hyperref[def:cellular-homology-group]{homology groups} are equal to the \hyperref[def:cellular-chain-complex]{chain groups}
	because the boundary maps are all zero. Hence, we have
	\[
		H_{k}(T) = \begin{dcases}
			\mathbb{\MakeUppercase{z}} ,     & \text{ if } k = 0, 2 ; \\
			\mathbb{\MakeUppercase{z}} ^{2}, & \text{ if } k = 1;     \\
			0,                               & \text{ otherwise}.
		\end{dcases}
	\]
\end{explanation}

\begin{eg}[Cellular homology group of \(\Sigma _g\)]
	Calculate the \hyperref[def:cellular-homology-group]{cellular homology group} of a genus \(g\) surface \(\Sigma _g\).
\end{eg}
\begin{explanation}
	A genus \(g\) surface \(\Sigma_g\) has the \hyperref[def:CW-Complex]{CW complex} structure as
	\begin{itemize}
		\item \(1\) \hyperref[def:cell]{\(0\)-cell} \(x\).
		\item \(2g\) \hyperref[def:cell]{\(1\)-cells} \(a_1, b_1, a_2, b_2, \ldots\).
		\item \(1\) \hyperref[def:cell]{\(2\)-cell} \(D\) glued along \([a_1, b_2][a_2, b_2]\cdots[a_g, b_g]\) (a product of commutators)
	\end{itemize}

	For \(\partial _1\), we have
	\[
		\partial_1(a_i) = \partial_1(b_i) = x - x = 0.
	\]

	Furthermore, by the principle discussed above, we know that every \hyperref[def:cell]{\(1\)-cell} appears once in the \hyperref[def:word]{word}, and its inverse appears once,
	so all the coefficients of \hyperref[def:cell]{\(1\)-cells} in \(\partial_2(D)\) are zero, so \(\partial_2(D) = 0\). This means we have a \hyperref[def:cellular-chain-complex]{chain complex}
	\[
		\begin{tikzcd}
			0 & {\mathbb{Z}} & {\mathbb{Z}^{2g}} & {\mathbb{Z}} & 0
			\arrow[from=1-1, to=1-2]
			\arrow["0", from=1-2, to=1-3]
			\arrow["0", from=1-3, to=1-4]
			\arrow[from=1-4, to=1-5]
		\end{tikzcd}
	\]
	And so then we have that
	\[
		H_k(\Sigma _{g} ) = \begin{dcases}
			\mathbb{\MakeUppercase{z}} ,      & \text{ if } k = 0, 2 ; \\
			\mathbb{\MakeUppercase{z}} ^{2g}, & \text{ if } k = 1;     \\
			0,                                & \text{ otherwise}.
		\end{dcases}
	\]
\end{explanation}

\begin{exercise}
	Calculate the \hyperref[def:cellular-homology-group]{cellular homology group} of \(\mathbb{\MakeUppercase{r}} P^n\).
\end{exercise}

\begin{eg}[Torus example: \(\partial_2\) in more detail]
	We're going to work through this example a bit more carefully.
	\begin{figure}[H]
		\centering
		\incfig{eg:more-careful-torus-cellular}
		\label{fig:eg:more-careful-torus-cellular}
	\end{figure}
	Let's zoom in on these two preimage points and use \emph{local homology} to compute this:
	\todo{Fill this up!}
\end{eg}