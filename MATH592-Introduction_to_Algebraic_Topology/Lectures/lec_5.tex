\lecture{5}{14 Jan. 10:00}{Operation on Spaces}
\subsection{Operations on CW Complexes}
\subsubsection{Products}
We can consider the product of two CW complexes given by a CW complex structure. Namely, given \(X\) and \(Y\)
two CW complexes, we can take two cells \(e^n_{\alpha }\) from \(X\) and \(e^m_{\beta }\) from \(Y\) and
form the product space \(e^n_{\alpha }\times e^m_{\beta }\), which is homeomorphic to an \(n+m\)-cell. We then
take these products as the cells for \(X\times Y\).

Specifically, given \(X\), \(Y\) are CW complexes, then \(X\times Y\) has a cell structure
\[
	\left\{e_{\alpha}^m \times e_{\alpha}^n\colon e^m_{\alpha}\text{ is a \(m\)-cell on \(X\)}, e^n_{\alpha}\text{ is a \(n\)-cell on \(Y\)}\right\}.
\]
\begin{remark}
	The product topology may not agree with the weak topology on the \(X\times Y\). However, they do agree if
	\(X\) or \(Y\) is locally compact \underline{or} if \(X\) and \(Y\) both have at most countably many cells.
\end{remark}

\begin{note}
	Notice that if the product is wild enough, then the product topology may not agree with the weak topology.
\end{note}
\subsubsection{Wedge Sum}
Given \(X\), \(Y\) are CW complexes, and \(x_0\in X^0\), \(y_0\in Y^0\) (only points). Then we define
\[
	X\vee Y = X\coprod Y
\]
with quotient topology.
\begin{remark}
	\(X\lor Y\) is a CW complex.
\end{remark}

\subsubsection{Quotients}
Let \(X\) be a CW complex, and \(A\subseteq X\) subcomplex (closed union of cells), then
\[
	\quotient{X}{A}
\]
is a quotient space collapse \(A\) to one point and inherits a CW complex structure.

\begin{remark}
	\(\quotient{X}{A} \) is a CW complex.
\end{remark}
\(0\)-skeleton
\[
	(X^0 - A^0)\coprod *
\]
where \(*\) is a point for \(A\). Each cell of \(X-A\) is attached to \(\left(\quotient{X}{A} \right)^n\)
by attaching map
\[
	\begin{tikzcd}
		S^n \ar[r,"\phi_{\alpha}"] & X^n \ar[r, "\text{quotient}"] & \quotient{X^n}{A^n}
	\end{tikzcd}
\]

\begin{eg}
	Here is some interesting examples.
	\begin{enumerate}
		\item We can take the sphere and squish the equator down to form a wedge of two spheres.
		      \begin{figure}[H]
			      \centering
			      \incfig{eg:quotient-cw-complex-sphere}
			      \label{fig:eg:quotient-cw-complex-sphere}
		      \end{figure}
		\item We can take the torus and squish down a ring around the hole.
		      \begin{figure}[H]
			      \centering
			      \incfig{eg:quotient-cw-complex-torus}
			      \caption{We see that \(\quotient{X}{A}\) is homotopy equivalent to a \(2\)-sphere
				      wedged with a \(1\)-sphere via extending the red point into a line, and then
				      sliding the left point to the line along the \(2\)-sphere towards the other
				      point, forming a circle.}
			      \label{fig:eg:quotient-cw-complex-torus}
		      \end{figure}
	\end{enumerate}
\end{eg}
