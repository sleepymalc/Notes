\lecture{8}{24 Jan. 10:00}{}
\begin{eg}
	In category \(\underline{\mathrm{Ab}}\) free Abelian group on a set \(S\) is
	\[
		\bigoplus_S \mathbb{\MakeUppercase{z}}.
	\]
	In category of fields, no such thing as \textbf{free field on \(\bm{S} \) }.
\end{eg}

\subsection{Constructing the free group \(F_S\)}
Fix a set \(S\), and we define a \underline{word} as a finite sequence (possibly \(\varnothing \))
in the formal symbols
\[
	\left\{s, s ^{-1} \mid s\in S\right\}.
\]
Then we see that elements in \(F_S\) are equivalence classes of words with the equivalence relation being
\begin{itemize}
	\item delete \(s s ^{-1} \) or \(s ^{-1} s\). i.e.,
	      \[
		      \begin{split}
			      vs ^{-1} s w&\sim vw\\
			      v s s ^{-1}  w &\sim vw
		      \end{split}
	      \]
	      for every word \(v, w, s\in S\),
\end{itemize}
with the group operation being concatenation.

\begin{exercise}
	Check that \(F_S\) satisfies the universal property.
\end{exercise}

\section{Fundamental Group}
We start with the definition.
\begin{definition}[Path]
	A \emph{path} in a space \(X\) is a continuous map \(\gamma\colon I\to X\), \(I = [0, 1]\).
	A \emph{homotopy of paths} \(\gamma_0\), \(\gamma_1\) is a homotopy \(\mathrm{rel} \{0, 1\}\).
	\begin{figure}[H]
		\centering
		\incfig{def:homotopy-of-paths}
		\label{fig:def:homotopy-of-paths}
	\end{figure}
\end{definition}
\begin{eg}
	Fix \(x_1, x_0\in X\), then \underline{\(\exists\) homotopy of paths} is an equivalence relation on paths
	from \(x_0\) to \(x_1\) (i.e., \(\gamma\) with \(\gamma(0)=x_0, \gamma(1)=x_1\)).
\end{eg}

\begin{definition}
	For paths \(\alpha , \beta \) in \(X\) with \(\alpha (1) = \beta (0)\), the composition \(\alpha \cdot \beta \) is
	\[
		(\alpha \cdot \beta )(t) \coloneqq \begin{dcases}
			\alpha (2t),  & \text{ if } t\in \left[0, \frac{1}{2}\right]  \\
			\beta (2t-1), & \text{ if } t\in \left[\frac{1}{2}, 1\right].
		\end{dcases}
	\]
	\begin{figure}[H]
		\centering
		\incfig{def:path-composition}
		\label{fig:def:path-composition}
	\end{figure}
\end{definition}

\begin{definition}[reparameterization]
	Let \(\gamma\colon I\to X\) be a path, then a \emph{reparameterization} of \(\gamma\) is a path
	\[
		\gamma ^\prime \colon I\overset{\phi }{\longrightarrow} I\overset{\gamma}{\longrightarrow} X
	\]
	where \(\phi \) is continuous and
	\[
		\phi (0) = 0,\quad \phi (1) = 1.
	\]
\end{definition}

\begin{definition}[Fundamental Group]
	Let \(X\) denotes the space and let \(x_0\in X\) be the base point. The fundamental group \(\pi_1(X, x_0)\) of \(X\)
	based at \(x_0\) is a group with
	\begin{itemize}
		\item elements: homotopy classes of paths \([\gamma]\) where \(\gamma\) is a loop with \(\gamma(0) = \gamma(1) = x_0\)
		      \begin{figure}[H]
			      \centering
			      \incfig{def:fundamental-group-elements}
			      \label{fig:def:fundamental-group-elements}
		      \end{figure}
		\item operation: composition of paths
		\item constant loop:
		      \[
			      \gamma\colon I\to X,\quad t\mapsto x_0
		      \]
		\item inverses.
		      \[
			      \overline{\gamma} (t) = \gamma(1-t).
		      \]
	\end{itemize}
\end{definition}
\begin{proof}
	We need to prove that the above define a group. \todo{HW.}
\end{proof}

\begin{theorem}
	If \(X\) is path-connected, then
	\[
		\forall x_0, x_1\in X\quad \pi_1(X, x_0)\cong \pi _1(X, x_1).
	\]
\end{theorem}
\begin{remark}
	We often write \(\pi _1(X)\).
\end{remark}
\begin{proof}
	\todo{HW.}
\end{proof}

