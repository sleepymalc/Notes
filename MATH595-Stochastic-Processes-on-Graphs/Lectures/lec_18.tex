\lecture{18}{27 Mar.\ 9:30}{}
\begin{prev}
	For random graphs \((G_n)_{n \geq 1}\), we have the following notion of local convergence so far as \(n \to \infty \).
	\begin{itemize}
		\item Weakly: \((G_n, o_n) \overset{D}{\to} (G, o) \sim \overline{\mu} \) if \(\mathbb{E}_{G_n}[\mathbb{E}_{o_n}[h(G_n, o_n)] ] \to \mathbb{E}_{\overline{\mu} }[h(G, o)] \eqqcolon \overline{\mu} (h)\) where \(\overline{\mu} \) is a probability on \((\mathcal{F} _{\ast} , d_{\ast} )\).
		\item In probability: \((G_n, o_n) \overset{p}{\to} (G, o) \sim \overline{\mu} \) if \(\mathbb{E}_{o_n}[h(G_n, o_n)] \overset{p}{\to} \mathbb{E}_{\overline{\mu} }[h(H, o)] \) for all \(h\) bounded continuous on \((\mathcal{F} _{\ast} , d_{\ast} )\).
		\item Almost surely: \((G_n, o_n) \overset{\text{a.s.} }{\to} (G, o) \sim \overline{\mu} \) if \(\mathbb{E}_{o_n}[h(G_n, o_n)] \overset{\text{a.s.} }{\to} \mathbb{E}_{\overline{\mu} }[h(H, o)] \) for all \(h\) bounded continuous on \((\mathcal{F} _{\ast} , d_{\ast} )\).
	\end{itemize}
\end{prev}

\begin{remark}
	Typically, to show local convergence in probability, we first show that it converges weakly (i.e., mean convergence), and then show that the variance goes to \(0\) (i.e., \(L^2\)-convergence).
\end{remark}

\begin{remark}
	Instead of arbitrary \(h\) bounded and continuous, it's enough to check for \(h = \mathbbm{1}_{H_{\ast} } \) where \(H_{\ast} \) is a finite and connected rooted graph with radius \(r\). In this case,
	\[
		\mathbb{E}_{o_n}[h(G_n, o_n)]
		= \frac{1}{\lvert V_n \rvert } \sum_{v \in V_n} \mathbbm{1}_{B_r^{(G_n)}(o_n) \cong H_{\ast} } .
	\]
	If we further know that \(\overline{\mu} \) is supported on a smaller subset \(\mathcal{S} _{\ast} \subseteq \mathcal{F} _{\ast} \) (i.e., \(\overline{\mu} (s_{\ast} ) = 1\)), then we only need to check for \(H_{\ast} \in \bigcup_{r \geq 0} \mathcal{S} _{\ast} (r)\).
\end{remark}

\begin{eg}
	For all \(\lambda > 0\), \(\operatorname{ER}(n, \lambda / n) \overset{p}{\to} \operatorname{GWBP}(\operatorname{Pois}(\lambda ) ) \).
\end{eg}

\begin{eg}
	\(\operatorname{CM}(\undertilde{d} \sim D) \) where with \(\nu = \mathbb{E}_{}[D(D-1)] / \mathbb{E}_{}[D] \), \(\Pr_{}\left(\text{simple} \right) \approx e^{-\nu / 2 - \nu ^2 / 4}\). Recall that this is a branching process with progeny distribution \(\hat{D} \) where \(\Pr_{}\left(\hat{D} = k - 1 \right) = k \cdot \Pr_{}\left(D = k\right) / \mathbb{E}_{}[D] \) with mean \(\nu \).
\end{eg}

\begin{eg}
	For a random \(d\)-regular graph \(\operatorname{RRG}(n, d) \) (\(\operatorname{CM}(d \cdot \undertilde{1}) \) condition on being simple), \(D \equiv d\)  and \(\hat{D} \equiv d - 1\). Then, \(\Pr_{}\left(\text{simple} \right) \to e^{- \frac{d-1}{2} - \left( \frac{d-1}{2} \right)^2 } > 0\). Thus, any even \(A\) satisfying \(\Pr_{}\left(A\text{ in } \operatorname{CM}(d \cdot \undertilde{1}) \right) \to 0\) satisfies \(\Pr_{}\left(A \text{ in } \operatorname{RRG}(n, d) \right) \to 0\).

	\(\operatorname{CM}(d \cdot \undertilde{1}) \overset{p}{\to} T_\infty ^d\), i.e., \(d\)-regular infinite tree. Hence, \(\operatorname{RRG}(n, d) \overset{p}{\to} T_\infty ^d\).\footnote{Another way to show this is to show that there are no cycles up to distance \(r\) from a typical vertex.}
\end{eg}

\begin{eg}
	\(\operatorname{IRG}(\mu , \kappa ) \overset{p}{\to} \operatorname{MTBP} \) where \(\mu \) is the vertex weight distribution and \(\kappa \) is the kernel, and \(\operatorname{MTBP} \) is a multi-type branching process, with the root having weight \(u_{\phi } \sim \mu \), and the number of children being \(\operatorname{Pois}(\kappa (x, \cdot)) \) where \(\kappa (x, \cdot) = \int \kappa (x, y) \mu (\mathrm{d} y)\) and their weights are independent according to \(\kappa (x, y) \mu (\mathrm{d} y) / \kappa (x, \cdot)\).
\end{eg}

\begin{eg}
	\(\operatorname{RRT}_n = \operatorname{UA}_n \overset{p}{\to} \) the infinite spine (in \(\mathcal{T} _{\ast} \)) with yule trees \(\tau \) hanging, as we have discussed before. We can further break the spine and inject the infinite spine into \(\mathcal{T} _{\ast} ^{\mathbb{N} }\).
\end{eg}

\begin{note}[Consequence]
	Local convergence in probability gives the in \(\Pr\)-limit of \(\lvert \{ v \mid B_r^{(G_n)}(v) \approx H_{\ast} \} \rvert / n\).
\end{note}

\begin{note}
	\(\#\text{clusters} / n \to \mathbb{E}_{\overline{\mu} }[\mathbbm{1}_{\lvert G \rvert < \infty } / \lvert V(G) \rvert ] \) since
	\[
		\frac{\#\text{clusters}}{n}
		= \frac{1}{n} \sum_{\mathcal{C} \colon \text{cluster} } 1
		= \frac{1}{n} \sum_{v=1}^{n} \frac{1}{\lvert \mathcal{C} (v) \rvert }
		= \mathbb{E}_{o_n}\left[\frac{1}{\lvert \mathcal{C} (o_n) \rvert }\right]
		\to \mathbb{E}_{\overline{\mu} }\left[\frac{\mathbbm{1}_{\lvert \mathcal{C} (o) \rvert < \infty } }{\lvert \mathcal{C} (o) \rvert }\right]
	\]
	where \((G, o) \sim \overline{\mu} \). Also,
	\[
		\frac{1}{n} \sum_{\mathcal{C} \colon \text{cluster} } f(\mathcal{C} )
		\to \mathbb{E}_{\overline{\mu} }\left[\frac{f(G)}{\lvert G \rvert }\right]
	\]
	where \(f(G) / \lvert G \rvert \leq K\) for some \(K < \infty \).\footnote{We need some regularity condition such as the existence of \(\lim_{r \to \infty} f(B_r^{(G)}(o)) / \lvert B_r^{(G)}(o) \rvert \).}
\end{note}

\subsection{Spin System}
Given a finite graph \(G = (V, E)\), the spin value at vertex \(v\), denoted as \(\sigma _v \in \{ \pm 1 \} \) (or in general, \(\sigma _v \in [q]\) or \(\sigma _v \in S^1\), etc.), the Gibbs measure (inverse temperature \(\beta \))
\[
	\Pr_{\beta , G}(\sigma )
	= \frac{\exp (\beta \cdot H(\sigma ))}{\sum_{\tau } \exp (\beta \cdot H(\tau ))},
\]
where \(H(\sigma ) \) is the Hamiltonian, depending on \(G\).

\begin{eg}[Ising model]
	Given \(G = (V, E)\) with \(V = [n]\), consider \(H(\sigma ) \coloneqq \sum_{(i, j) \in E} \sigma _i \sigma _j\), where \((\sigma _1, \dots , \sigma _n) \in \{ \pm 1 \} ^n\). \(H(\sigma )\) is simply the number of edges of equal spin at endpoints minus the number of edges with unequal spins.

	For \(\beta = 0\), \(\Pr _{0, G}\) is the uniform spin; for \(\beta = \infty \), \(\Pr _{\infty , G} = \bigotimes_{\mathcal{C} \colon \text{cluster} } (\delta _{\text{all spins equal \(1\) on } \mathcal{C} } / 2 + \delta _{\text{all spins equal \(-1\) on } \mathcal{C} } / 2)\). There is usually a phase transition between \(\beta = 0\) to \(\beta = \infty \).
\end{eg}

\begin{eg}[Ferromagnetic Ising model]
	We can also consider \(H(\sigma ) = \sum_{(i, j) \in E} J_{ij} \sigma _i \sigma _j\) for some \(J_{ij} > 0\).
\end{eg}

\begin{eg}[Anti-ferromagnetic Ising model]
	We can also consider \(H(\sigma ) = - \sum_{(i, j) \in E} \sigma _i \sigma _j\), which is the number of edges of unequal spin at endpoints minus the number of edges with equal spins. There are nontrivial ground states if there is some odd cycle.
\end{eg}

\begin{eg}[Hardcore model]
	\(H(\sigma ) = -\infty \cdot \#\text{edges with equal spin} \). In this case, \(\beta \) doesn't matter and \(\Pr (\sigma ) \propto \mathbbm{1}_{\text{all edges are unequal spin} } \).
\end{eg}

\begin{eg}[Spin glass model]
	\(H(\sigma ) = \sum_{(i, j) \in E} J_{aij} \sigma _i \sigma _j \), where \(J_{ij} \overset{\text{i.i.d.} }{\sim } \mathcal{N} (0, 1)\) (or \(\mathcal{U} (\{ \pm 1 \} )\)).

	To actually analyze this, we first fix the graph sequence, and assign \(J_{ij} \overset{\text{i.i.d.} }{\sim } \mathcal{N} (0, 1)\), then look at the Ising model \(\Pr _{\beta , G_n, J}(\sigma )\).
\end{eg}

The partition function \(Z_n(\beta ) = \sum_{\sigma } \exp (\beta H(\sigma ))\). Typically, we will focus on the free energy \(\frac{1}{n}\log Z_n(\beta )\), and look at what's its limiting behavior as \(n \to \infty \).