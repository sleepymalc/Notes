\lecture{2}{23 Jan.\ 9:30}{Erdős-Rényi Random Graph Model}
\begin{prev}
	We mainly focus on the following three types of questions for both the degree and components:
	\begin{enumerate}
		\item Typical (local) behavior: single/multiple points view.
		\item Global behavior: empirical behavior.
		\item Extremal behavior: maxima or minima of various objects.
	\end{enumerate}
\end{prev}

Toward proving \autoref{thm:Erdős-Rényi-phase-transition} \autoref{thm:Erdős-Rényi-phase-transition-a}, we will need the following idea:

\begin{definition}[Stochastic domination]\label{def:stochastic-domination}
	Let \(X\) and \(Y\) be two real-valued random variables. We say that \(X\) is \emph{stochastically dominated} by \(Y\), denoted as \(X \preceq Y\), if there exists a coupling of \(X, Y\) such that \(X \leq Y\).
\end{definition}

The reason why \hyperref[def:stochastic-domination]{stochastic domination} is useful is because of the following:

\begin{exercise}\label{ex:stochastic-domination}
	\(X \preceq Y\) if and only if \(\Pr_{}(X > t) \leq \Pr_{}(Y > t) \) for all \(t \in \mathbb{R} \).
\end{exercise}

Here we give some elementary examples of \hyperref[def:stochastic-domination]{stochastic domination}.

\begin{eg}
	\(\operatorname{Bin}(n, p) \preceq \operatorname{Bin}(m, p) \) for \(m \geq n\).
\end{eg}
\begin{explanation}
	Since we have \(\operatorname{Bin}(m, p) \overset{D}{=} \operatorname{Bin}(n, p) + \operatorname{Bin}(m - n, p)\).
\end{explanation}

\begin{eg}
	\(\operatorname{Ber}(p) \preceq \operatorname{Ber}(r) \) if \(p \leq r\).
\end{eg}

\begin{eg}
	\(\operatorname{Ber}(p) \preceq \operatorname{Pois}(\theta ) \) by letting \(\theta e^{-\theta } = p\). More generally, we just need \(1 - p \geq e^{-\theta }\).
\end{eg}
\begin{explanation}
	This follows from an application of the above \hyperref[ex:stochastic-domination]{exercise}.
\end{explanation}

As we will soon see, by using \hyperref[def:stochastic-domination]{stochastic domination}, one can provide a nice bound for proving \autoref{thm:Erdős-Rényi-phase-transition} easily.

\begin{intuition}
	We will often construct objects that are \hyperref[def:stochastic-domination]{stochastically dominating} the object of interest, such that bounds on the dominating quantity imply bounds on the desired quantities in the graph.
\end{intuition}

\section{Degree in Sparse Regime}
As a warm-up toward proving \autoref{thm:Erdős-Rényi-phase-transition} \autoref{thm:Erdős-Rényi-phase-transition-a}, let's first look at an easier problem: the degree.

\subsection{Single Point Viewpoint}
When \(p = \lambda / n\), recall what we have proven.

\begin{prev}
	The expected degree of any vertex \(v\) is approximately \(\lambda \in (0, \infty )\). We also have \(\deg_{G_n}(v) \overset{D}{\to} \operatorname{Pois}(\lambda ) \) as \(n \to \infty \) where \(G_n \sim \operatorname{ER}(n, \lambda / n) \).
\end{prev}

\subsection{Joint Distribution}
This is for a single point, what about their joint behaviors?

\begin{claim}
	For any finite \(k\), \((\deg(1), \deg(2), \dots , \deg(k)) \overset{D}{\to} (\operatorname{Pois}(\lambda ) , \operatorname{Pois}(\lambda ), \dots , \operatorname{Pois}(\lambda ) )\).
\end{claim}
\begin{explanation}
	Consider any two vertices \(i, j\), we see that
	\[
		\deg(i) = \mathbbm{1}_{(i, j) \in E} + \sum_{v \neq j} \mathbbm{1}_{(i, v) \in E} \text{ and }
		\deg(j) = \mathbbm{1}_{(i, j) \in E} + \sum_{v \neq i} \mathbbm{1}_{(j, v) \in E}.
	\]
	Note that the remaining parts, \(\sum_{v \neq j} \mathbbm{1}_{(i, v) \in E} \) and \(\sum_{v \neq i} \mathbbm{1}_{(j, v) \in E} \), are independent. The same argument generalizes to any fixed \(k\) vertices.

	Moreover, for any fixed \(k\), the number of edges among these \(k\) vertices follows \(\operatorname{Bin}(\binom{k}{2}, \lambda / n) \), which goes to \(0\) as \(n \to \infty \). Hence, only the remaining parts in the above degree expression survive, which are independent. As \(k\) is finite, the remaining parts again follow \(\operatorname{Pois}(\lambda ) \).
\end{explanation}

\begin{intuition}
	Since the graph is sparse, for any fixed, finite \(k\),  \(n \to \infty \), independence emerges.
\end{intuition}

The above is for finite \(k\), serving as the multiple points view.

\subsection{Extremal Viewpoint}
For a global view of the degree distribution, recall the following:

\begin{prev}
	Consider the empirical distribution of degree, defined as \(\frac{1}{n} \sum_{v=1}^{n} \delta _{\deg(v)}\), converges to \(\operatorname{Pois}(\lambda ) \) in the total variation distance.
\end{prev}

The last question is the extremal behavior, where we are interested in either bounding or approximating the maximum degree \(\deg _{\max , n} \coloneqq \max _{v \in V} \deg (v)\) for \(G \sim \operatorname{ER}(n, p)\).

\begin{proposition}\label{prop:Erdős-Rényi-max-degree}
	Consider the \hyperref[def:Erdős-Rényi-random-graph]{Erdős-Rényi random graph} model \(\operatorname{ER}(n, \lambda / n)\) for \(\lambda \in (0, \infty )\). Then for all \(\epsilon > 0\), as \(n \to \infty \), we have
	\[
		\Pr_{}\left( \deg _{\max , n} \geq (1 + \epsilon ) \frac{\log n}{\log \log n} \right)
		\to 0.
	\]
\end{proposition}
\begin{proof}
	By a simple union bound, for any \(x \in \mathbb{R} \), we have
	\[
		\Pr_{}\left( \max _{v \in [n]} \deg (v) \geq x\right)
		= \Pr_{}\left( \bigcup_{v=1}^{n} \{ \deg (v) \geq x \} \right)
		\leq n \Pr_{}(\deg (1) \geq x) .
	\]
	Now we focus on \(\Pr_{}(\deg (1) \geq x) \). With the Chernoff-Cramér method, for any \(\theta > 0\), we have
	\begin{align*}
		\Pr_{}(\deg (1) \geq x)
		 & \leq e^{-\theta x} \mathbb{E}_{}[e^{\theta \deg (1)}]                                             \\
		 & = e^{-\theta x} \cdot \left( 1 - \frac{\lambda}{n} + \frac{\lambda}{n} e^{\theta } \right) ^{n-1} \\
		 & \leq \exp (-\theta x + (n-1) \frac{\lambda}{n} (e^\theta - 1))
		\leq \exp (-\theta x + \lambda (e^\theta - 1)). \tag*{(\(1 + t \leq e^t\))}
	\end{align*}
	Optimizing \(\theta \), we see that \(\theta _0 = \ln (x / \lambda )\) minimizes the above, and it's positive if \(x > \lambda \). In the end, we have an upper bound \(\exp (-x \ln (x / \lambda ) + x - \lambda )\). Plugging it back, we have
	\[
		\Pr_{}\left( \max _{v \in [n]} \deg (v) \geq x\right)
		\leq n \exp (- x \ln \frac{x}{\lambda } + x - \lambda ).
	\]
	By choosing \(x = (1 + \epsilon ) \log n / \log \log n\), the upper bound goes to \(0\) as \(n \to \infty \).
\end{proof}

\begin{remark}
	The proof technique of \autoref{prop:Erdős-Rényi-max-degree} will be used extensively in this course.
\end{remark}

Let's summarize all results we have for degree so far in the following:

\begin{theorem}[Degree of sparse Erdős-Rényi graph]\label{thm:degree-of-sparse-Erdős-Rényi-graph}
	Let \(G \sim \operatorname{ER}(n, \lambda / n) \) for some \(\lambda \in (0, \infty )\).
	\begin{enumerate}[(a)]
		\item\label{thm:degree-of-sparse-Erdős-Rényi-graph-a} \(\deg (1) \overset{D}{\to} \operatorname{Pois}(\lambda ) \) as \(n\to \infty \).
		\item\label{thm:degree-of-sparse-Erdős-Rényi-graph-b} For any finite \(k\), \((\deg (1), \dots , \deg (k)) \overset{D}{\to} \operatorname{Pois}(\lambda ) \otimes \dots \otimes \operatorname{Pois}(\lambda ) \) as \(n\to \infty \).\footnote{I.e., the joint distribution of \(k\) many i.i.d.\ \(\operatorname{Pois}(\lambda ) \).}
		\item\label{thm:degree-of-sparse-Erdős-Rényi-graph-c} The empirical degree distribution \(\frac{1}{n} \sum_{v=1}^{n} \delta _{\deg (v)} \overset{D}{\to} \operatorname{Pois}(\lambda ) \) as \(n\to \infty \).
		\item\label{thm:degree-of-sparse-Erdős-Rényi-graph-d} For any \(\epsilon > 0\), as \(n \to \infty \), we have
		      \[
			      \Pr_{}\left( \deg _{\max , n} \geq (1 + \epsilon ) \frac{\log n}{\log \log n} \right)
			      \to 0.
		      \]
	\end{enumerate}
\end{theorem}

\section{Size of Connected Component in Sparse Regime}
Getting back to \autoref{thm:Erdős-Rényi-phase-transition}, we start by consider the \emph{subcritical regime}, i.e., when \(\lambda < 1\).

\subsection{Subcritical Regime \(\lambda < 1\)}
We start from a similar technique and argument from \autoref{thm:degree-of-sparse-Erdős-Rényi-graph}, without loss of generality we consider \(\lvert \mathcal{C} (1) \rvert \). To see how to compute the size of a connected component, consider the breadth-first search algorithm starting from vertex \(1\).

\begin{intuition}
	We see that the induced distance tree \(\mathcal{T} \) is in some sense \emph{dominated} by the tree \(\mathcal{T} \) where we do not mark the already explored vertices.
\end{intuition}

The latter is considered as a \href{https://en.wikipedia.org/wiki/Galton%E2%80%93Watson_process}{Galton-Watson branching process} with progeny \(\operatorname{Bin}(n-1, p) \), denoted as \(\operatorname{GWBP}(\operatorname{Bin}(n-1, p) ) \). The crucial observation is that, the size of this branching process \hyperref[def:stochastic-domination]{stochastically dominates} the size of \(\mathcal{C} (1)\). We can now see some intuition of how to prove \autoref{thm:Erdős-Rényi-phase-transition} \autoref{thm:Erdős-Rényi-phase-transition-a}, where we aim to show that \(\Pr_{}(\lvert \mathcal{C} _{\max _1} \rvert \geq a \log n) \to 0\) as \(n \to \infty \) if \(a(\lambda - 1 - \log \lambda ) > 1\).

\begin{intuition}[Proof intuition of \autoref{thm:Erdős-Rényi-phase-transition} \autoref{thm:Erdős-Rényi-phase-transition-a}]
	For any \(t\), as we discussed above, we will have \(\Pr_{}(\lvert \mathcal{C} (1) \rvert \geq t) \leq \Pr_{}(\lvert \operatorname{GWBP}(\operatorname{Bin}(n-1, \lambda / n) )  \rvert \geq t) \). Next, we observe that we can maintain the number of vertices in the queue when we do the breadth-first search, we see that the tree \(\mathcal{T} \) can be (uniquely) embedded in a sequence.

	Formally, let \((u_i)\) be the sequence of vertices ordered in terms of the order of exploration. Then, consider the size of the tree \(\mathcal{T} \), which is the length of the sequence \((s_n)\) that records the number of vertices in the queue,\footnote{Note that the sequence stops whenever the exploration stops, i.e., an entire connected component is explored.} where \(s_0 = 1\), \(s_k = s_{k-1} + (x_k - 1)\) such that \(x_k\) is the number of children of \(u_k\) in \(\mathcal{T} \). It is easy to verify that this embedding is indeed a bijection. Finally, we see that the size of the tree is the hitting time to \(0\), i.e., \(\lvert \mathcal{T} \rvert = \inf \{ n \geq 1 \mid s_n = 0 \} \).

	With the above two ingredients, consider the branching process. In this case, the embedded sequence has i.i.d.\ increments, and is therefore a random walk given by \(s_0 = 1\), \(s_k = s_{k-1} + X_k - 1\) with \(X_k \sim \operatorname{Bin}(n-1, p) \) for all \(k\). The final observation is that when \(\lambda < 1\), the above process has a negative drift, hence the hitting time is almost surely finite and can be bounded.
\end{intuition}