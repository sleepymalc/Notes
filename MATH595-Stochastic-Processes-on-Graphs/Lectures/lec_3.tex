\lecture{3}{28 Jan.\ 9:30}{Component Size in Subcritical Regime}
\begin{prev}
	Consider \(\operatorname{ER}(n, \lambda / n) \) for some \(\lambda > 0\). As in \autoref{thm:degree-of-sparse-Erdős-Rényi-graph}, we have proved:
	\begin{enumerate*}
		\item[\autoref{thm:degree-of-sparse-Erdős-Rényi-graph-a}] \(\deg (1) \overset{D}{\to} \operatorname{Pois}(\lambda ) \) and
		\item[\autoref{thm:degree-of-sparse-Erdős-Rényi-graph-b}] \((\deg (1), \dots , \deg (k)) \overset{D}{\to} \operatorname{Pois}(\lambda ) \otimes \dots \otimes \operatorname{Pois}(\lambda ) \) as \(n \to \infty \).
		\item[\autoref{thm:degree-of-sparse-Erdős-Rényi-graph-c}] Also, the empirical distribution \(\frac{1}{n}\sum_{v=1}^{n} \delta _{\deg (v)} \overset{D}{\to} \operatorname{Pois}(\lambda ) \).
		\item[\autoref{thm:degree-of-sparse-Erdős-Rényi-graph-d}] Finally, we have a maximum degree bound such that for any \(\epsilon > 0\), \(\Pr_{}(\deg _{\max , n}) \geq (1 + \epsilon ) \log n / \log \log n \).
	\end{enumerate*}
\end{prev}

With the build-up from the previous lecture, we are almost ready to prove \autoref{thm:Erdős-Rényi-phase-transition} \autoref{thm:Erdős-Rényi-phase-transition-a}. However, as noted above, it's expected that the result will depend on \(\operatorname{Pois}(\lambda ) \) in various ways. Hence, we note the following results from standard probability analysis.

\begin{exercise}
	Let \(X_1, \dots , X_r \overset{\text{i.i.d.} }{\sim } \operatorname{Pois}(\lambda ) \). Prove the following.
	\begin{enumerate}[(a)]
		\item As \(n \to \infty \), we have
		      \[
			      \max _{i = 1, \dots , n} X_i \cdot \frac{\log \log n}{\log n}
			      \overset{p}{\to} 1.
		      \]
		\item Moreover, we can show that as \(n \to \infty \),
		      \[
			      \log \log n \left( \max _{i = 1, \dots , n} X_i \cdot \frac{\log \log n}{\log n} - 1 \right)
		      \]
		      converges in distribution to some non-trivial limit.
		\item Similarly, we can prove that as \(n \to \infty \),
		      \[
			      \deg _{\max , n} \cdot \frac{\log \log n}{\log n}
			      \overset{p}{\to} 1,
		      \]
		      and
		      \[
			      \log \log n \left( \deg _{\max , n} \cdot \frac{\log \log n}{\log n} - 1\right)
		      \]
		      converges in distribution to some non-trivial limit.
	\end{enumerate}
\end{exercise}
\begin{answer}
	We quickly sketch the proof for \(\deg _{\max , n}\), specifically for the lower bound. Consider a bipartition \((V_1, V_2)\) of \(V\), each with \(n / 2\) vertices. Then, for any \(v \in V_1\), \(\deg _{V_2}(v)\) lower bounds \(\deg _{\max , n}\). Analyzing \(\deg _{V_2}(v)\) turns out to be manageable.
\end{answer}

Now, we're ready to prove \autoref{thm:Erdős-Rényi-phase-transition} \autoref{thm:Erdős-Rényi-phase-transition-a}.

\begin{lemma}[Component of subcritical Erdős-Rényi graph]\label{lma:component-of-subcritical-Erdős-Rényi-graph}
	Let \(G \sim \operatorname{ER}(n, \lambda / n) \) with \(\lambda < 1\).
	\begin{enumerate}[(a)]
		\item\label{lma:component-of-subcritical-Erdős-Rényi-graph-a}  \(\mathcal{C} (1) \overset{D}{\to} \operatorname{BP}(\operatorname{Pois}(\lambda ) ) \). In particular, \(\lvert \mathcal{C} (1) \rvert \overset{D}{\to} \lvert \mathcal{T} _\lambda  \rvert \) where \(\mathcal{T} _\lambda \sim \operatorname{BP}(\operatorname{Pois}(\lambda ) )\).
		\item\label{lma:component-of-subcritical-Erdős-Rényi-graph-b} For any finite \(k\), \((\mathcal{C} (1), \dots , \mathcal{C} (k)) \overset{D}{\to} \mathcal{T} _\lambda \otimes \dots \otimes \mathcal{T} _\lambda \), where \(\mathcal{T} _\lambda \sim \operatorname{BP}(\operatorname{Pois}(\lambda ) )\).
		\item\label{lma:component-of-subcritical-Erdős-Rényi-graph-c} The empirical distribution of components converges weakly to \(\operatorname{BP}(\operatorname{Pois}(\lambda ) ) \).
		\item\label{lma:component-of-subcritical-Erdős-Rényi-graph-d} \(\lvert \mathcal{C} _{\max , n} \rvert \leq (1 / I_\lambda + \epsilon ) \cdot \log n\) with high probability where \(I_\lambda = \lambda - 1 - \log \lambda > 0\).\footnote{Note that \(I_\lambda \) equals to \(0\) at \(1\), and diverges to \(\infty \) at both \(+\infty \) and \(-\infty \).}
	\end{enumerate}
\end{lemma}
\begin{proof}
	Let's prove \autoref{lma:component-of-subcritical-Erdős-Rényi-graph-a} first. Last time, we have shown that \(\lvert \mathcal{C} _n(1) \rvert \preceq \lvert \operatorname{BP} (\operatorname{Bin}(n-1, p) )\rvert \) for \(p = \lambda /n\). To show that in general, \(\mathcal{C} (1) \overset{D}{\to} \operatorname{BP}(\operatorname{Pois}(\lambda ) ) \), we need to compute the pmf of \(\mathcal{C} (1)\). Obviously, the support of the distribution of \(\mathcal{C} (1)\) is on the set of connected rooted graphs \(G^{\prime}  = (V^{\prime} , E^{\prime} )\). Suppose \(G^{\prime} \) is not a tree such that \(\lvert V^{\prime} \rvert = k\) and \(\lvert E^{\prime} \rvert = d \geq k\). By a simple counting argument, we have
	\[
		\Pr_{}(C_n(1) = G^{\prime} )
		= \left( \frac{\lambda }{n} \right) ^d \left( 1 - \frac{\lambda}{n} \right) ^{k (n-k) + \binom{k}{2} - d} \cdot \binom{n-1}{k-1} f(G^{\prime} )
		\to 0,
	\]
	where \(f(G)\) is the number of automorphisms of \(G\), which is finite for any fixed \(G^{\prime} \). Hence, we see that only when \(k = d-1\), this probability is not \(0\). That is to say, in the limit, the component will be a tree. In fact, with the same calculation, we have the following.

	\begin{claim}[Borel-Tenner distribution]
		For \(\lambda \leq 1\), and for \(k \geq 1\), we have
		\[
			\Pr_{}(\lvert \mathcal{T} _\lambda \rvert = k)
			= e^{-\lambda k} \frac{(\lambda k)^{k-1}}{k!}.
		\]
		Moreover, \(\Pr_{}(\mathcal{C} _n(1) = \mathcal{T} ) \to \Pr_{}(\operatorname{BP}(\operatorname{Pois}(\lambda ) ) = \mathcal{T} ) \) for all rooted finite tree \(\mathcal{T} \).
	\end{claim}
	\begin{explanation}
		For a leveled tree \(\mathcal{T} \) of \(k\) vertices with \(k-1\) edges,\footnote{Note that we don't need \(f(\mathcal{T} )\) since we're considering leveled tree, which is already labeled, making it unique (in terms of automorphisms). In some sense \(f(\mathcal{T} )\) is handled by the Cayley's formula below.}
		\[
			\Pr_{}(\mathcal{C} (1) = \mathcal{T} )
			= \left( \frac{\lambda}{n} \right) ^{k-1} \cdot \left( 1 - \frac{\lambda}{n} \right) ^{k(n-1) + \binom{k}{2}  - (k-1)} \cdot \binom{n-1}{k-1}
			\to \frac{\lambda ^{k-1}}{(k-1)!} e^{-\lambda k}.
		\]
		From Cayley's formula, the number of leveled trees on \(k\) nodes is \(k^{k-2}\), proving the claim.
	\end{explanation}

	We omit \autoref{lma:component-of-subcritical-Erdős-Rényi-graph-b} and \autoref{lma:component-of-subcritical-Erdős-Rényi-graph-c} since they can be easily shown. To prove \autoref{lma:component-of-subcritical-Erdős-Rényi-graph-d}, we have
	\[
		\Pr_{}(\lvert \mathcal{C} _{\max , n} \rvert \geq t)
		\leq n \cdot \Pr_{}(\lvert \mathcal{C} _n(1) \rvert \geq t)
		\leq n \cdot \Pr_{}(\lvert \operatorname{BP}(\operatorname{Bin}(n, \lambda / n) )  \rvert \geq t).
	\]

	\begin{notation}
		Recall our algorithmic notation, where we denote the set of active vertices as \(\mathcal{A} _t\) at time \(t\), and \(A_t \coloneqq \lvert \mathcal{A} _t \rvert \).
	\end{notation}

	Specifically, we have \(A_0 = \lvert \mathcal{A} _0 \rvert = 1\), \(A_1 = \lvert \mathcal{A} _1 \rvert = X_1 + 1 - 1 = A_0 + (X_1 - 1)\), \(A_2 = A_1 + (X_2 - 1)\), etc., where \(X_i \overset{\text{i.i.d.} }{\sim } \operatorname{Bin}(n, \lambda / n) \). Then, \(\lvert \operatorname{BP}(\operatorname{Bin}(n, \lambda / n) )  \rvert \) is the hitting time at \(0\), \(H^{\{ 0 \} } = \inf \{ t \geq 1 \mid A_t = 0 \} \). Hence, we have \(\Pr_{}(\lvert \mathcal{C} _n(1) \rvert > t) \leq \Pr_{}(A_t \geq 1)\), where \(A_t = 1 + (X_1 - 1) + \dots  + (X_t - 1)\). Combining the above, for all \(\theta > 0\),
	\[
		\begin{split}
			\Pr_{}(\lvert \mathcal{C} _{\max , n} \rvert \geq t)
			 & \leq n \cdot \Pr_{}\left( \sum_{i=1}^{t} (X_i - 1) \geq 0 \right) \\
			 & \leq n \left( \mathbb{E}_{}[e^{\theta (X_1 - 1)}]  \right) ^t
			= n \left( e^{-\theta } \left( 1 - \frac{\lambda}{n} + \frac{\lambda}{n} e^{\theta } \right)^n \right) ^t
			\leq n \exp (t (-\theta + \lambda (e^\theta - 1)))
		\end{split}
	\]
	Minimizing over \(\theta \), we have \(\lambda e^\theta = 1\), hence \(\theta = \log 1 / \lambda > 0\), which gives
	\[
		n \exp (-t (-\log \lambda - 1 + \lambda ))
		\eqqcolon n \exp (- t I_\lambda )
		= \exp (\log n -t I_\lambda )
	\]
	By taking \(t = (1 / I_\lambda + \epsilon ) \log n\), the probability goes to \(0\), proving the result.
\end{proof}

\begin{note}
	We note that for \autoref{lma:component-of-subcritical-Erdős-Rényi-graph-a}, we can also prove it by observing that in the exploration tree, each vertex has \(\operatorname{Bin}(n-1-c, p) \) children where \(c\) is some constant depending on the same level. Hence, as long as we're considering a fixed level neighborhood, everything converges to \(\operatorname{Pois}(\lambda ) \). In all, for any finite connected rooted tree \(\mathcal{T} \), we have
	\[
		\Pr_{}(\mathcal{C} (1) = \mathcal{T} )  \to \Pr_{}(\operatorname{BP}(\operatorname{Pois}(\lambda ) ) = \mathcal{T} ).
	\]
\end{note}

\begin{note}
	From \autoref{lma:component-of-subcritical-Erdős-Rényi-graph-b}, for \(\lambda < 1\), \(\lvert C_{\max , n} \rvert / \log n \overset{D}{\to} 1 / I_\lambda \) as \(n \to \infty \).
\end{note}

\paragraph{Regular Sparse Regime \(\lambda > 1\).}
Next, we consider the regime when \(\lambda > 1\). Specifically, we want to show \autoref{thm:Erdős-Rényi-phase-transition} \autoref{thm:Erdős-Rényi-phase-transition-b}. This is proved in \autoref{lma:component-of-regular-supercritical-Erdős-Rényi-graph} below.

\begin{lemma}[Component of supercritical Erdős-Rényi graph]\label{lma:component-of-regular-supercritical-Erdős-Rényi-graph}
	Let \(G \sim \operatorname{ER}(n, \lambda / n) \) with \(\lambda > 1\).
	\begin{enumerate}[(a)]
		\item\label{lma:component-of-regular-supercritical-Erdős-Rényi-graph-a} \(\lvert C_{\max _1, n} \rvert / n \overset{p}{\to} \zeta _\lambda\) where \(\zeta _\lambda = \Pr_{}(\operatorname{BP}(\operatorname{Pois}(\lambda ) ) \text{ survives forever} ) \).
		\item\label{lma:component-of-regular-supercritical-Erdős-Rényi-graph-b} \(\lvert C_{\max _2, n} \rvert / \log n \overset{p}{\to} 1 / I_\lambda \) where \(I_\lambda = \lambda - 1 - \log \lambda \).
		\item\label{lma:component-of-regular-supercritical-Erdős-Rényi-graph-c} Outside \(\mathcal{C} _{\max _1, n}\), the graph looks like \(\operatorname{ER}(m, \mu / m) \) for some \(m \approx n (1 - \zeta _\lambda )\) with \(\mu < 1\).
	\end{enumerate}
\end{lemma}

To prove \autoref{lma:component-of-regular-supercritical-Erdős-Rényi-graph}, we divide it into three steps. Fix \(k = k_n \approx A \log n\) for some large \(A\). Define \(Z_{\geq k_n} \coloneqq \sum_{v=1}^{n} \mathbbm{1}_{\lvert \mathcal{C} _n(v) \rvert \geq k_n} \) as the number of vertices that has a component size greater than \(k_n\). Then:
\begin{enumerate}[(i)]
	\item \(\mathbb{E}_{}[Z_{\geq k_n}] \approx n \cdot \zeta _\lambda + o(n^{1 - \epsilon })\) and \(\Var_{}[Z_{\geq k_n}] \ll (\mathbb{E}_{}[Z_{\geq k_n}] )^2\). By using the second-moment method to control \(\Pr_{}(Z_{\geq k_n} = 0) \), e.g., Chebyshev's inequality, we have a concentration bound.
	\item \(\Pr_{}(B \log n \leq \lvert \mathcal{C} (1) \rvert \leq a n) \to 0\) for some \(B > 0\) and for any \(\zeta _\lambda > a\), i.e., either the component is small or large.
	\item \(Z_{\geq k_n} \approx \lvert \mathcal{C} _{\max , n} \rvert \). Since \(Z_{\geq k_n} = \sum_{v=1}^{n} \mathbbm{1}_{\lvert \mathcal{C} (v) \rvert \geq k_n} = \sum_{v \colon \lvert \mathcal{C} (v) \rvert \geq k_n } \lvert \mathcal{C} (v) \rvert \).
\end{enumerate}

Now, to analyze \(Z_{\geq k_n}\), we need to consider the exploration algorithm again. However, for convenience, we will now maintain three sets \((\mathcal{A} , \mathcal{U} , \mathcal{R} )\), corresponding to \emph{active}, \emph{unexplored}, and \emph{already explored} set.

\begin{intuition}
	We see that:
	\begin{itemize}
		\item At time \(0\), \(\mathcal{A} _0 = \{ 1 \} \), \(\mathcal{U} _0 = \{ 2, \dots , n \} \), \(\mathcal{R} _0 = \varnothing \) with \(A_0 = 1\), \(U_0 = n-1\), \(R_0 = 0\).
		\item At time \(1\), \(A_1 = A_0 + \operatorname{Bin}(U_0, p) - 1\), \(U_1 = U_0 - \operatorname{Bin}(U_0, p) - 1\), \(R_1 = 1\).
		\item At time \(2\), \(A_2 = A_1 + \operatorname{Bin}(U_1, p) - 1\), \(U_2 = U_1 - \operatorname{Bin}(U_1, p) \), and \(R_2 = 2\).
		\item In general, \(A_{t+1} - A_t \overset{D}{\sim } \operatorname{Bin}(U_t, p) - 1 \), \(U_{t+1} - U_t = \operatorname{Bin}(U_t, p) \), and \(R_{t+1} = t+1\) for all \(t \geq 1\).
	\end{itemize}
	Eventually, the above \((A, U, R)\) structure embeds the graph. Then, we can simply look at \(R_t\), and look at the parts whenever it hits \(0\) to determine the components.
\end{intuition}