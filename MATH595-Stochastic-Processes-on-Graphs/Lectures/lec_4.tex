\lecture{4}{30 Jan.\ 9:30}{Component Size in Supercritical Regime}
Let's first simplify the notations. At time \(t\), we choose \(i_t \in \mathcal{A} _t\), and let \(\mathcal{C} _{t+1} = \operatorname{Children}(i_t, \mathcal{U} _t) \). Then,
\[
	\mathcal{A} _{t+1} = (\mathcal{A} _t \setminus \{ i_t \} ) \cup \mathcal{C} _{t+1}, \quad
	\mathcal{U} _{t+1} = \mathcal{U} _t \setminus \mathcal{C} _{t+1}, \quad
	\mathcal{R} _{t+1} = \mathcal{R} _{t} \cup \{ i_t \} ,
\]
where \(\mathcal{A} _0 = \{ 1 \} \), \(\mathcal{U} _0 = [n] \setminus \mathcal{A} _0\), and \(\mathcal{R} _0 = \varnothing \). Let \(\xi _{t+1} \coloneqq \lvert \mathcal{C} _{t+1} \rvert \sim \operatorname{Bin}(U_t, p) \) where \(p = \lambda / n\), then
\[
	A_{t+1} = A_t - 1 + \xi _{t+1}, \quad
	U_{t+1} = U_t - \xi _{t+1} \sim \operatorname{Bin}(U_t, 1 - p), \quad
	R_{t+1} = R_t + 1 = t+1.
\]

\begin{claim}
	For all \(t \geq 0\), \(U_t \sim \operatorname{Bin}(U_0, (1 - p)^t) = \operatorname{Bin}(n-1, (1 - p)^t)  \). In particular, \(A_t = n - U_t - R_t = n - t - U_t\) with
	\[
		\mathbb{E}_{}[A_t]
		= n - t - (n - 1) \cdot (1 - p)^t
		= (n-1) \left[ 1 - \left( 1- \lambda / n \right) ^t \right]  - t + 1
		\cong n \left( 1 - e^{-\lambda \cdot \frac{t}{n}} - t / n \right)
	\]
	and \(\Var_{}[A_t] = (n-1)(1 - p)^t \cdot [1 - (1 - p)^t] \approx \lambda t e^{-\lambda t / n}\).
\end{claim}

Let \(f_\lambda (x) = 1 - e^{-\lambda x} - x\), hence, \(\mathbb{E}_{}[A_t] \cong n f_\lambda (t / n)\). Also, let \(\zeta _\lambda > 0\) be the unique positive solution of \(f_\lambda (x) = 0\), where one can easily check that when \(\lambda > 1\), \(f(x) = 0\) for \(x > 0\) has a unique solution.

\begin{intuition}
	Since \(\Var_{}[A_t] < (\mathbb{E}_{}[A_t]) ^2 \), so \(A_t \neq 0\) with high probability away from \(t = 0\) and \(t = \zeta _\lambda \).
\end{intuition}

We prove that this is indeed the case, i.e., there are no intermediate clusters.

\begin{lemma}\label{lma:supercritical-Erdős-Rényi-graph}
	When \(\lambda > 1\) and \(0 < t < n \zeta _{\lambda }\), \(\Pr_{}(A_t = 0) \leq \exp (-t \cdot I_{g(t / n)})\)\footnote{Recall that we define \(I_x = x - 1 - \log x\).} where \(g(x) = \frac{1 - e^{-\lambda x}}{x}\).
\end{lemma}
\begin{proof}
	We see that
	\begin{align*}
		\Pr_{}(A_t = 0)
		 & = \Pr_{}(n-1 - U_t = t-1)                                                                                                        \\
		 & \leq \Pr_{}(\operatorname{Bin}(n-1, 1 - (1 - p)^t) \leq t-1) \tag*{(\(n-1 - U_t \sim \operatorname{Bin}(n-1, 1 - (1 - p)^t)\))}  \\
		 & \leq \Pr_{}(\operatorname{Bin}(n-1, 1 - e^{-p t}) \leq t-1) \tag*{(\(1 - p \leq e^{-p}\))}                                       \\
		 & \leq \Pr_{}(\operatorname{Bin}(n, 1 - e^{-p t}) \leq t)                                                                          \\
		 & \leq \inf _{\theta  > 0} e^{\theta t} \cdot \left[ \mathbb{E}_{}[e^{-\theta \cdot \operatorname{Ber}(1 - e^{-p t}) }] \right] ^n \\
		 & = \inf _{\theta > 0} e^{\theta t} \cdot \left[ 1 - (1 - e^{-\theta } ) (1 - e^{-p t} )\right] ^n                                 \\
		 & \leq \inf _{\theta > 0} \exp (n \cdot \frac{\theta t}{n} - n (1 - e^{-\theta }) (1 - e^{- \frac{\lambda t}{n}}))
		= \inf _{\theta > 0} \exp (n \cdot (\theta a - (1 - e^{-\theta }) (1 - e^{-\lambda a}))),
	\end{align*}
	where we let \(a \coloneqq t / n\). Setting \(a \coloneqq e^{-\theta } \cdot (1 - e^{-\lambda a})\) will minimize the above, i.e., \(e^{\theta } = \frac{1 - e^{-\lambda a}}{a}\). When \(a \in [0, c)\) for some \(c\), we can find \(\theta > 0\) since \(f(x) = 1 - e^{-\lambda x} - x = x (\frac{1-e^{\lambda x}}{x} - 1)\). Specifically, we want \(0 < a = t / n < \zeta _\lambda \), where \(\zeta _\lambda > 0\) is the solution of \(1 - e^{-\lambda \zeta _\lambda } = \zeta _\lambda \). With such \(a\), we have
	\[
		\begin{split}
			\Pr_{}(A_t = 0)
			 & \leq \exp (n \theta a -  n(1 - e^{-\lambda a}) + na) \\
			 & = \exp (-t (g(a) - 1 - \ln g(a)))
			= \exp (-t (g(t / n) - 1 - \ln g(t / n)))
			= \exp (-t I_{g(t / n)}),
		\end{split}
	\]
	where \(\ln g(a) = \theta \) since \(e^\theta = g(a) = \frac{1 - e^{-\lambda a}}{a}\).
\end{proof}

From \autoref{lma:supercritical-Erdős-Rényi-graph}, the following is immediate.

\begin{corollary}\label{col:supercritical-Erdős-Rényi-graph}
	When \(\lambda > 1\) and \(t / n \leq \alpha < \zeta _\lambda \), \(\Pr_{}(A_t = 0) \leq e^{-t c(\alpha )} \) where \(c(\alpha ) \coloneqq I_{g(\alpha )}\).
\end{corollary}
\begin{proof}
	Since \(g^{\prime} (x) = \frac{e^{-\lambda x}}{x^{2} } (1 + \lambda x - e^{\lambda x}) < 0\) when \(x > 0\), we have \(1 < g(\alpha ) \leq g(t / n)\), which implies \(I_{g(\alpha )} \leq I_{g(t / n)}\). The first result then follows from \autoref{lma:supercritical-Erdős-Rényi-graph}.
\end{proof}

From \autoref{col:supercritical-Erdős-Rényi-graph}, the following is immediate.

\begin{corollary}\label{col:supercritical-Erdős-Rényi-graph-intermediate-cluster}
	When \(\lambda > 1\) and \(\alpha < \zeta _\lambda \), there is some \(k > 0\) such that
	\[
		\Pr_{}(\exists t\colon k \log n \leq t \leq \alpha n \text{ such that } A_t = 0)
		\leq \frac{n^{-k c(\alpha )}}{1 - e^{-c(\alpha )}}.
	\]
\end{corollary}

By using a union bound, \autoref{col:supercritical-Erdős-Rényi-graph-intermediate-cluster} leads to the following key lemma.

\begin{lemma}\label{lma:intermediate-component-supercritical-Erdős-Rényi-graph}
	When \(\lambda > 1\) and \(\alpha < \zeta _{\lambda }\), as \(k > 1 / c(\alpha )\), as \(n \to \infty \),
	\[
		\Pr_{}(\exists v \in [n] \colon k \log n \leq \lvert \mathcal{C} (v) \rvert \leq \alpha n)
		\leq n \Pr_{}(k \log n \leq \lvert \mathcal{C} (1) \rvert \leq \alpha n)
		\lesssim n^{1 - k c(\alpha )}
		\to 0.
	\]
\end{lemma}

\begin{remark}
	\autoref{lma:intermediate-component-supercritical-Erdős-Rényi-graph} basically proves that when \(\lambda > 1\), if a cluster survives after some initial size (\(k \log n\)), it'll stay alive until it reaches a size of a constant fraction of \(n\) with high probability.
\end{remark}

Now, we just need to worry about the size of \(\mathcal{C} _{\max }\). To do this, define a random variable that counts the number of vertices having a small component:
\[
	Z_{\leq k_n}
	\coloneqq \sum_{v \in [n]} \mathbbm{1}_{\lvert \mathcal{C} (v) \rvert \leq k_n},
\]
where we let \(k_n \coloneqq k \log n\).

\begin{lemma}\label{lma:small-component-supercritical-Erdős-Rényi-graph}
	When \(\lambda > 1\), \(\lvert Z_{\leq k_n} - n(1 - \zeta _\lambda ) \rvert \leq n^{1 / 2 + \epsilon }\) with high probability for all \(\epsilon > 0\).
\end{lemma}
\begin{proof}
	Consider using mean control as our primary tool. We see that
	\[
		\begin{split}
			\mathbb{E}_{}[Z_{\leq k_n}]
			 & = n \Pr_{}(\lvert \mathcal{C} (1) \rvert \leq k_n)                                                                                                                                                                                                  \\
			 & = n \Pr_{}(\lvert \operatorname{BP}(\operatorname{Bin}(n-1, p) )  \rvert \leq k_n) + O(k_n \cdot p)                                                                                                                                                 \\
			 & = n \Pr_{}(\lvert \operatorname{BP}(\operatorname{Pois}(\lambda ) )  \rvert \leq k_n) + O(k_n \cdot p)                                                                                                                                              \\
			 & = n ( 1 - \underbrace{\Pr_{}(\lvert \mathcal{T} _\lambda  \rvert = \infty )}_{\zeta _\lambda }  - \underbrace{\Pr_{}(k_n < \lvert \mathcal{T} _\lambda \rvert < \infty )}_{e^{-k_n \cdot I _\lambda }} ) + O\left( \frac{\lambda \log n}{n} \right)
			= n (1 - \zeta _\lambda ) + O\left( \frac{\log n}{n} \right).
		\end{split}
	\]
	For the variance, we have
	\[
		\begin{split}
			\Var_{}[Z_{\leq k_n}]
			 & = \sum_{u, v=1}^{n} \Cov_{}[\mathbbm{1}_{\lvert \mathcal{C} (1) \rvert \leq k_n} , \mathbbm{1}_{\lvert \mathcal{C} (u) \rvert \leq k_n} ]                                                                                                                               \\
			 & = n\mathbb{E}_{}\left[\mathbbm{1}_{\lvert \mathcal{C} (1) \rvert \leq k_n} \sum_{u=1}^{n} \left( \mathbbm{1}_{\lvert \mathcal{C} (u) \rvert \leq k_n} - \Pr_{}(\lvert \mathcal{C} (1) \rvert \leq k_n)  \right)  \right]                                                \\
			 & = n \mathbb{E}_{}\left[\mathbbm{1}_{\lvert \mathcal{C} (1) \rvert \leq k_n} \sum_{u=1}^{n} \mathbbm{1}_{1 \leftrightarrow u} \left( \mathbbm{1}_{\lvert \mathcal{C} (u) \rvert \leq k_n} - \Pr_{}(\lvert \mathcal{C} (1) \rvert \leq k_n)  \right)  \right]             \\
			 & \qquad + n \mathbb{E}_{}\left[\mathbbm{1}_{\lvert \mathcal{C} (1) \rvert \leq k_n} \sum_{u=1}^{n} \mathbbm{1}_{1 \not\leftrightarrow u} \left( \mathbbm{1}_{\lvert \mathcal{C} (u) \rvert \leq k_n} - \Pr_{}(\lvert \mathcal{C} (1) \rvert \leq k_n)  \right)  \right].
		\end{split}
	\]
	Let's first look at the first term, where we have
	\[
		\begin{split}
			 & \mathbb{E}_{}\left[\mathbbm{1}_{\lvert \mathcal{C} (1) \rvert \leq k_n} \sum_{u=1}^{n} \mathbbm{1}_{1 \leftrightarrow u} \left( \mathbbm{1}_{\lvert \mathcal{C} (u) \rvert \leq k_n} - \Pr_{}(\lvert \mathcal{C} (1) \rvert \leq k_n)  \right)  \right] \\
			 & = \mathbb{E}_{}\left[\mathbbm{1}_{\lvert \mathcal{C} (1) \rvert \leq k_n} \sum_{u=1}^{n} \mathbbm{1}_{1 \leftrightarrow u} (1 - \Pr_{}(\lvert \mathcal{C} (1) \rvert \leq k_n) ) \right]
			= \mathbb{E}_{}\left[\mathbbm{1}_{\lvert \mathcal{C} (1) \rvert \leq k_n} \lvert \mathcal{C} (1) \rvert \right] ( 1 - \Pr_{}(\lvert \mathcal{C} (1) \rvert \leq k_n) )
			\leq k_n.
		\end{split}
	\]
	As for the second term, we see that
	\[
		\begin{split}
			 & \mathbb{E}_{}\left[\mathbbm{1}_{\lvert \mathcal{C} (1) \rvert \leq k_n} \sum_{u=1}^{n} \mathbbm{1}_{1 \not\leftrightarrow u} \left( \mathbbm{1}_{\lvert \mathcal{C} (u) \rvert \leq k_n} - \Pr_{}(\lvert \mathcal{C} (1) \rvert \leq k_n)  \right)  \right] \\
			 & = \sum_{u=2}^{n} \sum_{\ell =1}^{k_n} \Pr_{}(\lvert \mathcal{C} (1) \rvert = \ell ) \cdot \Pr_{}(1 \not\leftrightarrow u \mid \lvert \mathcal{C} (1) \rvert = \ell )                                                                                        \\
			 & \qquad \cdot \left( \Pr_{}(\lvert \mathcal{C} (u) \rvert \leq k_n \mid 1 \not\leftrightarrow u, \lvert \mathcal{C} (1) \rvert = \ell ) - \Pr_{}(\lvert \mathcal{C} (1) \rvert \leq k_n) \right)                                                             \\
			 & \leq \sum_{u=2}^{n} \sum_{\ell =1}^{k_n} \Pr_{}(\lvert \mathcal{C} (1) \rvert = \ell ) \cdot 1 \cdot \ell k_n \frac{\lambda}{n}                                                                                                                             \\
			 & = \frac{(n-1) k_n \lambda}{n} \mathbb{E}_{}[\mathbbm{1}_{\lvert \mathcal{C} (1) \rvert \leq k_n} \lvert \mathcal{C} (1) \rvert ]
			\leq \lambda k_n^2,
		\end{split}
	\]
	where the first inequality comes from the fact that when \(\lvert \mathcal{C} (1) \rvert = \ell \) and \(1 \not\leftrightarrow u\), \(\lvert \mathcal{C} (u) \rvert \) follows the law of \(\lvert \mathcal{C} (1) \rvert \) in \(\operatorname{ER}(n-\ell , p) \). Then, we couple \(\operatorname{ER}(n-\ell , p) \) and \(\operatorname{ER}(n, p) \) by adding vertices \(\{ n-\ell +1, \dots , n \} \) of \(\operatorname{ER}(n-\ell , p) \) and missing edges (sampled i.i.d.\ from \(\operatorname{Ber}(p) \)). Hence,
	\[
		\begin{split}
			 & \Pr_{}(\lvert \mathcal{C} (u) \rvert \leq k_n \mid 1 \not\leftrightarrow u, \lvert \mathcal{C} (1) \rvert = \ell ) - \Pr_{}(\lvert \mathcal{C} (1) \rvert \leq k_n)               \\
			 & = \Pr_{}(\lvert \mathcal{C} (1) \rvert \leq k_n \text{ in } \operatorname{ER}(n-\ell , p) ) - \Pr_{}(\lvert \mathcal{C} (1) \rvert \leq k_n \text{ in } \operatorname{ER}(n, p) ) \\
			 & = \Pr_{}(\lvert \mathcal{C} (1) \rvert \leq k_n \text{ in } \operatorname{ER}(n-\ell , p) \text{ and } \lvert \mathcal{C} (1) \rvert > k_n \text{ in } \operatorname{ER}(n, p) )  \\
			 & \leq \Pr_{}(\exists u \in \{ n-\ell +1, \dots , n \} \colon 1 \leftrightarrow u)
			\leq \ell k_n p,
		\end{split}
	\]
	since there are at most \(\ell k_n\) many edges between \(\mathcal{C} (1)\) in \(\operatorname{ER}(n-\ell , p) \) and \(\{ n-\ell +1, \dots , n \} \) in \(\operatorname{ER}(n, p) \). Putting everything together, we have, \(\Var_{}[Z_{\leq k_n}] \leq n (k \log n + \lambda k^2 \log ^2 n)\), which gives
	\[
		\Pr_{}(\lvert Z_{\leq k_n} - \mathbb{E}_{}[Z_{\leq k_n}]  \rvert \geq n^{1 / 2 + \epsilon })
		\leq \frac{n \log ^2 n}{n^{1 + 2 \epsilon }}
		\to 0
	\]
	as \(n \to \infty \) from the Chebyshev's Inequality.
\end{proof}

Combining \autoref{lma:intermediate-component-supercritical-Erdős-Rényi-graph} and \autoref{lma:small-component-supercritical-Erdős-Rényi-graph}, we have the following.

\begin{corollary}
	For \(\lambda > 1\), for all \(\alpha \) such that \(0 < \alpha < \zeta _\lambda \), \(\lvert Z_{\geq \alpha n} - n \zeta _\lambda \rvert \leq n^{1 / 2 + \epsilon }\). Moreover, \(\lvert \mathcal{C} _{\max } \rvert = Z_{\geq \alpha n}\) with high probability for all \(\alpha \in (\zeta _\lambda / 2, \zeta _\lambda )\).
\end{corollary}

Putting all results we have, we see that:
\begin{enumerate}[(i)]
	\item No middle ground: no clusters between \([k \log n, \alpha n]\) for \(\alpha < \zeta _\lambda \) (\autoref{lma:intermediate-component-supercritical-Erdős-Rényi-graph}).
	\item The number of vertices with \(\lvert \mathcal{C} (v) \rvert \leq k \log n\) is concentrated at \(n (1 - \zeta _\lambda )\) (\autoref{lma:small-component-supercritical-Erdős-Rényi-graph}).
	\item Everything else is in a single component.
\end{enumerate}

Putting everything together, \autoref{lma:component-of-regular-supercritical-Erdős-Rényi-graph} \autoref{lma:component-of-regular-supercritical-Erdős-Rényi-graph-a} and \autoref{lma:component-of-regular-supercritical-Erdős-Rényi-graph-b} are proved.