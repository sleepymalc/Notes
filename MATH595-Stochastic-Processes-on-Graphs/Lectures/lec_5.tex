\lecture{5}{4 Feb.\ 9:30}{Component Size in Critical Regime}
\subsection{Critical Regime \(\lambda = 1\)}
What is left is the \emph{critical regime}, where we want to prove \autoref{thm:Erdős-Rényi-phase-transition} \autoref{thm:Erdős-Rényi-phase-transition-c}: the random vector \(\frac{1}{n^{2 / 3}} (\lvert \mathcal{C} _{\max _1} \rvert , \lvert \mathcal{C} _{\max _2} \rvert , \dots )\) converges in distribution to a non-trivial limit. To analyze the component size when \(\lambda = 1\), as what we have done previously, we have \(\lvert \mathcal{C} (1) \rvert \preceq \operatorname{Bin}(n-1, 1 / n) \). Moreover,
\begin{itemize}
	\item \(\operatorname{Ber}(p) \preceq \operatorname{Pois}(\theta ) \) for \(\theta = - \log (1 - p)\);
	\item \(\operatorname{Bin}(n-1, p) \preceq \operatorname{Pois}(- (n-1) \log (1 - p)) \preceq \operatorname{Pois}(1) \) with \(p = 1 / n\) since
	      \[
		      - (n-1) \log \left( 1 - \frac{1}{n} \right)
		      = (n-1) \left( \frac{1}{n} + \frac{1}{2n^2} + \frac{1}{3n^3} + \dots \right)
		      \leq (n-1) \cdot \frac{1}{n} \cdot \frac{1}{1-\frac{1}{n}}
		      = 1.
	      \]
\end{itemize}
Hence, for \(\lambda = 1\), we have \(\lvert \mathcal{C} (1) \rvert \preceq \operatorname{Pois}(1) \). This gives the following.

\begin{claim}
	For any \(k > 0\), \(\Pr_{}(\lvert \mathcal{C} (1) \rvert \geq k) \leq 1 / \sqrt{k} \).
\end{claim}
\begin{explanation}
	Let \(\mathcal{T} _1 \sim \operatorname{BP}(\operatorname{Pois}(1) ) \). We see that
	\[
		\Pr_{}(\lvert \mathcal{C} (1) \rvert \geq k)
		\leq \Pr_{}(\lvert \mathcal{T} _1 \rvert \geq k)
		= \sum_{i=k}^{\infty} e^{-i} \frac{(1 \cdot i)^{i-1}}{i!}
		\leq \sum_{i=k}^{\infty} \frac{e^{-i} \cdot i^{i-1}}{\sqrt{2\pi } \cdot i^{1 / 2 + i} \cdot e^{-i}}
		= \frac{1}{\sqrt{2\pi } } \sum_{i=k}^{\infty} \frac{1}{i^{3 / 2}}
		\leq \frac{1}{\sqrt{k} },
	\]
	where we use the Stirling approximation with \(i! \geq \sqrt{2 \pi i} \cdot e^{-i} \cdot i^i\).
\end{explanation}

Given the above bound, if we want to use the usual union bound to bound the maximum component size, the bound is too weak. However, we can improve upon the union bound in this case as
\[
	\Pr_{}(\lvert \mathcal{C} _{\max } \rvert \geq k)
	= \Pr_{}(Z_{\geq k} \geq k)
	\leq \frac{1}{k} \mathbb{E}_{}[Z_{\geq k}]
	= \frac{n}{k} \Pr_{}(\lvert \mathcal{C} (1) \rvert \geq k)
	\leq \frac{n}{k^{3 / 2}},
\]
hence \(k = a \cdot n^{2 / 3}\) for some \(a > 0\) suffices. We now restate and prove \autoref{thm:Erdős-Rényi-phase-transition} \autoref{thm:Erdős-Rényi-phase-transition-c} in \autoref{lma:component-of-regular-critical-Erdős-Rényi-graph}:

\begin{lemma}[Component of critical Erdős-Rényi graph]\label{lma:component-of-regular-critical-Erdős-Rényi-graph}
	Let \(G \sim \operatorname{ER}(n, \lambda / n) \) with \(\lambda = 1\).
	\begin{enumerate}[(a)]
		\item\label{lma:component-of-regular-critical-Erdős-Rényi-graph-a} For any \(\epsilon > 0\), for some large \(a = a(\epsilon )\), \(\liminf_{n \to \infty} \Pr_{}\left( n^{2 / 3} / a \leq \lvert \mathcal{C} _{\max } \rvert \leq a \cdot n^{2 / 3} \right) \geq 1 - \epsilon\).
		\item\label{lma:component-of-regular-critical-Erdős-Rényi-graph-b} For any \(k>0\), \(\frac{1}{n^{2 / 3}} (\lvert \mathcal{C} _{\max _1} \rvert , \lvert \mathcal{C} _{\max _2} \rvert , \dots , \lvert \mathcal{C} _{\max _k} \rvert ) \) converges in distribution to some non-degenerated random vectors as \(n \to \infty \).
	\end{enumerate}
\end{lemma}
\begin{proof}
	We already proved the upper bound part of \autoref{lma:component-of-regular-critical-Erdős-Rényi-graph-a}. For the lower bound, consider \(Z_{\geq n^{2 / 3} / a}\). We can show that it is concentrated at the mean tightly, and as \(a \to \infty \), the mean is small.

	For \autoref{lma:component-of-regular-critical-Erdős-Rényi-graph-b}, recall the exploration algorithm, where we maintain \((\mathcal{A} _t, \mathcal{U} _t, \mathcal{R} _t)\). We know that \(U_t \sim \operatorname{Bin}(n-1, (1-p)^t) \) and \(A_t = n-1-U_t\) with \(A_0 = 1\). We want to study when \(A_t = 0\) for some \(t\) since this indicates the completion of the exploration of the component.

	However, since we want to control \(k\) components at once, after a component is fully explored, we continue the exploration by adding a new random vertex as the seed into the set of active vertices. Hence, the corresponding process is defined as
	\[
		\hat{A} _t
		\coloneqq A_t + \text{\#0 hitting in \([0, t-1]\) in \(\hat{A}_t\)}
		= A_t - \min _{s < t} A_s + 1,
	\]
	where we \emph{add one} to \(A_t\) after the current component is fully explored (\(\hat{A} _t = 0\)).

	\begin{intuition}
		We see that \(\hat{A} _t\) again encodes the entire graph into a single path.
	\end{intuition}

	To make sense of \(\hat{A} _t \coloneqq A_t - \min _{s < t} A_s + 1\), it's worth recalling that \(A_t \to -\infty \) as \(t \to \infty \):

	\begin{prev}
		We have \(A_t \overset{D}{=} n-t-\operatorname{Bin}(n-1, (1-p)^t) \) with
		\[
			\mathbb{E}_{}[A_t]
			= n - t - (n-1) \left( 1 - \frac{1}{n} \right) ^t
			\approx 1 - \frac{t}{n} + \frac{t^2}{2n} + \dots.
		\]
		When \(t \geq \sqrt{n} \), the quadratic term dominates, contributing a negative drift. Moreover,
		\[
			\Var_{}[A_t] = (n-1)\left( 1 - \frac{1}{n} \right) ^t \left( 1 - \left( 1 - \frac{1}{n} \right) ^t \right)
			\approx t e^{-t / n}.
		\]
	\end{prev}

	One can check that when \(1 \ll t \ll n\), by CLT, the standard Binomial converges in distribution to \(\mathcal{N} (0, 1)\) if and only if \(n p (1 - p) \to \infty \). This implies \(A_t \approx - t^2 / 2n + \sqrt{t} \cdot \mathcal{N} (0, 1) \).

	\begin{intuition}
		The timescale where \(t^2 / n \approx \sqrt{t} \), i.e., \(t \approx n^{2 / 3}\), is necessary to maintain the balance between the subcritical and supercritical behavior.
	\end{intuition}

	From the recursive definition of \(U_t\), we can make a martingale. Let \(B_s\) denotes the standard Brownian motion, then with martingale CLT, one can prove that
	\[
		\left( \frac{1}{n^{1 / 3}} A_{\lfloor s \cdot n^{2 / 3} \rfloor } \right) _{s \geq 0}
		\overset{D}{\to} \left( - \frac{s^2}{2} + B_s \right) _{s \geq 0}.
	\]
	Hence, for \(\hat{A} _t = A_t - \min _{s < t} A_s + 1\), scaling by \(n^{2 / 3}\) now, we have
	\[
		\left( \frac{1}{n^{2 / 3}} \hat{A} _{s \cdot n^{2 / 3}} \right) _{s \geq 0}
		\overset{D}{\to} \left( \left( B_s - \frac{s^2}{2} \right) - \inf _{t \leq s} \left( B_t - \frac{t^2}{2} \right)  \right)_{s \geq 0},
	\]
	which is precisely the reflected (and thus non-negative) version of the process \((B_s - s^2 / 2)_{s\geq 0}\). With this, we can use a martingale CLT argument to show that the component sizes converge to the excursion lengths of the \(\hat{A} _t\) process.
\end{proof}

In particular, from the proof of \autoref{lma:component-of-regular-critical-Erdős-Rényi-graph}, we know that \(\lvert \mathcal{C} _{\max _1} \rvert = \Theta _p(n^{2 / 3})\). Furthermore, we can zoom in at the critical region and study what will happen when \(\lambda \) is very close to \(1\).

\begin{remark}[Critical window]
	When \(\lambda = 1 + \theta / n^{1 / 3} \) for some fixed \(\theta \in \mathbb{R} \), the above becomes
	\[
		\left( \frac{1}{n^{1 / 3}} A_{\lfloor s \cdot n^{2 / 3} \rfloor } \right) _{s \geq 0}
		\overset{D}{\to} \left( - \frac{s^2}{2} + B_s + \theta s\right) _{s \geq 0},
	\]
	and
	\[
		\left( \frac{1}{n^{2 / 3}} \hat{A} _{s \cdot n^{2 / 3}} \right) _{s \geq 0}
		\overset{D}{\to} \left( \left( B_S - \frac{s^2}{2} + \theta s\right) - \inf _{t \leq s} \left( B_t - \frac{t^2}{2} + \theta t \right)  \right)_{s \geq 0}.
	\]
	Hence, when \(\lambda \) is in a small window \([1 - \theta / n^{1 / 3}, 1 + \theta / n^{1 / 3}]\) around \(1\), we're effectively in the critical regime where the phase transition happens.
\end{remark}

This concludes the discussion for the component sizes on the sparse regime where \(\lambda = \Theta (1)\).

\section{Structure Counting in Various Regime}
Next, we're interested in understanding the structural emergence behavior as \(\lambda \) varies.

\begin{eg}[Disconnected edge]
	Again consider \(\operatorname{ER}(n, \lambda / n) \) for some \(\lambda \in (0, \infty )\). Then
	\[
		\mathbb{E}_{}[\text{\#disconnected edge} ]
		= \frac{n(n-1)}{2} \cdot \frac{\lambda}{n} \left( 1 - \frac{\lambda}{n} \right) ^{2 (n-2)}.
	\]
\end{eg}

We have done such a counting several times. For instance, one can consider other structures such as \(3\)-chains, \hyperref[def:cycle]{cycles}, etc. In general, we have the following:

\begin{eg}
	Let \(k\) and \(\ell \) be the number of vertices and edges of a specific disconnected structure \(S\), respectively. Then we see that
	\[
		\mathbb{E}_{}[\# S \text{ in } \operatorname{ER}(n, \lambda / n) ]
		= \binom{n}{k} \left( \frac{\lambda}{n} \right) ^{\ell } \left( 1 - \frac{\lambda}{n} \right) ^{k(n-k)}
		\approx \frac{n^k}{k!} \frac{\lambda ^\ell }{n^{\ell } } e^{-\lambda k}.
	\]
\end{eg}

\begin{intuition}
	We see that it gets increasingly difficult (with \(k\) grows) for \(k\)-components to remain isolated. This hints that the bottleneck to connectivity of a graph are isolated vertices.
\end{intuition}

It turns out that we can characterize this. In particular, we will see that the connectivity threshold is \(\log n\): as \(\lambda < \log n\), there are single vertices, while after \(\lambda > \log n\), the whole graph is connected. We will also study the behavior when \(\lambda = \Theta (n)\) later.