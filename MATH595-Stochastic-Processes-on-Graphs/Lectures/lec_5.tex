\lecture{5}{4 Feb.\ 9:30}{Component Size in Critical Regime}
\subsection{Critical Regime \(\lambda = 1\)}
What is left is the \emph{critical regime}, where we want to prove \autoref{thm:Erdős-Rényi-phase-transition} \autoref{thm:Erdős-Rényi-phase-transition-c}: the random vector \(\frac{1}{n^{2 / 3}} (\lvert \mathcal{C} _{\max _1} \rvert , \lvert \mathcal{C} _{\max _2} \rvert , \dots )\) converges in distribution to a non-trivial limit. To analyze the component size when \(\lambda = 1\), as what we have done previously, we have \(\lvert \mathcal{C} (1) \rvert \preceq \operatorname{Bin}(n-1, 1 / n) \). Moreover,
\begin{itemize}
	\item \(\operatorname{Ber}(p) \preceq \operatorname{Pois}(\theta ) \) for \(\theta = - \log (1 - p)\);
	\item \(\operatorname{Bin}(n-1, p) \preceq \operatorname{Pois}(- (n-1) \log (1 - p)) \preceq \operatorname{Pois}(1) \) with \(p = 1 / n\) since
	      \[
		      - (n-1) \log \left( 1 - \frac{1}{n} \right)
		      = (n-1) \left( \frac{1}{n} + \frac{1}{2n^2} + \frac{1}{3n^3} + \dots \right)
		      \leq (n-1) \cdot \frac{1}{n} \cdot \frac{1}{1-\frac{1}{n}}
		      = 1.
	      \]
\end{itemize}
Hence, for \(\lambda = 1\), we have \(\lvert \mathcal{C} (1) \rvert \preceq \operatorname{Pois}(1) \). This gives the following.

\begin{claim}
	For any \(k > 0\), \(\Pr_{}(\lvert \mathcal{C} (1) \rvert \geq k) \leq 1 / \sqrt{k} \).
\end{claim}
\begin{explanation}
	Let \(\mathcal{T} _1 \sim \operatorname{BP}(\operatorname{Pois}(1) ) \). We see that
	\[
		\Pr_{}(\lvert \mathcal{C} (1) \rvert \geq k)
		\leq \Pr_{}(\lvert \mathcal{T} _1 \rvert \geq k)
		= \sum_{i=k}^{\infty} e^{-i} \frac{(1 \cdot i)^{i-1}}{i!}
		\leq \sum_{i=k}^{\infty} \frac{e^{-i} \cdot i^{i-1}}{\sqrt{2\pi } \cdot i^{1 / 2 + i} \cdot e^{-i}}
		= \frac{1}{\sqrt{2\pi } } \sum_{i=k}^{\infty} \frac{1}{i^{3 / 2}}
		\leq \frac{1}{\sqrt{k} },
	\]
	where we use the Stirling approximation with \(i! \geq \sqrt{2 \pi i} \cdot e^{-i} \cdot i^i\).
\end{explanation}

Given the above bound, if we want to use the usual union bound to bound the maximum component size, the bound is too weak. However, we can improve upon the union bound in this case as
\[
	\Pr_{}(\lvert \mathcal{C} _{\max } \rvert \geq k)
	= \Pr_{}(Z_{\geq k} \geq k)
	\leq \frac{1}{k} \mathbb{E}_{}[Z_{\geq k}]
	= \frac{n}{k} \Pr_{}(\lvert \mathcal{C} (1) \rvert \geq k)
	\leq \frac{n}{k^{3 / 2}},
\]
hence \(k = a \cdot n^{2 / 3}\) for some \(a > 0\) suffices. We now restate and prove \autoref{thm:Erdős-Rényi-phase-transition} \autoref{thm:Erdős-Rényi-phase-transition-c} in \autoref{lma:component-of-regular-critical-Erdős-Rényi-graph}:

\begin{lemma}[Component of critical Erdős-Rényi graph]\label{lma:component-of-regular-critical-Erdős-Rényi-graph}
	Let \(G \sim \operatorname{ER}(n, \lambda / n) \) with \(\lambda = 1\).
	\begin{enumerate}[(a)]
		\item\label{lma:component-of-regular-critical-Erdős-Rényi-graph-a} For any \(\epsilon > 0\), for some large \(a = a(\epsilon )\), \(\liminf_{n \to \infty} \Pr_{}\left( n^{2 / 3} / a \leq \lvert \mathcal{C} _{\max } \rvert \leq a \cdot n^{2 / 3} \right) \geq 1 - \epsilon\).
		\item\label{lma:component-of-regular-critical-Erdős-Rényi-graph-b} For any \(k>0\), \(\frac{1}{n^{2 / 3}} (\lvert \mathcal{C} _{\max _1} \rvert , \lvert \mathcal{C} _{\max _2} \rvert , \dots , \lvert \mathcal{C} _{\max _k} \rvert ) \) converges in distribution to some non-degenerated random vectors as \(n \to \infty \).
	\end{enumerate}
\end{lemma}
\begin{proof}
	We already proved the upper bound part of \autoref{lma:component-of-regular-critical-Erdős-Rényi-graph-a}. For the lower bound, consider \(Z_{< n^{2 / 3} / a}\). We can show that it is concentrated at the mean tightly, and as \(a \to \infty \), the mean is small.

	For \autoref{lma:component-of-regular-critical-Erdős-Rényi-graph-b}, recall the exploration algorithm, where we maintain \((\mathcal{A} _t, \mathcal{U} _t, \mathcal{R} _t)\). We know that \(U_t \sim \operatorname{Bin}(n-1, (1-p)^t) \) and \(A_t = n-1-U_t\) with \(A_0 = 1\). However, since we want to control \(k\) components at once, after we have finished exploring a component, we want to continue exploring the graph by choosing a new random vertex as the seed. Hence, we consider a different process
	\[
		\hat{A} _t
		\coloneqq A_t + \text{\#0 hitting in \([0, t-1]\) in \(\hat{A}_t\)}
		= A_t - \min _{s < t} A_s + 1,
	\]
	where we \emph{add one} (corresponding to starting at an unexplored vertex in another unexplored component) to \(A_t\) after the current component is fully explored (\(\hat{A} _t = 0\)).

	\begin{prev}
		We have \(A_t \overset{D}{=} n-t-\operatorname{Bin}(n-1, (1-p)^t) \) with \(\mathbb{E}_{}[A_t] \approx n (1 - e^{- t / n} - t / n)\) and \(\Var_{}[A_t] = (n-1)(1- 1 / n)^t (1 - (1 - 1 / n)^t) \approx t e^{-t / n}\).
	\end{prev}

	One can check that when \(1 \ll t \ll n\), \(A_t \approx - t^2 / 2n + \sqrt{t} \cdot \mathcal{N} (0, 1) \). Moreover, since \(U_t\) is defined recursively, we can make a martingale. With martingale CLT, one can prove that
	\[
		\left( \frac{1}{n^{1 / 3}} A_{\lfloor s \cdot n^{2 / 3} \rfloor } \right) _{s \geq 0}
		\overset{D}{\to} \left( - \frac{s^2}{2} + B_s \right) _{s \geq 0},
	\]
	where \(B_s\) denotes the standard Brownian motion. Hence, for \(\hat{A} _t = A_t - \min _{s < t} A_s + 1\), we have
	\[
		\left( \frac{1}{n^{1 / 3}} \hat{A} _{s \cdot n^{2 / 3}} \right) _{s \geq 0}
		\overset{D}{\to} \left( \left( B_s - \frac{s^2}{2} \right) - \inf _{t \leq s} \left( B_t - \frac{t^2}{2} \right)  \right)_{s \geq 0}.
	\]\todo{finish the argument}
\end{proof}

In particular, from the proof of \autoref{lma:component-of-regular-critical-Erdős-Rényi-graph}, we know that \(\lvert \mathcal{C} _{\max _1} \rvert = \Theta _p(n^{2 / 3})\).

\begin{remark}[Critical window]
	If we let \(\lambda = 1 + \theta / n^{1 / 3} \) for some fixed \(\theta \in \mathbb{R} \), the above becomes
	\[
		\left( \frac{1}{n^{1 / 3}} A_{\lfloor s \cdot n^{2 / 3} \rfloor } \right) _{s \geq 0}
		\overset{D}{\to} \left( - \frac{s^2}{2} + B_s + \theta s\right) _{s \geq 0},
	\]
	and
	\[
		\left( \frac{1}{n^{1 / 3}} \hat{A} _{s \cdot n^{2 / 3}} \right) _{s \geq 0}
		\overset{D}{\to} \left( \left( B_S - \frac{s^2}{2} + \theta s\right) - \inf _{t \leq s} \left( B_t - \frac{t^2}{2} + \theta t \right)  \right)_{s \geq 0}.
	\]
	Hence, when \(\lambda \) is in a small window around \(1\), we're effectively in the critical regime.
\end{remark}

This concludes the discussion for the component sizes on the sparse regime where \(\lambda = \Theta (1)\).

\section{Structure Counting in Various Regime}
Next, we're interested in understanding the structural emergence behavior as \(\lambda \) varies.

\begin{eg}[Disconnected edge]
	Again consider \(\operatorname{ER}(n, \lambda / n) \) for some \(\lambda \in (0, \infty )\). Then
	\[
		\mathbb{E}_{}[\text{\#disconnected edge} ]
		= \frac{n(n-1)}{2} \cdot \frac{\lambda}{n} \left( 1 - \frac{\lambda}{n} \right) ^{2 (n-2)}.
	\]
\end{eg}

We have done such a counting several times. For instance, one can consider other structures such as \(3\)-chains, \hyperref[def:cycle]{cycles}, etc. In general, we have the following:

\begin{eg}
	Let \(k\) and \(\ell \) be the number of vertices and edges of a specific disconnected structure \(S\), respectively. Then we see that
	\[
		\mathbb{E}_{}[\# S \text{ in } \operatorname{ER}(n, \lambda / n) ]
		= \binom{n}{k} \left( \frac{\lambda}{n} \right) ^{\ell } \left( 1 - \frac{\lambda}{n} \right) ^{k(n-k)}
	\]
\end{eg}

It turns out that we can do much more. In particular, we will show, for example, as \(\lambda < \log n\), there are single vertices, while after \(\lambda > \log n\), the whole graph is connected. We will also study the behavior when \(\lambda = \Theta (n)\).