\lecture{6}{6 Feb.\ 9:30}{Subgraph Counting in Sparse Regime}
\subsection{Stein-Chen Method for Poisson Approximation}
Let's first summarize some common proof techniques we have seen so far:

\begin{prev}
	For some counting random variable \(Z\) (i.e., non-negative integer-valued):
	\begin{itemize}
		\item \(\Pr_{}(Z > 0) = \Pr_{}(Z \geq 1) \leq \mathbb{E}_{}[Z] \). For example, \(\Pr_{}(\lvert \mathcal{C} _{\max } \rvert \geq k) = \Pr_{}(Z_{\geq k} \geq k) \leq \mathbb{E}_{}[Z_{\geq k}] / k\).
		\item \(\Pr_{}(Z = 0) = \Pr_{}(Z - \mathbb{E}_{}[Z] = - \mathbb{E}_{}[Z] ) \leq \Pr_{}(\lvert Z - \mathbb{E}_{}[Z] \rvert \geq \mathbb{E}_{}[Z] ) \leq \Var_{}[Z] / (\mathbb{E}_{}[Z] )^2\).
	\end{itemize}
\end{prev}

To proceed, we will need some tools on \(\operatorname{Pois}(\lambda ) \). The following notation will be heavily used.

\begin{notation}
	For \(X, k \in \mathbb{N} \), we let \((X)_k \coloneqq X! / k! = X (X-1) \cdots (X-k+1)\).
\end{notation}

By some calculation, the following can be shown.

\begin{lemma}\label{lma:Poisson-moment}
	For \(X \sim \operatorname{Pois}(\lambda ) \), \(\mathbb{E}_{}[(X)_k] \coloneqq \mathbb{E}_{}[X (X - 1) \dots (X-k+1)] = \lambda ^k\) for all \(k = 1, 2, \dots \).
\end{lemma}

Surprisingly, if all moments of a random variable converges to what is stated in \autoref{lma:Poisson-moment}, then it indeed will converge in distribution to a Poisson random variable.

\begin{lemma}\label{lma:Poisson-moment-convergence}
	For a non-negative integer random variable \(X_n\), if \(\mathbb{E}_{}[(X_n)_k] \to \lambda ^k\) as \(n \to \infty \) for all \(k = 1, 2, \dots \), then \(X_n \overset{D}{\to} \operatorname{Pois}(\lambda ) \).
\end{lemma}

With \autoref{lma:Poisson-moment-convergence}, the main tool we will be utilized can be proven (omit due to its length):

\begin{theorem}[Stein-Chen method]\label{thm:Stein-Chen-method}
	Let \((A_i)_{i \geq 1}^{n}\) be a sequence of events with \(p_i = \Pr_{}(A_i) \) for all \(i \in [n]\), and let \(X = \sum_{i=1}^{n} \mathbbm{1}_{A_i} \) with \(\lambda = \mathbb{E}_{}[X] = \sum_{i=1}^{n} p_i\). If \((A_i)_{i \geq 1}^{n}\)'s are positively associated, i.e., \((A_i)_{i \neq j} \mid A_j \geq (A_i)_{i \neq j}\) for all \(j\), then,
	\[
		d_{\mathrm{TV} }(X, \operatorname{Pois}(\lambda ) )
		= \frac{1}{2} \sum_{k \geq 0} \left\lvert \Pr_{}(X = k) - e^{-\lambda } \frac{\lambda ^k }{k!}  \right\rvert
		\leq \min ( 1, 1 / \lambda ) \left( \Var_{}[X] - \lambda + 2 \sum_{i=1}^{n} p_i^2 \right).
	\]
	On the other hand, if \((A_i)_{i\geq 1}^{n}\) are negatively associated, i.e., \((A_i)_{i \neq j} \mid A_j \leq (A_i)_{i \neq j}\)  for all \(j\), then
	\[
		d_{\mathrm{TV} }(X, \operatorname{Pois}(\lambda ) )
		= \frac{1}{2} \sum_{k \geq 0} \left\lvert \Pr_{}(X = k) - e^{-\lambda } \frac{\lambda ^k }{k!}  \right\rvert
		\leq \min ( 1, 1 / \lambda ) \left( \lambda - \Var_{}[X] \right).
	\]
\end{theorem}

\subsection{Injective Cycle Counting}
Consider the \hyperref[prb:cycle-counting]{cycle counting problem} for \(\operatorname{ER}(n, \lambda / n) \) for some \(\lambda > 0\):

\begin{problem}[Cycle counting]\label{prb:cycle-counting}
For some fixed \(k \geq 3\), we're interested in controlling
\[
	X_k
	\coloneqq \sum_{\quotient{\substack{(v_1, \dots , v_k), v_i \in [n] \\ v_i \neq v_{i^{\prime} } \text{ for \(i \neq i^{\prime}\) }}}{\substack{\text{starting point}\\ \text{orientation} }}} \mathbbm{1}_{(v_1, \dots , v_k) \text{ is a \(k\)-cycle} },
\]
where the summation is over all \(k\) distinct vertices modulo the starting one and the orientation.
\end{problem}

\begin{note}
	For the \hyperref[prb:cycle-counting]{cycle counting} problem, it's okay that the \(k\)-cycle has additional edges, i.e., we care about induced subgraph rather than the exact structure.
\end{note}

The following is easy to see.

\begin{lemma}\label{lma:Erdős-Rényi-cycle-counting-mean}
	Let \(G \sim \operatorname{ER}(n, \lambda / n) \). When \(\lambda < 1\), the expected number of cycles is less than \(\sum_{k=3}^{\infty} \lambda ^k / 2k < \infty \). Moreover, the expected number of vertices in a cycle is less than \(\sum_{k=3}^{\infty} \mathbb{E}_{}[k \cdot X_k] \leq \sum_{k=3}^{\infty} \lambda ^k / 2 < \infty \).
\end{lemma}
\begin{proof}
	By a simple counting argument, we see that
	\[
		\mathbb{E}_{}[X_k]
		= \left( \frac{\lambda}{n} \right) ^k \cdot \binom{n}{k} \frac{k!}{2 \cdot k}
		= \frac{\lambda ^k}{n^k} \cdot \frac{n(n-1) \dots (n-k+1)}{2k}
		= \frac{\lambda ^k}{2k} \prod_{i=1}^{k} \left( 1 - \frac{i}{n} \right)
		\approx \frac{\lambda ^k}{2k} e^{- \frac{k (k-1)}{2n}}.
	\]
	Hence, when \(k \ll \sqrt{n} \), we have \(\mathbb{E}_{}[X_k] \lessapprox \lambda ^k / 2k\), which becomes more vacuous as \(k\) increases.
\end{proof}

We can also calculate the variance of \(X_k\). In particular, we see that
\begin{equation}\label{lma:Erdős-Rényi-cycle-counting-variance}
	\Var_{}[X_k]
	= \binom{n}{k} \frac{k!}{2k} \cdot \left( \frac{\lambda}{n} \right) ^k \left( 1 - \left( \frac{\lambda}{n} \right) ^k \right) + O\left( \sum_{s=1}^{k-2} \binom{n}{k} \cdot \frac{k!}{2k} \cdot n^{k - s - 1} \cdot \left( \frac{\lambda}{n} \right)^{k + k - s} \right),
\end{equation}
where the big-\(O\) (second) term is the covariance: If two cycles don't share edges, then the covariance is \(0\). Otherwise, it is strictly greater than \(0\), with \(s\) being the number of shared edges between these two cycles. In particular, we can show that as \(n \to \infty \), we have
\[
	\sum_{s=1}^{k-2} \binom{n}{k} \cdot \frac{k!}{2k} \cdot n^{k - s - 1} \left( \frac{\lambda }{n} \right) ^{k + k - s}
	\leq \sum_{s=1}^{k-2} \frac{\lambda ^{2k - s}}{2k} \cdot \frac{1}{n} \to 0.
\]

From the \hyperref[thm:Stein-Chen-method]{Stein-Chen method}, we can show the following.

\begin{theorem}\label{thm:Erdős-Rényi-cycle-counting}
	Let \(G \sim \operatorname{ER}(n, \lambda / n) \). For a fixed \(\lambda > 0\) and \(k \geq 3\), we have
	\begin{itemize}
		\item \(X_k \overset{D}{\to} \operatorname{Pois}(\lambda ^k / 2k) \) as \(n \to \infty \).
		\item For any fixed \(d\), \((X_k)_{k = 3}^d \overset{D}{\to} \bigotimes_{k=3}^d \operatorname{Pois}(\lambda ^k / 2k) \) as \(n \to \infty \).
		\item For any fixed \(d\), \(\sum_{k=3}^{d} X_k \overset{D}{\to} \operatorname{Pois}(\sum_{k=3}^{d} \lambda ^k / 2k) \) for all \(\lambda > 0\).
		\item If \(\lambda < 1\), the above converges, i.e., \(\sum_{k=3}^{\infty} X_k \overset{D}{\to} \operatorname{Pois}(\sum_{k=3}^{\infty} \lambda ^k / 2k) \).
	\end{itemize}
\end{theorem}

\subsection{Injective Tree Counting}
The next elementary object after cycles might be trees. Let \(G \sim \operatorname{ER}(n, \lambda / n) \), consider the problem of degree counting, which simply corresponds to the star graph. Fix \(k \geq 0\), then the number of vertices with degree \(k\) is defined as
\[
	N_k = \sum_{v=1}^{n} \mathbbm{1}_{\deg (v) = k}.
\]
We see that
\begin{equation}\label{eq:Erdős-Rényi-degree-counting-mean}
	\mathbb{E}_{}[N_k]
	= n \cdot \binom{n-1}{k} \cdot \left( \frac{\lambda}{n} \right) ^k \left( 1 - \frac{\lambda}{n} \right) ^{n-1-k}
	\approx n \cdot \frac{n^k}{k!} \cdot \frac{\lambda ^k}{n^k} e^{-\lambda }
	= n \cdot \frac{\lambda ^k}{k!} e^{-\lambda }.
\end{equation}
Hence, for all \(k \geq 0\), as \(n \to \infty \),
\[
	\frac{1}{n} \mathbb{E}_{}[N_k]
	\to \frac{\lambda ^k}{k!} e^{-\lambda }.
\]
We can also calculate the variance of \(N_k\), which is
\begin{equation}\label{eq:Erdős-Rényi-degree-counting-variance}
	\begin{split}
		\Var_{}[N_k]
		 & = n \left( \frac{\lambda ^k e^{-\lambda }}{k!} \left( 1 - \frac{\lambda ^k e^{-\lambda }}{k!} \right) + o(1) \right) \\
		 & \qquad\qquad + n(n-1) \left( \Pr_{}(\deg (1) = k, \deg (2) = k) - \Pr_{}(\deg (v) = k)^2 \right).
	\end{split}
\end{equation}
We see that \(\Pr_{}(\deg (v) = k)^2 = \Pr_{}(\operatorname{Bin}(n-1, p) =k)^2\) and
\[
	\Pr_{}(\deg (1) = k, \deg (2) = k)
	= \frac{\lambda}{n} \Pr_{}(\operatorname{Bin}(n-2, p) = k-1 ) ^2 + \left( 1 - \frac{\lambda}{n} \right) \Pr_{}(\operatorname{Bin}(n-2, p) = k )^2.
\]
Overall, we have \(\Pr_{}(\deg (1) = k, \deg (2) = k) - \Pr_{}(\deg (v) = k)^2 \approx c_\lambda / n + \dots \).

\begin{theorem}
	Let \(G \sim \operatorname{ER}(n, \lambda / n) \). For any fixed \(k \geq 0\), \(\lambda > 0\), and \(\ell \geq 1\), as \(n \to \infty \), we have
	\[
		\left( \frac{N_k - \mathbb{E}_{}[N_k] }{\sqrt{n} } \right) _{k=1}^{\ell }
		\overset{D}{\to} \mathcal{N} _\ell (0, D)
	\]
	where \(D \in \mathbb{R} ^{\ell \times \ell }\) is a positive definite covariance matrix.
\end{theorem}

\begin{note}
	In the above calculation, we assume that there can be edges presented between the non-center vertices of the star.
\end{note}

\begin{remark}
	For the counting problem, the Poisson approximation holds if and only if \(\Var_{}[Z] / \mathbb{E}_{}[Z] \to 1\) as \(n \to \infty \).
\end{remark}

Using the same idea, we can consider any given tree structure.

\begin{intuition}
	Consider the following tree with \(5\) vertices and \(4\) edges:
	\begin{center}
		\incfig{subgraph-count-1}
	\end{center}
	We see that the number of components with this tree structure has mean \(\approx n^5 \cdot \lambda ^4 / n^4 = n \lambda ^4\).
\end{intuition}

\subsection{Connected Component Counting and Connectivity Threshold}
One can actually consider a more restrictive version of the structure counting, where we require the structure to be presented \emph{exactly}, i.e., in the exact connected component sense. In general, one can show that for any tree \(\mathcal{T} \), let \(N_{\mathcal{T} } \) to be the number of clusters that look like \(\mathcal{T} \) in \(\operatorname{ER}(n, p) \). Then,
\[
	\left( \frac{N_{\mathcal{T} } - \mathbb{E}_{}[N_{\mathcal{T} }] }{\sqrt{n} } \right) _{\mathcal{T} }
	\overset{D}{\to} \mathcal{N} _\ell (0, D)
\]
for some non-degenerate \(D\), where \(\ell \) is the number of trees \(\mathcal{T} \)'s we considered jointly.

\begin{eg}
	Consider a tree \(\mathcal{T} \) that is a \(3\)-chain. Then, \(\mathbb{E}_{}[N_{\mathcal{T} }] \approx n \lambda ^2 e^{-3\lambda }\), where \(N_{\mathcal{T} } \) counts the number of induced subgraph of \(3\) vertices being \(\mathcal{T} \).
	\begin{center}
		\incfig{subgraph-count-2}
	\end{center}
\end{eg}
\begin{explanation}
	We see that since there are \(3\) possible edge configurations for \(\mathcal{T} \) among three vertices,
	\[
		\mathbb{E}_{}[N_{\mathcal{T} } ]
		= \binom{n}{3} \cdot 3 \cdot \left( \frac{\lambda }{n} \right) ^{2} \left( 1 - \frac{\lambda}{n} \right) ^{3(n-3) + 1}
		\approx n \lambda ^2 e^{-3\lambda },
	\]
	with the difference being not allowing extra edges present between the two vertices at the end.
\end{explanation}

\begin{eg}
	More generally, given a cluster \(\mathcal{T} _k\) of \(k\) nodes and \(k-1\) edges, for some small constant \(c_k\),
	\[
		\mathbb{E}_{}[N_{\mathcal{T} _k} ]
		= n \lambda ^{k-1} e^{-k \lambda } \cdot \Theta _{k} (1)
		= \exp (\log n - k \lambda + (k-1) \log \lambda + c_k).
	\]
\end{eg}

The above calculation has some interesting implications. If \(\lambda > (1 + \epsilon ) \log n / k\), then \(\mathbb{E}_{}[N_{\mathcal{T} _k}] \to 0\) as \(n \to \infty \), i.e., no \(\mathcal{T} _k\) will appear. In particular, we will show that as \(\lambda > \log n\), i.e., when isolating vertices stop appearing, \(G\) becomes connected.

\begin{intuition}
	Consider \(\lambda >(1 + \epsilon ) \log n / k\) and \(k\) decreases from a large number until \(k = 1\). We see that \(\mathcal{T} _k\) stop showing in order, and in the end, the graph becomes connected.
\end{intuition}