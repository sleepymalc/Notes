\lecture{6}{6 Feb.\ 9:30}{Cycle Counting, Star Counting, and Tree Counting}
Let's first summarize some common proof techniques we have seen so far:

\begin{prev}
	For some counting random variable \(Z\):
	\begin{itemize}
		\item \(\Pr_{}(Z > 0) = \Pr_{}(Z \geq 1) \leq \mathbb{E}_{}[Z] \). For example, \(\Pr_{}(\lvert \mathcal{C} _{\max } \rvert \geq k) = \Pr_{}(Z_{\geq k} \geq k) \leq \mathbb{E}_{}[Z_{\geq k} / k] \).
		\item \(\Pr_{}(Z = 0) = \Pr_{}(Z - \mathbb{E}_{}[Z] = - \mathbb{E}_{}[Z] ) \leq \Pr_{}(\lvert Z - \mathbb{E}_{}[Z] \rvert \geq \mathbb{E}_{}[Z] ) \leq \Var_{}[Z] / (\mathbb{E}_{}[Z] )^2\).
	\end{itemize}
\end{prev}

\subsection{Stein-Chen Method for Poisson Approximation}
To proceed, we will need some tools on \(\operatorname{Pois}(\lambda ) \). The following notation will be heavily used.

\begin{notation}
	For \(X, k \in \mathbb{N} \), we let \((X)_k \coloneqq X! / k! = X (X-1) \cdots (X-k+1)\).
\end{notation}

By some calculation, the following can be shown.

\begin{lemma}\label{lma:Poisson-moment}
	For \(X \sim \operatorname{Pois}(\lambda ) \), \(\mathbb{E}_{}[(X)_k] \coloneqq \mathbb{E}_{}[X (X - 1) \dots (X-k+1)] = \lambda ^k\) for all \(k \geq 1\).
\end{lemma}

Surprisingly, if all moments of a random variable converges to what is stated in \autoref{lma:Poisson-moment}, then it indeed will converge in distribution to a Poisson random variable.

\begin{lemma}\label{lma:Poisson-moment-convergence}
	For an integer random variable \(X_n\), if \(\mathbb{E}_{}[(X_n)_k] \to \lambda ^k\) as \(n \to \infty \) for all \(k = 1, 2, \dots \), then \(X_n \overset{D}{\to} \operatorname{Pois}(\lambda ) \).
\end{lemma}

With \autoref{lma:Poisson-moment-convergence}, the main tool we will be utilized can be proven (omit due to its length):

\begin{theorem}[Stein-Chen method]\label{thm:Stein-Chen-method}
	Let \((A_i)_{i \geq 1}^{n}\) be a sequence of events with \(p_i = \Pr_{}(A_i) \) for all \(i \in [n]\), and let \(X = \sum_{i=1}^{n} \mathbbm{1}_{A_i} \) with \(\lambda = \mathbb{E}_{}[X] = \sum_{i=1}^{n} p_i\). If \((A_i)_{i \geq 1}^{n}\)'s are positively associated, i.e., \((A_i)_{i \neq j} \mid A_j \geq (A_i)_{i \neq j}\) for all \(j\), then,
	\[
		d_{\mathrm{TV} }(X, \operatorname{Pois}(\lambda ) )
		= \frac{1}{2} \sum_{k \geq 0} \left\lvert \Pr_{}(X = k) - e^{-\lambda } \frac{\lambda ^k }{k!}  \right\rvert
		\leq \min ( 1, 1 / \lambda ) \left( \Var_{}[X] - \lambda + 2 \sum_{i=1}^{n} p_i^2 \right).
	\]
	On the other hand, if \((A_i)_{i\geq 1}^{n}\) are negatively associated, then
	\[
		d_{\mathrm{TV} }(X, \operatorname{Pois}(\lambda ) )
		= \frac{1}{2} \sum_{k \geq 0} \left\lvert \Pr_{}(X = k) - e^{-\lambda } \frac{\lambda ^k }{k!}  \right\rvert
		\leq \min ( 1, 1 / \lambda ) \left( \lambda - \Var_{}[X] \right).
	\]
\end{theorem}

\begin{remark}

\end{remark}

\subsection{Cycle Counting}
Consider the \hyperref[prb:cycle-counting]{cycle counting problem} for \(\operatorname{ER}(n, \lambda / n) \) for some \(\lambda > 0\):

\begin{problem}[Cycle counting]\label{prb:cycle-counting}
For some fixed \(k \geq 3\), we're interested in controlling
\[
	X_k
	\coloneqq \sum_{\quotient{\substack{(v_1, \dots , v_k) \in [n] \\ v_j \neq v_{j^{\prime} } \text{ for \(j \neq j^{\prime}\) }}}{\substack{\text{starting point}\\ \text{orientation} }}} \mathbbm{1}_{(v_1, \dots , v_k) \text{ is a cycle} },
\]
where the summation is over all \(n\) distinct vertices modulo the starting one and the orientation.
\end{problem}

\begin{note}
	For the \hyperref[prb:cycle-counting]{cycle counting} problem, it's okay that the \(k\)-cycle has additional edges, i.e., we care about induced subgraph rather than the exact structure.
\end{note}

The following is easy to see.

\begin{lemma}\label{lma:cycle-counting-mean}
	When \(\lambda < 1\), the expected number of cycles is less than \(\sum_{k=3}^{\infty} \lambda ^k / 2k < \infty \). Moreover, the expected number of vertices in a cycle is less than \(\sum_{k=3}^{\infty} \mathbb{E}_{}[k \cdot X_k] \leq \sum_{k=3}^{\infty} \lambda ^k / 2 < \infty \).
\end{lemma}
\begin{proof}
	By a simple counting argument, we see that
	\[
		\mathbb{E}_{}[X_k]
		= \left( \frac{\lambda}{n} \right) ^k \cdot \binom{n}{k} \frac{k!}{2 \cdot k}
		= \frac{\lambda ^k}{n^k} \cdot \frac{n(n-1) \dots (n-k+1)}{2k}
		= \frac{\lambda ^k}{2k} \prod_{i=1}^{k} \left( 1 - \frac{i}{n} \right)
		\approx \frac{\lambda ^k}{2k} e^{- \frac{k (k-1)}{2n}}.
	\]
	Hence, when \(k \ll \sqrt{n} \), we have \(\mathbb{E}_{}[X_k] \lessapprox \lambda ^k / 2k\), which becomes more vacuous as \(k\) increases.
\end{proof}

We can also calculate the variance of \(X_k\). In particular, we see that
\[
	\Var_{}[X_k]
	= \binom{n}{k} \frac{k!}{2k} \cdot \left( \frac{\lambda}{n} \right) ^k \left( 1 - \left( \frac{\lambda}{n} \right) ^k \right) + O\left( \sum_{s=1}^{k-2} \binom{n}{k} \cdot \frac{k!}{2k} \cdot n^{k - s - 1} \cdot \left( \frac{\lambda}{n} \right)^{k + k - s} \right),
\]
where the second term is the covariance, which goes to \(0\) as we desired. From the \hyperref[thm:Stein-Chen-method]{Stein-Chen method}, we can show the following.

\begin{theorem}\label{thm:cycle-counting}
	For \(\lambda > 0\) fixed, \(k \geq 3\) fixed, we have \(X_k \overset{D}{\to} \operatorname{Pois}(\lambda ^k / 2k) \) as \(n \to \infty \). Moreover,
	\begin{itemize}
		\item \((X_k)_{k = 3}^d \overset{D}{\to} \bigotimes_{k=3}^d \operatorname{Pois}(\lambda ^k / 2k) \) as \(n \to \infty \).
		\item \(\sum_{k=3}^{d} X_k \overset{D}{\to} \operatorname{Pois}(\sum_{k=3}^{d} \lambda ^k / 2k) \) for all \(\lambda > 0\).
		\item If \(\lambda < 1\), \(\sum_{k=3}^{\infty} X_k \overset{D}{\to} \operatorname{Pois}(\sum_{k=3}^{\infty} \lambda ^k / 2k) \).
	\end{itemize}
\end{theorem}

\subsection{Tree Counting}
Now, consider the problem of degree counting. Fix \(k \geq 0\), consider
\[
	N_k = \sum_{v=1}^{n} \mathbbm{1}_{\deg (v) = k}.
\]
We see that
\[
	\mathbb{E}_{}[N_k]
	= n \cdot \binom{n}{k} \cdot \left( \frac{\lambda}{n} \right) ^k \left( 1 - \frac{\lambda}{n} \right) ^{n-1-k}
	\approx n \cdot \frac{n^k}{k!} \cdot \frac{\lambda ^k}{n^k} e^{-\lambda }
	\approx n \cdot \frac{\lambda ^k}{k!} e^{-\lambda }.
\]
Hence,
\[
	\frac{1}{n} \mathbb{E}_{}[N_k]
	\to \frac{\lambda ^k}{k!} e^{-\lambda }
\]
as \(n \to \infty \) for all \(k \geq 0\). We also have that
\[
	\Var_{}[N_k]
	= n \left( \frac{\lambda ^k e^{-\lambda }}{k!} \left( 1 - \frac{\lambda ^k e^{-\lambda }}{k!} \right) + o(1) \right) + n(n-1) \left( \Pr_{}(\deg (1) = k, \deg (2) = k) - \Pr_{}(\deg (v) = k)^2 \right)
\]
We see that \(\Pr_{}(\deg (v) = k)^2 = \Pr_{}(\operatorname{Bin}(n-1, p) =k)^2\) and
\[
	\Pr_{}(\deg (1) = k, \deg (2) = k)
	= \frac{\lambda}{n} \Pr_{}(\operatorname{Bin}(n-2, p) = k-1 ) ^2 + \left( 1 - \frac{\lambda}{n} \right) \Pr_{}(\operatorname{Bin}(n-2, p) = k )^2.
\]
Overall, we have \(\Var_{}[N_k] / n \approx c_\lambda / n + \dots \).

\begin{theorem}
	\[
		\left( \frac{N_k - \mathbb{E}_{}[N_k] }{\sqrt{n} } \right) _{k=1}^{\ell }
		\overset{D}{\to} \mathcal{N} _\ell (0, D)
	\]
	where \(D \in \mathbb{R} ^{\ell \times \ell }\) is a positive definite matrix.
\end{theorem}

\begin{remark}
	For the subgraph counting problem, the Poisson approximation holds if and only if \(\Var_{}[Z] / \mathbb{E}_{}[Z] \to 1\) as \(n \to \infty \).
\end{remark}

\begin{eg}
	Using the same idea, we can consider any tree or any cluster of a given finite tree.
\end{eg}

Consider a tree \(T\) that is a \(3\)-chain. Then,
\[
	\mathbb{E}_{}[N_T]
	= n^3 \cdot c_3 \cdot \frac{\lambda ^2}{n^2} \left( 1 - \frac{\lambda}{n} \right) ^{3 (n-3) + 1}
	\approx n \lambda ^2 e^{-3\lambda }.
\]
More generally, given a tree \(T\) of \(k\) nodes,
\[
	\mathbb{E}_{}[N_T]
	= n \lambda ^{k-1} e^{-k \lambda } \cdot \Theta _{k} (1)
	= \exp (\log n - k \lambda + (k-1) \log \lambda ).
\]
We see that if \(\lambda > (1 + \epsilon ) \log n / k\), \(\mathbb{E}_{}[N_k] \to 0\).

If \(\lambda \in (\log n / k, \log n / (k-1))\), no \(T_k\) will appear. In particular, when \(\lambda > \log n\), the graph becomes connected.