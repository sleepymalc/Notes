\lecture{8}{13 Feb.\ 9:30}{Existence Threshold using Chaos Decomposition}
\subsection{Existence Threshold for General Structure}
We now consider using chaos decomposition to compute the subgraph counts of any finite connected graph \(F = (V_F, E_F)\) with \(v_F = \lvert V_F \rvert \) and \(e_F = \lvert E_F \rvert\).

\begin{note}
	We're interested in the \emph{injective} subgraph rather than the induced subgraph.
\end{note}

Let \(X_n(F)\) be the number of copies of \(F\) in a graph \(G = (V, E)\) with \(n = \lvert V \rvert \), which is
\[
	X_n(F)
	= \sum_{\quotient{i_1, \dots , i_{v_F} \text{ distinct}}{\operatorname{Aut}(F)}} \mathbbm{1}_{(i_1, \dots , i_{v_F})\text{ contains edges in } F}.
\]
\begin{remark}[\(F\)-density]
	Clearly, the maximum value of \(X_n(F)\) is \((n)_{v_F} / \lvert \operatorname{Aut}(F)  \rvert \) when \(G\) is a complete graph. Hence, we can define the density of \(F\) (called \emph{\(F\)-density}) in \(G\) as
	\[
		t(F, G)
		\coloneqq \frac{X_n(F)}{(n)_{v_F} / \lvert \operatorname{Aut}(F)  \rvert }.
	\]
\end{remark}

If \(G \sim \operatorname{ER}(n, p) \), then the expectation of \(X_n(F)\) is
\[
	\mathbb{E}_{}[X_n(F)]
	= \frac{(n)_{v_F}}{\lvert \operatorname{Aut}(F)  \rvert } p^{e_F}
	\approx n^{v_F} p^{e_F}.
\]
We see that to decide the existence threshold for the first moment, we want to see \(n^{v_F} p^{e_F} \to 0\), which happens if and only if \(p \ll 1 / n^{v_F / e_F}\). Hence, the existence threshold of \(F\) is \(p = 1 / n^{v_F / e_F}\).

\begin{eg}
	If \(F\) is a tree, \(v_F / e_F = v_F / (v_F - 1) = 1 + 1 / (v_F - 1) > 1\), hence \(1 / n^{v_F / e_F} \ll 1 / n\). This implies that trees always exist in all regimes when \(p \cdot n\) is bounded away from \(0\).
\end{eg}

\begin{eg}
	If \(F\) is connected but not a tree, \(v_F / e_F \leq v_F / v_F = 1\), hence \(1 / n^{v_F / e_F} \geq 1 / n\). This means that the regime must be super-critical for non-tree components to exist with positive probability.
\end{eg}

To get concentration of \(X_n\), we need to compute the variance. First, we have
\[
	\hat{X} _n(F)
	\coloneqq X_n(F) - \mathbb{E}_{}[X_n(F)]
	= \sum_{\varnothing \neq H \subseteq F} n^{v_F - v_H} \cdot p^{e_F - e_H} \cdot c_n(H, F) \cdot \hat{X} _n(H),
\]
where \(c_n(H, F)\) is the number of copies of \(H\) in \(F\).

\begin{note}
	We can interpret that \(\hat{X} _n(\cdot)\) being the version of \(X_n(\cdot)\) where all edge variables \(\omega \) is replaced by its centered version \(\omega - p\).
\end{note}

From the chaos decomposition, we have independence, and hence
\[
	\Var_{}[X_n(F)]
	= \sum_{\varnothing \neq H \subseteq F} c_n(H, F)^2 n^{2 (v_F - v_H)} p^{2(e_F - e_H)} n^{v_H} p^{e_H} (1 - p)^{e_H}
	\approx \sum_{_{\varnothing \neq H \subseteq F}} n^{2 v_F - v_H} p^{2 e_F - e_H} (1 - p)^{e_H}.
\]
For \(p\) bounded away from \(1\), we have \((1 - p)^{e_H} \approx 1\). Hence, the ratio of the variance to the square of the mean is
\[
	\frac{\Var_{}[X_n(F)] }{(\mathbb{E}_{}[X_n(F)] )^2}
	\approx \sum_{\varnothing \neq H \subseteq F} \frac{1}{n^{v_H} p^{e_H}}.
\]
From Chebyshev's inequality, for any \(\epsilon > 0\), we have
\[
	\Pr_{}\left(\left\lvert \frac{X_n(F)}{\mathbb{E}_{}[X_n(F)] } - 1 \right\rvert > \epsilon \right)
	= \Pr_{}\left(\lvert X_n(F) - \mathbb{E}_{}[X_n(F)] \rvert > \epsilon \mathbb{E}_{}[X_n(F)] \right)
	\leq \frac{\Var_{}[X_n(F)] }{\epsilon ^2 (\mathbb{E}_{}[X_n(F)] )^2}.
\]
Combining the above, if we want concentration for \(X_n(F)\) to \(\mathbb{E}_{}[X_n(F)]\), we need \(n^{v_H} p_{e_H} \to \infty\) for all \(\varnothing \neq H \subseteq F\). Equivalently,
\[
	n \cdot \min _{\varnothing \neq H \subseteq F} p^{e_H / v_H}
	= n \cdot p^{\max _{\varnothing \neq H \subseteq F} e_H / v_H}
	\to \infty.
\]

\begin{notation}
	Define the \emph{concentration parameter} \(\theta (F) \coloneqq \max _{\varnothing \neq H \subseteq F} e_H / v_H \geq e_F / v_F\).
\end{notation}

We can then rewrite the condition for concentration as \(n p^{\theta (F)} \to \infty \), or equivalently, \(p \gg 1 / n^{1 / \theta (F)}\). Hence, we conclude that the concentration threshold of \(X_n(F)\) is \(p = 1 / n^{1 / \theta (F)}\).

\begin{remark}[Balanced subgraph]
	If the above maximum is taken when \(H = F\) with \(\theta _F = e_F / v_F\), then \(1 / n^{v_F / e_F} = 1 / n^{1 / \theta (F)}\), i.e., the concentration threshold and existence threshold are the same. In this case, we say \(F\) is balanced.

	However, for an unbalanced \(F\), then there is a gap between the existence threshold and concentration threshold. We need to look at the maximum \(H \subsetneq F\) in this case.
\end{remark}

If we now consider the contribution to the variance from edges (i.e., \(H\) with \(v_H=2\) and \(e_H=1\)), that is, \(n^{2v_F - 2} p^{2e_F - 1} (1 - p) = O(n^{2v_F - 2} p^{2e_F - 1})\), we see that its contribution to \(\Var_{}[X_n(F)] \) is of order
\[
	\frac{\Var_{}[X_n(F)] }{n^{2v_F - 2} p^{2e_F - 1}}
	\approx \Theta (1) + \sum_{H \subseteq F\colon e_H > 1} n^{2 - v_H} p^{1 - e_H}.
\]
We see that the term from edges will dominate \(\Var_{}[X_n(F)]\) when \(n^{v_H - 2} p^{e_H - 1} \to \infty \) for all \(H \subseteq F\) with \(e_H > 1\), or equivalently, as \(n \to \infty \),
\[
	n \cdot \min _{H \subseteq F\colon e_H > 1} p^{\frac{e_H - 1}{v_H - 2}}
	\to \infty.
\]

\begin{notation}
	Define the \emph{edge-dominant parameter} \(\theta _1(F) \coloneqq \max _{H \subseteq F \colon e_H > 1} \frac{e_H - 1}{v_H - 2} \geq \theta (F)\).
\end{notation}

Now, recall that \(	\mathbb{E}_{}[X_n(F)] \approx n^{v_F} p^{e_F}\), hence as the edge term is dominating in \(\Var_{}[X_n(F)] \), we have
\[
	\Var_{}\left[\frac{X_n(F)}{\mathbb{E}_{}[X_n(F)] }\right]
	\leq \frac{c_F \cdot n^{2v_F - 2} p^{2e_F - 1}}{n^{2v_F p^{2e_F}}}
	\leq \frac{c_F}{n^{2} p}
	\to 0
\]
for some constant \(c_F\) as \(n \to \infty \). Now, observe that for a fixed \(p \in (0, 1)\), \(\mathbb{E}_{}[X_n(F)] \approx n^{v_F}\). Hence, we conclude that the density of \(F\) converges in probability to \(p^{e_F}\) as \(n \to \infty \) for all \(F\), i.e., as \(n \to \infty \),
\[
	\frac{X_n(F)}{\mathbb{E}_{}[X_n(F)]}
	\overset{p}{\to} p^{e_F}.
\]

\section{Dense Graph Limit}
We digress a bit to talk about a limit of the dense graph sequence. The motivation is that consider a sequence of dense graphs \(G_n\) with increasing number of vertices, and we want to have a unified way to define the limiting object for such a sequence. \hyperref[def:graphon]{Graphon} normalized vertex set to \([0, 1]\) and define a function \(W(x, y)\) on \([0, 1]^2\) that generalizes the adjacency matrix structure.

\begin{definition}[Graphon]\label{def:graphon}
	A \emph{graphon} is a symmetric measurable function \(W \colon [0, 1]^2 \to [0, 1]\), i.e., \(W(x, y) = W(y, x)\) for all \(x, y \in [0, 1]\).
\end{definition}

We see that we can embed any fixed, simple graph \(F\) as a \hyperref[def:graphon]{graphon} \(W_F\).

\begin{eg}
	For a finite simple graph \(G = (V, E)\) with \(V = [n]\), its associated graphon \(W_G(x, y)\) can be defined as
	\[
		W_G(x, y)
		\coloneqq \begin{dcases}
			0, & \text{ if \(x, y\) belongs to a region representing an edge} ; \\
			1, & \text{ otherwise} .
		\end{dcases}
	\]
	That is, we partition \([0, 1]\) into \(n\) equal-length intervals \(I_i = [(i-1) / n, i / n]\) for \(i \in [n]\), and define \(W_G(x, y) = 1\) if and only if \((i, j) \in E\) for \(x \in I_i\) and \(y \in I_j\).
\end{eg}

\begin{eg}[Constant graphon]
	An important \hyperref[def:graphon]{graphon} is the constant \hyperref[def:graphon]{graphon} \(W(x, y) = p\cdot \mathbbm{1}_{[0, 1]^2} \) with some \(p \in (0, 1)\).
\end{eg}

Intuitively, a constant \hyperref[def:graphon]{graphon} should correspond to a uniform random graph where each edge is present independently with probability \(p\), i.e., \(\operatorname{ER}(n, p) \). We will see this later.

\subsection{Space of Graphon}
Since this embedding is not unique due to various vertex relabeling, we define the space of \hyperref[def:graphon]{graphons} as follows.

\begin{definition}[Graphon space]\label{def:graphon-space}
	The \emph{graphon space}, denoted as \(\widetilde{\mathcal{W} } \), is defined as
	\[
		\widetilde{\mathcal{W} }
		\coloneqq \quotient{\{ W\colon [0, 1]^2 \to [0, 1] \mid W \text{ is a \hyperref[def:graphon]{graphon}} \}}{\sim } ,
	\]
	where the equivalence reaction \(W \sim W^{\prime} \) holds if there exists a measure-preserving bijection \(\varphi \colon [0, 1] \to [0, 1]\) such that \(W^{\prime} (x, y) = W(\varphi (x), \varphi (y))\) for almost all \((x, y) \in [0, 1]^2\).
\end{definition}

Now, to introduce the distance over the \hyperref[def:graphon-space]{graphon space}, we introduce the following first.

\begin{definition}[Cut norm]\label{def:cut-norm}
	The \emph{cut norm} of a \hyperref[def:graphon]{graphon} \(W\) is defined as
	\[
		\lVert W \rVert _{\square }
		= \sup _{S, T \subseteq [0, 1]} \left\lvert \int _{S \times T} W(x, y) \,\mathrm{d} x \,\mathrm{d} y \right\rvert .
	\]
\end{definition}

It's probably trivial to see that in \autoref{def:cut-norm}, the definition is a bit redundant since we only care about \hyperref[def:graphon]{graphons}, which is non-negative, i.e., we always have
\[
	\lVert W \rVert _{\square }
	= \int_{[0, 1]^2} W(x, y) \,\mathrm{d}x \,\mathrm{d} y.
\]
However, what we really care is the ``metric'' induced by this norm:

\begin{definition}[Cut metric]\label{def:cut-metric}
	The \emph{cut metric} between two \hyperref[def:graphon]{graphons} \(\widetilde{W} , \widetilde{Y} \in \widetilde{\mathcal{W} } \) is defined as
	\[
		d_{\square}(\widetilde{W} , \widetilde{Y} ) = \inf _{\varphi \colon [0, 1] \to [0, 1]} \lVert W - Y \circ \varphi \rVert _{\square},
	\]
	where the infimum is taken over all measure-preserving bijections \(\varphi \) and \(W\) and \(Y\) are any representation \hyperref[def:graphon]{graphon} of \(\widetilde{W} \) and \(\widetilde{Y} \), respectively.
\end{definition}

\begin{remark}
	It's also possible to consider defining \hyperref[def:graphon-space]{graphon space} by first defining the \hyperref[def:cut-metric]{cut metric} and identify two \hyperref[def:graphon]{graphons} \(W_1, W_2\) to be the same when \(d_{\square }(W_1 , W_2) = 0\).
\end{remark}

\subsection{Homomorphism Density}
If \(W\) is a \hyperref[def:graphon]{graphon}, we can define the \emph{\(F\)-density} (or \emph{homomorphism density}) in \(W\) similarly as in the usual graph \(G\):
\[
	t(F, W)
	\coloneqq \int_{[0, 1]^{v_F}} \prod_{(i, j) \in E(F)} W(x_i, x_j) \,\mathrm{d} x_i \cdots \,\mathrm{d} x_{v_F},
\]
which is the continuous version of the \(F\)-density in a graph.

\begin{intuition}
	This again measures how frequently a given finite simple graph \(F\) appears as a subgraph within a \hyperref[def:graphon]{graphon} \(W\).
\end{intuition}

It turns out that the \(F\)-density characterizes a \hyperref[def:graphon]{graphon} exactly:

\begin{theorem}[Graphon convergence]\label{thm:graphon-convergence}
	Given a \hyperref[def:graphon]{graphon} sequence \((W_n) \in \widetilde{\mathcal{W}} \), \(d_{\square}(W_n, W) \to 0\) as \(n \to \infty \) if and only if \(t(F, W_n) \to t(F, W)\) for all finite connected graph \(F\).
\end{theorem}

\begin{prev}
	For a fixed \(p \in (0, 1)\), \(X_n(F) / \mathbb{E}_{}[X_n(F)] \overset{p}{\to} p^{e_F}\) in \(G_n \sim \operatorname{ER}(n, p) \).
\end{prev}

The above is equivalent to \(t(F, G_n) \overset{p}{\to} p^{e_F}\). Now, let \(W \equiv p\), a constant function, then \(t(F, W) = p^{e_F}\) as well. Then, from \autoref{thm:graphon-convergence}, \(d_{\square} (G_n, p \cdot \mathbbm{1}_{[0, 1]^2} ) \overset{p}{\to} 0\) as \(n \to \infty \).