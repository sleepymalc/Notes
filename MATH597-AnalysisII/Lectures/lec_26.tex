\lecture{26}{16 Mar.\ 11:00}{Bounded Linear Transformations}
We now prove \autoref{thm:lec-25}, completing the proof of \autoref{thm:Riesz-Fischer-theorem} since the latter relies on this result.

\begin{proof}[Proof of \autoref{thm:lec-25}]\label{pf:thm:lec25}
	We prove it by proving two directions.
	\paragraph{\((\implies)\)} Suppose \(V\) is \hyperref[def:complete]{complete}, and fix an absolutely convergent series \(\sum_n v_n\). Define \(s_N = \sum_{n=1}^N v_n\).
	It suffices to show the partial sums are a \hyperref[def:Cauchy-sequence]{Cauchy Sequence}.

	Fix \(\epsilon > 0\), then because \(\sum_{n=1}^\infty \left\lVert v_n\right\rVert  < \infty\), there is a \(K \in \mathbb{N}\) so that
	\[
		\sum_{n=K}^\infty \left\lVert v_n\right\rVert  < \epsilon.
	\]

	Now let \(M > N > K\), we see that
	\[
		\left\lVert s_M - s_N\right\rVert = \left\lVert \sum_{n = N + 1}^M v_n\right\rVert \leq \sum_{n=N+1}^M \left\lVert v_n\right\rVert  \leq \sum_{n=N}^\infty \left\lVert v_n\right\rVert  < \epsilon,
	\]
	so this is \hyperref[def:Cauchy-sequence]{Cauchy}.

	\paragraph{\((\impliedby)\)} Now suppose \(v_n, n \in \mathbb{N}\) is a \hyperref[def:Cauchy-sequence]{Cauchy sequence}. For all \(j \in \mathbb{N}\),
	there exists an \(N_j \in \mathbb{N}\) such that
	\[
		\left\lVert v_n - v_m\right\rVert  < \frac{1}{2^j}
	\]
	for all \(n, m \geq N_j\). Without loss of generality, we may assume \(N_1 < N_2 < \dots \).

	Let \(w_1 = v_{N_1}\), \(w_j = v_{N_j} - v_{N_j - 1}\) for \(j \geq 2\). Therefore,
	\[
		\sum_{j=1}^\infty \left\lVert w_j\right\rVert \leq \left\lVert v_{N_1}\right\rVert + \sum_{j=2}^\infty \frac{1}{2^{j-1}} < \infty.
	\]
	Thus, \(\sum_{j=1}^k w_j \to s \in V\) as \(k \to \infty\). But by telescoping, we have
	\[
		v_{N_k} = \sum_{j=1}^k w_j \to s.
	\]

	Now we claim that since \(v_n\) is \hyperref[def:Cauchy-sequence]{Cauchy}, so \(v_n \to s\).

	Explicitly, take \(\epsilon > 0\), and let \(k\) be large enough so that \(\left\lVert v_{N_k} - s\right\rVert < \epsilon\) and \(1/2^k < \epsilon\). Then if \(n > N_k\) then
	\[
		\left\lVert v_n - s\right\rVert \leq \left\lVert v_n - v_{N_k}\right\rVert + \left\lVert v_{N_k} - s\right\rVert < \epsilon + \epsilon = 2\epsilon.
	\]
	Thus, \(v_n \to s\).
\end{proof}

\section{Bounded Linear Transformations}
\begin{definition}[Bounded linear transformation]\label{def:bounded-linear-transformation}
	Given two \hyperref[def:norm]{normed} vector spaces \((V,\left\lVert \cdot\right\rVert )\), \((W,\left\lVert \cdot\right\rVert^\prime )\),
	a linear map \(T \colon V \to W\) is called a \emph{bounded map} if there exists \(c \geq 0\) such that
	\[
		\left\lVert Tv\right\rVert^\prime \leq c\left\lVert v\right\rVert
	\]
	for all \(v \in V\).
\end{definition}

\begin{proposition}
	Suppose \(T \colon (V, \left\lVert \cdot\right\rVert) \to (W, \left\lVert \cdot\right\rVert^\prime )\) is a linear map. Then the following are equivalent.
	\begin{enumerate}[(a)]
		\item \(T\) is continuous.
		\item \(T\) is continuous at \(0\).
		\item \(T\) is a \hyperref[def:bounded-linear-transformation]{bounded map}.
	\end{enumerate}
\end{proposition}
\begin{proof}
	\((1)\implies(2)\) is clear.
	\begin{claim}
		\((2)\implies(3)\).
	\end{claim}
	\begin{explanation}
		Take \(\epsilon = 1\), then there exists a \(\delta > 0\) such that \(\left\lVert Tu\right\rVert^\prime < 1\)
		if \(\left\lVert u\right\rVert < \delta\).

		Now take an arbitrary \(\left\lVert v \right\rVert\in V\), \(v \neq 0\). Let \(u = \frac{\delta}{2\left\lVert v\right\rVert}v\). Then \(\left\lVert u\right\rVert < \delta\).
		Therefore,
		\[
			\left\lVert Tu\right\rVert ^\prime <1 \implies \frac{\delta }{2\left\lVert v\right\rVert }\left\lVert Tv\right\rVert ^\prime < 1\implies \left\lVert Tv\right\rVert ^\prime < \frac{2}{\delta }\left\lVert v\right\rVert .
		\]
		Then \(2/\delta\) is our constant.
	\end{explanation}
	\begin{claim}
		\((3)\implies (1)\).
	\end{claim}
	\begin{explanation}
		Fix \(v_0 \in V\). Then for some constant \(c\)
		\[
			\left\lVert Tv - Tv_0\right\rVert^\prime = \left\lVert T(v - v_0)\right\rVert ^\prime \leq c\left\lVert v - v_0\right\rVert.
		\]
		Thus, \(T\) is continuous, as when \(v \to v_0\) the right-hand side goes to zero, and so \(Tv \to Tv_0\).
	\end{explanation}
\end{proof}

\begin{eg}
	Let's see some examples.
	\begin{enumerate}[(a)]
		\item We can look at
		      \[
			      \begin{split}
				      T \colon \ell^1 & \to \ell^1               \\
				      (a_1,a_2,\dots) & \mapsto (a_2,a_3,\dots).
			      \end{split}
		      \]
		      Then clearly \(\left\lVert Ta\right\rVert_1 \leq \left\lVert a\right\rVert _1\), so \(T\) is a \hyperref[def:bounded-linear-transformation]{bounded linear transformation}.
		\item We can also look at \(S \colon (C([-1,1]),\left\lVert \cdot\right\rVert _1) \to \mathbb{C}\), where \(Sf = f(0)\). \(S\) is not a \hyperref[def:bounded-linear-transformation]{bounded linear transformation},
		      because we can make
		      \[
			      \begin{dcases}
				      \left\lVert Sf\right\rVert  & = \left\vert f(0) \right\vert  = n \\
				      \left\lVert f\right\rVert_1 & = 1
			      \end{dcases}
		      \]
		      for every \(n \in \mathbb{N}\) (take \(f\)'s graph to be a skinny triangle shooting up to \(n\) at \(0\)).
		\item But \(U \colon (C([-1,1]), \left\lVert \cdot\right\rVert _\infty) \to \mathbb{C}\) defined by \(Uf = f(0)\) is a \hyperref[def:bounded-linear-transformation]{bounded linear transformation},
		      because \(\left\vert f(0) \right\vert \leq \left\lVert f\right\rVert_\infty\).
		\item Let \(A\) be an \(n \times m\) matrix. Then \(T \colon \mathbb{R}^m \to \mathbb{R}^n\) defined by \(v \mapsto Av\) is a
		      \hyperref[def:bounded-linear-transformation]{bounded linear transformation}.

		      Explicitly this is
		      \[
			      (Tv)_i = (Av)_i = \sum_{j=1}^m A_{ij}v_j.
		      \]
		\item Let \(K(x,y)\) be a continuous function on \([0,1] \times [0,1]\). We'll define
		      \[
			      T \colon (C[0,1],\left\lVert \cdot\right\rVert _\infty) \to (C[0,1], \left\lVert \cdot\right\rVert _\infty)
		      \]
		      by
		      \[
			      (Tf)(x) = \int_0^1 K(x,y)f(y) \,\mathrm{d} y.
		      \]
		      This is an analogue of matrix multiplication (\(K\) is like a continuous matrix). This is a \hyperref[def:bounded-linear-transformation]{bounded linear transformation}.
		\item Let us look at \(T \colon L^1(\mathbb{R}) \to (C(\mathbb{R}),\left\lVert \cdot\right\rVert _\infty)\) defined by
		      \[
			      (Tf)(t) = \int_{-\infty}^\infty e^{-itx}f(x) \,\mathrm{d} x
		      \]
		      that is the Fourier transform of \(f\).
		\item \(T \colon (C^\infty[0,1],\left\lVert \cdot\right\rVert _\infty) \to (C^\infty[0,1], \left\lVert \cdot\right\rVert _\infty)\). Define
		      \[
			      (Tf)(x) = f^\prime (x).
		      \]
		      This is \underline{not} a \hyperref[def:bounded-linear-transformation]{bounded linear transformation}. In contrast, \(S\), defined on the same spaces
		      \[
			      (Sf)(x) = \int_0^x f(t) \,\mathrm{d} t
		      \]
		      is bounded.
	\end{enumerate}
\end{eg}

\begin{definition}[Operator norm]\label{def:operator-norm}
	Let \(L(V, W)\) be defined as a vector space such that
	\[
		L(V,W) \coloneqq \{T \colon V \to W \mid T \text{ is a \hyperref[def:bounded-linear-transformation]{bounded linear transformation}}\}.
	\]

	Then for \(T \in L(V,W)\), the \emph{operator norm} of \(T\) is
	\[
		\begin{split}
			\left\lVert T\right\rVert & \coloneqq \inf\{c \geq 0 \mid \left\lVert Tv\right\rVert^{\prime\prime}  \leq c\left\lVert v\right\rVert^\prime  \text{ for all } v \in V\} \\
			                          & = \sup\left\{ \frac{\left\lVert Tv\right\rVert^{\prime\prime}}{\left\lVert v\right\rVert^\prime } \mid v \neq 0, v \in V\right\}            \\
			                          & = \sup\left\{ \left\lVert Tv\right\rVert^{\prime\prime} \mid \left\lVert v\right\rVert^\prime  = 1, v \in V\right\}.
		\end{split}
	\]
\end{definition}

\begin{lemma}
	We have that
	\begin{enumerate}[(a)]
		\item The \hyperref[def:operator-norm]{three definitions} of \(\left\lVert T\right\rVert\) above are all equal.
		\item \((L(V,W), \left\lVert \cdot\right\rVert )\) is indeed a \hyperref[def:norm]{normed} space.
	\end{enumerate}
\end{lemma}
\begin{proof}
	\todo{DIY}
\end{proof}