\lecture{32}{30 Mar. 11:00}{Lebesgue Differentiation Theorem for Regular Borel Measures on \(\mathbb{\MakeUppercase{r}} ^d\)}
\begin{eg}[Lebesgue-Radon-Nikodym]
	Let \(\mu = m\), \(\nu = \mu_F\) (\hyperref[def:Lebesgue-Stieltjes-measure]{Lebesgue-Stieltjes measure} for \(F\)). We'll define \(F(x)\) by
	\[
		F(x) = \begin{dcases}
			e^{3x}, & \text{ if } x\leq 0 ;   \\
			1,      & \text{ if } 0 < x < 1 ; \\
			5,      & \text{ if } x \geq 1 .
		\end{dcases}
	\]
	Then we will have that
	\[
		\mu_F(E) = \int_{E \cap \mathbb{R}_{< 0}} 3e^{3x} \,\mathrm{d}x + 4 \delta_1(E).
	\]
	It is enough to check on \((-\infty,x]\) because these are \hyperref[def:locally-finite]{locally finite} \hyperref[def:Borel-measure]{Borel measures} on \(\mathbb{R}\).

	Then we have \(\mu_F = \,\mathrm{d}\rho + \,\mathrm{d}\lambda = f\,\mathrm{d}m + \,\mathrm{d}\lambda\) where \(f = 1_{\mathbb{R}_{< 0}}3e^{3x}\) and \(\lambda = 4\delta_1\),
	\(\lambda \perp m\).
\end{eg}

\emph{Read}: Theorem 3.5, Proposition 3.9, Corollary 3.10 of section 3.2 of \cite{folland1999real}.

\emph{Skip}: Complex measures (section 3.3) of \cite{folland1999real}.

\begin{prev}
	If \(\nu = \nu^+ - \nu^-\), we defined the \hyperref[def:total-variation]{total variation} \(\left\vert \nu \right\vert = \nu^+ + \nu^-\).

	Then we have \(\left\vert \nu(E) \right\vert \leq \left\vert \nu \right\vert(E)\).
\end{prev}

\subsection{Lebesgue Differentiation Theorem for Regular Borel Measures on \(\mathbb{R} ^d\)}
See page 99 of \cite{folland1999real}.
\begin{definition}[Regular]\label{def:regular}
	A Borel \hyperref[def:signed-measure]{signed measure} \(\nu\) on \(\mathbb{R}^d\) is called \emph{regular} if
	\begin{enumerate}
		\item\label{def:regular-compact-finite} (compact finite) \(\left\vert \nu \right\vert(K) < \infty\) for all compact \(K\).
		\item\label{def:regular-outer-regularity} (outer regularity) We have \hyperref[thm:regularity]{outer regularity}
		\[
			\left\vert \nu \right\vert(E) = \inf\{\left\vert \nu \right\vert(U) \mid \text{ open } U \supseteq E\}
		\]
		for every \hyperref[def:Borel-set]{Borel set} \(E\).
	\end{enumerate}
\end{definition}

\begin{eg}
	Any \hyperref[def:Lebesgue-Stieltjes-measure]{Lebesgue-Stieltjes measure} on \(\mathbb{R}\) has this property (see \autoref{thm:regularity}),
	so is the difference between two of them (at least if one of them ii \hyperref[def:finite-signed-measure]{finite}).

	The \hyperref[def:Lebesgue-measure]{Lebesgue measure} on \(\mathbb{R}^d\) is \hyperref[def:regular]{regular}.
\end{eg}

\begin{note}
	From \hyperref[def:regular-compact-finite]{compact finiteness}, if \(\nu\) is \hyperref[def:regular]{regular} then \(\nu\) is \hyperref[def:finite-signed-measure]{\(\sigma\)-finite}. Also,
	if \(\,\mathrm{d}\nu = f\,\mathrm{d}m\) is \hyperref[def:regular]{regular}, then
	\[
		\left\vert \nu \right\vert(K) = \int_K \left\vert f \right\vert \,\mathrm{d}m < \infty
	\]
	for all compact \(K\). Thus, \(f \in L^1_{\mathrm{loc}}(\mathbb{R}^d)\).
\end{note}

\begin{lemma}
	\(f \in L^1_{\mathrm{loc}}(\mathbb{R}^d)\) if and only if \(\,\mathrm{d}\nu = f \,\mathrm{d}m\) is \hyperref[def:regular]{regular}.
\end{lemma}
\begin{proof}
	\todo{DIY}
\end{proof}

\begin{prev}
	Recall the \hyperref[thm:Lebesgue-differentiation-theorem]{Lebesgue differentiation theorem}, here we had that if \(f \in L^1_{\mathrm{loc}}(\mathbb{R}^d)\)
	implies that for Lebesgue \hyperref[def:mu-almost-everywhere]{almost every} \(x\),
	\[
		\lim_{r \to 0} \frac{1}{m(E_r)} \int_{E_r} f(y) \,\mathrm{d}y = f(x)
	\]
	for any \(E_r\) \hyperref[def:shrink-nicely]{shrinks nicely} to \(x\) (think of \(B_r(x)\)).
\end{prev}

\begin{corollary}
	Let \(\rho\) be a \hyperref[def:regular]{regular} \hyperref[def:signed-measure]{signed Borel measure} on \(\mathbb{R}^d\). Suppose \(\rho \ll m\). Then
	\(\,\mathrm{d}\rho = f\,\mathrm{d}m\) for some \(f \in L^1_{\mathrm{loc}}(\mathbb{R}^d)\), So then for Lebesgue \hyperref[def:mu-almost-everywhere]{almost every} \(x\)
	we have
	\[
		\lim_{r \to 0} \frac{1}{m(E_r)} \int_{E_r} f(y) \,\mathrm{d}y = f(x).
	\]
	Writing this nicely, using established notation, this is
	\[
		\lim_{r \to 0} \frac{\rho(E_r)}{m(E_r)} = \frac{\,\mathrm{d}\rho}{\,\mathrm{d}m}(x)
	\]
	for every \(E_r\) \hyperref[def:shrink-nicely]{shrinks nicely} to \(x\).
\end{corollary}

\begin{proposition}
	Let \(\lambda\) be a \hyperref[def:regular]{regular} \hyperref[def:signed-measure]{positive Borel measure} on \(\mathbb{R}^d\). Suppose \(\lambda \perp m\).

	For Lebesgue \hyperref[def:mu-almost-everywhere]{almost every} \(x\), we have
	\[
		\lim_{r \to 0} \frac{\lambda(E_r)}{m(E_r)} = 0
	\]
	for every \(E_r\) \hyperref[def:shrink-nicely]{shrinking to \(x\) nicely} (equivalently, \hyperref[def:shrink-nicely]{shrinking to \(0\) nicely}).
\end{proposition}

\begin{proof}
	It is enough to consider \(E_r = B(x,r)\). We wish to prove that
	\[
		G \coloneqq \left\{x \mid \limsup_{r \to 0} \frac{\lambda(E_r)}{m(E_r)} \neq 0\right\}  = \bigcup_{n = 1}^\infty G_n
	\]
	where
	\[
		G_n \coloneqq \left\{x \mid \limsup_{r \to 0} \frac{\lambda(E_r)}{m(E_r)} > \frac{1}{n}\right\}.
	\]
	It is enough to show \(m(G_n) = 0\) for all \(n\).

	\(\lambda \perp m\), so we know \(\mathbb{R}^d = A \cup B\) disjoint, \(\lambda(A) = 0\), \(m(B) = 0\). Thus, it suffices to show \(m(G_n \cap A) = 0\).

	Fix \(\epsilon> 0\), since \(\lambda\) is \hyperref[def:regular]{regular}, there exists an open set \(U \supseteq A\) such that \(\lambda(U) \leq \lambda(A) + \epsilon= \epsilon \).

	For every \(x \in G_n \cap A\), there is an \(r_x > 0\) such that \(\lambda(B(x,r_x))/m(B(x,r_x)) > 1/n\) and \(B(x,r_x) \subseteq U\).

	Let \(K \subseteq G_n \cap A\), compact. Then \(K \subseteq \bigcup_{x \in K} B(x,r_x)\). By compactness, we can take a finite sub-cover, and then use
	\autoref{lma:Vitali-type-covering-lemma} to find \(B_1,B_2,\ldots,B_N\) disjoint each of type \(B(x,r_x)\) such that \(K \subseteq \bigcup_i 3B_i\).

	Therefore,
	\[
		m(K) \leq 3^d \sum_{i=1}^N m(B_i) \leq 3^dn\sum_{i=1}^N \lambda(B_i) = 3^d n\lambda\left( \bigcup_i B_i \right) \leq 3^dn\lambda(U) = 3^dn \epsilon .
	\]
	By inner regularity, \(m(G_n \cap A) \leq 3^d n\epsilon\) for any \(\epsilon> 0\). Taking \(\epsilon\to 0\) yields \(m(G_n \cap A) = 0\), so then \(m(G_n) = 0\) as desired.
\end{proof}
