\lecture{35}{6 Apr. 11:00}{Continue on Functions of Bounded Variation}
\begin{definition}[Normalized bounded variation]\label{def:normalized-bounded-variation}
	A function \(G \in BV\) is said to have \emph{normalized bounded variation}, denoted as \(G \in NBV\) provided that \(G\) is right continuous and \(G(-\infty) = 0\).
\end{definition}

\begin{eg}
	If \(F\) is increasing and bounded, \(F\) right continuous, \(F(-\infty) = 0\).

	\(F(x) = \int_{-\infty}^x f(t) \,\mathrm{d}t, f \in L^1(\mathbb{R})\). Midterm gave \(F\) is uniformly continuous.
\end{eg}

\begin{lemma}
	If \(F \in BV\) is right continuous, then \(T_F \in NBV\).
\end{lemma}
\begin{proof}
	\(T_F\) is bounded, increasing, and satisfies \(T_F(-\infty) = 0\) by \autoref{lma:lec34-1}. Thus, \(T_F \in BV\).

	Hence, we just need to check that \(T_F\) is right continuous. Suppose not, then there is a point \(a \in \mathbb{R}\) such that \(c \coloneqq T_F(a^+) - T_F(a) > 0\).

	Fix \(\epsilon > 0\), since \(F(x)\) and \(g(x) \coloneqq T_F(x^+)\) are right-continuous, there exists a \(\delta > 0\) such that for \(y \in (a,a+\delta]\) we have
	\[
		\left\vert F(y) - F(a) \right\vert - \epsilon ,\quad \left\vert g(y) - g(a) \right\vert < \epsilon .
	\]
	We then have that
	\[
		T_F(y) - T_F(a^+) \leq T_F(y^+) - T_F(a^+) < \epsilon .
	\]
	There exist \(a = x_0 < x_1 < \cdots < x_n = a + \delta\) such that
	\[
		\sum_{i=1}^n \abs{F(x_i) - F(x_{i-1})}  \geq T_F(a + \delta) - T_F(a) - \frac{c}{4} \geq T_F(a^+) - T_F(a) - \frac{c}{4} = \frac{3c}{4}.
	\]
	Then \(\left\vert F(x_1) - F(a) \right\vert < \epsilon \) so we have
	\[
		\sum_{i=2}^n \abs{F(x_i) - F(x_{i-1})} \geq \frac{3}{4} - \epsilon.
	\]
	There exist \(a = t_0 < \cdots < t_k = x_1\) such that
	\[
		\sum_{i=1}^k \left\vert F(t_i) - F(t_{i-1}) \right\vert \geq T_F(x_1) - T_F(a) - \frac{c}{4} \geq \frac{3}{4}c.
	\]
	Then as \([a,a+\delta] = [a,x_1] \cup [x_1,a+\delta]\) we see that
	\[
		T_F(a + \delta) - T_F(a) \geq \sum_{j=1}^k \left\vert F(t_j) - F(t_{j-1}) \right\vert + \sum_{i=2}^n \left\vert F(x_i) - F(x_{i-1}) \right\vert \geq \frac{3}{4}c - \epsilon + \frac{3}{4}c = \frac{3}{2}c - \epsilon.
	\]
	Thus
	\[
		\epsilon + c  \geq T_F(a + \delta) - T_F(a^+) + T_F(a^+) - T_F(a)  = T_F(a + \delta) - T_F(a) \geq \frac{3}{2}c - \epsilon
	\]
	and
	\[
		c \leq 4 \epsilon.
	\]
	Thus taking \(\epsilon \to 0\) yields \(c = 0\), which is a contradiction.
\end{proof}

\begin{corollary}
	\(F \in NBV\) if and only if \(F = F_1 - F_2\), \(F_1,F_2 \in NBV\) and increasing.
\end{corollary}
\begin{proof}
	\(F = (T_F + F)/2 - (T_F - F)/2\).
\end{proof}

\begin{theorem}\label{thm:nbv-measures}
	We have that
	\begin{enumerate}[(1)]
		\item Suppose that \(\mu\) is a \hyperref[def:finite-signed-measure]{finite signed \hyperref[def:Borel-measure]{Borel measure}}
		      on \(\mathbb{R}\), then \(F(x) = \mu((-\infty, x]) \in NBV\).
		\item \(F \in NBV\) implies that there exists a unique \hyperref[def:finite-signed-measure]{finite signed \hyperref[def:Borel-measure]{Borel measure}}
		      on \(\mathbb{R}\) satisfying \(\mu_F((-\infty,x]) = F(x)\).
	\end{enumerate}
\end{theorem}

\begin{proof}
	We have
	\begin{enumerate}[(1)]
		\item Let \(\mu = \mu^+ - \mu^-\), then \(F = F^+ - F^-\), where \(F^{\pm}(x) = \mu^{\pm}((-\infty,x])\), which are bounded,
		      right continuous, \(F^{\pm}(-\infty) = 0\), so \(F^{\pm} \in NBV\).
		\item Let \(F \in NBV\), then \(F = F_1 - F_2\), \(F_1,F_2 \in NBV\) and increasing. Then define \(\mu_{F_1},\mu_{F_2}\) by
		      \hyperref[def:Lebesgue-Stieltjes-measure]{Lebesgue-Stieltjes measure}, and set \(\mu_F \coloneqq \mu_{F_1} - \mu_{F_2}\).\todo{Show Uniqueness in HW}
	\end{enumerate}
\end{proof}

\begin{proposition}\label{prop:nbv-ftoc}
	We have the following.
	\begin{enumerate}[(1)]
		\item If \(F \in NBV\), then \(F\) is differentiable \hyperref[def:mu-almost-everywhere]{almost everywhere}, \(F^\prime  \in L^1(\mathbb{R},m)\).
		\item \(\,\mathbb{d}\mu_F + \,\mathbb{d}\lambda + F^\prime \,\mathbb{d}m\) for some measure \(\lambda\) satisfying \(\lambda \perp m\).
		\item \(\mu_F \perp m\) if and only if \(F^\prime = 0\) Lebesgue \hyperref[def:mu-almost-everywhere]{almost everywhere}.
		\item \(\mu_F \ll m\) if and only if \(\int_{-\infty}^x F^\prime (t) \,\mathbb{d}t = F(x) - F(-\infty) = F(x)\).
	\end{enumerate}
\end{proposition}

\begin{proof}
	\todo{Check (1), (2), (3)}

	For (4), we have
	\[
		\begin{split}
			\mu_F \ll m & \iff \lambda = 0 \\
			&\iff \,\mathbb{d}\mu_F = F^\prime \,\mathbb{d}m                                          \\
			& \iff \mu_F(E) = \int_E F^\prime  \,\mathbb{d}m \;\; \forall \text{ Borel } E                               \\
			& \iff F(x) = \mu_F((-\infty,x]) = \int_{-\infty}^x F^\prime (t) \,\mathbb{d}t,\quad \forall x \in \mathbb{R}.
		\end{split}
	\]
	The last converse comes from the uniqueness of \autoref{thm:nbv-measures} above.
\end{proof}