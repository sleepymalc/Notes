\section{Additional Proofs}\label{Apx:Additional-Proofs}
\subsection{\hyperref[sec:Measure]{Measure}}
This section gives all additional proofs in \autoref{sec:Measure}.
\begin{theorem}[\autoref{thm:Caratheodory-extension-Thm} 3.]\label{thm:Caratheodory-extension-Thm:3.}
	Under the setup of \autoref{thm:Caratheodory-extension-Thm}, \((X, \mathcal{\MakeUppercase{a}} , \mu )\) is a \hyperref[def:complete-measure-space]{complete measure space}.
\end{theorem}
\begin{proof}
	We see this in two parts.
	\begin{enumerate}
		\item \textbf{Claim:} If \(A\subset X\) satisfies \(\mu ^{*} (A) = 0\), then \(A\) is \hyperref[def:C-measurable]{Carathéodory measurable} with respect to \(\mu ^{*} \).
		      \begin{proof}
			      If \(A\subset X\) and \(\mu^{*} (A) = 0\), where \(\mu^{*} \) is an outer measure on \(X\), we'll show that \(A\) is \hyperref[def:C-measurable]{Carathéodory measurable}
			      with respect to \(\mu^{*} \).

			      \par Equivalently, we want to show that for any \(E\subset X\),
			      \[
				      \mu^{*} (E) = \mu^{*} (E\cap A) + \mu^{*} (E \setminus A).
			      \]
			      Firstly, noting that \((E\cap A)\subset A\), and by \hyperref[def:outer-measure-montonicity]{monotonicity} of \(\mu^{*} \), we see that
			      \[
				      \mu^{*} (E\cap A)\leq \mu^{*} (A) = 0,
			      \]
			      and since \(\mu^{*} \geq 0\), hence \(\mu^{*} (E\cap A) = 0\). Now, we only need to show that
			      \[
				      \mu^{*} (E) = \mu^{*} (E\setminus A).
			      \]
			      Since \(E\setminus A = E\cap A^{c} \), and hence we have \(E\cap A^{c} \subset E\), so
			      \[
				      \mu^{*} (E)\geq \mu^{*} (E\setminus A).
			      \]
			      To show another direction, we note that
			      \[
				      \mu^{*} (E)\leq \mu^{*} (E\cup A) = \mu^{*} ((E\setminus A) \cup A) \leq \mu^{*} (E\setminus A),
			      \]
			      hence we conclude that \(A\) is \hyperref[def:C-measurable]{Carathéodory measurable} with respect to \(\mu^{*} \) if \(\mu^{*} (A)=0\).
		      \end{proof}
		\item \textbf{Claim:} If \(A\) is \hyperref[def:mu-subnull-set]{\(\mu\)-subnull}, then \(A\in \mathcal{\MakeUppercase{a}} \).
		      \begin{proof}
			      Let \(\mathcal{A} \) denotes the \hyperref[thm:Caratheodory-extension-Thm]{Carathéodory \(\sigma\)-algebra}, and \(\mu\coloneqq \at{\mu^{*} }{\mathcal{A} }{} \). We want to show if
			      \(A\) is \hyperref[def:mu-subnull-set]{\(\mu\)-subnull}, then \(A\in\mathcal{A} \).

			      \par Firstly, if \(A\) is \hyperref[def:mu-subnull-set]{\(\mu \)-subnull}, then there exists a \(B\in \mathcal{A} \) such that \(A\subset B\) and \(\mu (B) = 0\). But since from
			      the \hyperref[def:outer-measure-montonicity]{monotonicity} of \(\mu ^{*} \), we further have
			      \[
				      0 = \mu(B) = \mu ^{*} (B) \geq \mu ^{*} (A),
			      \]
			      hence \(\mu ^{*} (A) = 0\).

			      \par  From the first claim, we immediately see that \(A\) is \hyperref[def:C-measurable]{Carathéodory measurable} with respect to \(\mu ^{*}\),
			      which implies \(A\) is in \hyperref[thm:Caratheodory-extension-Thm]{Carathéodory \(\sigma\)-algebra}, hence \(A\in \mathcal{A} \).
		      \end{proof}
	\end{enumerate}
	We see that the second claim directly proves that \((X, \mathcal{\MakeUppercase{a}} , \mu )\) is a \hyperref[def:complete-measure-space]{complete measure space}.
\end{proof}

\begin{lemma}\label{lma:Cantor-Function-is-distribution-function}
	\hyperref[sssec:Cantor-Function]{Cantor Function} is a \hyperref[def:distribution-function]{distribution function}
\end{lemma}
\begin{proof}
	\par We define
	\[
		F_{n}(x) = \begin{dcases}
			1, & \text{ if } x\geq r_{n} ; \\
			0, & \text{ if } x<r_{n}
		\end{dcases}
	\]
	where \(\{r_1, r_2, \ldots  \}= \mathbb{\MakeUppercase{q}} \), and
	\[
		F(x) = \sum\limits_{n=1}^{\infty} \frac{F_{n}(x)}{2^n} = \sum\limits_{n;r_{n}\leq x}\frac{1}{2^n}
	\]
	is both increasing and right-continuous.

	\begin{itemize}
		\item Increasing. Consider \(x<y\). We see that
		      \[
			      F(x) = \sum\limits_{n;r_{n}\leq x} \frac{1}{2^n} \leq \sum\limits_{n;r_{n}\leq y} \frac{1}{2^n} = F(y)
		      \]
		      clearly.\footnote{This is trivial since we're always going to sum more strictly positive terms in \(F(y)\) than in \(F(x)\).}
		\item Right-continuous. We want to show \(F(x^+) = F(x)\). Let \(x^+(\epsilon )\coloneqq x + \epsilon \) with \(\epsilon >0\), we'll show that
		      \[
			      \lim\limits_{\epsilon\to 0}F(x^+(\epsilon )) =  \lim\limits_{\epsilon \to 0} F(x + \epsilon ) = F(x).
		      \]

		      Firstly, we have
		      \[
			      F(x^+(\epsilon )) - F(x) = \sum\limits_{n;r_{n}\leq x+\epsilon } \frac{1}{2^n} - \sum\limits_{n;r_{n}\leq x}\frac{1}{2^n} = \sum\limits_{n;x<r_{n}\leq x+\epsilon \footnotemark}\frac{1}{2^n},
			      \footnotetext{The strict is crucial to show the result, since if \(x = r_k\) for some fixed \(k\), then we can't make the summation arbitrarily small.}
		      \]
		      and we want to show
		      \[
			      \lim\limits_{\epsilon \to 0}F(x^+(\epsilon )) - F(x) = \lim\limits_{\epsilon \to 0}\sum\limits_{n;x< r_{n}\leq x+\epsilon }\frac{1}{2^n} = 0.
		      \]

		      \par Before we show how we choose \(\epsilon \),\footnote{To be precise, how \(\epsilon \) depends on \(r_n\).} we see that
		      \[
			      \sum\limits_{n=k}^{\infty }\frac{1}{2^n} = 2^{1-k}.
		      \]
		      Now, we simply observe that
		      \[
			      \sum\limits_{n;x< r_{n}\leq x+\epsilon }\frac{1}{2^n}\leq \sum\limits_{n=\underset{k}{\arg\mathop{\min}}\ x< r_{k}\leq x+\epsilon }^{\infty }\frac{1}{2^n} = 2^{1-k}.
		      \]
		      With this observation, it should be fairly easy to see that we can choose \(\epsilon \) based on how small we want to make \(2^{1-k}\) be,\footnote{We're referring to \(\epsilon -\delta \) proof approach.}
		      and we indeed see that
		      \[
			      \lim\limits_{k \to \infty} 2^{1-k} = 0,
		      \]
		      which implies that \(F\) is right-continuous by squeeze theorem.
	\end{itemize}
\end{proof}