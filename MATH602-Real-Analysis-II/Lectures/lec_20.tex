\lecture{20}{8 Nov. 14:30}{Spectrum Theory}
\hyperref[thm:Fredholm-alternative]{Fredholm alternative} does not hold if \(T\) is not \hyperref[def:compact-op]{compact}, see the following example.
\begin{eg}[Shift operator]
	Consider the shift operators on \(\ell _2 = \left\{ (a_1, a_2, \ldots ), a_j\in \mathbb{R}, \sum_{j=1}^{\infty}  a_j^2 < \infty \right\} \). The right shift \(T_r\) is defined as
	\[
		T_r(a_1, a_2, \ldots) = (0, a_1, a_2, \ldots ),
	\]
	which is injective but not surjective; while the left shift \(T_{\ell} \) is defined as
	\[
		T_{\ell } (a_1, a_2, \ldots ) = (a_2, a_3, \ldots ),
	\]
	which is surjective but not injective.
\end{eg}

The name \hyperref[thm:Fredholm-alternative]{Fredholm alternative} is explained by the following application to solving linear equations of the form
\[
	\lambda x - Tx = b
\]
where \(T\in \mathcal{K} (X, X)\), \(\lambda \in \mathbb{C} \), \(b\in X\). One is interested in existence and uniqueness of solution. \hyperref[thm:Fredholm-alternative]{Fredholm alternative} states that exactly one of the following statements holds for every \(\lambda \neq 0\):
\begin{itemize}
	\item \emph{either} the homogeneous equation \(\lambda x - Tx = 0\) has a nontrivial solution,
	\item \emph{or} the inhomogeneous equation \(\lambda x - Tx = b\) has a solution for every \(b\), where this solution is automatically unique.
\end{itemize}

Also, this \hyperref[thm:Fredholm-alternative]{alternative} is particularly useful for studying integral equations, since for the integral operator
\[
	(Tf)(t) = \int_{0}^{1} k(t, x)f(s) \,\mathrm{d}s ,
\]
the homogeneous Fredholm equation is
\[
	\lambda f(t) - \int_{0}^{1} k(t, x)f(s) \,\mathrm{d}s = 0,
\]
while the inhomogeneous Fredholm equation (\emph{of second kind}) is
\[
	\lambda f(t) - \int_{0}^{1} k(t, s) f(s) \,\mathrm{d}s = b(t).
\]

\section{Spectrum of Bounded Linear Operators}
Studying \hyperref[def:linear-op]{linear operators} through their spectral properties is a powerful approach in analysis and mathematical physics. Recall from linear algebra that the \hyperref[def:spectrum-point]{spectrum} of a \hyperref[def:linear-op]{linear operator} \(T\) acting on \(\mathbb{C} ^n\) consists of the \emph{eigenvalues} of \(T\), which are the numbers \(\lambda \in \mathbb{C} \) such that \(Tx = \lambda x\) for some nonzero vector \(x\in \mathbb{C} ^n\); such \(x\) are called the \emph{eigenvectors} of \(T\).

There are at most \(n\) eigenvalues of \(T\), or one can say exactly \(n\) counting multiplicities.\footnote{Eigenvalues always exist by the \href{https://en.wikipedia.org/wiki/Fundamental_theorem_of_algebra}{fundamental theorem of algebra}, as they are the roots of the characteristic polynomial \(\det(T - \lambda I) = 0 \).} Eigenvectors corresponding to different eigenvalues are linearly independent. However, the eigenvalues do not need to form a basis of \(\mathbb{C} ^n\). The dimension of the span of eigenvectors corresponding to a given eigenvalue (the eigenspace) may be strictly less than the multiplicity of that root. This happens, for example, for the Jordan block
\[
	T = \begin{pmatrix}
		\lambda & 1       \\
		0       & \lambda \\
	\end{pmatrix}.
\]
An \hyperref[def:orthonormal-basis]{orthonormal basis} of eigenvectors exists if and only if \(T\) is \hyperref[def:normal-op]{normal}, i.e. \(T^{\ast} T = TT^{\ast} \).

To start our formal discussion, we first formally define the notion of \hyperref[def:spectrum-point]{spectrum points}.

\begin{definition*}
	Let \(X\) be a complex \hyperref[def:Banach-space]{Banach space}, and \(T\colon X \to X\) be a \hyperref[def:bounded-linear-op]{bounded linear operator}.
	\begin{definition}[Regular point]\label{def:regular-point}
		A number \(\lambda \in \mathbb{C} \) is called a \emph{regular point} of \(T\) if the operator \(T-\lambda I\) is invertible, i.e., \((T-\lambda I)^{-1} \in \mathcal{L} (X, X)\).
	\end{definition}
	\begin{definition}[Spectrum point]\label{def:spectrum-point}
		A number \(\lambda \in \mathbb{C} \) is called a \emph{spectrum point} of \(T\) if it's not a \hyperref[def:regular-point]{regular point}.
	\end{definition}
\end{definition*}

\begin{notation}
	The set of \hyperref[def:regular-point]{regular points} for \(T\) is denoted as \(\rho (T)\), while the set of \hyperref[def:spectrum-point]{spectrum points} is denoted as \(\sigma (T)\).
\end{notation}

\begin{remark}[Resolvent point]
	We sometimes called a \hyperref[def:regular-point]{regular point} as a \emph{resolvent point}.
\end{remark}

From definitions, we know that \(\sigma (T) = \mathbb{C} - \rho (T)\).

\subsection{Classification of Spectrum}
For operators \(T\) acting on a finite dimensional space, the \hyperref[def:spectrum-point]{spectrum} consists of eigenvalues of \(T\); in infinite dimensions, this is not true, as there are various reason why \(T - \lambda I\) may be non-invertible.

Let's first see some examples illustrating the fact that the \hyperref[def:spectrum-point]{spectrum} is a richer concept in infinite-dimensional spaces than in finite-dimensional spaces.

\begin{eg}[Uncountable number of eigenvalues]
	Consider the differential operator \(T = \mathrm{d} / \mathrm{d}t \) acting on \(C^1(\mathbb{C} )\). Suppose \(\lambda \) is an eigenvalue of \(T\) with eigenvector \(u\in C^1(\mathbb{C} )\). This implies that the ordinary differential equation \(u^\prime = \lambda u\) holds. The solution has the form \(u(t) = C e^{\lambda t}\), so every \(\lambda \in \mathbb{C} \) is an eigenvalue of \(T\).
\end{eg}

\begin{eg}[No eigenvalues]
	Consider a multiplication operator on \(L^2([0, 1])\) acting as
	\[
		(Tf)(t) = tf(t).
	\]
	Suppose \(\lambda \) is an eigenvalue of \(T\) with eigenvector \(f\in L^2([0, 1])\). This implies
	\[
		tf(t) = \lambda f(t)
	\]
	for all \(t\in [0, 1]\). But this means \(f=0\), so \(T\) has no eigenvalues.
\end{eg}

These reasons cause the following different types of \hyperref[def:spectrum-point]{spectrum}.

\begin{definition*}[Classification of spectrum]
	Let \(T\) be a \hyperref[def:bounded-linear-op]{bounded linear operator} on \(X\).
	\begin{definition}[Point spectrum]\label{def:point-spectrum}
		The \emph{point spectrum} \(\sigma _p(T)\) contains \(\lambda \) such that \(\ker(T - \lambda I) \neq \left\{ 0 \right\} \), i.e., \(T - \lambda I\) is not injective, i.e., \(\lambda \in \sigma _p(T)\) if \(\lambda \) is an eigenvalue of \(T\).
	\end{definition}
	\begin{definition}[Continuous spectrum]\label{def:continuous-spectrum}
		The \emph{continuous spectrum} \(\sigma _c(T)\) contains \(\lambda \) such that \(\ker(T - \lambda I) = \left\{ 0 \right\} \) (\(\lambda \notin \sigma _p(T)\)) and \(\im(T-\lambda I)\) is dense in \(X\), i.e., \(\lambda \in \sigma _c(T)\) if \(\lambda \) is not an eigenvalue and \(\im(T-\lambda I)\neq X\) but \(\overline{\im(T-\lambda I)}=X\).\footnote{Note that by \hyperref[thm:inverse-mapping]{inverse mapping theorem}, if \(\im(T - \lambda I)=X\), then \(\lambda \in \rho (T)\).}
	\end{definition}
	\begin{definition}[Residual spectrum]\label{def:residual-spectrum}
		The \emph{residual spectrum} \(\sigma _r(T)\) is defined as \(\sigma _r(T)\coloneqq \sigma (T) - \sigma _p(T) - \sigma _c(T)\), i.e., \(\lambda \in \sigma _r(T)\) if \(\lambda \) is not an eigenvalue and \(\overline{\im(T-\lambda I)}\neq X\).
	\end{definition}
\end{definition*}

From the above, we have
\[
	\sigma (T) = \sigma _p (T) \sqcup \sigma _c(T) \sqcup \sigma _r(T).
\]
Let's now compute and classify the \hyperref[def:spectrum-point]{spectrum} of some basic \hyperref[def:linear-op]{linear operators}.

\begin{eg}[Diagonal operator on \(\ell _2\)]
	Let \(\left\{ \lambda _k \right\} _{k \geq 1}\) be a sequence in \(\mathbb{C} \setminus \left\{ 0 \right\} \) such that \(\lim\limits_{k \to \infty} \lambda _k = 0\). Define \(T\colon \ell _2 \to \ell _2\) by
	\[
		T(\left\{ x_k \right\} _{k\geq 1})= \left\{ \lambda_k x_k \right\} _{k\geq 1}
	\]
	where the sequence \(\left\{ \lambda _k \right\} _{k\geq 1}\) is bounded, hence \(T\) is a \hyperref[def:bounded-linear-op]{bounded linear operator}. Then, \((T-\lambda I) x = \left\{ (\lambda _k - \lambda )x_k \right\} _{k\geq 1}\), so given \(y = \left\{ y_k \right\} _{k\geq 1}\),
	\[
		(T-\lambda I)^{-1} y = \left\{ \frac{y_k}{\lambda _k - \lambda } \right\} _{k\geq 1},
	\]
	which implies \((T-\lambda I)^{-1} \) is \hyperref[rmk:bounded-op]{bounded} on \(\ell _2\) if \(\sup _{k\geq 1} \vert \lambda _k - \lambda \vert ^{-1} < \infty \). Indeed, since \(\lim\limits_{k \to \infty} \lambda _k = 0\), \(\sup _{k\geq 1} \vert \lambda _k - \lambda \vert ^{-1} < \infty\) if \(\lambda \notin \left\{ \lambda _k \right\}_{k\geq 1} \cup \left\{ 0 \right\}\). Further,
	\begin{itemize}
		\item if \(\lambda =\lambda _k\) then \(\ker(T - \lambda I) \neq \left\{ 0 \right\} \);
		\item if \(\lambda=0\), then \(\im(T - \lambda I)\) is dense in \(\ell _2\), and notice that \(\ker(T-\lambda I) = \left\{ 0 \right\} \),\footnote{Notice that \(\lambda _k \in \mathbb{C} \setminus \left\{ 0 \right\} \).} so \(0\) is in the \hyperref[def:continuous-spectrum]{continuous spectrum}.
	\end{itemize}

	Hence, \(\sigma (T) = \left\{ \lambda _k \right\}_{k\geq 1} \cup \left\{ 0 \right\}\) with \(\sigma _p(T) = \left\{ \lambda _k \right\}_{k\geq 1} \), \(\sigma _c(T) = \left\{ 0 \right\} \) with \(\sigma _r(T) = \varnothing \).
\end{eg}

Let's revisit previous example on the multiplication operator on \(L^2\).

\begin{eg}[Multiplication operator on \(L^2\)]
	Consider the multiplication on \(L^2([0, 1])\) such that \(Tf(t) = tf(t)\) for \(0 < t < 1\). Then
	\[
		(T-\lambda I)f(t) = (t - \lambda )f(t) \implies (T-\lambda I)^{-1} g(t) = \frac{g(t)}{t - \lambda }
	\]
	for \(0 < t < 1\). Hence, \(T-\lambda I\) is invertible if \(\lambda \notin [0, 1]\), so \(\sigma (T) = [0, 1]\). And if \(\lambda \in [0, 1]\), then \(\ker(T - \lambda I) = 0\), i.e., \(\sigma _p(T)=\varnothing \). Lastly, since \(\im(T-\lambda I)\) is dense in \(L^2([0, 1])\) if \(\lambda \in [0, 1]\), hence \(\sigma _c(T) = [0, 1]\), so \(\sigma _r(T) = \varnothing \).
\end{eg}

\subsection{Properties of Spectrum}
In this section, let \(X\) denotes a \hyperref[def:Banach-space]{Banach space} and \(T\in \mathcal{L} (X, X)\). We'll see that studying the \hyperref[def:spectrum-point]{spectrum} of \(T\) is convenient via the so-called \hyperref[def:resolvent-op]{resolvent operator}.

\begin{definition}[Resolvent operator]\label{def:resolvent-op}
	To each \hyperref[def:regular-point]{regular point} \(\lambda \in \rho (T)\) for \(T\in \mathcal{L} (X, X)\), the associated \emph{resolvent operator} \(R(\lambda) = (T-\lambda I)^{-1}\) is defined by \(R\colon \rho (T) \to \mathcal{L} (X, X)\).\footnote{This is the so-called resolvent function.}
\end{definition}

The \hyperref[def:resolvent-op]{resolvent operator} can be computed in terms of series expansion involving \(T\). This technique is based on the following simple lemma.

\begin{lemma}[Von Neumann]\label{lma:Von-Neumann}
	Let \(S\in \mathcal{L} (X, X)\) such that \(\lVert S \rVert < 1\), then \(I-S\) is invertible and \((I-S)^{-1} \) is given by the geometric series
	\[
		(I-S)^{-1} = \sum_{k=0} ^{\infty} S^k \text{ and }\lVert (I-S)^{-1} \rVert \leq \frac{1}{1 - \lVert S \rVert} .
	\]
\end{lemma}
\begin{proof}
	From the inequality \(\lVert S^k \rVert \leq \lVert S \rVert ^k\) for all \(k\), \(\sum_{k=0}^{\infty} S^k\) converges absolutely. Hence,
	\[
		(I-S) \sum_{k=0}^{\infty} S^k = \sum_{k=0}^{\infty} S^k(I-S) = I
	\]
	by telescoping series. Finally,
	\[
		\lVert (I-S)^{-1} \rVert \leq \sum_{k=0}^{\infty} \lVert S \rVert ^k \leq \frac{1}{1 - \lVert S \rVert }.
	\]
\end{proof}

\begin{proposition}\label{prop:lec20-1}
	The \hyperref[def:regular-point]{resolvent} set \(\rho (T)\subseteq \mathbb{C} \) is open and containing \(\left\{ \lambda \in \mathbb{C} \colon \vert \lambda  \vert > \lVert T \rVert  \right\} \), with \(\lVert R(\lambda ) \rVert \leq 1 / (\vert \lambda  \vert - \lVert T \rVert )\).
\end{proposition}
\begin{proof}
	Since
	\[
		(T-\lambda I)^{-1}
		= - \lambda ^{-1} (I-\lambda ^{-1} T)^{-1}
		= - \lambda ^{-1} (I-S),
	\]
	by letting \(S = T / \lambda \), we have \(\lVert S \rVert < 1\) if \(\vert \lambda \vert > \lVert T \rVert \), hence by \autoref{lma:Von-Neumann},
	\[
		\lVert (T-\lambda I)^{-1} \rVert
		\leq \frac{1}{\vert \lambda  \vert } \frac{1}{1 - \lVert S \rVert }
		= \frac{1}{\vert \lambda  \vert } \frac{1}{1 - \vert \lambda  \vert^{-1} \lVert T \rVert  }
		= \frac{1}{\vert \lambda  \vert - \lVert T \rVert },
	\]
	i.e., if \(\vert \lambda  \vert > \lVert T \rVert \), then \(\lambda \in \rho (T)\). To show \(\rho (T)\) is open, since
	\[
		\frac{1}{x - \lambda } - \frac{1}{x-\mu } = \frac{\lambda -\mu }{(x-\lambda )(x-\mu )}
	\]
	for all \(\mu , \lambda \in \mathbb{C} \) and \(x\in \mathbb{C} \), we can generalize this to\footnote{This is known as the resolvent identity.}
	\[
		R(\lambda ) - R(\mu ) = (\lambda -\mu )R(\lambda )R(\mu )
	\]
	since \(R(\lambda ), R(\mu )\) commutes. Hence,
	\[
		R(\mu )
		= \left[ I-(\mu -\lambda )R(\lambda ) \right] ^{-1} R(\lambda )
		= (I-S)^{-1} R(\lambda ),
	\]
	so \(R(\mu )\) is \hyperref[rmk:bounded-op]{bounded} if \(\lVert S \rVert < 1\), i.e., \(\vert \mu -\lambda  \vert \lVert R(\lambda ) \rVert < 1\). We see that \(\lambda \in \rho (T)\) implies that the disk \(D(\lambda , r) \subseteq \rho (T)\) if \(r \lVert R(\lambda ) \rVert < 1\), i.e., \(\rho (T)\) is open.
\end{proof}

\autoref{prop:lec20-1} implies the following.

\begin{proposition}\label{prop:lec20-2}
	The \hyperref[def:spectrum-point]{spectrum} \(\sigma (T)\) is a closed bounded set with
	\[
		\sigma (T) \subseteq \left\{ \lambda \in \mathbb{C} \mid \vert \lambda \vert \leq \lVert T \rVert \right\}.
	\]
\end{proposition}
\begin{proof}
	From \autoref{prop:lec20-1}, we have shown that \(\sigma (T)\) is a closed set since \(\rho (T)\) is open and \(\mathbb{C} \setminus \rho (T) = \sigma (T)\), together with \(\sigma (T)\) being bounded such that \(\sigma (T) \subseteq \overline{D(0, \lVert T \rVert )}\).
\end{proof}

\subsection{Spectral Radius}
The \hyperref[def:spectrum-point]{spectrum} of any operator \(T\in \mathcal{L} (X, X)\) is a bounded set by \autoref{prop:lec20-2}, and moreover, we have a quantitative bound \(\vert \lambda  \vert \leq \lVert T \rVert  \) for all \(\lambda \in \sigma (T)\). This bound is not always sharp, and we will try to come up with a sharper bound.

To do this, we first introduce the notion of \hyperref[def:spectral-radius]{spectral radius}.

\begin{definition}[Spectral radius]\label{def:spectral-radius}
	The \emph{spectral radius} of an operator \(T\in \mathcal{L} (X, X)\) is defined as
	\[
		r(T) \coloneqq \max _{\lambda\in \sigma (T) }\vert \lambda \vert .
	\]
\end{definition}

From \autoref{prop:lec20-2}, \(r(T) \leq \lVert T \rVert \), but the equality is not always achieved. However, we have the \hyperref[thm:Gelfand-formula]{Gelfand's formula}, which characterizes the asymptomatic behavior of \(r(T)\), essentially stating that
\[
	\lVert T^n \rVert \sim r(T)^n.
\]

\begin{theorem}[Gelfand's formula]\label{thm:Gelfand-formula}
	Let \(T\in \mathcal{L} (X, X)\) on a \hyperref[def:Banach-space]{Banach space} \(X\), then
	\[
		r(T) = \lim_{n \to \infty} \lVert T^n \rVert ^{1 / n} = \inf _{n\geq 1}\lVert T^n \rVert ^{1 / n}.
	\]
\end{theorem}
\begin{proof}\let\qed\relax
	Let \(r\) be a large integer and \(m\geq 1\) an integer with \(r = am+b\) for \(a \geq 0\), \(0 \leq b < m\). Then
	\[
		\lVert T^r \rVert
		= \lVert T^{am} T^b\rVert
		\leq \lVert T^m \rVert ^a \lVert T^b \rVert,
	\]
	hence
	\[
		\lVert T^r \rVert ^{1 / r}
		\leq \lVert T^m \rVert ^{a / r} \lVert T^b \rVert ^{1 / r}
		= \left( \lVert T^m \rVert ^{1 / m} \right) ^{\frac{a}{a+ b / m}} \lVert T^b \rVert ^{1 / r}.
	\]
	Let \(r \to \infty \), we have \(\limsup_{r \to \infty} \lVert T^r \rVert ^{1 / r} \leq \lVert T^m \rVert ^{1 / m}\) for all \(m \geq 1\) since \(a / (a + b / m) \to 1\) and \(\lVert \cdot \rVert ^{1 / r} \to 1\), hence
	\[
		\limsup_{r \to \infty} \lVert T^r \rVert ^{1 / r}
		= \liminf_{r \to \infty} \lVert T^r \rVert ^{1 / r}
		= \inf _{m\geq 1}\lVert T^m \rVert ^{1 / m}.
	\]
\end{proof}