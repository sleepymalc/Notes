\lecture{23}{17 Nov. 14:30}{}
\begin{lemma}[Invertibility criterion]\label{lma:invertibility-criterion}
	Let \(T\) be a \hyperref[rmk:bounded-op]{bounded} \hyperref[def:self-adjoint-op]{self-adjoint operator} on \(\mathcal{\MakeUppercase{h}} \), then \(T\) is invertible if and only if \(T\) is bounded below, i.e., for all \(x\in \mathcal{\MakeUppercase{h}} \), there exists \(c>0\) such that \(\lVert Tx \rVert \geq c\lVert x \rVert \).
\end{lemma}
\begin{proof}
	If \(T\) is invertible, then \(T\) is bounded below with \(c = \lVert T^{-1}  \rVert ^{-1} \). Conversely, if \(T\) is bounded below, then \(T\) is injective and \(\im T\) is closed. Since \(\sigma _r(T)\) is empty, we conclude that \(\im T=\mathcal{\MakeUppercase{h}} \), with \hyperref[thm:open-mapping]{open mapping theorem}, \(T\) is invertible with \(\lVert T^{-1}  \rVert \leq c^{-1} \).
\end{proof}

We see that by applying the \hyperref[lma:invertibility-criterion]{invertibility criterion} for the operator \(T - \lambda I\), we immediately obtain the following.

\begin{corollary}[Criterion of spectrum points]\label{col:criterion-of-spectrum-points}
	Let \(T\in \mathcal{\MakeUppercase{l}} (\mathcal{\MakeUppercase{h}} )\) be a \hyperref[def:self-adjoint-op]{self-adjoint operator}, then \(\lambda \in \sigma (T)\) if and only if the operator \(T - \lambda I\) is not \hyperref[rmk:bounded-op]{bounded} below.
\end{corollary}

\begin{remark}[Approximate point spectrum]
	\(\lambda \in \sigma (T)\) if and only if there exists a sequence \(\left\{ x_n \right\} _{n\geq 1}\) in \(\mathcal{\MakeUppercase{h}} \) and \(\lVert x_n \rVert = 1\), and
	\[
		\lim_{n \to \infty} \lVert Tx_n - \lambda x_n \rVert = 0.
	\]
\end{remark}

\subsection{The Spectrum Interval}
We now compute the tightest interval that contains the \hyperref[def:spectrum-point]{spectrum} of a \hyperref[def:self-adjoint-op]{self-adjoint operator} \(T\). This interval can be computed from the quadratic form of \(T\) as follows.

\begin{theorem}[Spectrum interval]\label{thm:spectrum-interval}
	Suppose \(T \in \mathcal{\MakeUppercase{l}} (\mathcal{\MakeUppercase{h}} )\) for \(T\) being a \hyperref[def:self-adjoint-op]{self-adjoint operator}. Then
	\begin{enumerate}[(a)]
		\item \(\sigma (T) \subseteq [m, M]\) for \(m\coloneqq \inf _{\lVert x \rVert = 1}\left\langle Tx, x \right\rangle \), \(M\coloneqq \sup _{\lVert x \rVert = 1} \left\langle Tx, x \right\rangle \).
		\item The endpoints \(m, M\in \sigma (T)\).
	\end{enumerate}
\end{theorem}
\begin{proof}
	We prove this one by one.
	\begin{enumerate}[(a)]
		\item Let \(\lambda \in \mathbb{\MakeUppercase{c}} -[m, M]\), and set \(d\) be
		      \[
			      d\coloneqq \mathop{\mathrm{dist}}(\lambda , [m, M])
			      = \inf _{m \leq y\leq M} \vert \lambda -y \vert > 0.
		      \]
		      Then we have
		      \[
			      \lVert (T-\lambda I)x \rVert
			      \geq \vert \left\langle (T-\lambda I)x, x \right\rangle  \vert
			      = \vert \left\langle Tx, x \right\rangle - \lambda  \vert
			      \geq d
		      \]
		      since \(\lVert x \rVert = 1\), which implies \(T - \lambda I\) is bounded below, and hence \(\lambda \in \rho (T)\).
		\item Without loss of generality, assume that \(0 \leq m \leq M\) by considering \(T - mI\) instead of \(T\). Now, choose a sequence \(\left\{ x_n \right\} _{n\geq 1}\) in \(\mathcal{\MakeUppercase{h}} \) where \(\lVert x_n \rVert = 1\) such that
		      \[
			      \lim_{n \to \infty} \left\langle Tx_n, x_n \right\rangle = M.
		      \]
		      By the \hyperref[lma:parallelogram-law]{parallelogram law},
		      \[
			      \lVert (T-MI)x_n \rVert ^{2}
			      = \left\langle (T-MI)x_n, (T-MI)x_n \right\rangle
			      = \lVert Tx_n \rVert ^2 - 2M\left\langle Tx_n, x_n \right\rangle + M^{2} \lVert x_n \rVert ^{2}.
		      \]
		      Since we already showed that \(\lVert T \rVert = M\), i.e., \(\lVert T \rVert = \sup _{\lVert x \rVert = 1}\left\langle Tx, x \right\rangle \) from \(\left\langle Tx, x \right\rangle \geq 0\), by letting \(n \to \infty \), the right-hand side goes to \(\leq 0\) since \(\lVert Tx_n \rVert ^{2} \leq M^{2} \lVert x_n \rVert ^{2} \) and \(\left\langle Tx_n, x_n \right\rangle \to M\), we may conclude that
		      \[
			      \lim_{n \to \infty} \lVert (T-MI)x_n \rVert = 0,
		      \]
		      and hence \(M\in \sigma (T)\).
	\end{enumerate}
\end{proof}

\begin{remark}
	It follows that \(\sigma (T) = \lVert T \rVert \).
\end{remark}

\begin{theorem}
	Let \(T\) be a \hyperref[def:compact-op]{compact} \hyperref[def:self-adjoint-op]{self-adjoint operator} on \(\mathcal{\MakeUppercase{h}} \) and \(\mathcal{\MakeUppercase{h}} \) is \hyperref[def:separable]{separable}. Then there exists an \hyperref[def:orthonormal-system]{orthonormal} basis of \(\mathcal{\MakeUppercase{h}} \) consisting of eigenvectors of \(T\).
\end{theorem}
\begin{proof}
	We already showed that \(\sigma (T)\) is nonempty, which also follows from \autoref{thm:spectrum-interval} since \(\lVert T \rVert \) or \(-\lVert T \rVert \) is in \(\sigma (T)\). Now, since \(T\) is \hyperref[def:compact-op]{compact}, we know
	\[
		\sigma (T) = \sigma _p(T) \cup \left\{ 0 \right\}.
	\]
	Hence, if \(\sigma _p(T)= \varnothing \), \(r(T) = \lVert T \rVert = 0\), i.e., \(T \equiv 0\). In this case, any \hyperref[def:orthonormal-system]{orthonormal} basis gives a basis of eigenvectors, so the result follows by showing that if \(E\) is a closed subspace of \(\mathcal{\MakeUppercase{h}} \), then \(T(E) \subseteq E\) implies \(T(E^{\perp} ) \subseteq E^{\perp} \). To see this, let \(x \perp y \in E\), then \(T(E) \subseteq E\) implies \(\left\langle x, Ty \right\rangle = 0\) for \(y\in E\), i.e., \(\left\langle Tx, y \right\rangle = 0\) for \(y\in E\), so \(Tx\in E^{\perp} \).
\end{proof}

\begin{definition}[Normal operator]\label{def:normal-op}
	An operator \(T\) is \emph{normal} if \(T T^{\ast} = T^{\ast} T\).
\end{definition}

\begin{remark}
	The spectral theorems for \hyperref[def:self-adjoint-op]{self-adjoint operators} extend to \hyperref[def:normal-op]{normal operator}.
\end{remark}

However, spectrum of \hyperref[def:normal-op]{normal operators} do not have to be real,\footnote{For example, the unitary operators \(U^{\ast} U = U U^{\ast} = I\).} we only have \(\sigma (T) \subseteq \left\{ \lambda \in \mathbb{\MakeUppercase{c}} \colon \vert \lambda  \vert =1 \right\} \).

\begin{theorem}
	Let \(p, q\) be \(2\) polynomials, then
	\begin{enumerate}[(a)]
		\item \((ap + bq)(T) = ap(T) + bq(T)\) for \(a, b\in \mathbb{\MakeUppercase{c}} \).
		\item \((pq)(T) = p(T) q(T)\).
		\item \(p(T)^{\ast} = \overline{p} (T^{\ast} )\).\footnote{\(\overline{p} \) is the polynomial with coefficients that are complex conjugates of the coefficients of \(p\).}
	\end{enumerate}
\end{theorem}

\begin{lemma}
	Suppose \(T\in \mathcal{\MakeUppercase{l}} (\mathcal{\MakeUppercase{h}} )\), then the operator \(\rho (T)\) is invertible if and only if \(p(t) \neq 0\) for all \(t\in \sigma (T)\).
\end{lemma}
\begin{proof}
	Let
	\[
		p(t) = a_n (t-t_1)(t-t_2)\ldots  (t-t_n)
	\]
	where \(t_1, \ldots  , t_n\) are zeros of \(p(\cdot)\). Then
	\[
		p(T) = a_n (T-t_1 I)\ldots  (T-t_n I).
	\]
	Next, observe that if \(S, R\in \mathcal{\MakeUppercase{l}} (\mathcal{\MakeUppercase{h}} )\) and \(SR\) is invertible, then both \(S\) and \(R\) are invertible.\footnote{Since if \(SR\) is invertible, then \(S^{-1} = R(SR)^{-1} \).} Hence, if \(p(T)\) is invertible, then \(t_1, \ldots  , t_n\in \rho (T)\), and conversely, i.e., \(p(t) \neq 0\) for \(t\in \sigma (T)\).
\end{proof}

\begin{proposition}
	Let \(p(\cdot)\) be a polynomial and \(T\in \mathcal{\MakeUppercase{l}} (\mathcal{\MakeUppercase{h}} )\). Then
	\[
		\sigma (p(T)) = p(\sigma (T)).
	\]
\end{proposition}
\begin{proof}
	Since \(\lambda \in \sigma (p(T))\) if and only if \(p(T) - \lambda I\) is not invertible, which is equivalent to say \(p(t) - \lambda = 0\) for some \(t\in \sigma (T)\), i.e., \(\lambda \in p(\sigma (T))\).
\end{proof}

\begin{corollary}\label{col:lec23}
	Suppose \(T\) is a \hyperref[rmk:bounded-op]{bounded} \hyperref[def:self-adjoint-op]{self-adjoint operator} on \(\mathcal{\MakeUppercase{h}} \) and \(p(\cdot)\) is a polynomial with real coefficients. Then \(p(T)\) is \hyperref[def:self-adjoint-op]{self-adjoint operator} and
	\[
		\lVert p(T) \rVert = \sup _{t\in \sigma (T)}\vert p(t) \vert.
	\]
\end{corollary}
\begin{proof}
	We see that \(p(T)\) is \hyperref[def:self-adjoint-op]{self-adjoint} since \(\overline{p} = p\), and
	\[
		\lVert p(T) \rVert = \sup _{t\in \sigma (p(T))} \vert t \vert = \max _{s\in \sigma (T)} \vert p(s) \vert .
	\]
\end{proof}