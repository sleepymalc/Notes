\lecture{25}{29 Nov. 14:30}{Proofs of Borel Functional Calculus Theorem}
We start by proving \autoref{thm:Borel-functional-calculus}.
\begin{proof}
	Let's prove this one by one.
	\begin{enumerate}[(a)]
		\item We construct \(f(T)\)  using \hyperref[thm:Riesz-representation-for-C-K]{Riesz representation for \(C(K)\)} where \(K = [m, M]\) is \hyperref[def:compact]{compact} Hausdorff. For \(x, y\in \mathcal{H} \), define functional \(F_{x, y} \colon C(K) \to \mathbb{C} \) by
		      \[
			      F_{x, y} (f) = \left\langle f(T)x, y \right\rangle
		      \]
		      where \(f\in C([m, M])\). Since \(F_{x, y} \in C(K)^{\ast} \), we hav e
		      \[
			      \vert F_{x, y}(f) \vert
			      \leq \lVert f(T) \rVert  \lVert x \rVert \lVert y \rVert
			      = \lVert f \rVert _{\infty } \lVert x \rVert \lVert y \rVert,
		      \]
		      so \(\lVert F_{x, y} \rVert \lVert x \rVert \lVert y \rVert \). Then \hyperref[thm:Riesz-representation-for-C-K]{Riesz representation} implies that there exists a unique Borel measure \(\mu _{x, y} \) on \([m, M]\) such that
		      \[
			      \left\langle f(T) x, y  \right\rangle
			      = \int_{m}^{M} f(\lambda ) \,\mathrm{d}\mu _{x, y}(\lambda )
		      \]
		      for \(f\in C([m, M])\). Now, we extend \(F_{x, y}\) to all Borel measurable functions \(f\colon [m, M] \to \mathbb{R} \) with \(\lVert f \rVert _\infty < \infty \) by
		      \[
			      Bf(x, y) = \int_{m}^{M} f(\lambda ) \,\mathrm{d}\mu _{x, y} (\lambda ).
		      \]
		      Notice that \(TV(\mu _{x, y}) \leq \lVert x \rVert \lVert y \rVert \) by \hyperref[thm:Riesz-representation-for-C-K]{Riesz representation}, and hence \(\vert Bf(x, y) \vert \leq \lVert f \rVert _\infty \lVert x \rVert \lVert y \rVert \).

		      Note that the function \([x, y] \to Bf(x, y)\) is linear in \(x\) and \href{https://en.wikipedia.org/wiki/Sesquilinear_form}{sesquilinear}\footnote{Recall that if \(f(T)\) is sesquilinear, then \(Bf(x, y) = \overline{Bf(y, x)}\).} in \(y\). Hence, there exists \(f(T) \in \mathcal{L} (\mathcal{H} )\) such that
		      \[
			      Bf(x, y) = \left\langle f(T)x, y \right\rangle
		      \]
		      by \hyperref[thm:Riesz-representation]{Riesz representation for Hilbert space}, so \(f(T)\) is defined uniquely by the sesquilinear property.
		\item \(\lVert f(T) \rVert \leq \lVert f \rVert _\infty \) follows from
		      \[
			      \vert Bf(x, y) \vert  \leq \lVert f \rVert _\infty \lVert x \rVert \lVert y \rVert
		      \]
		      for \(x, y\in \mathcal{H} \).
		\item Consider \(f_n, f\colon [m, M] \to \mathbb{R} \) such that \(\sup _{n\geq 1} \lVert f_n \rVert _\infty < \infty \) with \(f_n(t) \to f(t)\) for \(t\in [m, M]\). Then we have
		      \[
			      \left\langle f_n (T)x, y \right\rangle = \int_{m}^{M} f_n(\lambda ) \,\mathrm{d}\mu _{x, y}.
		      \]
		      From the dominated convergence theorem, we have
		      \[
			      \lim_{n \to \infty} \int_{m}^{M} f_n(\lambda ) \,\mathrm{d}\mu _{x, y} = \int_{m}^{M} f(\lambda ) \,\mathrm{d}\mu _{x, y},
		      \]
		      hence
		      \[
			      \lim_{n \to \infty} \left\langle f_n(T)x, y \right\rangle = \left\langle f(T)x, y \right\rangle
		      \]
		      for all \(x, y\in \mathcal{H} \), which is saying that \(f_n(T)\) \hyperref[def:weakly-convergence]{converges weakly} to \(f(T)\). To show \hyperref[def:strongly-convergence]{strong convergence}, note that
		      \[
			      \begin{split}
				      \lVert (f_n(T) - f(T)) x \rVert ^2
				      &= \left\langle (f_n(T) - f(T)) x, (f_n(T) - f(T)) x \right\rangle \\
				      &= \left\langle (f_n(T) - f(T))^2 x, x \right\rangle \\
				      &= \int_{m}^{M} (f_n(T) - f(T))^2 \,\mathrm{d}\mu _{x, x}
				      \to 0
			      \end{split}
		      \]
		      by dominated convergence theorem. Hence, we conclude that
		      \[
			      \lim_{n \to \infty} \lVert (f_n(T) - f(T)) x \rVert = 0
		      \]
		      for all \(x\in \mathcal{H} \).
		\item The commutativity follows from two approximation arguments. Firstly, it holds if \(f, g\) are polynomials, so by \hyperref[thm:Weierstrass-approximation]{Weierstrass approximation theorem}, this holds for continuous \(f, g\colon [m, M] \to \mathbb{R} \). Then, if this holds for continuous functions, it also follows for Borel functions by approximating Borel functions with continuous functions.\footnote{This is the so-called \href{https://en.wikipedia.org/wiki/Lusin's_theorem}{Lusin's theorem}.}
	\end{enumerate}
\end{proof}

\subsection{Spectral Measures}
From \hyperref[thm:Borel-functional-calculus]{Borel functional calculus}, we have
\[
	\left\langle Tx, y \right\rangle = \int_{m}^{M} \lambda  \,\mathrm{d}\mu _{x, y}(\lambda )
\]
for \(x, y\in \mathcal{H} \). We can abstract this result to construct integrals with respect to spectral projections.

Let \(E \subseteq [m, M]\) be a Borel set, and \(\mathbbm{1}_{E} \) be the indicator function such that
\[
	\mathbbm{1}_{E} (\lambda ) = \begin{dcases}
		1, & \text{ if } \lambda \in E ;    \\
		0, & \text{ if } \lambda \notin E .
	\end{dcases}
\]
Set \(P_E = \mathbbm{1}_{E}(T) \), from \((\mathbbm{1}_{E} )^2 = \mathbbm{1}_{E} \), we have \(P_E^2 = P_E\), so \(P_E\) is indeed a projection.

\begin{proposition}
	The spectral projections \(E \to P_E\) for \(E\in \mathcal{B} ([m, M])\) satisfies \(P_{[m, M]} = I\). Furthermore, if \(E = \bigcup_{k=1}^{\infty} E_k\) and \(E_k\) are disjoint for all \(k\geq 1\), then
	\[
		P_E = \sum_{k=1}^{\infty} P_{E_k}
	\]
	where convergence is in the \hyperref[def:strongly-convergence]{strong} sense.
\end{proposition}
\begin{proof}
	It follows \autoref{thm:Borel-functional-calculus} where if \(f_n \to f\), then \(f_n(T) \to f(T)\) \hyperref[def:strongly-convergence]{strongly}.
\end{proof}

\begin{remark}
	The mapping \(E \to P_E\) for \(E\in \mathcal{B} ([m, M])\) is an operator valued measures.
\end{remark}

\begin{theorem}[Spectral theorem]\label{thm:spectral}
	Let \(T\in \mathcal{L} (\mathcal{H} )\) be \hyperref[def:self-adjoint-op]{self-adjoint} and \(E \to P_E\) be the associated spectral measure. Then
	\[
		T = \int_{-\infty}^{\infty} \lambda  \,\mathrm{d}P_\lambda.
	\]
\end{theorem}
\begin{proof}
	This is just a way of interpretation, i.e.,
	\[
		\left\langle Tx, y \right\rangle = \int_{-\infty}^{\infty} \lambda \, \left[\mathrm{d} P_\lambda x, y \right]
	\]
	for \(x, y\in \mathcal{H} \) where \(\left[ \mathrm{d} P_\lambda x, y \right] = d_{\mu _{x, y}}\).
\end{proof}

\chapter{Epilogue}
Lastly, we prove some left-out theorems, start with the \hyperref[thm:Riesz-representation-for-C-K]{Riesz representation for \(C(K)\)}.

\section{Proof of Riesz Representation for \(C(K)\)}\label{pf:Riesz-representation-for-C-K}
This section is the proof about \hyperref[thm:Riesz-representation-for-C-K]{Riesz representation for \(C(K)\)}. Let's first restate the theorem.

\begin{prev}[Riesz representation for \(C(K)\)]
	Let \(E = C(K)\) be the space of continuous functions on \hyperref[def:compact]{compact} Hausdorff space \(K\). Then we have the following.
	\begin{enumerate}[(a)]
		\item For every Borel regular signed measure on \(K\), the \hyperref[def:linear-functional]{functional} \(F(f) = \int _K f\,\mathrm{d} \mu \) is a \hyperref[def:bounded-linear-functional]{bounded linear functional} on \(K\).
		\item Every \hyperref[def:bounded-linear-functional]{bounded linear functional} on \(C(K)\) can be expressed as \(F(f) = \int _K f\,\mathrm{d} \mu \) for some measure \(\mu \), and \(\left\lVert F\right\rVert = \left\vert \mu  \right\vert (K) \), i.e., \(TV(K)\).
	\end{enumerate}
\end{prev}
Let's first outline the proof. The main theorem we're going to use is the \hyperref[thm:Urysohn-lemma]{Urysohn's lemma} for the construction of continuous functions. \hyperref[thm:Urysohn-lemma]{Urysohn's lemma} allows us to construct means of \hyperref[def:positive-op]{positive} linear functional on \(C(X)\) where \(X\) is \hyperref[def:locally-compact]{locally compact} \hyperref[def:Hausdorff]{Hausdorff}. Then, let \(\Lambda \colon C(X) \to \mathbb{R} \) where \(\Lambda \) is linear and \(\Lambda (f) \geq 0\) if \(f(x) \geq 0\) for all \(x\in X\). Now, we define means of an open set \(V \subseteq X\) where
\[
	\mu (V) = \sup \left\{ \Lambda (f) \colon 0 \leq f \leq \mathbbm{1}_{V} \right\}.
\]
Then, we define \(\mu \) for any subset \(E \subseteq X\),
\[
	\mu (E) = \inf \left\{ \mu (V)\colon E \subseteq V, V \text{ open}  \right\},
\]
where \(\mu \) is the outer measure, i.e., subadditive.