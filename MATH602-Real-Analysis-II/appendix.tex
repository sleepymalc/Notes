\chapter{Additional Proofs}
\section{Additional Proofs}\label{Apx:Additional-Proofs}
\chapter{Review}
\section{Midterm Review}
\subsection{Normed Space}
Recall the normed spaces, and the properties of which. In particular, focus on convexity and note that \(x\mapsto \left\lVert x\right\rVert \) is a convex function.

\begin{eg}[Normed spaces]
	The spaces \(\ell _p\) for \(1 \leq p \leq \infty \) of sequences and \(L^p(\Omega , \mathcal{\MakeUppercase{f}} , \mu )\) of integrable functions. Also, the space of continuous functions on compact Hausdorff space with supremum norm \(C(K)\). Notice that
	\[
		C(K) \subseteq L^{\infty} (K, \mathcal{\MakeUppercase{f}}).
	\]
\end{eg}

\subsection{Legendre Transform}
The Legendre transform of convex functions. Recall the most general form is that let \(X\) be a Banach space and \(X^{\ast} \) its dual space with a convex function \(f\colon X\to \mathbb{\MakeUppercase{r}} \) and \(f^{\ast} \colon X^{\ast} \to \mathbb{\MakeUppercase{r}} \). We have
\[
	f^{\ast} (y^{\ast} ) = \sup _{x\in X}\left[ y^{\ast} (x) - f^{\ast} (x) \right].
\]

Notice that \(f^{\ast} \) is convex and lower semi-continuous where \(f^{\ast} \colon X^{\ast} \to \mathbb{\MakeUppercase{r}} \cup \left\{ +\infty  \right\} \).

\subsection{Quotient Space}
Let \(X\) be a normed space and \(E\) be a subspace of \(X\). Then \(\quotient{X}{E} = \left\{ [x] = x+E\colon x\in X \right\}  \) if \(E\) is closed, then \(\quotient{X}{E} \) is also a normed space with the norm
\[
	\left\lVert [x]\right\rVert \coloneqq \inf _{y\in E}\left\lVert x- y\right\rVert.
\]
\begin{remark}
	We need \(E\) to be closed since \(\left\lVert [x]\right\rVert = 0 \implies [x] = 0\).
\end{remark}

\subsection{Banach Space}
Any normed space \(e\) can be completed to a Banach space \(\hat{E} \).

\begin{eg}
	\(\ell _p\) and \(L^p\) are Banach spaces. For \(x\in \ell _p\), \(x= \left\{ x_n, n\geq 1 \right\} \) with
	\[
		\left\lVert x\right\rVert _p = \left( \sum_{n=1} ^{\infty} \left\vert x_n \right\vert ^p \right) ^{1 / p}.
	\]
\end{eg}

Notice that Minkowski inequality is the triangle inequality for \(\ell _p\) and \(L^p\). We can prove this using Holder's inequality where we have
\[
	\left\lVert fg\right\rVert _1 \leq \left\lVert f\right\rVert _p \left\lVert g\right\rVert _q
\]
for \(1 / p + 1 / q = 1\).

\begin{proof}[Proof of completeness of the \(\ell _p\) spacees]
	This is easy for \(\ell _p\), but for \(L^p\), we need to use dominated convergence theorem.
\end{proof}

\section{Inner Product Space}
Notice that the Hilbert spaces are the complete of inner product spaces. Recall the parallelogram law:
\[
	\left\lVert x e y\right\rVert ^{2} + \left\lVert x - y\right\rVert ^{2} = 2\left\lVert x\right\rVert ^{2} + 2 \left\lVert y\right\rVert ^{2}
\]
and the Schwartz inequality:
\[
	\left\vert \left\langle x, y \right\rangle  \right\vert \leq \left\lVert x\right\rVert \left\lVert y\right\rVert.
\]

\subsection{Orthogonality}
Recall the orthogonal projection \(P_E\) onto a closed subspace \(E \subseteq \mathcal{\MakeUppercase{h}} \) is \(P_E x = x(y)\) where \(x(y)\) is the minimizer of \(\min _{y\in E} \left\lVert x - y\right\rVert \).

\begin{remark}
	\(P_E\) is the projection, i.e., \(P_E ^{2} g P_E\), and \(I-P_E\) is proaction onto the orthogonal complement \(E^\perp\) of \(E\) in \(\mathcal{\MakeUppercase{h}} \) such that \(\mathcal{\MakeUppercase{h}} = E \oplus E^{\perp} \). We see that \(\left\lVert x\right\rVert ^{2} = \left\lVert P_E x\right\rVert ^{2} + \left\lVert (I - P_E)x\right\rVert ^{2}\) for \(x\in \mathcal{\MakeUppercase{h}} \).
\end{remark}

Consider the orthogonal and orthonormal sets of vectors \(x_k\), \(k = 1, 2, \ldots  \) in \(\mathcal{\MakeUppercase{h}} \) corresponding Fourier series is defined as
\[
	S_n(x) \coloneqq \sum_{k=1} ^n \left\langle x, x_k \right\rangle x_k
\]
such that
\[
	\left\lVert S_n(x)\right\rVert ^{2} = \sum_{k=1}^n \left\vert \left\langle x, x_k \right\rangle  \right\vert ^{2}.
\]

If the set \(\left\{ x_k \right\} _{k=1}^{\infty} \) is orthonormal, then \(S_n = P_{E_n}\) where \(E_n\) is the span of \(\left\{ x_1, \ldots , x_n  \right\} \), and
\[
	\left\lVert S_n x\right\rVert ^{2} = \left\lVert P_{E_n}x\right\rVert ^{2} \leq \left\lVert x\right\rVert ^{2},
\]
which is the so-called Bessel's inequality.

\begin{remark}
	\(S_n x \to S_\infty x\) in \(\mathcal{\MakeUppercase{h}} \) where \(S_\infty = P_{E_\infty }\) and \(E_\infty \) is the closure of spaces \(E_n\), \(n \geq 1\).
\end{remark}

The orthonormal system \(x_k\), \(k \geq 1\) is complete if \(E_\infty = \mathcal{\MakeUppercase{h}}\). In that case, \(\left\lVert x\right\rVert ^{2} = \left\lVert P_{E_\infty }x\right\rVert ^{2} = \sum_{k=1}^{\infty} \left\vert \left\langle x, x_k \right\rangle  \right\vert ^{2} \).

\begin{remark}
	Proving completeness can be difficult.
\end{remark}

\begin{eg}[Haar basis]
	The Haar basis for \(L^2([0, 1])\) is the Fourier basis \(e^{2\pi ni x}\), \(n \in \mathbb{\MakeUppercase{z}} \) for \(L^2([0, 1])\).
\end{eg}
\begin{explanation}
	Let \(x_k\), \(k \geq 1\) be any arbitrary sequence of vectors in \(\mathcal{\MakeUppercase{h}} \). We can then construct an orthonormal sequence \(y_k\), \(k \geq 1\) by applying Gram-Schmidt procedure.
\end{explanation}

\section{Bounded Linear Functionals}
Consider bounded linear functionals on a Banach space \(E\), \(f\in E^{\ast} \), \(\left\lVert f\right\rVert = \sup _{\left\lVert x\right\rVert = 1}\left\vert f(x) \right\vert \) and \(E^{\ast} \) is Banach space. Recall that \(f(\cdot)\) is essentially defined by \(H = \ker(f)\) where \(H\) is a closed subspace of \(E\) with \(\mathop{\mathrm{codim}}(H) = 1\), i.e., \(\dim \quotient{E}{H} = 1 \) and we have
\[
	\widetilde{f} \colon \quotient{E}{H} \to \mathbb{\MakeUppercase{r}}
\]
is defined via \(\widetilde{f} ([x]) = f(x)\) for \(x\in E\), and \(\widetilde{f} (a[x]) = ca\)  for some constant \(c\).

\section{Representation Theorem}
The important representation theorem for bounded linear functionals is the Riesz representation theorem. The easiest case is \(E = \mathcal{\MakeUppercase{h}} \) being a Hilbert space and \(E^{\ast} \equiv \mathcal{\MakeUppercase{h}} \).this implies Radon-Nikydom theorem, where we have \(\nu \ll \mu \), then
\[
	\nu (E) = \int _E f\,\mathrm{d} \mu ,\quad f = \frac{\mathrm{d}\nu }{\mathrm{d}\mu } \in L^1(\mu )
\]
for \(\nu , \mu \) being finite measures. Furthermore, the Radon-Nikydom theorem implies the Riesz representation theorem for \(\ell _p\) and \(L^p\) with \(1 \leq p < \infty \).

\begin{remark}
	We have \(E^{\ast} = \ell _q\) or \(L^q\) with \(1 / p + 1 / q = 1\) for \(1 \leq p < \infty \), and remarkably, \(\ell _1^{\ast} = \ell _\infty \) but \(\ell _\infty ^{\ast} \neq \ell _1\).
\end{remark}

\begin{remark}
	The Riesz representation theorem for \(C(K)\) is space of bounded Borel measures where for \(g\in C(K)^{\ast} \),
	\[
		g(f) = \int _K f\,\mathrm{d} \mu
	\]
	for \(f\in C(K)\).
\end{remark}

\section{Hahn-Banach Theorem}
Let \(E\) be a Banach space and \(E_0\) be a subspace such that \(f_0 \colon E_0 \to \mathbb{\MakeUppercase{r}} \) a bounded linear functional on \(E_0\) such that \(\left\lVert f_0\right\rVert < \infty \). Then there exists an extension \(f\) of \(f_0\) to \(e\) with \(\left\lVert f\right\rVert = \left\lVert f_0\right\rVert \).

\begin{remark}
	\(f\) is not necessary unique. Nevertheless, it is unique for Hilbert spaces, or \(\ell _p\), \(L^p\) with \(1 < p < \infty \).
\end{remark}

\section{Reflexivity}
Consider the embedding \(E\to E^{\ast\ast} \) such that \(x\mapsto x^{\ast\ast}\), then \(E\) is reflexive if the embedding is surjective. Also, \(E\) is reflexive implies that
\[
	\left\lVert f\right\rVert = \sup _{\left\lVert x\right\rVert = 1}\left\vert f(x) \right\vert = f(x_f)
\]
for some \(x_f\in E\) with \(\left\lVert x_f\right\rVert = 1\) for every \(f\in E^{\ast} \).

\begin{remark}
	This is one way of showing some spaces is not reflexive.
\end{remark}

\section{Separation Theorem}
We first consider the separation theorem for convex sets. Given a convex set \(K\) and a point \(x_0 \notin K\), there is a hyperplane such that \(f(x_0) > f(k)\) for all \(k\in K\) The Minkowski functional for convex set essentially makes convex sets unit ball for some semi-norm.