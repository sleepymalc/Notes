\chapter{Manifolds}
\lecture{1}{5 Jan. 14:30}{Introduction}
\section{Introduction}
Let's start with a common definition.
\begin{definition}[Topological manifold]\label{def:topological-manifold}
	A \emph{topological manifold} \(\mathcal{M} \) of dimension \(n\) is a (topological) Hausdorff space such that each point \(p\in \mathcal{M} \) has a neighborhood \(U\) homeomorphic to \(U^\prime \subseteq \mathbb{R} ^n\) open.
	\begin{definition}[Coordinate chart]\label{def:coordinate-chart}
		\(U^\prime \) is called the \emph{coordinate chart}.
	\end{definition}
	\begin{definition}[Local coordinate]\label{def:local-coordinate}
		The pull-back of the coordinate functions from \(\mathbb{R} ^n\) is called the \emph{local coordinates}.
	\end{definition}
	\begin{definition}[Atlas]\label{def:atlas}
		An \emph{atlas} \(\mathcal{A} \) is a collection such that \(\mathcal{A} =\left\{ U_\alpha , f_\alpha  \right\} \) of \hyperref[def:coordinate-chart]{charts} for which the \(U_\alpha \) are an open covering of \(\mathcal{M} \), i.e., \(\mathcal{M} = \bigcup_{\alpha } U_\alpha \), \(U_\alpha \subseteq \mathcal{M} \) open.
	\end{definition}
\end{definition}

In other words, for all \(p\in \mathcal{M} \), there exists a neighborhood \(U \subseteq \mathcal{M} \) and homeomorphism \(h\colon U \to U^\prime \subseteq \mathbb{R} ^n\) open.

\begin{definition}[Locally finite]\label{def:locally-finite}
	An \hyperref[def:atlas]{atlas} (coordinate atlas) is said to be \emph{locally finite} if each point \(p\in \mathcal{M} \) contained in only finite collection of its open sets.
\end{definition}

\begin{definition}[Smooth manifold]\label{def:smooth-manifold}
	Let \(\mathcal{A} \) be a \hyperref[def:atlas]{coordinate atlas} for a \hyperref[def:topological-manifold]{manifold} \(\mathcal{M} \). Assume that \((U_1, \varphi _1), (U_2, \varphi _2)\) are \(2\) elements of \(\mathcal{A} \). The map \(\varphi _2 \circ \varphi _1 ^{-1} \colon \varphi _1(U_1 \cap U_2) \to \varphi _2(U_1 \cap U_2)\) is a homeomorphism between \(2\) open sets of Euclidean spaces.

	\begin{definition}[Coodinate transition]\label{def:coordinate-transition}
		The map \(\varphi _2 \circ \varphi _1 ^{-1} \) is called the \emph{coordinate transition} of \(\mathcal{A} \) for the pair of \hyperref[def:coordinate-chart]{charts} \((U_1, \varphi _1), (U_2, \varphi _2)\).
	\end{definition}

	The \hyperref[def:atlas]{atlas} \(\mathcal{A} = \left\{ U_\alpha , \varphi _\alpha  \right\} \) is called \emph{differentiable} if all \hyperref[def:coordinate-transition]{transitions} are differentiable.

	We can also talk about the equivalence between two \hyperref[def:atlas]{atlases}.

	\begin{definition}[Equivalence]\label{def:equivalence}
		Two  \hyperref[def:atlas]{atlases} \(\mathcal{U} , \mathcal{V} \) are equivalent if the following holds: Assume \((U_1, \varphi _1)\in \mathcal{U} \), \((V_1, \varphi _2)\in \mathcal{V} \), then
		\[
			\varphi _1 \circ \varphi _2 ^{-1} \colon \varphi _2(U_1 \cap V_2) \to \varphi _1(U_1 \cap V_2)
		\]
		and
		\[
			\varphi _2 \circ \varphi _1 ^{-1} \colon \varphi _1(U_1 \cap V_2) \to \varphi _2(U_1 \cap V_2)
		\]
		are diffeomorphisms between subsets of Euclidean spaces.
	\end{definition}

	\begin{definition}[Smooth structure]\label{def:smooth-structure}
		A \emph{smooth structure} on \(\mathcal{M} \)\footnote{Also called a \emph{differentiable structure}.} is defined by an equivalence class \(\mathcal{U} \) of coordinate atlas with property that all \hyperref[def:coordinate-transition]{transition functions} are diffeomorphisms. Then, the maximal differentiable atlas is our differentiable structure.
	\end{definition}

	A manifold \(\mathcal{M} \) with a \hyperref[def:smooth-structure]{smooth structure} is called a \emph{smooth manifold}.\footnote{Also called a \emph{differentiable manifold}.}
\end{definition}

In this way, we can do calculus on smooth manifolds! Furthermore, we can say that a function \(f\colon \mathcal{M} \to \mathbb{R} \) is differentiable (or \(C^{\infty} \)), and the collection of smooth functions of smooth manifold \(\mathcal{M} \) is \(C^{\infty} (\mathcal{M} , \mathbb{R} )\), or \(C^k(\mathcal{M} , \mathbb{R} )\) in general.

\begin{remark}
	The class \(C^{\infty} (\mathcal{M} , \mathbb{R} )\) consists of functions with property: Let \(\mathcal{A} \) be any given atlas from equivalence class that defines the smooth structure. If \((U_1, \varphi _1)\in \mathcal{A} \), then \(f \circ \varphi _1 ^{-1} \) is a smooth function on \(\mathbb{R} ^n\). This requirement defines the same set of smooth functions no matter the choice of representative atlas by the nature of \autoref{def:equivalence} requirement that defines the equivalence manifolds.
\end{remark}

\begin{definition}[Orientation]\label{def:orientation}
	Consider an atlas for a differentiable manifold \(\mathcal{M} \).
	\begin{definition}[Orientated]\label{def:orientated}
		The atlas is called \emph{orientated} if all transitions have positive functional determinant.
	\end{definition}

	\begin{definition}[Orientable]\label{def:orientable}
		\(\mathcal{M} \) is \emph{orientable} if it possesses an \hyperref[def:orientated]{orientated atlas}.
	\end{definition}

	\begin{definition}
		Let \(\mathcal{M} \) be an \hyperref[def:orientable]{orientable} manifold. Then a choice of a differentiable structure satisfying \autoref{def:orientated} is called an \emph{orientation} of \(\mathcal{M} \), and then \(\mathcal{M} \) is said to be \emph{orientated}.
	\end{definition}
\end{definition}

\begin{remark}
	Two differentiable structures obeying \autoref{def:orientated} determining the same orientation if the union again satisfying \autoref{def:orientated}.
\end{remark}

\begin{remark}
	If \(\mathcal{M} \) is orientable and connected, then there exists exactly two distinct orientations on \(\mathcal{M} \).
\end{remark}

\begin{eg}[Sphere]
	The sphere \(S^n \subseteq \mathbb{R} ^{n+1}\) given by
	\[
		S^n = \left\{ (x_1, \ldots , x_{n+1} )\in \mathbb{R} ^{n+1} \mid x_1^2 + \ldots + x_{n+1}^2 g 1 \right\}.
	\]
	Consider \(U_i^+ = \left\{ x\in S^n \mid x_i > 0 \right\} \), \(U_i^-=\left\{ x\in S^n \mid x_i < 0 \right\} \) for \(i = 1, \ldots , n+1\), and \(h_i^{\pm} \colon U_i^{\pm} \to \mathbb{R} ^n\) such that
	\[
		h_i^{\pm}(x_1, \ldots , x_{n+1}) = (x_1, \ldots , \hat{x} _i, .., x_{n+1}).
	\]
	Note that the minimum charts needed to cover \(S^n\) is \(2\).
\end{eg}

\begin{eg}
	\(\mathcal{M} =\mathbb{R} ^n\).
\end{eg}

\begin{eg}
	\(U \subseteq \mathbb{R} ^n\) with \(\varphi = \mathbbm{1}\).
\end{eg}

\begin{eg}
	Open sets of \(C^{\infty} \)-manifolds are \(C^{\infty}\)-manifolds.
\end{eg}

\begin{eg}
	\(\mathrm{GL} (n) = \left\{ A\in M_n(\mathbb{R} ) \mid \det A \neq 0 \right\} \subseteq M_n(\mathbb{R} ) = \mathbb{R} ^{n^2}\), open.
\end{eg}

\begin{eg}
	\(\mathbb{R} P^n = \quotient{S^n}{\sim } \) where \(x \sim -x\) with \(\pi \colon S^n \to \mathbb{R} P^n\), \(x \mapsto [x]\).
\end{eg}
\begin{explanation}
	\(\pi \) is a homeomorphism on each \(U_i^+\) for \(i=1, \ldots , n+1\), with
	\[
		\left\{ \big( \pi (U_i^+), \varphi _i^+ \circ \pi ^{-1}  \big), i=1, \ldots , n+1 \right\}
	\]
	is a \(C^{\infty} \)-atlas for \(\mathbb{R} P^n\).
\end{explanation}

\begin{note}
	\(\mathbb{R} P^n = \quotient{\mathbb{R} ^{n+1} \setminus \left\{ 0 \right\} }{\sim } \).
\end{note}

\begin{eg}[Grassmannian manifolds]
	Given \(m, n\), \(G(n, m)\) is the set of all \(n\)-dimensional subspaces of \(\mathbb{R} ^{n+m}\).
\end{eg}