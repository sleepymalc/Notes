\lecture{10}{7 Feb.\ 13:00}{Curvatures and Flow of Vector Fields}
\section{Riemannian Curvatures}
Given all these definitions, we can now introduce the notion of ``curvatures.'' Consider the following.\footnote{In do Carmo~\cite{flaherty2013riemannian}, the corresponding definition of \autoref{def:Riemannian-curvature} differs by a sign.}

\begin{definition}[Riemannian curvature]\label{def:Riemannian-curvature}
	The \emph{Riemannian curvature} \(R\) of a \hyperref[def:Levi-Civita-connection]{Levi-Civita connection} \(\nabla \) is the \hyperref[def:tensor-field]{\((1, 3)\)-tensor field}\footnote{\(R\) is indeed \(C^{\infty} \)-linear in each entry (see \autoref{section:C-infty-module-viewpoint-of-tensor-fields}) although we omit the proof here.}
	\[
		R(\omega , Z, X, Y) \coloneqq \omega \left( \nabla _X \nabla _Y Z - \nabla _Y \nabla _X Z - \nabla _{[X, Y]}Z \right) .
	\]
\end{definition}

\begin{notation}
	We usually write this as \(R(X, Y)Z\) by emphasizing \(Z\) and neglecting \(\omega \).
\end{notation}

\begin{eg}[Euclidean space]
	If \(\mathcal{M} = \mathbb{R} ^n\) (with the ``flat'' \(\nabla\)), \(R(X, Y)Z = 0\) for all \(X, Y, Z \in \Gamma (T \mathbb{R} ^n)\).
\end{eg}
\begin{explanation}
	Since given \(Z=(z_1, \dots , z_n)\) with the components from natural coordinates of \(\mathbb{R} ^n\), \(\nabla _X Z = (Xz_1, \dots , Xz_n)\), then \(\nabla _Y \nabla _X Z = (YX z_1, \dots , YXz_n)\), hence \(R(X, Y)Z = 0\).
\end{explanation}

Hence, we see the following.

\begin{intuition}
	\(R(X, Y)Z\) is trying to measure how much \(\mathcal{M} \) deviates from being Euclidean.
\end{intuition}

Another way to look at this is the following.

\begin{intuition}\label{int:lec10}
	Consider a system of \hyperref[def:local-coordinate]{coordinates} \(\left\{ x_i \right\} \) around \(p\in \mathcal{M} \). Since \([\partial / \partial x_i , \partial / \partial x_j] = 0\),
	\[
		R\left( \frac{\partial }{\partial x_i} , \frac{\partial }{\partial x_j} \right) \frac{\partial }{\partial x_k}
		= ( \nabla _{\frac{\partial }{\partial x_i} } \nabla _{\frac{\partial }{\partial x_j} } - \nabla _{\frac{\partial }{\partial x_j} } \nabla _{\frac{\partial }{\partial x_i} } ) \frac{\partial }{\partial x_k},
	\]
	i.e., \(R(X, Y)Z\) is trying to measure the non-commutativity of the \hyperref[def:covariant-derivative]{covariant derivative}.
\end{intuition}

\subsection{Local Expressions}
It;s convenient to express things in a \hyperref[def:local-coordinate]{local coordinates}. Consider a \hyperref[def:local-coordinate]{chart} \((U, x)\) at \(p\in \mathcal{M} \) and let \(\partial /\partial x_i = X_i\). We define \(R_{ijk}^{\ell }\) as\footnote{This is how we define \hyperref[not:connection-coefficient]{connection coefficients}, i.e., \(R^{\ell }_{ijk} \) are components of \(R\) in \((U, x)\).}
\[
	R^{\ell }_{ijk} X_{\ell } \coloneqq R(X_i, X_j) X_k.
\]
If \(X = u^i X_i, Y = v^j X_j, Z = w^k X_k\), from the linearity of \(R\),
\[
	R(X, Y)Z = R^{\ell }_{ijk} u^i v^j w^k X_{\ell }.
\]
Then the above \hyperref[int:lec10]{intuition} can be rewritten as follows.

\begin{remark}[Algebraic significant of Riemannian curvature]
	Since
	\[
		\left( \nabla _X \nabla _Y Z - \nabla _Y \nabla _X Z \right) = R(\cdot, Z, X, Y) + \nabla _{[X, Y]} Z,
	\]
	by letting \(\nabla _i \coloneqq \nabla _{\frac{\partial }{\partial x^i} }, \nabla _j \coloneqq \nabla _{\frac{\partial }{\partial x^j} }\), in a \hyperref[def:coordinate-chart]{chart} \((U, x)\), we have
	\[
		(\nabla _i \nabla _j Z)^k - (\nabla _j \nabla _i Z)^k
		= R^k_{\ell ij} Z^{\ell } + \underbrace{\nabla _{\left[ \frac{\partial }{\partial x^i} , \frac{\partial }{\partial x^j} \right] }}_{=0} Z
		= R^k_{\ell ij} Z^{\ell },
	\]
	i.e., the components of \(R\) contains all the information of how \(\nabla _i\) and \(\nabla _j\) fail to commute.
\end{remark}

We can also express \(R^{\ell }_{ijk} \) in terms of \(\Gamma ^k_{ij}\) by observing
\[
	R(X_i, X_j)X_k
	= \nabla _{X_i} \nabla _{X_j} X_k - \nabla _{X_j} \nabla _{X_i} X_k
	= \nabla _{X_i} (\Gamma ^\ell _{jk} X_{\ell } ) - \nabla _{X_j} (\Gamma ^\ell _{ik} X_{\ell } ),
\]
hence,
\[
	R^s_{ijk} = \Gamma _{jk}^{\ell }  \Gamma ^s _{i \ell } - \Gamma ^\ell _{ik} \Gamma ^s _{j \ell } + \Gamma ^s _{jk, i} - \Gamma ^s _{ik, j}.
\]
Lastly, we write
\[
	\langle R(X_i, X_j) X_k, X_{\ell } \rangle = R^{s}_{ijk} g_{\ell s} \eqqcolon R_{i j k \ell }.
\]

\subsection{Identities}
There are many important identities related to \(R\), and we should see some of them.

\begin{note}
	Although the above interpretations and intuitions are more or less formal, we should first get used to the formal properties of \(R\) and postpone a more geometric interpretation of \hyperref[def:Riemannian-curvature]{curvature} later.
\end{note}

The following two are due to Bianchi (both are proved in homework 2).

\begin{proposition}[First Bianchi identity]\label{prop:1st-Bianchi-identity}
	Given the \hyperref[def:Riemannian-curvature]{Riemannian curvature tensor} \(R\), for all \hyperref[def:vector-field]{vector fields} \(X, Y, Z\),
	\[
		R(X, Y)Z + R(Y, Z)X + R(Z, X)Y = 0;
	\]
	or equivalently, \(R_{k \ell i j} + R_{k ij \ell } + R_{k j \ell i} = 0\).
\end{proposition}
\begin{proof}
	See also do Carmo~\cite[Proposition 2.4]{flaherty2013riemannian}.
\end{proof}

\begin{proposition}[Second Bianchi identity]\label{prop:2nd-Bianchi-identity}
	Given the \hyperref[def:Riemannian-curvature]{Riemannian curvature tensor} \(R\),
	\[
		\frac{\partial }{\partial x^h} R_{k \ell i j} + \frac{\partial }{\partial x^k} R_{\ell h i j} + \frac{\partial }{\partial x^{\ell } } R_{h k i j} = 0;
	\]
	or equivalently, \(\nabla _{[\alpha} R_{\beta \gamma ]\delta \epsilon }\coloneqq \nabla _\alpha R_{\beta \gamma \delta \epsilon }+ \nabla _\beta R_{\gamma \alpha \delta \epsilon }+ \nabla _\gamma R_{\alpha \beta \delta \epsilon }= 0\).\footnote{This notation is a bit cryptic: see \href{https://en.wikipedia.org/wiki/Ricci_calculus}{Ricci calculus}.}
\end{proposition}

Moreover, we can also talk about exchanging two indices.

\begin{proposition}
	Given the \hyperref[def:Riemannian-curvature]{Riemannian curvature tensor} \(R\),
	\begin{enumerate}[(a)]
		\item \(R(X, Y)Z = -R(Y, X)Z\), i.e., \(R_{k \ell i j} = -R_{k \ell j i}\);
		\item \(\left\langle R(X, Y)Z, W \right\rangle = -\left\langle R(X, Y)W, Z \right\rangle\), i.e., \(R_{k \ell i j} = -R_{\ell k i j}\);
		\item \(\left\langle R(X, Y)Z, W \right\rangle = -\left\langle R(Y, X)Z, W \right\rangle\), i.e., \(R_{k \ell i j} = -R_{\ell k j i}\);
		\item \(\left\langle R(X, Y)Z, W \right\rangle = -\left\langle R(Z, W)X, Y \right\rangle\), i.e., \(R_{k \ell i j} = R_{i j \ell k}\).
	\end{enumerate}
\end{proposition}
\begin{proof}
	See also do Carmo~\cite[Proposition 2.5]{flaherty2013riemannian}.
\end{proof}

\subsection{Other Curvatures}
There are other notions of curvature, but they all depend on the \hyperref[def:Riemannian-curvature]{Riemannian curvature}, and appearing to be some sorts of ``average'' of \(R\). We have already seen the first one.

\begin{definition}[Riemannian-Christoffel curvature]\label{def:Riemannian-Christoffel-curvature}
	The \emph{Riemannian-Christoffel curvature}	is defined by
	\[
		R_{k \ell ij}
		\coloneqq g_{km} R^m_{\ell ij}
		= \left\langle R\left( \frac{\partial }{\partial x^i}, \frac{\partial }{\partial x^j} \right) \frac{\partial }{\partial x^{\ell } }, \frac{\partial }{\partial x^k} \right\rangle.
	\]
\end{definition}

\begin{definition}[Ricci curvature]\label{def:Ricci-curvature}
	The \emph{Ricci curvature} is defined by \(R_{ab} = g^{cm}R_{c a m b} = R^m_{amb}\).
\end{definition}

\begin{definition}[Ricci scalar curvature]\label{def:Ricci-scalar-curvature}
	The \emph{(Ricci) scalar curvature} is defined by \(R = g^{ab}R_{ab}\).
\end{definition}

\begin{note}
	For a more formal treatment, see do Carmo~\cite[\defaultS 4.4]{flaherty2013riemannian}.\footnote{Notice that the order in do Carmo~\cite{flaherty2013riemannian} is a bit different: it introduces \hyperref[def:sectional-curvature]{sectional curvature} first.}
\end{note}

\section{Flows of Vector Fields}
Let \(\mathcal{M} \) be a \hyperref[def:smooth-manifold]{smooth manifold}, and \(X\) a \hyperref[def:vector-field]{vector field} on \(\mathcal{M} \). Then \(X\) defines a first order differential equation\footnote{If \(\dim \mathcal{M} > 1\), it is a system of first order differential equations.}
\[
	\dot{c} = X(c).
\]
And this ODE has a solution, as guaranteed by \autoref{prop:vector-ODE-has-solution}.

\begin{proposition}\label{prop:vector-ODE-has-solution}
	For all \(p\in \mathcal{M} ^d\), there exists an open interval \(I = I_p \subseteq \mathbb{R} \) with \(0\in I_p\) such that a \hyperref[def:curve]{smooth curve} \(c\colon I_p \to \mathcal{M} \) solves
	\[
		\begin{dcases}
			\frac{\mathrm{d}c(t)}{\mathrm{d}t} = X(c(t)), & t\in I; \\
			c(0) = p.
		\end{dcases}
	\]
	Further, the solution depends smoothly on the initial data (i.e., \(p\)).\footnote{This directly follows from ODE theory.}
\end{proposition}
\begin{proof}
	For all \(p\in \mathcal{M} \), we want to find an open interval \(I = I_p\) around \(0\in \mathbb{R} \) and a solution of the following ODE for \(c\colon I \to \mathcal{M} \):
	\[
		\begin{dcases}
			\frac{\mathrm{d}c(t)}{\mathrm{d}t} = X(c(t)), & t\in I; \\
			c(0) = p.
		\end{dcases}
	\]
	We can check in \hyperref[def:coordinate-chart]{local coordinates} that this is a system of ODE. In such \hyperref[def:coordinate-chart]{coordinates}, let \(c(t)\) be given by \(c(t) = \big(c^1(t), c^2(t), \dots , c^d(t)\big)\). Let \(X \eqqcolon X^i \partial / \partial x^i\), then the above system becomes
	\[
		\frac{\mathrm{d}c^i(t)}{\mathrm{d}t} = X^i\big(c(t)\big) ,\quad i = 1, \dots , d.
	\]
	From the \href{https://en.wikipedia.org/wiki/Picard%E2%80%93Lindel%C3%B6f_theorem}{Picard-Lindelöf theorem}, with the initial data \(c(0)=p\), there is a unique solution.	
\end{proof}

\begin{proposition}\label{prop:vector-ODE-has-solution-in-nbh}
	For all \(p\in \mathcal{M} \), there exists an open neighborhood \(U\) of \(p\) and an open interval \(I_p\) with \(0\in I_p\) such that for all \(q\in U\), the \hyperref[def:curve]{curve} \(c_q\) with
	\[
		\dot{c}_q(t) = X(c_q(t)),\quad c_q(0) = q
	\]
	is defined on \(I\) and the map \(c\colon I \times U \to \mathcal{M} \), \((t, q) \mapsto c_q(t)\) is smooth.
\end{proposition}

\autoref{prop:vector-ODE-has-solution-in-nbh} suggests the following definition.

\begin{definition}[Local flow]\label{def:local-flow}
	The map \(c_q(t) \colon I\times U \to \mathcal{M} \), \((t, q)\mapsto c_q(t)\) from \autoref{prop:vector-ODE-has-solution-in-nbh} is called the \emph{local flow} of the \hyperref[def:vector-field]{vector field} \(X\).
\end{definition}

\begin{center}
	\incfig{local-flow}
\end{center}

\begin{definition}[Integral curve]\label{def:integral-curve}
	The \hyperref[def:local-flow]{local flow} \(c_q(t)\) is called the \emph{integral curve} of \(X\) through \(q\).
\end{definition}

\subsection{Local \(1\)-Parameter Groups}
Now, given a \hyperref[def:local-flow]{local flow} \(c_q(t)\) of a \hyperref[def:vector-field]{vector field} \(X\), by fixing \(t\), we can vary \(q\) and see the following.

\begin{theorem}\label{thm:local-1-parameter-group}
	Let \(\varphi _t(q) \coloneqq c_q(t)\) such that \(\varphi _t \circ \varphi _s(q) = \varphi _{t+s}(q)\) for \(s, t, (t+s)\in I_q\). If \(\varphi _t\) is defined on \(U \subseteq \mathcal{M} \), it maps \(U\) \hyperref[def:diffeomorphic]{diffeomorphically} onto its image.
\end{theorem}

We see that \(\varphi _t\) defines a family of \hyperref[def:diffeomorphism]{diffeomorphism} around \(p\), which gives the following.

\begin{definition}[Local \(1\)-parameter group]\label{def:local-1-parameter-group}
	A family \((\varphi _t)_{t\in I}\) of \hyperref[def:diffeomorphism]{diffeomorphism} from \(\mathcal{M} \) to \(\mathcal{M} \) satisfying \autoref{thm:local-1-parameter-group} is called a \emph{local \(1\)-parameter group} of \hyperref[def:diffeomorphism]{diffeomorphisms}.
\end{definition}

In general, a \hyperref[def:local-1-parameter-group]{local \(1\)-parameter group} needs not be extendible to a group because the maximum interval \(I = I_q\) in \autoref{def:local-1-parameter-group} need not be all of \(\mathbb{R} \).

\begin{eg}
	Let \(\mathcal{M} = \mathbb{R} \), \(X(t) = \tau ^2 \mathrm{d} / \mathrm{d} \tau \). Then the solution of \(\dot{c}(t) = c^2(t)\) is not defined over all \(\mathbb{R} \).
\end{eg}

To get the whole group structure, consider the following.

\begin{theorem}
	Let \(X\) be a \hyperref[def:vector-field]{vector field} on a \hyperref[def:smooth-manifold]{smooth manifold} \(\mathcal{M} \) with a compact support. Then the corresponding \hyperref[def:local-flow]{local flow} is defined for every \(q\in \mathcal{M} \) and \(t\in \mathbb{R} \), and the \hyperref[def:local-1-parameter-group]{local \(1\)-parameter group} becomes a group of \hyperref[def:diffeomorphism]{diffeomorphisms}.
\end{theorem}
\begin{proof}
	By using \(\supp(X) \subseteq K\), \(K\) compact, we can cover \(K\) by a finite covering, then using \autoref{prop:vector-ODE-has-solution-in-nbh}, we're done.
\end{proof}

This leads to the following.

\begin{corollary}
	On a compact \hyperref[def:smooth-manifold]{differentiable manifold} \(\mathcal{M} \), any \hyperref[def:vector-field]{vector field} generates a \hyperref[def:local-1-parameter-group]{local \(1\)-parameter group}.
\end{corollary}