\lecture{11}{9 Feb. 14:30}{Parallel Transport}
Let's first transform \autoref{eq:geodesic} into \(1^{st} \) order system on the \hyperref[def:cotangent-space]{cotangent} \hyperref[def:vector-bundle]{bundle} \(T^{\ast} \mathcal{M} \), and locally trivialize \(T^{\ast} \mathcal{M} \) by \hyperref[def:coordinate-chart]{chart} \(\at{T^{\ast} \mathcal{M} }{U}{} \cong U \times \mathbb{R} ^d\) with coordinates \((x^1, \ldots , x^d, p_1, \ldots , p_d)\). Set
\begin{equation}\label{eq:Hamilton}
	H(x, p) = \frac{1}{2} g^{ij} (x) p_i p_j,
\end{equation}
recall \(g^{ik} g_{kj} = \delta ^i_j \).

\begin{theorem}
	\autoref{eq:geodesic} is equivalent to the system on \(T^{\ast} \mathcal{M} \):
	\begin{equation}\label{eq:cogeodesic}
		\begin{dcases}
			\dot{x}^i = \frac{\partial H}{\partial p_i} g^{ij}(x) p_j; \\
			\dot{p}_i = -\frac{\partial H}{\partial x^i} = -\frac{1}{2} g^{jk}_{, i}(x)p_j p_k.
		\end{dcases}
	\end{equation}
\end{theorem}

\begin{remark}[Cogeodesic flow]\label{rmk:cogeodesic-flow}
	The \hyperref[def:local-flow]{flow} determined by \autoref{eq:cogeodesic} is called the \emph{cogeodesic flow}.
\end{remark}

\begin{note}
	The \hyperref[def:geodesic]{geodesic} \hyperref[def:local-flow]{flow} on \(T\mathcal{M} \) is obtained from the \hyperref[rmk:cogeodesic-flow]{cogeodesic flow} by the first equation in \autoref{eq:cogeodesic}.
\end{note}

\begin{remark}[Hamiltonian flow]
	The \hyperref[rmk:cogeodesic-flow]{cogeodesic flow} is a \emph{Hamiltonian flow} for the Hamiltonian \(H\).
\end{remark}
\begin{explanation}
	By \autoref{eq:cogeodesic}, along the \hyperref[def:integral-curve]{integral curves},
	\[
		\frac{\mathrm{d}H}{\mathrm{d}t}
		= H_{x^i} \dot{x}^i + H_{p_i}\dot{p}^i
		= -\dot{p}_i x\dot{x}^i + \dot{x}^i \dot{p}_i = 0.
	\]
	Observe that the \hyperref[rmk:cogeodesic-flow]{cogeodesic flow} maps the set
	\[
		E_\lambda \coloneqq \left\{ (x, p)\in T^{\ast} \mathcal{M} \mid H(x, p) = \lambda  \right\}
	\]
	onto itself for all \(\lambda \geq 0\).
\end{explanation}

\begin{remark}
	If \(\mathcal{M} \) is compact, then all \(E_\lambda \) are compact, then all \hyperref[def:geodesic]{geodesic} \hyperref[def:local-flow]{flow} define don all of \(E_\lambda \) for all \(\lambda \). Also, \(\mathcal{M} = \bigcup_{\lambda \geq 0} P E_\lambda \) for \(P\) being the projection.
\end{remark}

\begin{definition}[Vector field along a curve]\label{def:vector-field-along-a-curve}
	A \hyperref[def:vector-field]{vector field} along a \hyperref[def:curve]{curve} \(c\colon I \subseteq \mathbb{R} \to \mathcal{M} \) is \(X\colon I \to T \mathcal{M} \) such that \(X(t)\in T_{c(t)} \mathcal{M} \) for all \(t\in I\).
\end{definition}

\begin{notation}
	Denote the set of smooth \hyperref[def:vector-field-along-a-curve]{vector field along \(c\)}  by \(\chi _c(\mathcal{M} )\).
\end{notation}

\begin{theorem}\label{thm:lec11}
	Let \((\mathcal{M} , g)\) be a \hyperref[def:Riemannian-manifold]{Riemannian manifold}, \(\mathrm{D}\) is the canonical \hyperref[def:Levi-Civita-connection]{(Levi-Civita) connection} and \(c\) a \hyperref[def:curve]{curve} in \(\mathcal{M} \). Then there exists a unique operator \(\mathrm{D} / \mathrm{d} t\) defined as the vector space of \hyperref[def:vector-field-along-a-curve]{vector field along \(c\)} satisfying
	\begin{enumerate}[(i)]
		\item \label{thm:lec11-i}\begin{enumerate}[(a)]
			      \item \label{thm:lec11-i-a} \(\frac{\mathrm{D}}{\mathrm{d}t} (fY)(t) = f^\prime (t) Y(t) + f(t) \frac{\mathrm{D} }{\mathrm{d} t} Y(t)\) for all real function \(f\) on \(I\).
			      \item \label{thm:lec11-i-b} \(\frac{\mathrm{D} }{\mathrm{d} t} (V+W) = \frac{\mathrm{D} V}{\mathrm{d} t} + \frac{\mathrm{D} W}{\mathrm{d} t}\).
		      \end{enumerate}
		\item \label{thm:lec11-ii} If there exists a neighborhood of in \(I\) such that \(Y\) is the restriction to \(c\) of a \hyperref[def:vector-field]{vector field} \(X\) defined on a neighborhood of \(c(t_0)\) in \(\mathcal{M} \), then \(\frac{\mathrm{D} }{\mathrm{d} t} Y(t_0) = (\mathrm{D}_{c(t_0)} X) _{c(t_0)}\).
	\end{enumerate}
\end{theorem}

\begin{problem}
Why not just define \(\mathrm{D} Y / \mathrm{d} t\) by \autoref{thm:lec11-ii}?
\end{problem}
\begin{answer}
	A \hyperref[def:vector-field-along-a-curve]{vector field \(Y\) along a curve} may not always be extended to a neighborhood of \(c\) in \(\mathcal{M} \). But, in \hyperref[def:coordinate-chart]{local coordinates},
	\[
		Y(t) = \sum_{i=1}^{n} Y^i (t) \left( \frac{\partial }{\partial x^i}  \right) _{c(t)},
	\]
	which shows that a \hyperref[def:vector-field-along-a-curve]{vector field along \(c\)} always a linear combination of \hyperref[def:vector-field-along-a-curve]{vector fields along \(c\)} that can be extended.
\end{answer}

\begin{prev}
	Let \(X = X^i \partial _i\), \(V = V^k \partial _k\), and let \(\mathrm{D} \) be the \hyperref[def:Levi-Civita-connection]{Levi-Civita connection}. Then
	\[
		\mathrm{D}_V X
		= \mathrm{D} _V(X^i \partial _i)
		= V(X^i) \partial _i + X^i \underbrace{\mathrm{D} _V \partial _i}_{V^k D_{\partial _k} \partial _i}
		= V(X^i)\partial _i + V^k X^i \Gamma ^j_{ki}\partial _j.
	\]
\end{prev}
\begin{proof}
	The covariant derivative along \(c\) is defined by
	\[
		\frac{\mathrm{D}}{\mathrm{d}t} (Y^i (t)\partial _i)
		= \frac{\mathrm{d}V^i}{\mathrm{d}t} \partial _i + \dot{c}Y^i \Gamma _{ji}^k(c(t)) \partial _k,
	\]
	where \(\dot{c} = \dot{c}^k \partial _k\). This shows \autoref{thm:lec11-i} \autoref{thm:lec11-i-a} and \autoref{thm:lec11-i-b} hold. Next, to show \autoref{thm:lec11-ii}, let \(x\) be a smooth \hyperref[def:vector-field]{vector field} in \(\mathcal{M} \). Then the induced \hyperref[def:vector-field-along-a-curve]{vector field along \(c\)} is given by \(Y(t)=X_{c(t)}\), i.e., in terms of the coordinate basis, we have
	\[
		Y(t) = Y^i(t)\partial _i,\quad X_x= X^i(x)\partial _i,\quad Y^i(t) = X^i (c(t)).
	\]
	Then,
	\[
		\begin{split}
			\mathrm{D} _i X
			&= \mathrm{D} _i (X^i \partial _i)
			= \dot{c}(Xri)\partial _i + X^i \mathrm{D} _i \partial _i
			= X^i \dot{c}^k \underbrace{\mathrm{D} _{\partial _k} \partial _i}_{\Gamma _{ki}^{\ell } \partial _\ell }\\
			&= \partial _t(X^i \circ c)\partial _i + \dot{c}^k X^i \Gamma ^\ell _{ki}\partial _\ell
			= \partial _t(X^i \circ c)\partial _i + \dot{c}^k Y^i \Gamma ^\ell _{ki}\partial _\ell
			= \frac{\mathrm{D}}{\mathrm{d} t} Y.
		\end{split}
	\]
\end{proof}

\begin{definition}[Parallel]\label{def:parallel}
	A \hyperref[def:vector-field-along-a-curve]{vector field along \(c\)} is called \emph{parallel} if \(\mathrm{D} X/ \mathrm{d} t = 0\).
\end{definition}

\begin{definition}[Parallel transport]\label{def:parallel-transport}
	The \emph{parallel transport} from \(c(0)\) to \(c(t)\) along the \hyperref[def:curve]{curve} \(c\) in \((\mathcal{M} , g)\) is the linear map \(P_i \colon T_{c(0)} \mathcal{M} \to T_{c(t) }\mathcal{M} \) associating to \(v\in T_{c(0)} \mathcal{M} \) the vector \(X_v\) with \(X_v\) being the \hyperref[def:parallel]{parallel} \hyperref[def:vector-field-along-a-curve]{vector field along \(c\)} such that \(X_v (0) =v\).
\end{definition}

\begin{proposition}
	The \hyperref[def:parallel-transport]{parallel transport} defines for all \(t\) an isometry from \(T_{c(0)} \mathcal{M} \) onto \(T_{c(t)} \mathcal{M} \). More generally, if \(X, Y\) \hyperref[def:vector-field-along-a-curve]{vector fields along \(c\)}, then
	\[
		\frac{\mathrm{d}}{\mathrm{d}t} g(x(t), y(t))
		= g\left( \frac{\mathrm{D} X(t)}{\mathrm{d} t}, Y(t)\right) + g\left(X(t), \frac{\mathrm{D} Y(t)}{\mathrm{d} t} \right) .
	\]
\end{proposition}