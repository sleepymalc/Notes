\lecture{14}{21 Feb. 14:30}{Second Fundamental Forms}
\begin{proposition}
	Let \((\mathcal{N}, h )\) be a \hyperref[def:Riemannian-manifold]{Riemannian manifold} and \(G\) be a \href{https://en.wikipedia.org/wiki/Free_group}{free} and \href{https://mathworld.wolfram.com/ProperGroupAction.html}{proper} group of \hyperref[def:isometry]{isometries} of \((\mathcal{N} , h)\), then there exists a unique \hyperref[def:Riemannian-metric]{Riemannian metric} \(g\) on the quotient manifold \(\mathcal{M} = \quotient{\mathcal{N} }{G} \) such that the connected projection \(p\colon \mathcal{N}  \to  \mathcal{M} \) is a \hyperref[def:Riemannian-covering-map]{Riemannian covering map}.
\end{proposition}
\begin{proof}
	Let \(n, n^\prime \in \mathcal{N} \) such that \(n, n^\prime \in p ^{-1} (m)\) for \(m\in \mathcal{M} \). Hence, there exists an \hyperref[def:isometry]{isometry} \(f\in G\) such that \(f(n) = n^\prime \). Also, \(p \circ f = p\), and \(p\) is a local \hyperref[def:diffeomorphism]{diffeomorphism}, so we can define a scalar product \(g_m\) on \(T_m \mathcal{M} \): for all \(u, v\in Tvm \mathcal{M} \),
	\[
		g_m(u, v) = h_n\big( (T_n p)^{-1} u, (T_n p)^{-1} v\big)
	\]
	for \(n\in p ^{-1} (m)\). This does not depend on the choice of \(n\in p ^{-1} (m)\) since \((T_n p)^{-1} = T_n f \circ (T_n p)^{-1} \) and \(T_n f\) is an \hyperref[def:isometry]{isometry} of the Euclidean vector spaces  \(T_n \mathcal{N} \) and \(T_{n^\prime } \mathcal{N} \). It can be shown that \(g\) is smooth. Thus, we have constructed a \hyperref[def:Riemannian-metric]{metric} \(g\) on \(\mathcal{M} \) such that \(p\) is a \hyperref[def:Riemannian-covering-map]{Riemannian covering map}, which is unique.
\end{proof}

\begin{definition}[Totally geodesic]\label{def:totally-geodesic}
	A \hyperref[def:submanifold]{submanifold} \(\mathcal{M} \) of \((\widetilde{m} , \widetilde{g} )\) is called \emph{totally geodesic} if for all \(m\in \mathcal{M} \) and \(v\in T_m \mathcal{M} \), the \hyperref[def:geodesic]{geodesic} \(c\) of \((\widetilde{M}, g)\) with \(c(0) = m\) and \(c^\prime (0) = v\) is contained fully in \(\mathcal{M} \).
\end{definition}

\begin{proposition}
	Let \(p\colon (\mathcal{N} , h) \to (\mathcal{M} , g)\) be a \hyperref[def:Riemannian-covering-map]{Riemannian covering map}. The \hyperref[def:geodesic]{geodesic} of \((\mathcal{M} , g)\) are the projections of the \hyperref[def:geodesic]{geodesic} in \((\mathcal{N} , h)\); and the \hyperref[def:geodesic]{geodesic} of \((\mathcal{N} , h)\) are the liftings of those in \((\mathcal{M} , g)\).
\end{proposition}
\begin{proof}
	Since \(p\) is a \hyperref[def:local-isometry]{local isometry}, if \(\gamma \) is a \hyperref[def:geodesic]{geodesic} of \(\mathcal{N} \), then \(c = p \circ \gamma \) is also a \hyperref[def:geodesic]{geodesic} of \(\mathcal{M} \). From the \hyperref[thm:geodesic-existence-uniqueness]{uniqueness theorem} for \hyperref[def:geodesic]{geodesics} shows that these are indeed the only \hyperref[def:geodesic]{geodesics} on \(\mathcal{M} \). Conversely, if \(p \circ \gamma \) is a \hyperref[def:geodesic]{geodesic} in \(\mathcal{M} \), then \(\gamma \) is a \hyperref[def:geodesic]{geodesic} in \(\mathcal{N} \).
\end{proof}

\begin{eg}
	In Euclidean spaces, the \hyperref[def:totally-geodesic]{totally geodesic} \hyperref[def:submanifold]{submanifold} are affine linear subspaces and their open subsets.
\end{eg}

\begin{eg}
	Each closed \hyperref[def:geodesic]{geodesic} in \hyperref[def:Riemannian-manifold]{Riemannian manifolds} defines a \(1\)-dimensional compact \hyperref[def:totally-geodesic]{totally geodesic} \hyperref[def:submanifold]{submanifold}.
\end{eg}

\begin{eg}
	The \hyperref[def:totally-geodesic]{totally geodesic} \hyperref[def:submanifold]{submanifolds} of \(S^n \subseteq \mathbb{R} ^{n+1}\) are the intersections of \(S^n\) with linear subspaces of \(\mathbb{R} ^{n+1}\).
\end{eg}

\begin{eg}
	In general, \hyperref[def:Riemannian-manifold]{Riemannian manifolds} do not have any \hyperref[def:totally-geodesic]{totally geodesic} \hyperref[def:submanifold]{submanifolds} of dimensional \(> 1\).
\end{eg}

\begin{note}
	We will see that \(\mathcal{M} \) is \hyperref[def:totally-geodesic]{totally geodesic} in \(\widetilde{M} \) if and only if all the \hyperref[def:2nd-fundamental-form]{\(2^{nd}\)-fundamental forms} vanish identically.
\end{note}

\section{The Second Fundamental Form}
Let \(\mathcal{M} ^m \subseteq \mathcal{N} ^n\) be two \hyperref[def:Riemannian-manifold]{Riemannian manifolds}, and we know that a \hyperref[def:Riemannian-metric]{metric} on \(N\) induces a \hyperref[def:Riemannian-metric]{metric} on \(\mathcal{M} \) naturally. Now, we want to see that given the \hyperref[def:Levi-Civita-connection]{Levi-Civita connection} \(\nabla ^\mathcal{N} \) of \(\mathcal{N} \), how to get \(\nabla ^\mathcal{M} \) of \(\mathcal{M} \).

This is given by the central object \((\nabla ^\mathcal{N} _X Y)^T\) we will study in this chapter, where \(T\colon T_x \mathcal{N} \to T_x \mathcal{M} \) for \(x\in \mathcal{M} \) is the orthogonal projection. We see the following.

\begin{theorem}
	For \(X, Y\in \Gamma (T \mathcal{M} )\), \(\nabla ^\mathcal{M} _X Y = \left( \nabla ^\mathcal{N} _X Y \right)^{T}\).
\end{theorem}
\begin{proof}
	Firstly, we have to make sure that the right-hand side is defined. This can be done by extending \hyperref[def:vector-field]{vector fields} \(X, Y\) locally to a neighborhood of \(\mathcal{M} \) in \(\mathcal{N} \). Do this in the \hyperref[def:local-coordinate]{local coordinates} around \(x\in \mathcal{M} \) locally mapping \(\mathcal{M} \) to \(\mathbb{R} ^m \subseteq \mathbb{R} ^n\).

	Specifically, the extension of \(X = \xi ^i(x) \partial / \partial x^i\) is
	\[
		\widetilde{X} (x^1, \ldots , x^n) = \sum_{i=1}^{m} \xi ^i(x^1, \ldots , x^n) \frac{\partial }{\partial x^i}.
	\]
	Then \(\langle \widetilde{X} , \widetilde{Y} \rangle (x) = \left\langle X, Y \right\rangle (x)\) and \([\widetilde{X} , \widetilde{Y} ](x) = [X, Y](x)\). From \hyperref[thm:Levi-Civita]{Levi-Civita theorem}, the \hyperref[eq:Koszul-formula]{Koszul formula} holds for both \(\mathcal{N} \) and \(\mathcal{M} \). Finally, we see that
	\begin{itemize}
		\item \((\nabla _X^{\mathcal{N}} Y)^T\) does not depend on the chosen extensions: follows from the fact that the representation of \(\nabla ^\mathcal{N} \) is done by \(\Gamma \);
		\item \((\nabla _X^{\mathcal{N}} Y)^T\) defines a \hyperref[def:torsion-free]{torsion-free} \hyperref[def:linear-connection]{connection} on \(\mathcal{M} \): as \(\nabla ^\mathcal{N} _X Y - \nabla ^cal_Y X - [X, Y]\) vanishes, also the tangential part to \(\mathcal{M} \) has to vanish.
	\end{itemize}
\end{proof}

Let \(\nu (x)\) be a \hyperref[def:vector-field]{vector field} in a neighborhood of \(x_0\in \mathcal{M} \subseteq \mathcal{N} \) that is orthogonal to \(\mathcal{M} \), i.e., \(\left\langle \nu (x), X \right\rangle = 0\) for all \(X\in T_x \mathcal{M} \). Also, let \(T_x \mathcal{M} ^{\perp} \) be the orthogonal complement of \(T_x \mathcal{M} \) in \(T_x \mathcal{N} \), and \(T \mathcal{M} ^{\perp} \) with \hyperref[def:fiber]{fiber} \(T_x \mathcal{M} ^{\perp} \) of \(x\in \mathcal{M} \).

\begin{center}
	\incfig{submanifold-2nd-fundamental-form}
\end{center}

\begin{notation}[Normal bundle]\label{not:normal-bundle}
	\(T \mathcal{M} ^{\perp} \) is the \emph{normal bundle} of \(\mathcal{M} \) in \(\mathcal{N} \).
\end{notation}

We see that \(\left\langle \nu (x), X \right\rangle = 0\) for all \(X\in T_x \mathcal{M} \) means \(\nu (x)\in T_x \mathcal{M} ^{\perp} \).

\begin{lemma}
	\((\nabla _X ^\mathcal{N} \nu )^T(x)\) only depends on \(\nu (x)\).
\end{lemma}
\begin{proof}
	This follows directly from
	\[
		(\nabla _X ^\mathcal{N} f \nu )^T(x)
		= \left( X(f)(x) \nu (x) \right) ^T + f(x) (\nabla _X ^\mathcal{N} \nu )^T (x)
		= f(x) (\nabla _X ^\mathcal{N} \nu )^T (x)
	\]
	for \(f\) smooth, since \(\left( X(f)(x) \nu (x) \right) ^T = 0\).
\end{proof}

\begin{definition}[Second fundamental tensor]\label{def:2nd-fundamental-tensor}
	The \emph{second fundamental tensor} \(S\colon T_x \mathcal{M} \times T_x \mathcal{M} ^{\perp} \to T_x \mathcal{M} \) of \(\mathcal{M} \) at point \(x\in \mathcal{M} \) is defined by
	\[
		S(X, \nu ) = (\nabla ^\mathcal{N} _X \nu )^T.
	\]
\end{definition}

\begin{lemma}\label{lma:2nd-fundamental-form}
	For \(X, Y\in T_x \mathcal{M} \), \(\ell _\nu (X, Y) \coloneqq \left\langle S(X, \nu) , Y\right\rangle \) is symmetric in \(X, Y\).
\end{lemma}
\begin{proof}
	Since
	\[
		\ell _\nu (X, Y)
		= \langle (\nabla ^\mathcal{N} _X \nu )^T, Y \rangle
		= \langle \nabla _X^{\mathcal{N} } \nu , Y  \rangle
		= - \langle \nu , \nabla ^\mathcal{N} _X Y \rangle
	\]
	as \(\nabla ^\mathcal{N} \) is \hyperref[def:Riemannian]{metric} and \(\left\langle \nu , Y \right\rangle=0 \). Now, since \(\nabla ^\mathcal{N} \) is \hyperref[def:torsion-free]{torsion-free}, we further have
	\[
		\ell _\nu (X, Y)
		= - \langle \nu , \nabla ^\mathcal{N} _Y X + [X, Y] \rangle
		= - \langle \nu , \nabla ^\mathcal{N} _Y X \rangle - \langle \nu , [X, Y] \rangle
		= - \langle \nu , \nabla ^\mathcal{N} _Y X \rangle
	\]
	as \(\nu \in T_x \mathcal{M} ^{\perp}, [X, Y]\in T_x \mathcal{M}\), so \(\langle \nu , [X, Y] \rangle=0\). Finally, since again, \(\nabla ^\mathcal{N} \) is \hyperref[def:Riemannian]{metric},
	\[
		\ell _\nu (X, Y)
		= \langle \nabla _Y^{\mathcal{N} } \nu , X  \rangle
		= \langle (\nabla _Y^{\mathcal{N} } \nu)^T , X  \rangle
		= \ell _\nu (Y, X).
	\]
\end{proof}

\begin{definition}[Second fundamental form]\label{def:2nd-fundamental-form}
	The \emph{second fundamental form} \(\ell _\nu (\cdot, \cdot)\) of \(\mathcal{M} \) in \(\mathcal{N} \) is defined as \(\ell _\nu (X, Y) \coloneqq \left\langle S(X, \nu ), Y \right\rangle \).
\end{definition}

Now, fix a \hyperref[not:normal-bundle]{normal field} \(\nu \), and let \(S_\nu (X) \coloneqq S(X, \nu )\), then
\[
	S_\nu \colon T_x \mathcal{M} \to T_x \mathcal{M}
\]
is self-adjoint w.r.t.\ the \hyperref[def:Riemannian-metric]{metric} \(\left\langle \cdot, \cdot \right\rangle \) by \autoref{lma:2nd-fundamental-form}.

\begin{definition*}
	Assume that \(\left\langle \nu , \nu  \right\rangle \equiv 1\), i.e., \(\nu \) is the unit \hyperref[not:normal-bundle]{normal field}, then \(S_\nu \) has \(m\) real eigenvalues.

	\begin{definition}[Principal curvature]\label{def:principal-curvature}
		The eigenvalues are called \emph{principal curvatures} of \(\mathcal{M} \) in direction \(\nu \).
	\end{definition}

	\begin{definition}[Principal curvature vector]\label{def:principal-curvature-vector}
		The corresponding eigenvectors are called \emph{principal curvature vectors} of \(\mathcal{M} \) in direction \(\nu \).
	\end{definition}
\end{definition*}


\begin{definition}[Mean curvature]\label{def:mean-curvature}
	The \emph{mean curvature} of \(\mathcal{M} \) in direction \(\nu \) is defined by
	\[
		H_\nu \coloneqq \frac{1}{m} \Tr S_\nu .
	\]
\end{definition}

\begin{definition}[Gauss-Kronecker curvature]\label{def:Gauss-Kronecker-curvature}
	The \emph{Gauss-Kronecker curvature} of \(\mathcal{M} \) in direction \(\nu \) is defined by
	\[
		K _\nu \coloneqq \det S_\nu .
	\]
\end{definition}