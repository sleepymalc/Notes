\lecture{15}{23 Feb. 13:00}{The Second Fundamental Form}
\subsection{Totally Geodesic and Second Fundamental Form}
Let \(\dim \mathcal{N} = m+1\), \(\dim \mathcal{M} = m\), then for all \(x\in \mathcal{M} \), there are exactly \(2\) normal vectors \(\nu \in T_x \mathcal{M} ^{\perp} \) with \(\left\langle \nu, \nu \right\rangle \equiv 1\), i.e., \(\nabla _X^{\mathcal{N} } \nu \) always tangential to \(\mathcal{M} \). Now, we fix locally such a \hyperref[not:normal-bundle]{normal field} and drop the subscript \(\nu \) in the following discussion.

\begin{note}
	If we choose an opposite \hyperref[not:normal-bundle]{normal field}, then \(\ell \), \(S\), and \hyperref[def:mean-curvature]{mean curvature} will change their sign. However, for even \(m\), the \hyperref[def:Gauss-Kronecker-curvature]{Gauss-Kronecker curvature} does not depend on the choice of the direction of \(\nu \).
\end{note}

\begin{intuition}
	\(\nabla _X^{\mathcal{N} } \nu \) measures the ``tilting velocity'' with which \(\nu \) is tilted relative to a fixed \hyperref[def:parallel]{parallel} \hyperref[def:vector-field]{vector field} in \(\mathcal{N} \), when on \(\mathcal{M} \) in direction \(X\).
\end{intuition}

\begin{theorem}\label{thm:totally-geodesic-second-fundamental-form}
	Given \(\mathcal{M} \subseteq \widetilde{\mathcal{M}} \), then \(\mathcal{M} \) is \hyperref[def:totally-geodesic]{totally geodesic} in \(\widetilde{\mathcal{M}} \) if and only if all \hyperref[def:2nd-fundamental-form]{second fundamental form} of \(\mathcal{M} \) vanish identically.
\end{theorem}
\begin{proof}
	Let \(c\colon I \to \mathcal{M} \) be a \hyperref[def:geodesic]{geodesic} in \(\mathcal{M} \), i.e., \(\nabla _{\dot{c} }^{\mathcal{M}} \dot{c} = 0\). By \autoref{thm:immersion-induced-connection}, \(\nabla _{\dot{c} }^{\mathcal{M}} \dot{c} = (\nabla _{\dot{c} }^{\widetilde{\mathcal{M}} } \dot{c} )^{\top} = 0\), implying \(c\) is a \hyperref[def:geodesic]{geodesic} in \(\widetilde{\mathcal{M}} \) if and only if \((\nabla _{\dot{c} }^{\widetilde{\mathcal{M}} } \dot{c} )^{\top} = 0\), i.e., \(\langle \nabla _{\dot{c} }^{\widetilde{\mathcal{M}} } \dot{c} , \nu \rangle = 0\) for all \(\nu \in T \mathcal{M} ^{\perp} \). Notice that
	\begin{itemize}
		\item \(\langle \dot{c} , \nu \rangle = 0\), and hence
		\item \(\dot{c} \langle \dot{c} , \nu \rangle = \langle \nabla _{\dot{c} }^{\widetilde{\mathcal{M}} } \dot{c} , \nu \rangle + \langle \dot{c} , \nabla _{\dot{c} }^{\widetilde{\mathcal{M}} } \nu \rangle = 0\).
	\end{itemize}
	In all, we have \(0 = \langle \nabla _{\dot{c}}^{\widetilde{\mathcal{M}} } \dot{c} , \nu \rangle = -\langle \dot{c} , \nabla _{\dot{c} }^{\widetilde{\mathcal{M}} } \nu \rangle = -\ell _{\nu } (\dot{c}, \dot{c} )\), proving the theorem.
\end{proof}

\begin{note}
	\autoref{thm:totally-geodesic-second-fundamental-form} also holds for \hyperref[def:Lorentzian]{Lorentzian manifolds} \((\widetilde{\mathcal{M}} , \widetilde{g} )\).
\end{note}

\begin{eg}[Initial value problem for Einstein equations]
	Given a \((\widetilde{\mathcal{M}} ^4, \widetilde{g} )\) a \hyperref[def:Lorentzian]{Lorentzian manifolds} satisfying Einstein equations, and a \((\mathcal{M} ^3, g)\) non-degenerate \hyperref[def:Riemannian-manifold]{Riemannian manifold}. If the \hyperref[def:2nd-fundamental-form]{second fundamental form} of \(\mathcal{M} ^3\) in \(\widetilde{\mathcal{M}} ^4\) vanishes identically, then \(\mathcal{M} ^3\) is \hyperref[def:totally-geodesic]{totally geodesic}.\footnote{This is just a special case of \autoref{thm:totally-geodesic-second-fundamental-form}; in general, it does not vanish.}
\end{eg}

\autoref{thm:totally-geodesic-second-fundamental-form} allows us to get what is probably the best geometric interpretation of \hyperref[def:sectional-curvature]{sectional curvature}. Let \(\mathcal{M} \) be a \hyperref[def:Riemannian-manifold]{Riemannian manifold} and let \(p\in \mathcal{M} \). Let \(B \subseteq T_p \mathcal{M} \) be an open ball in \(T_p \mathcal{M} \) on which \(\exp_p\) is a \hyperref[def:diffeomorphism]{diffeomorphism}, and let \(\sigma \subseteq T_p \mathcal{M} \) be a subspace of dimension \(2\). Then, \(\exp_p(\sigma \cap B) = S\) is a \hyperref[def:submanifold]{submanifold} of dimension \(2\) of \(\mathcal{M} \) passing through \(p\).

\begin{intuition}
	\(S\) is the surface formed by ``small'' \hyperref[def:geodesic]{geodesics} that start from \(p\) and are tangent to \(\sigma \) at \(p\).
\end{intuition}

\begin{note}
	By \autoref{thm:totally-geodesic-second-fundamental-form}, \(S\) is \hyperref[def:geodesic]{geodesic} at \(p\), hence the \hyperref[def:2nd-fundamental-form]{second fundamental forms} of the inclusion \(\iota \colon S \subseteq \mathcal{M} \) vanish at \(p\).
\end{note}

As a \hyperref[def:submanifold]{submanifold} of \(\mathcal{M} \), \(S\) has an induced \hyperref[def:Riemannian-metric]{Riemannian metric} whose \hyperref[rmk:Gauss-curvature]{Gauss curvature} at \(p\) will be denoted by \(K_S\). It follows from the Gauss formula~\cite[\defaultS 6 Theorem 2.5]{flaherty2013riemannian}\footnote{Which is just a special case of \hyperref[thm:Gauss-equations]{Gauss' equations}.} that
\[
	K_S(p) = K(p, \sigma ),
\]
i.e., the \hyperref[def:sectional-curvature]{sectional curvature} \(K(p, \sigma )\) is the \hyperref[rmk:Gauss-curvature]{Gauss curvature}, at \(p\), of a small surface formed by \hyperref[def:geodesic]{geodesics} of \(\mathcal{M} \) that start from \(p\) and are tangent to \(\sigma \).

\begin{remark}
	This was exactly the way in which Riemann defined \hyperref[def:sectional-curvature]{sectional curvature}.
\end{remark}

\section{The Fundamental Equations}
Given an \hyperref[def:isometry]{isometric} \hyperref[def:immersion]{immersion} \(f\colon \mathcal{M} ^m \to \mathcal{N} ^n\) with \(n = m + k\), at each \(p\in \mathcal{M} \), we have
\[
	T_p \mathcal{N} = T_p \mathcal{M} \oplus (T_p \mathcal{M} ) ^{\perp},
\]
which varies differentiably with \(p\).

\begin{intuition}
	Locally, the portion of the \hyperref[def:tangent-bundle]{tangent bundle} \(T \mathcal{N} \) which sits over \(\mathcal{M} \) can be decomposed into the direct sum of the \emph{\hyperref[def:tangent-bundle]{tangent bundle}} \(T \mathcal{M} \) and the \emph{\hyperref[not:normal-bundle]{normal bundle}} \(T \mathcal{M} ^{\perp} \).
\end{intuition}

Everything about \hyperref[def:immersion]{immersions} occurs as if the geometry decomposes into two geometries: the geometry of the \hyperref[def:tangent-bundle]{tangent bundle} and the geometry of the \hyperref[not:normal-bundle]{normal bundle}, and these geometries are related by the \hyperref[def:2nd-fundamental-form]{second fundamental form} of the \hyperref[def:immersion]{immersions}.

\begin{notation}
	Greek indices \((\alpha , \beta , \dots )\) occurring twice are summed from \(1\) to \(k\) for \(X, Y, Z, W \in T_x \mathcal{M} \).
\end{notation}

\begin{theorem}[Gauss' equations]\label{thm:Gauss-equations}
	Let \(\mathcal{N} \) be a \hyperref[def:Riemannian-manifold]{Riemannian manifold} with \(\dim \mathcal{N} = n\), and let \(\mathcal{M} \subseteq \mathcal{N} \) be a \hyperref[def:submanifold]{submanifold} with \(\dim \mathcal{M} = m\). Let \(k = n - m\), and \(x\in \mathcal{M} \), \(\nu _1, \dots , \nu _k\) be an orthonormal basis of \((T_x \mathcal{M} )^{\perp} \), \(S_\alpha \coloneq S_{\nu _\alpha }\), \(\ell _\alpha \coloneqq \ell _{\nu _\alpha }\), \(\alpha = 1, \dots , k\). Then,
	\[
		R^{\mathcal{M}}(X, Y)Z - \left( R^{\mathcal{N} } (X, Y)Z \right) ^{\top}
		= \ell _\alpha (Y, Z)S_\alpha (X) - \ell _\alpha (X, Z)S_{\alpha }(Y).
	\]
	Thus, we also have
	\[
		\left\langle R^{\mathcal{M} }(X, Y)Z, W \right\rangle - \left\langle R^{\mathcal{N} }(X, Y)Z, W \right\rangle
		= \ell _\alpha (Y, Z) \ell _\alpha (X, W) - \ell _\alpha (X, Z) \ell _\alpha (Y, W).
	\]
\end{theorem}
\begin{proof}
	We can extend \(X, Y, Z, W\), ad \(\nu , \dots , \nu _k\) to \hyperref[def:vector-field]{vector fields} inn \(T_{\mathcal{M} } \) and \(T \mathcal{M} ^{\perp} \), respectively. Let \(\nu _\alpha \) be orthonormal, then
	\[
		\nabla _Y^{\mathcal{N} } Z
		= (\nabla _Y^{\mathcal{N} } Z)^{\top} = (\nabla _X^{\mathcal{N} } Z)^{\perp}
		= \nabla _Y ^\mathcal{M} Z + \left\langle \nu _\alpha , \nabla _Y^{\mathcal{N} } Z \right\rangle \nu _{\alpha }
	\]
	as \(\nu _\alpha \) form orthonormal basis of \(T \mathcal{M} ^{\perp} \). Hence,
	\[
		\nabla _X ^\mathcal{N} \nabla _Y ^\mathcal{N} Z
		= \nabla _X ^{\mathcal{N} }\nabla _Y^{\mathcal{M} }Z + X(\left\langle \nu _\alpha , \nabla _Y^{\mathcal{N} } Z \right\rangle ) \nu _\alpha + \left\langle \nu _\alpha , \nabla _Y^{\mathcal{N} } Z \right\rangle \nabla _X^{\mathcal{N} } \nu _\alpha.
	\]
	Then,
	\[
		(\nabla _X^{\mathcal{N} } \nabla _Y ^{\mathcal{N} }Z)^{\top}
		= \nabla _X ^\mathcal{M} \nabla _Y ^\mathcal{M} Z + \underbrace{\left\langle \nu _\alpha , \nabla _Y ^\mathcal{N} Z \right\rangle}_{-\ell _\alpha (Y, Z)} \underbrace{(\nabla _X ^\mathcal{N} \nu _\alpha )^{\top} }_{S_\alpha (X)}
		= \nabla _X ^\mathcal{M} \nabla _Y ^\mathcal{M} Z - \ell _\alpha (Y, Z) S_\alpha (X).
	\]
	Analogously, we have
	\[
		(\nabla ^\mathcal{N} _Y \nabla ^\mathcal{N} _X Z)^{\top}
		= \nabla _Y ^\mathcal{M} \nabla _X ^\mathcal{M} Z - \ell _\alpha (X, Z) S_\alpha (Y),
	\]
	and also, we have
	\[
		(\nabla ^\mathcal{N} _{[X, Y]} Z)^{\top} = \nabla ^\mathcal{M} _{[X, Y]} Z.
	\]
	By collecting terms, we have
	\[
		\begin{split}
			(\nabla ^\mathcal{N} _X \nabla ^\mathcal{N} _Y Z) ^{\top} &- (\nabla ^\mathcal{N} _Y \nabla ^\mathcal{N} _X Z) ^{\top} - (\nabla ^\mathcal{N} _{[X, Y]} Z) ^{\top}\\
			&= \nabla _X ^\mathcal{M} \nabla _Y ^\mathcal{M} Z - \nabla _Y ^{\mathcal{M} } \nabla _X ^\mathcal{M} Z - \nabla ^\mathcal{M} _{[X, Y]} Z - \ell _\alpha (Y, Z) S_\alpha (X) + \ell _\alpha (X, Z) S_\alpha (Y),
		\end{split}
	\]
	equivalently,
	\[
		R^{\mathcal{M}} (X, Y)Z - (R^{\mathcal{N} } (X, Y)Z)^{\top} = \ell _\alpha (Y, Z)S_\alpha (X) - \ell _\alpha (X, Z)S_{\alpha }(Y).
	\]
\end{proof}

\autoref{thm:Gauss-equations} tells us that for a surface \(\mathcal{M} \) in \(\mathbb{R} ^3\), the \hyperref[def:Gauss-Kronecker-curvature]{Gauss-Kronecker curvature} coincides with the \hyperref[def:Riemannian-curvature]{Riemannian curvature} of \(\mathcal{M} \), which is independent of the \hyperref[def:embedding]{embedding}. Therefore, \hyperref[def:Gauss-Kronecker-curvature]{Gauss-Kronecker curvature} does not depend on \hyperref[def:embedding]{embeddings} of \(\mathcal{M} \) into \(\mathbb{R} ^3\).

\begin{remark}[Codazzi equations]\label{rmk:Codazzi}
	Let \(\mathcal{M} ^m \subseteq \mathcal{N} ^{m+1}\) where \(N\) is unit normal on \(\mathcal{M} \). Then, the \emph{Codazzi equations} is defined as
	\begin{equation}\label{eq:Codazzi}
		\langle R(X, Y) e_j , N \rangle
		= (\nabla _X ^{\mathcal{M} } \ell ) (Y, e_j) - (\nabla _Y ^{\mathcal{M} } \ell )(X, e_j)
		= X^k Y^i \nabla _k^{\mathcal{M} } \ell _{ij} - Y^k X^i \nabla _k^{\mathcal{M} } \ell _{ij},
	\end{equation}
	i.e., \(\langle R(X, Y) Z, N \rangle = (\nabla _X^{\mathcal{M} } \ell )(Y, Z) - (\nabla _Y^{\mathcal{M} } \ell )(X, Z)\).
\end{remark}

The \hyperref[eq:Codazzi]{Codazzi equations}, together with \hyperref[thm:Gauss-equations]{Gauss' equations}, form the fundamental equations of the local theory of \hyperref[def:isometry]{isometric} \hyperref[def:immersion]{immersions}.