\chapter{Jacobi Fields}
\lecture{16}{7 Mar. 14:30}{Jacobi Field}

\section{Jacobi Fields}

\begin{definition}[Jacobi-field]\label{def:Jacobi-field}
	Let \(\mathcal{M} \) be a \(d\)-dimensional \hyperref[def:Riemannian-manifold]{Riemannian manifold}. Let \(c\colon I \to \mathcal{M} \) be a \hyperref[def:geodesic]{geodesic}. A \hyperref[def:vector-field-along-curve]{vector field \(X\) along \(c\)} is called a \emph{Jacobi field} if
	\begin{equation}\label{eq:Jacobi}
		\nabla _{\frac{\mathrm{d}}{\mathrm{d}t} }\nabla _{\frac{\mathrm{d}}{\mathrm{d}t} }X + R(X, \dot{c} )\dot{c} = 0.
	\end{equation}
\end{definition}

\begin{notation}
	We write \(\dot{X} \coloneqq \nabla _{\frac{\mathrm{d}}{\mathrm{d}t} }X\) and \(\ddot{X} \coloneqq \nabla _{\frac{\mathrm{d}}{\mathrm{d}t} } \nabla _{\frac{\mathrm{d}}{\mathrm{d}t} } X\). So the \hyperref[eq:Jacobi]{Jacobi equation} writes as
	\[
		\ddot{X} + R (X, \dot{c} ) \dot{c} = 0.
	\]
\end{notation}

Then, we can label \(c\) as follows
\begin{center}
	\incfig{2nd-variation}
\end{center}
Notice that we might either fix the endpoints or left them open. Formally, we have the following.

\begin{definition}[Geodesic variation]\label{def:geodesic-variation}
	Let \(\mathcal{M} \) be a \hyperref[def:Riemannian-manifold]{Riemannian manifold}. A variation of \hyperref[def:curve]{curves} \(c\colon (-\epsilon , \epsilon )\times I \to \mathcal{M} \) is called the \emph{geodesic variation} if for all \(s\in (-\epsilon , \epsilon )\), the \hyperref[def:curve]{curve} \(t\mapsto c_s(t) \coloneqq c(s, t)\) is a \hyperref[def:geodesic]{geodesic}.
\end{definition}

In general, for a \hyperref[def:curve]{smooth curve} \(c\colon [a, b] \to \mathcal{M} \) and \(\epsilon > 0\), a \hyperref[def:geodesic-variation]{geodesic variation} of \(c\) is a differentiable map
\[
	F\colon \underbrace{[a, b]}_{t} \times \underbrace{(-\epsilon , \epsilon )}_{s} \to \mathcal{M}
\]
such that \(F(t, 0) = c(t)\) for \(t\in [a, b]\).

\begin{definition}[Proper variation]\label{def:proper-variation}
	A \emph{proper variation} is a \hyperref[def:geodesic-variation]{geodesic variation} where the endpoints are fixed, i.e.,
	\[
		F(a, s) = c(a), F(b, s)=c(b)
	\]
	for all \(s\in (-\epsilon , \epsilon )\).
\end{definition}

\begin{notation}
	We set \(c_s(t) = c(t, s)=F(t, s)\), and
	\begin{itemize}
		\item \(\dot{c}(t, s) \coloneqq \frac{\partial }{\partial t} c(t, s) \), i.e., \(\mathrm{d} F(\partial / \partial t) c(t, s)\), and
		\item \(c^\prime (t, s) = \partial / \partial s c(t, s)\), i.e., \(\mathrm{d} F (\partial / \partial s)c(t, s)\).
	\end{itemize}
\end{notation}

Let \(\mathcal{M} \) be a \hyperref[def:Riemannian-manifold]{Riemannian manifold} of dimension \(d\), and \(\mathcal{H} \) be a \hyperref[def:smooth-manifold]{differentiable manifold}.\footnote{Often times, \(H\) is an interval \(I\subseteq \mathbb{R} \) or a square \(I\times I \subseteq \mathbb{R} ^2\).} Let \(f\colon \mathcal{H}  \to \mathcal{M} \), smooth.\footnote{\(f\) may not be injective.} What is the \hyperref[def:tangent-space]{tangent space} of \(f(\mathcal{H} )\) of point \(p\in f(\mathcal{H} )\)?
\begin{eg}
	Let \(p = f(x) = f(y)\) for \(x \neq y\). For \(f\) being an \hyperref[def:immersion]{immersion}, we may restrict \(f\) to a sufficiently small neighborhood \(U, V\) at \(x, y\), respectively, such that \(f(U), f(V)\) have well-defined \hyperref[def:tangent-space]{tangent spaces} at \(p\). Then, in a double point of \(f(\mathcal{H} )\), the \hyperref[def:tangent-space]{tangent space} can be specified by specifying the preimage (\(x\) or \(y\)), i.e., consider \(f^{\ast} (T \mathcal{M} )\), the \hyperref[def:fiber]{fiber} over \(x\in \mathcal{H} \) is \(T_{f(x)} \mathcal{M} \) introduce \hyperref[def:linear-connection]{connection} \(f^{\ast} (\nabla )\) on \(f^{\ast} (T \mathcal{M} )\). Let \(X\in T_x \mathcal{H} \), \(Y\) a \hyperref[def:section]{section} of \(f^{\ast} (T \mathcal{M} )\). Set
	\[
		(f^{\ast} \nabla )_x Y\coloneqq \nabla _{\mathrm{d} (x)}Y,
	\]
	where \(f^{\ast} (T \mathcal{M} _x)\) is identified with \(T_{f(x)}\mathcal{M} \) with \(\nabla \) for \(f^{\ast} \nabla \). All this means that a \hyperref[def:section]{section} of \(f^{\ast} (T \mathcal{M} )\) is a \hyperref[def:vector-field-along-curve]{vector field along \(f\)}.
\end{eg}

\subsection{First Variations}
Recall the following.

\begin{prev}
	The \hyperref[def:energy]{energy} is defined as
	\[
		E(s) \coloneqq \frac{1}{2} \int_{a}^{b} \left\langle \frac{\partial c(t, s)}{\partial t} , \frac{\partial c(t, s)}{\partial t}  \right\rangle  \,\mathrm{d}t,
	\]
	and the \hyperref[def:length]{length} is defined as
	\[
		L(s) \coloneqq \int_{a}^{b} \left\langle \frac{\partial c(t, s)}{\partial t} , \frac{\partial c(t, s)}{\partial t}  \right\rangle ^{1 / 2} \,\mathrm{d}t.
	\]
\end{prev}

Now, we consider
\begin{itemize}
	\item the first variations \(E^\prime (0) \) and \(L^\prime (0)\), i.e., the first derivatives;
	\item for \(c = c_0\) \hyperref[def:geodesic]{geodesic}, compute the second variations \(E^{\prime\prime}(0) \) and \(L^{\prime\prime}(0)\), i.e., the second derivatives.
\end{itemize}

\begin{lemma}
	If \(L(s)\), \(E(s)\) are differentiable w.r.t.\ \(s\), then
	\[
		L^\prime (0) = \int_{a}^{b} \left( \frac{\frac{\partial }{\partial t} \left\langle c^\prime , \dot{c}  \right\rangle }{\left\langle \dot{c} , \dot{c} \right\rangle ^{1 / 2}} - \frac{\left\langle c^\prime , \nabla _{\frac{\partial }{\partial t} } \dot{c} \right\rangle }{\left\langle \dot{c} , \dot{c} \right\rangle ^{1 / 2}}\right)  \,\mathrm{d}t,
	\]
	and
	\[
		E^\prime (0) = \left\langle c^\prime (b, 0), \dot{c} (b, 0) \right\rangle - \left\langle c^\prime (a, 0), \dot{c} (a, 0) \right\rangle - \int_{a}^{b} \left\langle \frac{\partial c}{\partial s} , \nabla _{\frac{\partial }{\partial t} }\frac{\partial c}{\partial t} (t, s) \right\rangle  \,\mathrm{d}t.
	\]
\end{lemma}
\begin{proof}
	We have already proved this in different notations.
\end{proof}

\begin{note}
	If \(c = c_0\) is parametrized proportionally to the arc-length, i.e., \(\lVert \dot{c} (t, 0) \rVert \) is a constant. Then \(L^\prime (0)\) becomes
	\[
		L^\prime (0) = \frac{1}{\left\langle \dot{c} , \dot{c}  \right\rangle ^{1 / 2}} \left( \at{\left\langle c^\prime , \dot{c} \right\rangle }{t=a, s=0}{t=b, s=0} - \int_{a}^{b} \left\langle c^\prime , \nabla _{\frac{\partial }{\partial t} }\dot{c}  \right\rangle  \,\mathrm{d}t \right).
	\]
\end{note}

If we consider the fixed endpoints case, we observe that \(E\) and \(L\) are stationary if and only if
\[
	\nabla _{\frac{\partial }{\partial t}} \dot{c} (t, 0) = 0,
\]
i.e., when \(c\) is a \hyperref[def:geodesic]{geodesic}.

\subsection{Second Variations}
Now, let \(c = c_0\) be a \hyperref[def:geodesic]{geodesic}. Then we compute the second derivatives w.r.t.\ \(s\) of \(E\) and \(L\) at \(s = 0\).

\begin{theorem}
	Let \(c\colon [a, b] \to \mathcal{M} \) be a \hyperref[def:geodesic]{geodesic}. Then
	\[
		E^{\prime\prime}(0)
		= \int_{a}^{b} \left\langle \nabla _{\frac{\partial }{\partial t} } c^{\prime} (t, 0), \nabla _{\frac{\partial }{\partial t} } c^{\prime} (t, 0)\right\rangle  \,\mathrm{d}t
		- \at{\int_{a}^{b} \left\langle R(\dot{c} , c^{\prime} ) c^{\prime} , \dot{c} \right\rangle \,\mathrm{d}t}{s= 0}{}
		+ \at{\left\langle \nabla _{\frac{\partial }{\partial s} } c^{\prime} , \dot{c}  \right\rangle }{t=a, s=0}{t=b, s=0} .
	\]
	By letting \(c^{\prime \perp } \coloneqq c^{\prime} - \left\langle \frac{\dot{c} }{\lVert \dot{c} \rVert } , c^{\prime} \right\rangle \frac{\dot{c} }{\lVert \dot{c} \rVert} \),\footnote{I.e., the component of \(c^{\prime} \) orthogonal to \(\dot{c} \).} we have
	\[
		L^{\prime\prime} (0)
		= \frac{1}{\lVert \dot{c} \rVert } \at{\left(
		\int_{a}^{b} \left\langle \nabla _{\frac{\partial }{\partial t} } c^{\prime\perp }, \nabla _{\frac{\partial }{\partial t} } c^{\prime\perp }\right\rangle \,\mathrm{d}t
		- \int_{a}^{b} \left\langle R(\dot{c}, c^{\prime\perp } )c^{\prime\perp }, \dot{c} \right\rangle \,\mathrm{d}t
		+ \at{\left\langle \nabla _{\frac{\partial }{\partial s} }c^{\prime} , \dot{c} \right\rangle }{t=a}{t=b} \right) }{s=0}{}.
	\]
\end{theorem}

\begin{remark}
	By keeping the endpoints fixed, if the \hyperref[def:sectional-curvature]{sectional curvature} of \(\mathcal{M} \)  is non-positive, then the \hyperref[def:Riemannian-curvature]{Riemannian curvature} in \(E^{\prime\prime} (0)\) and \(L^{\prime\prime} (0)\) are non-negative. This implies \(E^{\prime\prime} (0) > 0\), then \(E(c_s) > E(c_0)\) for small \(\vert s \vert \).
\end{remark}

\begin{corollary}
	On a manifold with non-positive \hyperref[def:sectional-curvature]{sectional curvature}, the \hyperref[def:geodesic]{geodesics} with fixed endpoints are always locally minimizing.
\end{corollary}

\subsection{Index Form}
Let \(X\) be a \hyperref[def:vector-field-along-curve]{vector field along \(c\)} where \(c\) is a \hyperref[def:geodesic]{geodesic}. Then, there exists a \hyperref[def:geodesic-variation]{geodesic variation}
\[
	c\colon [a, b] \times (-\epsilon , \epsilon )\to \mathcal{M}
\]
of \(c(t)\) with
\[
	\at{\frac{\partial c}{\partial s} }{s=0}{} = X.
\]
Put
\[
	I(X, X)
	\coloneqq \int_{a}^{b} \left( \left\langle \nabla _{\frac{\partial }{\partial t} }X, \nabla _{\frac{\partial }{\partial t} } X \right\rangle - \left\langle R(\dot{c}, X )X, \dot{c}  \right\rangle  \right)  \,\mathrm{d}t,
\]
i.e., \(I(X, X) = \frac{\mathrm{d}^2}{\mathrm{d}s^2} E(0)\) if \(X(a) = X(b) = 0\). Also, put
\[
	I(X, Y)
	\coloneqq \int_{a}^{b} \left( \left\langle \nabla _{\frac{\partial }{\partial t} } X, \nabla _{\frac{\partial }{\partial t} } Y \right\rangle - \left\langle R(\dot{c} , X) Y, \dot{c} \right\rangle \right)  \,\mathrm{d}t.
\]
We see that \(I(X, Y)\) is a bilinear symmetric in \(X, Y\) where \(Y\coloneqq \frac{\partial c}{\partial x} \)

\begin{definition}[Index form]\label{def:index-form}
	\(I\) defined above is called the \emph{index form} of \hyperref[def:geodesic]{geodesic} \(c\).
\end{definition}

\begin{prev}
	Recall the \hyperref[eq:Jacobi]{Jacobi equation}, i.e.,
	\[
		\nabla _{\frac{\mathrm{d}}{\mathrm{d}t} }\nabla _{\frac{\mathrm{d}}{\mathrm{d}t} }X + R(X, \dot{c} )\dot{c} = 0.
	\]
\end{prev}

\begin{proposition}[Jacobi field]\label{prop:Jacobi-field}
	A \hyperref[def:vector-field-along-curve]{vector field \(X\) along a \hyperref[def:geodesic]{geodesic} \(c\colon [a, b] \to \mathcal{M} \)} is a \hyperref[def:Jacobi-field]{Jacobi-field} if and only if the \hyperref[def:index-form]{index form} of \(c\) satisfies \(I(X, Y) = 0\) for all \hyperref[def:vector-field-along-curve]{vector fields \(Y\) along \(c\)} with \(Y(a) = Y(b) = 0\).
\end{proposition}
\begin{proof}
	Observe that
	\[
		\begin{split}
			I(X, Y)
			&\coloneqq \int_{a}^{b} \left( \left\langle \nabla _{\frac{\partial }{\partial t} } X, \nabla _{\frac{\partial }{\partial t} } Y \right\rangle - \left\langle R(\dot{c} , X) Y, \dot{c} \right\rangle \right)  \,\mathrm{d}t\\
			&= \int_{a}^{b} \left( \left\langle \nabla _{\frac{\partial }{\partial t} } X, \nabla _{\frac{\partial }{\partial t} } Y \right\rangle - \left\langle R(X, \dot{c}) \dot{c}, Y \right\rangle \right)  \,\mathrm{d}t
			= \int_{a}^{b} \left( \left\langle -\nabla _{\frac{\partial }{\partial t} } \nabla _{\frac{\partial }{\partial t} } X, Y \right\rangle - \left\langle R(X, \dot{c}) \dot{c}, Y \right\rangle \right)  \,\mathrm{d}t,
		\end{split}
	\]
	where the second inequality follows from the fact that \(\nabla \) is \hyperref[def:Riemannian]{Riemannian}, and \(Y(a) = 0 = Y(b)\). We see that the right-hand side of the above vanishes for every \(Y\) if and only if
	\[
		\nabla _{\frac{\mathrm{d}}{\mathrm{d}t} }\nabla _{\frac{\mathrm{d}}{\mathrm{d}t} }X + R(X, \dot{c} )\dot{c} = 0,
	\]
	which is just the \hyperref[eq:Jacobi]{Jacobi equation}, so the result follows.
\end{proof}

\begin{remark}
	\autoref{prop:Jacobi-field} is really where the \hyperref[eq:Jacobi]{Jacobi equation} comes from.
\end{remark}