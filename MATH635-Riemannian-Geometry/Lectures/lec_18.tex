\lecture{18}{14 Mar. 14:30}{Jacobi Fields and Geodesics}
\begin{theorem}\label{thm:lec18}
	Consider a \hyperref[def:geodesic]{geodesic} \(c\colon [0, 1] \to \mathcal{M} \), \(t \mapsto c(t)\), and the variation \(c\colon [0, 1] \times (-\epsilon , \epsilon ) \to \mathcal{M} \) of \(c\) for all curves \(c(\cdot, s) \eqqcolon c_s(\cdot)\) are \hyperref[def:geodesic]{geodesics}, Then \(X(t) \coloneqq \at{\frac{\partial }{\partial s} c(t, s)}{s=0}{} \) is a \hyperref[def:Jacobi-field]{Jacobi field} along \(c(t) = c_0(t)\).

	Conversely, every \hyperref[def:Jacobi-field]{Jacobi field} along \(c(t)\) can be obtained in this way, i.e., by variation of \hyperref[def:geodesic]{geodesics}.
\end{theorem}
\begin{proof}
	The forward direction is straightforward: since \(c(t, s)\) for a fixed \(s\) is a \hyperref[def:geodesic]{geodesic}, hence
	\[
		\nabla _{\frac{\partial }{\partial t} }\frac{\partial }{\partial t} c(t, s) = 0
	\]
	for all \(s\), hence \(\nabla _{\frac{\partial }{\partial s} } \nabla _{\frac{\partial }{\partial t} } \frac{\partial }{\partial t} c(t, s) = 0\). Then,
	\[
		\begin{split}
			&\nabla _{\frac{\partial }{\partial t} }\nabla _{\frac{\partial }{\partial s} }\frac{\partial }{\partial t} c(t, s)
			+ \left( - \nabla _{\frac{\partial }{\partial t} } \nabla _{\frac{\partial }{\partial s} } + \nabla _{\frac{\partial }{\partial s} } \nabla _{\frac{\partial }{\partial t} } \right) \frac{\partial }{\partial t} c(t, s)\\
			=& \nabla _{\frac{\partial }{\partial t} }\nabla _{\frac{\partial }{\partial t} } \frac{\partial }{\partial s} c(t, s) + R\left( \frac{\partial c}{\partial s} , \frac{\partial c}{\partial t} \right) \frac{\partial c}{\partial t}\\
			=& 0,
		\end{split}
	\]
	where we use the fact that \([\frac{\partial }{\partial s} , \frac{\partial }{\partial t} ] = 0\), so \(\nabla _{\frac{\partial }{\partial s} } \frac{\partial }{\partial t} = \nabla _{\frac{\partial }{\partial s} } \frac{\partial }{\partial s} \). Plugging in the definition of \(X\), we have
	\[
		\nabla _{\frac{\partial }{\partial t} }\nabla _{\frac{\partial }{\partial t} } X
		+ R\left( X, \frac{\partial c}{\partial t} \right) \frac{\partial c}{\partial t} = 0,
	\]
	i.e., \(X\) is a \hyperref[def:Jacobi-field]{Jacobi field}.

	The converse direction is left as a homework. As a hint, consider the following:
	\begin{center}
		\incfig{construct-geodesic}
	\end{center}
	Then, let
	\[
		c(t, s) = \exp _{\gamma (s)} \left( t(\dot{c} (0) + s\cdot V) \right)
	\]
	for some \(V\). Once we have this, we just let \(X (t)= \at{\frac{\partial }{\partial s} c(t, s)}{s = 0}{} \).
\end{proof}

\begin{intuition}
	The \hyperref[eq:Jacobi]{Jacobi equation} can be viewed as the linearization of the \hyperref[eq:geodesic]{geodesic equation}.
\end{intuition}

\begin{corollary}
	Every \hyperref[def:killing-field]{killing field} \(X\) on \(\mathcal{M} \) is a \hyperref[def:Jacobi-field]{Jacobi field} along any \hyperref[def:geodesic]{geodesic} in \(\mathcal{M} \).
\end{corollary}
\begin{proof}[Proof idea]
	Since we have a \hyperref[def:killing-field]{killing field} \(X\), we use it to construct \(\Phi _s \colon \mathcal{M} \to \mathcal{M} \), which is an \hyperref[def:isometry]{isometry} since \(X\) is a \hyperref[def:killing-field]{killing field}.
	\begin{center}
		\incfig{construct-geodesic-killing-field}
	\end{center}
	The idea is to consider \(c(t, s) = \Phi _s \circ c(t)\), and let \(X = \frac{\partial }{\partial s} c(t, s)\). By \autoref{thm:lec18}, we're done.
\end{proof}

\begin{corollary}
	Let \(c\colon [0, T] \to \mathcal{M} \) be a \hyperref[def:geodesic]{geodesic} with \(p = c(0)\), i.e., \(c(t) = \exp _p (t \dot{c} (0))\). For \(W\in Tvp \mathcal{M} \), the \hyperref[def:Jacobi-field]{Jacobi field} \(x\) along \(c\) with \(X(0) = 0\), \(\dot{X} ( 0) = W\) then is given as
	\[
		X(t) = \at{\mathrm{D} (\exp _p)}{(t \dot{c} (0))}{} (tW).
	\]
\end{corollary}
\begin{proof}
	This is a direct consequence of \autoref{thm:lec18}, since now \(X(0) = 0\), we don't need to worry about constructing \(\gamma (s)\), i.e., we have the following:
	\begin{center}
		\incfig{construct-geodesic-zero}
	\end{center}
	Now, we consider \(c(t, s) = \exp _p (t(\dot{c} (0) + s\cdot W))\), hence
	\[
		\frac{\partial }{\partial s} c(t, s) = \at{\frac{\partial }{\partial s} \exp _p(t \dot{c} + s\cdot W)}{s = 0}{} .
	\]
	This means that if we want to have \(\at{\mathrm{D} (\exp _p)}{V}{} (W)\), and construct a \hyperref[def:Jacobi-field]{Jacobi field} \(W\) such that \(X(0)=0\), \(\dot{X} (0) = W\).
\end{proof}

\subsection{Conjugate Points}
\begin{definition}[Conjugate point]\label{def:conjugate-point}
	Let \(c\colon I \to \mathcal{M} \) be a \hyperref[def:geodesic]{geodesic}. For \(t_0, t_1\in I\) with \(t_0 \neq t_1\), \(c(t_0)\) and \(c(t_1)\) are called \emph{conjugate} along \(c\) if there exists a \hyperref[def:Jacobi-field]{Jacobi field} \(X(t)\) along \(c\) which does not vanish identically and satisfies \(X(t_0) = 0 = X(t_1)\).
\end{definition}

\begin{note}
	We see that \(\left\langle X(t), \dot{c} (t) \right\rangle = 0\) for all \(t\).
\end{note}
\begin{explanation}
	Since \(\nabla _{\partial t} \langle X(t) , \dot{c} (t) \rangle = \langle \dot{X} , \dot{c} \rangle \), so
	\[
		\nabla _{\partial t} \nabla _{\partial t} \langle X(t), \dot{c} (t)\rangle
		= \langle \ddot{X}, \dot{c}  \rangle
		= -\langle R(X, \dot{c}) \dot{c}, \dot{c}\rangle = 0.
	\]
	This is a linear function, and if two endpoints are both \(0\), everything is \(0\).
\end{explanation}

\begin{note}
	If \(t_0, t_1\in I\), \(t_0 \neq t_1\) are not \hyperref[def:conjugate-point]{conjugate} along \(c\), then for \(V\in T_{c(t_0)} \mathcal{M}, W\in T_{c(t_1)} \mathcal{M} \), there exists a unique \hyperref[def:Jacobi-field]{Jacobi field} \(Y(t)\) along \(c\) such that \(Y(t_0) = V, Y(t_1)=W\).
\end{note}
\begin{explanation}
	Let \(\mathcal{J} _c\) be the \hyperref[def:Jacobi-field]{Jacobi fields} along \(c\), the construct the linear map
	\[
		A\colon \mathcal{J} _c \to T_{c(t_0)} \mathcal{M} \times T_{c(t_1)} \mathcal{M} ,\quad
		Y \mapsto (Y(t_0), Y(t_1)).
	\]
	Since \(\mathcal{J} _c\) is a vector space with \(\dim \mathcal{J} _c = 2n\), and the target space is also with dimension \(2n\), it suffices to show that \(\ker A = \left\{ 0 \right\} \). This is true because \(t_0 \neq t_1\) are not \hyperref[def:conjugate-point]{conjugate}.
\end{explanation}

\begin{eg}
	Any antipodal points of \(S^n\) are \hyperref[def:conjugate-point]{conjugate points}.
\end{eg}

\begin{eg}
	Consider \(\mathbb{R} ^n\) with flat \hyperref[def:Riemannian-metric]{metric} doesn't have \hyperref[def:conjugate-point]{conjugate points}.
\end{eg}

\begin{eg}
	\hyperref[def:Riemannian-manifold]{Riemannian manifolds} with non-positive \hyperref[def:sectional-curvature]{sectional curvature} has no \hyperref[def:conjugate-point]{conjugate points}.
\end{eg}

\subsection{Length-Minimizing Geodesics}

\begin{theorem}\label{thm:lec18-2}
	Let \(c\colon [a, b] \to \mathcal{M} \) be a \hyperref[def:geodesic]{geodesic}.
	\begin{enumerate}[(a)]
		\item If there does not exist a point \hyperref[def:conjugate-point]{conjugate} to \(c(a)\) along \(c(t)\), then there exists \(\epsilon > 0\) such that for all piecewise \hyperref[def:curve]{smooth curve} \(g\colon [a, b] \to \mathcal{M} \) with \(g(a) = c(a), g(b) = c(b)\) and \(d(g(t), c(t)) < \epsilon \) for all \(t\in [a, b]\), we have \(L(c) \leq L(g)\), and the equality holds when if and only if \(g\) is a reparametrization of \(c\).
		\item If there is \(\tau \in (a, b)\) such that \(c(a)\) and \(c(\tau )\) are \hyperref[def:conjugate-point]{conjugate points} along \(c\), then there exists a proper variation \(c(t, s) \colon [a, b] \times (-\epsilon , \epsilon ) \to \mathcal{M} \) such that \(L(c_s) < L(c)\) for \(s \in (-\epsilon , \epsilon ) \setminus \left\{ 0 \right\} \).
	\end{enumerate}
\end{theorem}

\begin{remark}
	Consider the cylinder, where we identify every line of \(\mathbb{R} ^n\) below.
	\begin{center}
		\incfig{cylinder-geodesic-minimizing}
	\end{center}
	There are two \hyperref[def:geodesic]{geodesics}, but one is strictly longer.
\end{remark}

\begin{corollary}\label{col:lec18}
	Let \(p\in \mathcal{M} \) and \(V\in T_p \mathcal{M} \) is contained in the domain of definition of \(\exp _p\). Let \(c(t) = \exp _p(t V)\), and \(\gamma \colon [0, 1] \to T_p \mathcal{M} \) be a piecewise \hyperref[def:curve]{smooth curve}  contained in the domain of \(\exp _p\) with \(\gamma (0) = 0, \gamma (1) = V\). Then
	\[
		L \left( \at{\exp _p(tV)}{t\in[0, 1]}{}  \right)
		\leq L \left( \exp _p \circ \gamma (t) \right)
	\]
	and the equality holds if and only if \(\gamma \) differs from the curve \(tV\), \(t\in[0, 1]\) only by reparametrization.
\end{corollary}
\begin{proof}[Proof hint]
	We directly estimate
	\[
		L(\exp \circ \gamma )
		= \int_{0}^{1} \left\vert \frac{\mathrm{d}}{\mathrm{d}t} \exp \circ \gamma \right\vert \,\mathrm{d}t
		= \int_{0}^{1} \vert \mathrm{D} \exp \circ \gamma  \vert  \,\mathrm{d}t.
	\]
\end{proof}