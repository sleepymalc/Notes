\lecture{24}{4 Apr. 13:00}{Introduction to the Rauch Comparison Theorem}
\begin{eg}[Tilted torus]
	Consider the tilted torus
	\begin{center}
		\incfig{tilted-torus}
	\end{center}
	We see that
	\begin{itemize}
		\item \(C_2 = \mathbb{Z} _2 [p_1]\);
		\item \(C_1 = \mathbb{Z} _2[p_2] \oplus \mathbb{Z} _2[p_3]\);
		\item \(C_0 = \mathbb{Z} _2[p_1]\).
	\end{itemize}
	Moreover, the chain complex is
	% https://q.uiver.app/?q=WzAsNSxbMCwwLCIwIl0sWzEsMCwiXFxtYXRoYmJ7Wn1bcF8xXSJdLFsyLDAsIlxcbWF0aGJie1p9XzJbcF8yXStcXG1hdGhiYntafV8yW3BfM10iXSxbMywwLCJcXG1hdGhiYntafV8yW3BfNF0iXSxbNCwwLCIwIl0sWzAsMSwiXFxwYXJ0aWFsXzMiXSxbMSwyLCJcXHBhcnRpYWxfMiJdLFsyLDMsIlxccGFydGlhbF8xIl0sWzMsNCwiXFxwYXJ0aWFsXzAiXV0=
	\[\begin{tikzcd}
			0 & {\mathbb{Z}[p_1]} & {\mathbb{Z}_2[p_2]+\mathbb{Z}_2[p_3]} & {\mathbb{Z}_2[p_4]} & 0
			\arrow["{\partial_3}", from=1-1, to=1-2]
			\arrow["{\partial_2}", from=1-2, to=1-3]
			\arrow["{\partial_1}", from=1-3, to=1-4]
			\arrow["{\partial_0}", from=1-4, to=1-5]
		\end{tikzcd}\]
\end{eg}

\section{The Rauch Comparison Theorem}
In this section, our goal is to compare \hyperref[def:Riemannian-manifold]{Riemannian manifolds} \((\mathcal{M} , g)\) with other \hyperref[def:Riemannian-manifold]{Riemannian manifolds} of constant \hyperref[def:sectional-curvature]{curvatures} model spaces, e.g., \(S^n\), \(\mathbb{R} ^n\), and \(\mathbb{H} ^n\).

\begin{notation}[Model space]
	The set of \emph{model spaces} is denoted as \(\mathcal{M} _m \in \left\{ S^n, \mathbb{R} ^n, \mathbb{H} ^n \right\} \).
\end{notation}

\subsection{Preliminary Estimations}
Let \(c(t)\) be a \hyperref[def:geodesic]{geodesic} with \(\lVert \dot{c} \rVert =1\), \(v\in T_{c(0)} \mathcal{M} \). Furthermore, let \(\mathcal{J} (t)\) be the \hyperref[def:Jacobi-field]{Jacobi field} along \(c(t)\) with \(\mathcal{J} (0) = 0\) and \(\dot{\mathcal{J} }(0) = v \) given by
\[
	\begin{dcases}
		(\sin t)v,  & \text{ for } S^n ;           \\
		tv,         & \text{ for } \mathbb{R} ^n ; \\
		(\sinh t)v, & \text{ for } \mathbb{H} ^n .
	\end{dcases}
\]
Now, consider \((\mathcal{M} , g)\) such that \(\lambda \leq \kappa \leq \mu \) with \(\lambda \leq 0\) and \(\mu \geq 0\).

\begin{notation}
	For \(\rho \in \mathbb{R} \),
	\[
		c_\rho (t) = \begin{dcases}
			\cos (\sqrt{\rho } t),   & \text{ if } \rho > 0 ; \\
			1,                       & \text{ if } \rho = 0 ; \\
			\cosh (\sqrt{-\rho } t), & \text{ if } \rho < 0 ,
		\end{dcases}
	\]
	and also,
	\[
		s_\rho (t) = \begin{dcases}
			\frac{1}{\sqrt{\rho } } \sin (\sqrt{\rho } t),   & \text{ if } \rho > 0 ; \\
			t,                                               & \text{ if } \rho = 0 ; \\
			\frac{1}{\sqrt{-\rho } }\sinh (\sqrt{-\rho } t), & \text{ if } \rho < 0 ,
		\end{dcases}
	\]
\end{notation}

These are solutions of \hyperref[eq:Jacobi]{Jacobi equations} for constant \hyperref[def:sectional-curvature]{sectional curvature} \(\rho \), i.e.,
\[
	\ddot{f}(t) + \rho f(t) = 0
\]
with corresponding initial values \(f(0) = 0\), \(\dot{f} (0) = 1\), respectively, \(f(0) = 1\), \(\dot{f} (0) = 0\).

\begin{theorem}
	Assume \(\kappa \leq \mu \) and \(\lVert \dot{c} \rVert \equiv 1\). Assume either \(\mu \geq 0\) or \(\mathcal{J} ^{\text{tan} } \equiv 0\). Let \(f_\mu \coloneqq \vert \mathcal{J} (0) \vert c_\mu + \vert \mathcal{J} \vert '(0) s_\mu \) solve
	\[
		\ddot{f} + \mu f = 0
	\]
	with \(f(0) = \vert \mathcal{J} (0) \vert \) and \(\dot{f} (0) = \vert \mathcal{J} \vert ^{\prime} (0)\). If \(f_\mu (t)> 0\) for \(0 < t < \tau \), then the following holds.
	\begin{enumerate}[(a)]
		\item \(\langle \mathcal{J} , \dot{\mathcal{J} }  \rangle f_\mu \geq \langle \mathcal{J} , \mathcal{J}  \rangle \dot{f} _\mu\) on \([0, \tau ]\).
		\item \(1 \leq \frac{\vert \mathcal{J} (t_0) \vert }{f_\mu (t_1)} \leq \frac{\vert \mathcal{J} (t_2) \vert }{f_\mu (t_2)}\) if \(0 < t_1 \leq t_2 < \tau \).
		\item \(\vert \mathcal{J} (0) \vert c_\mu (t) + \vert \mathcal{J}  \vert ^{\prime} (0) s_\mu (t) \leq \vert \mathcal{J} (t) \vert\) for \(0 \leq t \leq \tau \).
	\end{enumerate}
\end{theorem}
\begin{proof}
	Firstly, we have that
	\[
		\vert \mathcal{J}  \vert ^{\prime} = \frac{\langle \mathcal{J} , \dot{\mathcal{J} }  \rangle }{\vert \mathcal{J}  \vert },\quad
		\vert \mathcal{J}  \vert ^{\prime\prime} = \frac{\langle \dot{\mathcal{J} } , \dot{\mathcal{J} } \rangle }{\vert \mathcal{J}  \vert } + \frac{\langle \mathcal{J} , \ddot{\mathcal{J} }  \rangle }{\vert \mathcal{J}  \vert } - \frac{\langle \mathcal{J} , \dot{\mathcal{J} }  \rangle ^2}{\vert \mathcal{J} \vert^3 },
	\]
	so
	\[
		\vert \mathcal{J}  \vert ^{\prime\prime} + \mu \vert \mathcal{J} \vert
		= \frac{1}{\vert \mathcal{J} \vert } \left( - \langle R(\mathcal{J} , \dot{c} ) \dot{c}, \mathcal{J} \rangle + \mu \langle \mathcal{J} , \mathcal{J}  \rangle  \right) + \frac{1}{\vert \mathcal{J}  \vert ^3} \left( \vert \dot{\mathcal{J} } \vert^2 \vert \mathcal{J}  \vert ^2 - \langle \mathcal{J} , \dot{\mathcal{J} } \rangle ^2 \right) \geq 0
	\]
	since \(\kappa \leq \mu \) for \(0 < t < \tau \), provided \(\mathcal{J} \) has no zeros on \((0, \tau )\). Moreover,
	\[
		\left( \vert \mathcal{J}  \vert ^{\prime} f_\mu - \vert \mathcal{J}  \vert \dot{f} _\mu  \right) ^{\prime}
		= \vert \mathcal{J}  \vert ^{\prime\prime} f_\mu - \vert \mathcal{J}  \vert \ddot{f}_\mu \geq 0
	\]
	since \(\ddot{f}_\mu + \mu f_\mu = 0\) for \(f_\mu (t) \geq 0\). Also, we have \(\vert \mathcal{J}  \vert (0) = f_\mu (0)\), \(\vert \mathcal{J}  \vert ^{\prime} (0) = \dot{f} _\mu (0)\), implying
	\[
		\vert \mathcal{J}  \vert ^{\prime} f_\mu - \vert \mathcal{J}  \vert \dot{f} _\mu \geq 0,
	\]
	which proves the first claim.

	Furthermore,
	\[
		\left( \frac{\vert \mathcal{J}  \vert }{f_\mu } \right) ^{\prime} = \frac{1}{f^2_\mu } \left( \vert \mathcal{J}  \vert ^{\prime} f_\mu - \vert \mathcal{J}  \vert \dot{f} _\mu  \right) \geq 0,
	\]
	then since first zero of \(\mathcal{J}\) cannot occur before the first zero of \(f_\mu \), the second claim is proved. The last claim follows directly from this.
\end{proof}

\begin{remark}
	\(f_\mu (t) > 0\) for \(0 < t < \tau \) is necessary.
\end{remark}
\begin{explanation}
	Take \(S^{nn(\mu - \epsilon )} \)\todo{fix} with \(\mathcal{J} (0) = 0\). We see that \(f_\mu (t)\) has a zero at \(t = \pi / \sqrt{\mu } \) and \(\mathcal{J} (t)\) has one at \(t = \pi / \sqrt{\mu - \epsilon } \). For small \(\epsilon > 0\) and any \(t\), only slightly longer than \(\pi / \sqrt{\mu - \epsilon } \), we have \(\frac{\vert \mathcal{J} (t) \vert }{f(t)} < 1\).
\end{explanation}

\begin{corollary}
	Suppose that \(\kappa \leq \mu \), \(c_\mu \geq 0\) on \((0, \tau )\), and \(\mu \geq 0\) or \(\mathcal{J} ^{\text{tan} }\equiv 0\). Let \(\lVert \dot{c} \rVert \equiv 1\), \(\mathcal{J} (0) = 0\), \(\vert R \vert < \Lambda \) with \(R\) being the \hyperref[def:Riemannian-curvature]{curvature tensor}. Then,
	\[
		\vert \mathcal{J} (t) - t \dot{\mathcal{J} } (t) \vert \leq \vert \mathcal{J} (\tau ) \vert \frac{1}{2} \Lambda t^2.
	\]
\end{corollary}

\begin{theorem}
	Assume that \(\lambda \leq \kappa \leq \mu \), and either \(\lambda \leq 0\) or \(\mathcal{J} ^{\text{tan} } \equiv 0\), \(\lVert \dot{c} \rVert \equiv 1\), and \(\mathcal{J} (0)\), \(\dot{\mathcal{J} (0)} \) be linearly dependent. Finally, assume \(s_{(\lambda + \mu ) / 2} > 0\) on \((0, \tau )\). Then, for \(0 \leq t \leq \tau \),
	\[
		\vert \mathcal{J} (t) \vert \leq \vert \mathcal{J} (0) \vert c_\lambda (t) + \vert \mathcal{J}  \vert ^{\prime} (0) s_\lambda (t).
	\]
\end{theorem}
\begin{proof}[Proof idea]
	Let \(\rho \in \mathbb{R} \), \(\eta \coloneqq \max (\mu - \rho , \rho - \lambda )\). Let \(A\) be a \hyperref[def:vector-field-along-curve]{vector field along \(c\)} with \(\ddot{A} + \rho A = 0\), \(A(0) = \mathcal{J} (0)\), and \(\dot{A}(0) = \dot{\mathcal{J} } (0) \).

	Let \(a \colon I \to \mathbb{R} \) being a solution of \(\ddot{a} + (\rho -\eta )a = \eta \vert A \vert \), \(a(0) = \dot{a} (0) = 0\), and \(b \colon I \to \mathbb{R} \) solving \(\ddot{b} + \rho b = \eta \vert \mathcal{J}  \vert \), \(b(0) = \dot{b} (0) = 0\).
\end{proof}