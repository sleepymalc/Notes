\lecture{28}{18 Apr. 13:00}{Epilogue}
\begin{prev}
	W're looking at \(\Delta _g \omega - K = 0\) on \(\mathcal{M} \setminus p\) and \(\Delta _g \omega - K = -4 \pi \sigma _p\) on \(\mathcal{M} \).
\end{prev}

Consider choosing a polar normal coordinate \((r, \theta )\) in \(R_{2 \epsilon }(p)\) such that \(\mathrm{d} \rho = r\) and \(g = \mathrm{d} r^2 + R^2(r, \theta )\mathrm{d} \theta ^2\) such that
\[
	\int_{0}^{\pi } R(r, \theta ) \,\mathrm{d}\theta = L(r)
\]
be the perimeter of the geodesic circles. Then \(L(r) / r \to 2\pi \) as \(r \to 0\).

\begin{notation}
	The geodesic curvature of circles \(\kappa \) is defined as \(\kappa = \frac{1}{R} \frac{\partial R}{\partial r} \).
\end{notation}

Use \(\Delta \) in these coordinates, \(\Delta _g\).

\begin{prev}
	In \(B_{\epsilon } (p)\), \(\omega _0=-2\log r\).
\end{prev}

Then
\[
	\Delta _g \omega _0 = - \frac{2}{R} \frac{\partial R}{\partial r} \left( \frac{R}{r} \right) = - \frac{2\lambda }{r}
\]
with \(\lambda = \frac{1}{R} \frac{\partial R}{\partial r} - \frac{1}{r} = \kappa - \frac{1}{r}\), hence
\[
	\underbrace{\frac{\partial \kappa }{\partial r} + \kappa ^2}_{-K } = \frac{\partial \lambda }{\partial r} + \frac{2\lambda }{r}+ \lambda ^2 .
\]
Set \(\mu = r^2 \lambda \), observe that as \(r \to 0\), \(\mu \to 0\). Moreover, we have
\[
	\frac{\partial \mu }{\partial r} + \frac{\mu ^2}{r^2} = -r^2 K.
\]
For \(\mu (r, \theta )\) along any ray,
\[
	\mu (r, \theta )
	= - \int_{0}^{r} \frac{\mu (r^{\prime} , \theta )^2}{r^2} + {r^{\prime} }^2 K (r^{\prime} , \theta )  \,\mathrm{d}r^{\prime},
\]
we have \(\lambda (r, \theta ) = O(r)\). Moreover, \(\lambda / r \to - K_p / 3\) as \(r \to 0\). It follows that \(\Delta _r \omega _0\) is bounded and \(\Delta _g \omega _0 \to 2K_p / 3\) as approaching \(p\).

Now, set \(\omega = \omega _0 + \omega _1\), then \(\omega _1\) has to satisfy
\[
	\Delta _g \omega _1
	= \Delta _g \omega - \Delta _g \omega _0
	= K - \Delta _g \omega _0
\]
on \(\mathcal{M} \setminus p\). Let \(f = K - \Delta _g \omega _0\) be a function on \(\mathcal{M} \), and one can show that \(f\) extends continuously to a function on \(\mathcal{M} \). So \(\omega _1\) is a solution of \(\Delta _g \omega _1 = f\). This is unique up to additive constant if \(\int _\mathcal{M} f \,\mathrm{d} \mu _g = 0\).

\begin{intuition}
	To prove this, consider \(\mathcal{M} \setminus B_\delta (p)\) and let \(\delta \to 0\), we get two limits being \(-4\pi \) and \(4\pi \), which add up to \(0\).
\end{intuition}

\todo[inline]{See note for the full proof}

\subsection{Yamabe Problem}
\begin{problem}[Yamabe poroblem]\label{prb:Yamabe}
Given a compact \hyperref[def:Riemannian-manifold]{Riemannian manifold} \(\mathcal{M} , g\) of dimension \(n \geq 3\). Find a \hyperref[def:Riemannian-metric]{metric} \(\widetilde{g} \) \hyperref[def:conformal]{conformal} to \(g\) such that the \hyperref[def:Ricci-scalar-curvature]{scalar curvature} of \(\widetilde{g} \) is constant.
\end{problem}

If \(\mathcal{M} \) has no boundary, then Aubin, Sohaen, Trudinger solves it.

With boundary, Escober, Mouques, Bredt+Chen

If we write \(g = u^{\frac{4}{n-2}} g_0\), the \hyperref[def:Ricci-scalar-curvature]{scalar curvature} \(R_g\) is
\begin{equation}\label{eq:lec28}
	R_g
	= u^{-\frac{u+2}{n-2}} \left( - \frac{4(n-1)}{n-2} \Delta _{g_0} u + R_g u \right) .
\end{equation}
\(g\) has constant \hyperref[def:Ricci-scalar-curvature]{scalar curvature} \(c\) if and only if \(u\) is a solution of the \emph{Yamabe equation}
\begin{equation}\label{eq:Yamabe}
	\frac{4(n-1)}{n-2} \Delta _{g_0} u - R_{g_0} u + cu^{\frac{n+2}{n-2}} = 0.
\end{equation}


Variational approach:

\begin{definition}[Einstein-Hilbert action]\label{def:Einstein-Hilbert-action}
	The \emph{Einstein-Hilbert action} \(\mathcal{E} (g)\) is defined as
	\[
		\mathcal{E} (g) = \cfrac{\int _\mathcal{M} R_g \,\mathrm{d} \mathop{\mathrm{vol}} _g}{\mathop{\mathrm{vol}}(\mathcal{M} , g)^{\frac{n-2}{n}} }.
	\]
\end{definition}
\begin{definition}[Einstein metric]\label{def:Einstein-metric}
	A \hyperref[def:Riemannian-metric]{metric} \(g\) is called an \emph{Einstein metric} if \(\mathop{\mathrm{Ric}}(g) = c\cdot g \) for some constant \(c\).
\end{definition}
\begin{remark}
	A \hyperref[def:Riemannian-metric]{metric} \(g\) is a critical point of \(\mathcal{E} \) if and only if \(g\) is an \hyperref[def:Einstein-metric]{Einstein metric}.
\end{remark}

Given any positive function \(u\), the \emph{Yamabe functional}
\[
	\mathcal{E} _{g_0}(u) = \mathcal{E} (u^{\frac{4}{n-2}} g_0).
\]
\autoref{eq:lec28} implies
\[
	\mathcal{E} _{g_0}(u) = \frac{\int _\mathcal{M} \left( \frac{4(n-1)}{n-2}\vert \mathrm{d} u_{g_0} \vert ^2 + R_{g_0}u^2 \right) \,\mathrm{d} \mathop{\mathrm{vol}}_{g_0} }{\left( \int _\mathcal{M} u^{\frac{2n}{n-2}} \,\mathrm{d} \mathop{\mathrm{vol}}_{g_0} \right) ^{\frac{n-2}{n}}}.
\]
\(u\) is a critical point if and only if \(u\) satisfies the \hyperref[eq:Yamabe]{Yamabe equation}.

\section{Lorentzian Manifolds and General Relativity}
Consider
\[
	R_{\mu \nu } - \frac{1}{2}g_{\mu \nu } R = \frac{8 \pi G}{c^4} T_{\mu \nu }.
\]

\[
	g(X, X) = -(X^0)^2 + \sum_{i=1}^3 (X_i)^2.
\]

\begin{definition}[Arc length]
	The \emph{arc length} of causal curve \(\gamma \) between 2 points corresponding to parameter values \(\lambda = a\) and \(\lambda = b\) is
	\[
		L[\gamma ](a, b) = \int_{a}^{b} \sqrt{-g(\dot{\gamma }(\lambda ), \dot{\gamma }(\lambda )  )} \,\mathrm{d}\lambda .
	\]
\end{definition}

If \(q\in \mathcal{J} ^+(p)\), define temporal distance \(q\) from \(p\) as \(\tau (q, p) = \sup L[\gamma ]\) over  all future-directed casual curves \((p, q)\).

Recall that in the \hyperref[def:Riemannian-manifold]{Riemannian manifold} case, we have \hyperref[thm:Hopf-Rinow]{Hopf-Rinow theorem}. The analogous for Lorentzian is for maximization. Holds if spacetime admits Cauchy hypersurfaces when the supremum is achieved, and metric \(C^{\prime} \), maximizing curve is a causal geodesic.

\section{Ricci Flow}
\[
	\frac{\partial g(t)}{\partial t} = -2 \mathop{\mathrm{Ric}}(g) .
\]