\lecture{4}{17 Jan. 14:30}{Submanifolds, Vector Bundles, and Riemannian metrics}
\begin{definition*}
	Let \(\mathcal{M}^d \) be a \hyperref[def:smooth-manifold]{differentiable manifold} with a \hyperref[def:coordinate-chart]{chart} \(x\colon U \to \Omega \subseteq \mathbb{R} ^d\) and \(p\in U \subseteq \mathcal{M} \) where \(U\) is open. On \(\left\{ (x, v) \mid v\in T_{x(p)}\Omega \right\}\), we define an equivalence relation by \((x, v)\sim (y, w)\) if and only if \(w = \mathrm{d} (y \circ x ^{-1} )v\).

	\begin{definition}[Tangent space]\label{def:tangent-space}
		The space of equivalence classes is called the \emph{tangent space} \(T_p \mathcal{M} \) at point \(p\) to \(\mathcal{M} \).
	\end{definition}
	\begin{definition}[Tangent vector]\label{def:tangent-vector}
		The elements in the \hyperref[def:tangent-space]{tangent space} is called \emph{tangent vectors}.
	\end{definition}
\end{definition*}

\begin{remark}
	\(T_p \mathcal{M} \) naturally caries the structure of a vector space.
\end{remark}

Now, \(T \mathcal{M} \) is defined as
\[
	T\mathcal{M} \coloneqq \coprod _{p\in \mathcal{M} }T_p \mathcal{M} .
\]
Recall the projection \(\pi \colon T\mathcal{M} \to  \mathcal{M} \) with \(\pi (w) = p\) for \(w\in T_p \mathcal{M} \). Then we can define the following.

\begin{definition}[Derivation]\label{def:derivation}
	If \(x\colon U \to  \mathbb{R} ^d\) be a \hyperref[def:coordinate-chart]{chart} for \(\mathcal{M} \), and let \(TU = \coprod_{p\in U} T_p U\). Then we define the \emph{derivation} \(\mathrm{d} x\colon TU \to  T x(U) \coloneqq \coprod_{p\in x(U)} T_p \mathcal{M} \) by \(w \mapsto \mathrm{d} x(\pi (w))(w)\in T_{x(\pi (w))} x(U)\).
\end{definition}

The transition maps
\[
	\mathrm{d} x^\prime \circ (\mathrm{d} x)^{-1}
	= \mathrm{d} (x^\prime \circ x ^{-1} )
\]
are differentiable. \(\pi \) is local represented by \(x \circ \pi \circ \mathrm{d} x ^{-1} \) maps \((x_0, v)\in Tx(U)\) to \(x_0\).

\begin{definition}[Tangent bundle]\label{def:tangent-bundle}
	The triple \((T\mathcal{M} , \pi , \mathcal{M} )\) is called the \emph{tangent bundle} of \(\mathcal{M} \) of \(\mathcal{M} \).
\end{definition}
Consider the product of
\begin{definition}[Total space]\label{def:total-space}
	\(T\mathcal{M} \) is called the \emph{total space} of the \hyperref[def:tangent-bundle]{tangent bundle}.
\end{definition}

\section{Vector Fields and Brackets}
\subsection{Vector Fields}
We now introduce the notion of \hyperref[def:vector-field]{vector field}.

\begin{definition}[Vector field]\label{def:vector-field}
	A \emph{vector field} \(X\) on a \hyperref[def:smooth-manifold]{differentiable manifold} \(\mathcal{M} \) is a correspondence associating to each point \(p\in \mathcal{M} \) a vector \(X(p)\in T_p \mathcal{M} \), i.e., \(X\colon \mathcal{M} \to T\mathcal{M} \).
\end{definition}

\begin{remark}
	Naturally, we say that the \hyperref[def:vector-field]{field} \(X\) is differentiable if the map \(X\) is differentiable.
\end{remark}

Considering a \hyperref[def:coordinate-chart]{local chart} \(x\colon U \subseteq \mathbb{R} ^n \to \mathcal{M} \), we can write
\[
	X(p) = \sum_{i=1}^{n} a_i(p) \frac{\partial }{\partial x_i},
\]
where \(a_i\colon U \to \mathbb{R} \) are functions on \(U\) for \(i = 1, \ldots , n\), and \(\left\{ \partial / \partial x_i \right\}_{i} \) is the basis associated to \(x\).

\begin{remark}
	\(X\) is differentiable if and only if \(a_i\) are differentiable for some (and, therefore, for any) \(x\).
\end{remark}

It's convenient to think of a \hyperref[def:vector-field]{vector field} as a mapping \(X\colon \mathcal{D} \to \mathcal{F} \) from the set \(\mathcal{D} \) of differentiable functions on \(\mathcal{M} \) to the set \(\mathcal{F} \) of the functions on \(\mathcal{M} \), defined by
\[
	(Xf)(p) = \sum_{i=1}^{n} a_i(p)\frac{\partial f}{\partial x_i} (p),
\]
where \(f\) is implicitly denoting the expression of \(f\) in the \hyperref[def:coordinate-chart]{chart} \(x\).

\begin{intuition}
	This idea of a vector as a directional derivative is precisely what was used to define the notion of \hyperref[def:tangent-vector]{tangent vector}.
\end{intuition}

\begin{remark}
	\(Xf\) does not depend on the choice of \(x\).
\end{remark}

\begin{remark}
	\(X\) is differentiable if and only if \(X\colon \mathcal{D} \to \mathcal{D} \), i.e., \(Xf\in \mathcal{D} \) for all \(f\in \mathcal{D} \).
\end{remark}

Observe that if \(\varphi \colon \mathcal{M} \to \mathcal{M} \) is a \hyperref[def:diffeomorphism]{diffeomorphism}, \(v\in T_p \mathcal{M} \)  and \(f\) differentiable function in a neighborhood of \(\varphi (p)\), we have
\[
	(\mathrm{d} \varphi (v)f)\varphi (p) = v(f\circ \varphi )(p)
\]
since by letting \(\alpha \colon (-\epsilon , \epsilon ) \to \mathcal{M} \) be a differentiable \hyperref[def:curve]{curve} with \(\alpha ^\prime (0) = v\), \(\alpha (0)= p\),\footnote{This is the way do Carmo~\cite{flaherty2013riemannian} used to define \hyperref[def:tangent-vector]{tangent vectors}.} then
\[
	(\mathrm{d} \varphi (v) f) \varphi (p) = \at{\frac{\mathrm{d}}{\mathrm{d}t} (f \circ \varphi \circ \alpha )}{t=0}{} = v(f \circ \varphi )(p).
\]

\subsection{Brackets}
By viewing \(X\) as an operator on \(\mathcal{D} \), we can consider the iterates of \(X\), i.e, given differentiable \hyperref[def:vector-field]{fields} \(X\) and \(Y\) and \(f\colon M \to \mathbb{R} \) being a differentiable function, consider \(X(Y f)\) and \(Y(X f)\).

\begin{note}
	In general, \(X(Y f)\) (and hence \(Y(X f)\)) is not a \hyperref[def:vector-field]{field}.
\end{note}
\begin{explanation}
	It involves derivatives of order higher than one.
\end{explanation}

But we have the following.

\begin{lemma}
	Let \(X, Y\) be differentiable \hyperref[def:vector-field]{vector fields} on a \hyperref[def:smooth-manifold]{smooth manifold} \(\mathcal{M} \). Then there exists a unique \hyperref[def:vector-field]{vector field} \(Z\) such that for all \(f\in \mathcal{D} \), \(Zf = (XY - YX)f\).
\end{lemma}
\begin{proof}
	See do Carmo~\cite[Chapter 0, Lemma 5.2]{flaherty2013riemannian}.
\end{proof}

This \(Z\) is called the \hyperref[def:bracket]{bracket}.

\begin{definition}[Bracket]\label{def:bracket}
	Given two differentiable \hyperref[def:vector-field]{vector fields} \(X, Y\) on a \hyperref[def:smooth-manifold]{smooth manifold} \(\mathcal{M} \), the \emph{bracket} of \(X\) and \(Y\) is defined by
	\[
		[X, Y] \coloneqq XY - YX.
	\]
\end{definition}

Clearly, \([X, Y]\) is differentiable.

\begin{proposition}
	If \(X, Y\) and \(Z\) are differentiable \hyperref[def:vector-field]{vector fields} on \(\mathcal{M} \), \(a, b\in \mathbb{R} \), \(f, g\) are differentiable functions, then we have the following.
	\begin{enumerate}[(a)]
		\item \([X, Y] = -[Y, X]\) (\emph{anti-commutativity}),
		\item \([aX + bY, Z] = a[X, Z] + b[Y, Z]\) (\emph{linearity}),
		\item \([[X, Y], Z] + [[Y, Z], X] + [[Z, X], Y] = 0\) (\emph{Jacobi identity}),
		\item \([fX, gY] = fg[X, Y] + fX(g)Y - gY(f)X\).
	\end{enumerate}
\end{proposition}
\begin{proof}
	See do Carmo~\cite[Chapter 0, Proposition 5.3]{flaherty2013riemannian}.
\end{proof}


\section{Submanifolds, Immersions, Embeddings}
We now study the relation between \hyperref[def:smooth-manifold]{manifolds}.

\begin{definition}[Immersion]\label{def:immersion}
	Let \(\mathcal{M} ^m , \mathcal{N} ^n \) be \hyperref[def:smooth-manifold]{smooth manifolds}. A differentiable mapping \(\varphi \colon \mathcal{M} \to  \mathcal{N} \) is an \emph{immersion} if
	\[
		\mathrm{d} \varphi _p \colon T_p \mathcal{M} \to  T_{\varphi (p)} \mathcal{N}
	\]
	is injective for every \(p\in \mathcal{M} \).
\end{definition}

\begin{definition}[Embedding]\label{def:embedding}
	An \hyperref[def:immersion]{immersion} \(\varphi \colon \mathcal{M} \to \mathcal{N} \) is an \emph{embedding} if it is also a homeomorphism onto \(\varphi (\mathcal{M} )\subseteq \mathcal{N} \), with \(\varphi (\mathcal{M} )\) having the subspace topology induced from \(\mathcal{N} \).
\end{definition}

\begin{definition}[Submanifold]\label{def:submanifold}
	If the inclusion \(\iota \colon \mathcal{M} \hookrightarrow \mathcal{N} \) between two \hyperref[def:smooth-manifold]{manifolds} is an \hyperref[def:embedding]{embedding}, then \(\mathcal{M} \) is a \emph{submanifold} of \(\mathcal{N} \).
\end{definition}

\begin{figure}[H]
	\centering
	\begin{subfigure}[b]{0.3\textwidth}
		\centering
		\incfig{non-differentiable}
		\caption{Non-differentiable curve.}
	\end{subfigure}
	\hfill
	\begin{subfigure}[b]{0.3\textwidth}
		\centering
		\incfig{non-immersion}
		\caption{Non-\hyperref[def:immersion]{immersion} curve.}
	\end{subfigure}
	\hfill
	\begin{subfigure}[b]{0.3\textwidth}
		\centering
		\incfig{non-embedding}
		\caption{Non-\hyperref[def:embedding]{embedding} curve.}
	\end{subfigure}
	\caption{Three simple examples}
\end{figure}

\begin{lemma}
	Let \(f\colon \mathcal{M}^m \to \mathcal{N}^n \) to be an \hyperref[def:immersion]{immersion} and \(x\in \mathcal{M} \).\footnote{Hence, \(n \geq m\).} Then there exists a neighborhood \(U\) of \(x\) and a \hyperref[def:coordinate-chart]{chart} \((V, y)\) on \(\mathcal{N} \) with \(f(x)\in V\) such that \(\at{f}{U}{}\) is a differentiable \hyperref[def:embedding]{embedding} and \(y^{m+1}(p) = \ldots = y^n(p) = 0\) for all \(p\in f(U \cap V)\).
\end{lemma}
\begin{proof}
	In the local coordinates \((z^1, \ldots , z^n)\) on \(\mathcal{N} \), and \((x^1, \ldots , x^m)\) on \(\mathcal{M} \), without loss of generality,\footnote{Since \(\mathrm{d} f(x)\) is injective.} let
	\[
		\left( \frac{\partial z^\alpha (f(x))}{\partial x^i} \right) _{i, \alpha = 1, \ldots , m}
	\]
	be non-singular. Consider
	\[
		F(z, x) \coloneqq \left( z^1 - f^1(x), \ldots , z^n - f^n(x) \right),
	\]
	which has maximal rank in \(x^1, \ldots , x^m, z^{m+1}, \ldots , z^n\). By the \href{https://en.wikipedia.org/wiki/Implicit_function_theorem#Generalizations}{implicit function theorem}, locally, there exists a map \(\varphi \colon U \to \mathbb{R} ^n\) such that
	\[
		(z^1, \ldots , z^m) \mapsto (\varphi ^1(z^1, \ldots , z^m) , \ldots , \varphi ^n(z^1, \ldots , z^m)) = x
	\]
	such that \(F(z, x) = 0\), i.e.,
	\[
		\varphi ^i(z^1, \ldots , z^m) = \begin{dcases}
			x^i,   & \text{ if } i = 1, \ldots , m ;  \\
			z^{i}, & \text{ if } i = m+1, \ldots , n.
		\end{dcases}
	\]
	for which
	\[
		\left( \frac{\partial \varphi ^i}{\partial z^\alpha }  \right) _{\alpha , i = 1, \ldots , m}
	\]
	has maximal rank. Now, we choose a new coordinate
	\[
		\begin{split}
			(y^1, \ldots , y^n)
			= \big( &\varphi ^1(z^1, \ldots , z^m), \ldots , \varphi ^m (z^1, \ldots , z^m), \\
			&\quad z^{m+1} - \varphi ^{m+1}(z^1, \ldots , z^m), \ldots , z^n - \varphi ^n(z^1, \ldots , z^m) \big).
		\end{split}
	\]
	Then, we have \(z = f(x) \iff F(z, x) = 0\), i.e., \((y^1, \ldots , y^n) = (x^1, \ldots , x^n, 0, \ldots , 0)\), proving the result.
\end{proof}

\begin{lemma}
	Let \(f\colon \mathcal{M}^m \to  \mathcal{N}^n \) be a differentiable map such that \(m \geq n\) with \(p\in \mathcal{N} \). Let \(\mathrm{d} f(x)\) has rank \(n\) for all \(x\in \mathcal{M} \) with \(f(x) = p\). Then \(f^{-1} (p)\) is the union of differentiable \hyperref[def:submanifold]{submanifolds} of \(\mathcal{M} \) of dimension \(m - n\).
\end{lemma}

\begin{remark}
	Let \(\mathcal{N}^n \) be a \hyperref[def:smooth-manifold]{smooth manifold}, and let \(1 \leq m \leq n\). Then an arbitrary subset \(\mathcal{M} \subseteq \mathcal{N} \) has the structure of \hyperref[def:smooth-manifold]{differentiable} \hyperref[def:submanifold]{submanifold} of \(\mathcal{N} \) of dimension \(m\) if and only if for all \(p\in \mathcal{M}\), there exists a smooth \hyperref[def:coordinate-chart]{chart} \((U, \varphi )\) of \(\mathcal{N} \) such that \(p\in U\), \(\varphi (p) = 0\), \(\varphi (U)\) is open, and
	\[
		\varphi (U \cap \mathcal{M} ) = (-\epsilon , +\epsilon )^n \times \left\{0\right\} ^{n - m},
	\]
	where \((-\epsilon , +\epsilon )^n\) is the cube. Noticeably, the \hyperref[def:smooth-structure]{\(C^{\infty} \)-manifold structure} of \(\mathcal{M} \) is uniquely determined.
\end{remark}

\begin{remark}
	Let \(\mathcal{M} \subseteq \mathcal{N} \) be a \hyperref[def:smooth-manifold]{differentiable} \hyperref[def:submanifold]{submanifold} of \(\mathcal{N} \), and let \(\iota \colon \mathcal{M} \hookrightarrow \mathcal{N} \) be the inclusion. Then, for \(p\in \mathcal{M} \), \(T_p \mathcal{M} \) can be considered as subspace of \(T_p \mathcal{N} \), namely as the image of \(\mathrm{d} \iota (T_p \mathcal{M} )\).
\end{remark}

\begin{lemma}
	Let \(f\colon \mathcal{M}^m \to  \mathcal{N}^n \) be a differentiable map such that \(m \geq n\) with \(p\in \mathcal{N} \). Let \(\mathrm{d} f(x)\) has rank \(n\) for all \(x\in \mathcal{M} \) with \(f(x) = p\). For the \hyperref[def:submanifold]{submanifold} \(X = f^{-1} (p)\) and for \(q\in X\), it is true that
	\[
		T_q X = \ker \mathrm{d} f(q) \subseteq T_q \mathcal{M} .
	\]
\end{lemma}