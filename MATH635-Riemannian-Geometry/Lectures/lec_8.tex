\lecture{8}{31 Jan. 14:30}{Injectivity Radius and Vector Bundles}
In the proof we did last time, the last step can be shown via \cite[Corollary 3.9]{flaherty2013riemannian}.

\begin{proof}[Proof of \hyperlink{thm:Hopf-Rinow}{Hopf-Rinow theorem} (Continued)]
	We see that \autoref{thm:Hopf-Rinow-4} implies \autoref{thm:Hopf-Rinow-5}, hence we only need to show that \autoref{thm:Hopf-Rinow-1}, \autoref{thm:Hopf-Rinow-2}, \autoref{thm:Hopf-Rinow-3}, and \autoref{thm:Hopf-Rinow-4} are equivalent.
	\begin{itemize}
		\item \autoref{thm:Hopf-Rinow-4} \(\implies \) \autoref{thm:Hopf-Rinow-3} is trivial.
		\item \autoref{thm:Hopf-Rinow-3} \(\implies \) \autoref{thm:Hopf-Rinow-2}: Let \(K \subseteq \mathcal{M} \) be closed and bounded. As \(K\) bounded, \(K \subseteq B(p, r)\) for some \(r > 0\). Then any point in \(B(p, r)\) can be joined with \(p\) by \hyperref[def:geodesic]{geodesic} of length \(\leq r\), and \(B(p, r)\) is the image of the compact ball in \(T_p \mathcal{M} \) of radius \(r\) under continuous map \(\exp _p\), hence \(B(p, r)\) is compact. As \(K\) closed and \(K \subseteq B(p, r)\), \(K\) is compact.
		\item \autoref{thm:Hopf-Rinow-2} \(\implies \) \autoref{thm:Hopf-Rinow-1}: Let \((p_n)_{n \in \mathbb{N} } \subseteq \mathcal{M} \) be a Cauchy sequence, so it's bounded, and by \autoref{thm:Hopf-Rinow-2}, its closure is compact. It contains a convergent subsequence, so it converges, i.e., \(\mathcal{M} \) is \hyperref[def:geodesically-complete]{complete}.
		\item \autoref{thm:Hopf-Rinow-1} \(\implies \) \autoref{thm:Hopf-Rinow-4}: Let \(c\) be a \hyperref[def:geodesic]{geodesic} in \(\mathcal{M} \), parametrized by arc length defined on \(a\) maximal interval \(I\). Since \(I\) s non-empty, and we can show that \(I\) is both open and closed.\todo{Exercise}
	\end{itemize}
\end{proof}

It's worth mentioning that we do have uniqueness after choosing \(p_0\), in other words, after choosing \(p_0\), everything is fixed, so the non-uniqueness really comes from the initial choose of \(p_0\).

\begin{eg}
	Consider \(S^2\), after fixing \(p_0\), \(c(t_0)\) is extended uniquely.
	\begin{center}
		\incfig{Hopf-Rinow-uniqueness}
	\end{center}
\end{eg}

\section{Injectivity Radius}
Consider the following.

\begin{definition}[Injectivity radius]\label{def:injectivity-radius}
	Let \(\mathcal{M} \) be a \hyperref[def:Riemannian-manifold]{Riemannian manifold}, and \(p\in \mathcal{M} \). The \emph{injectivity radius} \(i(p)\) of \(p\) is
	\[
		i(p) \coloneqq \sup \left\{ \rho > 0 \mid \exp _p \text{ defined on \(B(0, \rho ) \subseteq T_p \mathcal{M} \) and injective}  \right\}.
	\]
	Similarly, the \emph{injectivity radius} \(i(\mathcal{M} )\) of \(\mathcal{M} \) is defined as \(i(\mathcal{M} )\coloneqq \inf _{p\in \mathcal{M} }i(p)\).
\end{definition}

\begin{eg}[Sphere]
	\(i(S^n) = \pi \).
\end{eg}

\begin{eg}[Torus]
	\(i(T^n) = 1 / 2\).
\end{eg}

\chapter{Affine and Riemannian Connections}

\begin{definition}[Vector bundle]\label{def:vector-bundle}
	A (differentiable) \emph{vector bundle} of rank \(n\) consists of a total space \(E\), a base \(\mathcal{M} \), a projection \(\pi \colon E\to \mathcal{M} \) with \(E, \mathcal{M} \) \hyperref[def:smooth-manifold]{differentiable manifolds}, \(\pi \) differentiable. Each fiber \(E_x\coloneqq \pi ^{-1} (x)\) for \(x\in \mathcal{M} \), carries structure of \(n\)-dimensional (real) vector space, and local triviality condition holds, i.e., for all \(x\in \mathcal{M} \), there exists a neighborhood \(U\) and diffeomorphism \(\varphi \colon \pi ^{-1} (U) \to U\times \mathbb{R} ^n\) with property such that for all \(x\in U\),
	\[
		\varphi _y \coloneqq \at{\varphi }{E_y}{} \colon E_y \to \left\{ y \right\} \times \mathbb{R} ^n
	\]
	is a vector space isomorphism.

	\begin{definition}[Bundle chart]\label{def:bundle-chart}
		\((\varphi , U)\) is the so-called \emph{bundle chart}.
	\end{definition}
\end{definition}

\begin{figure}[H]
	\centering
	\incfig{vector-bundle}
	\caption{title}
	\label{fig:vector-bundle}
\end{figure}

\begin{definition}
	A \emph{\(p\)-times contravariant} and \emph{\(q\)-times covariant tensor field} on a differentiable manifold \(\mathcal{M} \) is a section of
	\[
		\underbrace{T\mathcal{M} \otimes \ldots \otimes T\mathcal{M}}_{\text{\(p\)-times}} \otimes \underbrace{T^{\ast} \mathcal{M} \otimes \ldots \otimes T^{\ast} \mathcal{M}}_{\text{\(q\)-times}}.
	\]
\end{definition}