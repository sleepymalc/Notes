\lecture{11}{9 Feb. 14:30}{Algebraically Closed Fields}
\section{A Detour to Algebraically Closed Fields}
\hyperref[def:algebraically-closed]{Algebraically closed fields} are a great example of a \emph{tame} theory (as opposed to e.g., \(\mathbb{N} \), which are not \emph{tame}). We detour to discuss some important and related definitions for the future discussion.

\subsection{Rings}
All rings \(R\) we refer to will be commutative.

\begin{definition}[Ideal]\label{def:ideal}
	Let \(R\) be a ring. An \emph{ideal} \(I\) of \(R\) is a set \(I \subseteq R\) such that
	\begin{enumerate}[(a)]
		\item \(0\in I\);
		\item if \(a, b\in I\), then \(a+b\in I\);
		\item if \(a\in I\) and \(r\in R\), \(ra\in I\).
	\end{enumerate}
\end{definition}

\begin{definition}[Proper]\label{def:proper}
	An \hyperref[def:ideal]{ideal} is \emph{proper} if \(1 \notin I\), equivalently, \(I \notin R\).
\end{definition}

\begin{definition*}
	Let \(I\) be a \hyperref[def:proper]{proper} \hyperref[def:ideal]{ideal}.

	\begin{definition}[Radical]\label{def:radical}
		\(I\) is \emph{radical}	if \(a^n\in I\), then \(a\in I\).
	\end{definition}

	\begin{definition}[Prime]\label{def:prime}
		\(I\) is \emph{prime} if \(ab\in I\), then \(a\in I\) or \(b\in I\).
	\end{definition}

	\begin{definition}[Maximal]\label{def:proper-ideal-maximal}
		\(I\) is \emph{maximal} if there is no \hyperref[def:proper]{proper} \hyperref[def:ideal]{ideal} \(J \supsetneq I \).
	\end{definition}
\end{definition*}

\begin{remark}
	\hyperref[def:proper-ideal-maximal]{Maximal} \(\implies \) \hyperref[def:prime]{Prime} \(\implies \) \hyperref[def:radical]{Radical}.
\end{remark}

\begin{definition}[Polynomial ring]\label{def:polynomial-ring}
	Let \(R\) be a ring. Then \(R[x_1, \ldots , x_n]\) is the \emph{polynomial ring} with coefficients in \(R\) on indeterminates \(x_1, \ldots , x_n\).
\end{definition}

\begin{eg}
	Let \(K\) be a field, \(S \subseteq K^n\), and \(I \subseteq K[x_1, \ldots , x_n]\) defined as
	\[
		I=\left\{ f(\overline{x} ) \mid f(\overline{s} )= 0\text{ for all }\overline{s} \in S \right\}.
	\]
	Then \(I\) is a \hyperref[def:radical]{radical ideal}.
\end{eg}

\begin{theorem}
	\(K[x]\) is a principal ideal domain (PID): every \hyperref[def:ideal]{ideal} is generated by one element, \(I=(f(x)) = \left\{ g(x) f(x) \mid g(x) \in K[x] \right\} \).
\end{theorem}
\begin{proof}
	We can let \(g\) be the polynomial of least degree in \(I\). Then for any other \(h\in I\), by long division, \(h = gs + r\), with \(\deg (r) < \deg(g)\). But then \(r = h - gs \in I\), so if \(r\) has lower degree than \(g\), \(r = 0\), hence \(h = gs\in (g)\).
\end{proof}

\begin{definition}[Noetherian]\label{def:Noetherian}
	A ring \(R\) is \emph{Noetherian} if every \hyperref[def:ideal]{ideal} \(I\) of \(R\) is finitely generated.
\end{definition}

\begin{remark}
	Equivalently, there is no infinite proper ascending chain of \hyperref[def:ideal]{ideals}.
\end{remark}

\begin{theorem}[Hilbert basis theorem]\label{thm:Hilbert-basis}
	If \(R\) is a \hyperref[def:Noetherian]{Noetherian} ring, then \(R[x]\) is also \hyperref[def:Noetherian]{Noetherian}. In particular, \(K[x_1, \ldots , x_n]\) is \hyperref[def:Noetherian]{Noetherian} and so every \hyperref[def:ideal]{ideal} in \(K[x_1, \ldots , x_n]\) is finitely generated.
\end{theorem}

\begin{theorem}
	If \(\alpha \colon R \to S\) is a ring homomorphism, then \(\ker \alpha \) is an \hyperref[def:ideal]{ideal} of \(R\), and the induced map \(\overline{\alpha} \colon \quotient{R}{\ker \alpha } \to S\) is injective.
\end{theorem}

\begin{theorem}
	Let \(R\) be a ring, and \(I\) an \hyperref[def:ideal]{ideal} of \(R\).\footnote{Then \(\pi \colon R \to \quotient{R}{I} \) is a ring homomorphism with kernel \(I\).}
	\begin{enumerate}[(a)]
		\item \(\quotient{R}{I} \) is an integral domain\footnote{If \(ab=0\), then \(a=0\) or \(b=0\).} if and only if \(I\) is a \hyperref[def:prime]{prime}.
		\item \(\quotient{R}{I} \) is a field if and only if \(I\) is \hyperref[def:proper-ideal-maximal]{maximal}.
	\end{enumerate}
\end{theorem}

\subsection{Field Extensions}
\begin{definition}[Field extension]\label{def:field-extension}
	If \(K \subseteq L\) is a subfield of \(L\), we call \(\quotient{L}{K} \) a \emph{field extension}.
\end{definition}

Given a \hyperref[def:field-extension]{field extension} \(\quotient{L}{K} \), then we have that \(L\) is a \(K\)-vector space, which suggests the following natural notion.

\begin{definition}[Degree]\label{def:degree}
	The \emph{degree} of \(\quotient{L}{K} \) is the dimension of the \(K\)-vector space \(L\).
\end{definition}

\begin{notation}[Finite extension]
	If \([L\colon K]\) is finite, then we say \(\quotient{L}{K} \) is a \emph{finite extension}.
\end{notation}

\begin{theorem}\label{thm:lec11-1}
	If \(\quotient{M}{L} \) and \(\quotient{L}{K} \) are \hyperref[def:field-extension]{field extensions}, then
	\[
		[M\colon K] = [M \colon L] [L \colon K].
	\]
\end{theorem}

\begin{definition*}
	Let \(\quotient{L}{K} \) be a \hyperref[def:field-extension]{field extension}, and \(a\in L\).
	\begin{definition}[Algebraic]\label{def:algebraic}
		If there is a non-zero \(f(x)\in K[x]\) such that \(f(a) = 0\), then \(a\) is \emph{algebraic} over \(K\).
	\end{definition}

	\begin{definition}[Transcendental]\label{def:transcendental}
		If \(a\) is not \hyperref[def:algebraic]{algebraic}, then it is \emph{transcendental} over \(K\).
	\end{definition}
\end{definition*}

\begin{definition}[Minimal polynomial]\label{def:minimal-polynomial}
	If \(a\) is \hyperref[def:algebraic]{algebraic} over \(K\), there is a non-zero, monic\footnote{This is a common practice.} \(f(x)\in K[x]\) of least degree such that \(f(a) = 0\) which we call the \emph{minimal polynomial} of \(a\) over \(K\).
\end{definition}

\begin{remark}
	A \hyperref[def:minimal-polynomial]{minimal polynomial} is irreducible.
\end{remark}

\begin{remark}
	If \(f(x)\) is a \hyperref[def:minimal-polynomial]{minimal polynomial}, then
	\[
		(f(x)) = \left\{ g(x)\in K[x] \mid g(a) = 0 \right\}.
	\]
\end{remark}
\begin{eg}
	Consider a \hyperref[def:field-extension]{field extension} \(\quotient{\mathbb{R} }{\mathbb{Q} } \) with \(a = \sqrt{2} \in \mathbb{R} \). Then the \hyperref[def:minimal-polynomial]{minimal polynomial} is \(f(x) = x^2 - 2\).
\end{eg}

\begin{theorem}\label{thm:lec11-2}
	Let \(\quotient{L}{K} \) be a \hyperref[def:field-extension]{field extension} and \(a\in L\), then \(a\) is \hyperref[def:algebraic]{algebraic} over \(K\) if and only if \(n = [K(a) \colon K] < \infty \). Furthermore, if \(a\) is \hyperref[def:algebraic]{algebraic} over \(K\), then \(n\) is the degree of the \hyperref[def:minimal-polynomial]{minimal polynomial} of \(a\), and \(1, a, \ldots , a^{n-1}\) is a basis for \(K(a)\) as a \(K\)-vector space.
\end{theorem}

\begin{intuition}
	Think about \(f(a) = a^n + r_{n-1} a^{n-1} + \ldots + r_1 a + r_0 1 = 0\).
\end{intuition}

The following example illustrates how can we combine \autoref{thm:lec11-1} and \autoref{thm:lec11-2},

\begin{eg}
	Let \(f(x) = x^2 - 2\), \(\mathbb{Q} (\sqrt{2} ) = \left\{ a1+b\sqrt{2} \mid a, b\in \mathbb{Q} \right\} \).
	% https://q.uiver.app/?q=WzAsNCxbMSwyLCJcXG1hdGhiYntRfSJdLFsxLDAsIlxcbWF0aGJie1F9KFxcc3FydFs2XXsyfSkiXSxbMiwxLCJcXG1hdGhiYntRfShcXHNxcnRbM117Mn0pIl0sWzAsMSwiXFxtYXRoYmJ7UX0oXFxzcXJ0ezJ9KSJdLFswLDIsIjMiLDJdLFsyLDEsIjIiLDJdLFswLDEsIjYiLDJdLFswLDMsIjIiXSxbMywxLCIzIl1d
	\[
		\begin{tikzcd}
			& {\mathbb{Q}(\sqrt[6]{2})} \\
			{\mathbb{Q}(\sqrt{2})} && {\mathbb{Q}(\sqrt[3]{2})} \\
			& {\mathbb{Q}}
			\arrow["3"', from=3-2, to=2-3]
			\arrow["2"', from=2-3, to=1-2]
			\arrow["6"', from=3-2, to=1-2]
			\arrow["2", from=3-2, to=2-1]
			\arrow["3", from=2-1, to=1-2]
		\end{tikzcd}
	\]
\end{eg}

\begin{theorem}
	Let \(\quotient{L}{K} \) be a \hyperref[def:field-extension]{field extension}, \(a\in L\), and \(f(x)\in K[x]\) be the \hyperref[def:minimal-polynomial]{minimal polynomial} of \(a\) over \(K\).
	\begin{enumerate}[(a)]
		\item \(\quotient{K[x]}{(f(x))} \cong K(a)\).\footnote{Let \(x\in K[x]\), then \(\overline{x} = x+(f(x))\in \quotient{K[x]}{(f(x))} \), i.e., \(\overline{x} \) is a root of \(f\), hence the isomorphism is given by \(\overline{x} \mapsto a\).}
		\item If \(b\in L\) has the same \hyperref[def:minimal-polynomial]{minimal polynomial} as \(a\), then \(K(a) \cong \quotient{K[x]}{(f(x))} \cong K(b)\).
	\end{enumerate}
\end{theorem}

\begin{eg}
	Let \(a=\sqrt{2}, b=-\sqrt{2} \), and \(f(x) = x^2 - 2\) with \(K=\mathbb{Q} \). Then
	\[
		\begin{aligned}
			\mathbb{Q} (\sqrt{2} ) & \cong   &  & \quotient{\mathbb{Q} [x]}{(x^2 - 2)} & \cong   &  & \mathbb{Q} (-\sqrt{2} ); \\
			a+b\sqrt{2}            & \mapsto &  & [a+bx]                               & \mapsto &  & a-b\sqrt{2}.
		\end{aligned}
	\]
\end{eg}

\begin{definition}[Algebraic extension]\label{def:algebraic-extension}
	Let \(\quotient{L}{K} \) be a \hyperref[def:field-extension]{field extension}. Then \(L\) is an \emph{algebraic extension} of \(K\) if all \(a\in L\) are \hyperref[def:algebraic]{algebraic} over \(K\).
\end{definition}

If \(a\) is \hyperref[def:algebraic]{algebraic} over \(K\), then \(\quotient{K(a)}{K} \) is \hyperref[def:algebraic-extension]{algebraic}: If \(b\in K(a)\), then \(K(b) \subseteq K(a)\), so \([K(b) \colon K] \leq [K(a) \colon K] < \infty \), so \(b\) is \hyperref[def:algebraic]{algebraic} over \(K\).

\begin{theorem}
	If \(\quotient{M}{L} \) and \(\quotient{L}{K} \) are \hyperref[def:algebraic-extension]{algebraic extensions}, then \(\quotient{M}{K} \) is an \hyperref[def:algebraic-extension]{algebraic extension}.
\end{theorem}
\begin{proof}
	Let \(a\in M\), and let \(b_1, \ldots , b_n\in L\) be the coefficients of the \hyperref[def:minimal-polynomial]{minimal polynomial} of \(a\) over \(L\). Then \(b_1, \ldots , b_n\) are \hyperref[def:algebraic]{algebraic} over \(K\). Since
	\[
		\begin{split}
			[K(a) \colon K]
			&\leq [K(a, b_1, \ldots , b_n) \colon K]\\
			&= [K(a, b_1, \ldots , b_n) \colon K(b_1, \ldots , b_n)] \cdot [K(b_1, \ldots , b_n) \colon K(b_2, \ldots , b_n)] \cdots [K(b_n) \colon K].
		\end{split}
	\]
	Since each of these is a finite \hyperref[def:field-extension]{extension}, so \([K(a) \colon K] < \infty \).
\end{proof}

\begin{definition}[Algebraically closed]\label{def:algebraically-closed}
	A field \(L\) is \emph{algebraically closed} if any non-constant \(f(x)\in L[x]\) has a root in \(L\).
\end{definition}

\begin{definition}[Algebraically closure]\label{def:algebraically-closure}
	If \(\quotient{L}{K} \), then \(L\) is an \emph{algebraic closure} of \(K\) if \(L\) is \hyperref[def:algebraically-closed]{algebraically closed} and an \hyperref[def:algebraic-extension]{algebraic extension} of \(K\).
\end{definition}

\begin{eg}
	\(\mathbb{C} \) is \hyperref[def:algebraically-closed]{algebraically closed}, while \(\mathbb{R} \) is not.
\end{eg}

Given \(\quotient{L}{K} \), \(L\) is \hyperref[def:algebraically-closed]{algebraically closed} of \(K\) if \(L\) is \hyperref[def:algebraic-extension]{algebraic} over \(K\).

\begin{eg}
	\(\mathbb{C} \) is the \hyperref[def:algebraically-closed]{algebraic closure} of \(\mathbb{R} \), and \([\mathbb{C} \colon \mathbb{R} ] = 2\).
\end{eg}

\begin{eg}
	\(\mathbb{Q} ^{\text{alg} } = \left\{ a\in \mathbb{C} \mid a \text{ is \hyperref[def:algebraic]{algebraic} over \(\mathbb{Q} \)}\right\} \) is the \hyperref[def:algebraically-closed]{algebraic closure} of \(\mathbb{Q} \).
\end{eg}

If \(L\) is \hyperref[def:algebraically-closed]{algebraic closed}, any \(f(x) \in L[x]\) factors completely as \(f(x) = (x-a_1)\cdots (x-a_n)\) and \(a_1, \ldots , a_n\) are the only roots of \(f\).

\begin{theorem}
	Every field \(K\) has an \hyperref[def:algebraically-closed]{algebraic closure}. If \(\quotient{L}{K} \) and \(\quotient{M}{K} \) are \hyperref[def:algebraically-closed]{algebraic closures} over \(K\), then \(L \cong _{K} M\).\footnote{There exists \(\alpha \colon L \to M\) such that \(\alpha (a) = a\) for \(a\in K\).}
\end{theorem}
\begin{proof}
	First, we show the existence. Assume \(K\) . Let \(f_1, f_2, \ldots \) be the polynomials over \(K\). Start with \(K = K_0\), let \(g_1(x)\) be an irreducible factor of \(f_1(x)\). Let
	\[
		K_1 = \quotient{K_0[x]}{(g_1(x))}.
	\]
	Since \(g_1\) is irreducible, \((g_1(x))\) is \hyperref[def:proper-ideal-maximal]{maximal}, so \(K_1\) is a field with a root of \(f_1\). Now, we build
	\[
		K_1 \subseteq K_1 \subseteq K_2 \subseteq \ldots \subseteq K^{\ast} = \bigcup_{i} K_i.
	\]
	Since any \(f(x)\in K\) has a root in \(K^{\ast} \), so \(\quotient{K^{\ast} }{K} \) is \hyperref[def:algebraic-extension]{algebraic}.
	Now, take
	\[
		K\subseteq K^{\ast} \subseteq K^{\ast\ast } \subseteq K^{\ast\ast \ast } \subseteq \ldots \subseteq L = \bigcup K^{\ast \ldots },
	\]
	then \(L\) is the \hyperref[def:algebraically-closed]{algebraic closure} of \(K\).

	Now we prove the uniqueness.

	\begin{lemma}\label{lma:lec11-1}
		An \hyperref[def:algebraically-closed]{algebraic closed field} has no \hyperref[def:algebraic-extension]{algebraic extension}.
	\end{lemma}
	\begin{proof}
		If \(a\) is \hyperref[def:algebraic]{algebraic} over an \hyperref[def:algebraically-closed]{algebraic closed} \(L\), the \hyperref[def:minimal-polynomial]{minimal polynomial} of \(a\), \(f(x)\), factors completely and is irreducible, so \(f(x) = x-r\), \(r\in L\). Then \(f(a) = 0\) implies \(a = r\in L\).
	\end{proof}

	\begin{lemma}\label{lma:lec11-2}
		Let \(\quotient{L}{K} \) \hyperref[def:algebraic-extension]{algebraic}, \(\quotient{M}{K} \) \hyperref[def:algebraically-closed]{algebraic closed}. Then there is an embedding \(\alpha \colon L \to M\) fixing \(K\).
	\end{lemma}
	\begin{proof}[Proof idea]
		Consider the case that \(L = K(a)\), then let \(f(x)\) be the \hyperref[def:minimal-polynomial]{minimal polynomial} of \(\quotient{a}{K} \). Let \(b\) be a root of \(f\) in \(M\). \(K(a) \cong \quotient{K[x]}{(f)} \cong K(b) \subseteq L\), this is our \(\alpha \).

		Then, the full proof is to keep doing this and perhaps by using \hyperref[thm:Zorn]{Zorn's lemma} or transfinite induction.
	\end{proof}

	Now, if \(\quotient{L}{K} \) and \(\quotient{M}{K} \) are \hyperref[def:algebraically-closed]{algebraic closures} over \(K\), then by \autoref{lma:lec11-2}, there is \(\alpha \colon L \to M\). \(\quotient{M}{\alpha (L)} \) is an \hyperref[def:algebraic-extension]{algebraic extension}, and \(\alpha (L) \cong L\) is \hyperref[def:algebraically-closed]{algebraic closed}, by \autoref{lma:lec11-1}, \(M = \alpha (L)\), so \(\alpha \) is an isomorphism \(M \to L\) over \(K\).
\end{proof}