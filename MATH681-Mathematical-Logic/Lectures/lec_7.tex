\lecture{7}{26 Jan. 14:30}{Soundness, Completeness, and Compactness}
\begin{proposition}[Contraposition]\label{prop:contraposition}
	If \(\Gamma \cup \left\{ \varphi  \right\} \vdash \lnot \psi \), then \(\Gamma \cup \left\{ \psi  \right\} \vdash \lnot \varphi \).
\end{proposition}
\begin{proof}
	Suppose \(\Gamma \cup \left\{ \varphi  \right\} \vdash \lnot \psi \), by the \hyperref[thm:deduction]{deduction theorem} says that
	\[
		\Gamma \vdash \varphi \to \lnot \psi .
	\]
	From (A1), (A2), and (A3), we can \hyperref[def:proof]{prove} \((\varphi \to \lnot \psi ) \to (\psi \to \lnot \varphi )\). By \hyperref[def:rule-of-inference]{(MP)}, \(\Gamma \vdash \psi \to \lnot \varphi \). By the \hyperref[thm:deduction]{deduction theorem}, \(\Gamma \cup \left\{ \psi  \right\} \vdash \lnot \varphi \).
\end{proof}

\begin{definition}[Consistent]\label{def:consistent}
	A \hyperref[def:theory]{theory} \(T\) is \emph{consistent} if for all \(\varphi \), it is not the case that \(T \vdash \varphi \) and \(T \vdash \lnot \varphi \).
\end{definition}

\begin{definition}[Inconsistent]\label{def:inconsistent}
	If a \hyperref[def:theory]{theory} \(T\) is not \hyperref[def:consistent]{consistent}, then it's \emph{inconsistent}.
\end{definition}

\begin{remark}
	We could make the same definition for a set of \hyperref[def:formula]{formulas}.
\end{remark}

\begin{proposition}[Proof by contradiction]\label{prop:proof-by-contradiction}
	If \(\Gamma \cup \left\{ \varphi \right\}\) is \hyperref[def:inconsistent]{inconsistent}, then \(\Gamma \vdash \lnot \varphi \).
\end{proposition}
\begin{proof}
	There is \(\psi \) 	such that \(\Gamma \cup \left\{ \varphi  \right\} \vdash \psi \) and \(\Gamma \cup \left\{ \varphi  \right\} \vdash \psi \), so \(\Gamma \vdash \varphi \to \psi \) and \(\Gamma \vdash \varphi \to \lnot \psi \). Using (A1), (A2), and (A3), we can prove that
	\[
		(\varphi \to \psi ) \to \big( (\varphi \to \lnot \psi ) \to \lnot \varphi \big).
	\]
	By \hyperref[def:rule-of-inference]{(MP)}, \(\Gamma \vdash (\varphi \to \lnot \psi ) \to \lnot \varphi \), and by \hyperref[def:rule-of-inference]{(MP)} again, we have \(\Gamma \vdash \lnot \varphi \).
\end{proof}

\begin{proposition}
	If a \hyperref[def:theory]{theory} \(T\) is \hyperref[def:consistent]{consistent}, and \(\varphi \) is a \hyperref[def:sentence]{sentence}, then either \(T\cup \left\{ \varphi  \right\} \) or \(T\cup \left\{ \lnot \varphi \right\} \) is \hyperref[def:consistent]{consistent}.
\end{proposition}
\begin{proof}
	If they were both \hyperref[def:inconsistent]{inconsistent}, \(T\vdash \lnot \varphi \) and \(T\vdash \lnot \lnot \varphi \), so \(T\) would be \hyperref[def:inconsistent]{inconsistent}, but \(T\) is \hyperref[def:consistent]{consistent}\(\conta\)
\end{proof}

\begin{note}
	The above is also true for \hyperref[def:formula]{formula}.
\end{note}

\begin{proposition}
	If a \hyperref[def:theory]{theory} \(T\) is \hyperref[def:inconsistent]{inconsistent}, then \(T\cup \left\{ \varphi  \right\} \) is \hyperref[def:inconsistent]{inconsistent} for all \(\varphi \). Hence, \(T\vdash \varphi \) for all \(\varphi \).
\end{proposition}

\begin{definition}[Maximal]\label{def:maximal}
	A \hyperref[def:theory]{theory} \(T\) is called \emph{maximal} if it is \hyperref[def:consistent]{consistent} and for all \hyperref[def:sentence]{sentences} \(\varphi \), either \(\varphi \in T\) or \(\lnot \varphi \in T\).
\end{definition}

In particular, if \(T\vdash \varphi \), then \(\varphi \in T\).

\begin{theorem}[Zorn's lemma]\label{thm:zorn}
	Let \((P, \leq )\) be a \href{https://en.wikipedia.org/wiki/Partially_ordered_set}{partially ordered set}. If every non-empty chain in \(P\) has an upper bound, then \(P\) has a maximal element.
\end{theorem}

\begin{theorem}
	Any \hyperref[def:consistent]{consistent} \hyperref[def:theory]{theory} \(T\) can be extended to a \hyperref[def:maximal]{maximal} \hyperref[def:consistent]{consistent} \hyperref[def:theory]{theory} \(T^\prime \supseteq T\).
\end{theorem}
\begin{proof}
	We first consider the case that \(T\) is countable by considering \(\mathcal{L} \) is countable since if \(\mathcal{L} \) is countable, then there are only countable many \hyperref[def:formula]{formulas} since there are only countable many \hyperref[def:formula]{formulas} of each length.

	\begin{claim}
		The result holds for \(\mathcal{L} \) being countable.
	\end{claim}
	\begin{explanation}
		Firstly, list out all \hyperref[def:sentence]{sentences} \(\varphi _1, \varphi _2, \ldots \), start with \(T_0 = T\). Given \(T_i\) \hyperref[def:consistent]{consistent}, one of \(T_i \cup \left\{ \varphi _i \right\} \) or \(T_i \cup \left\{ \lnot \varphi _i \right\} \) is \hyperref[def:consistent]{consistent}. Let \(T_{i+1} \) be one of these that is \hyperref[def:consistent]{consistent}. Let \(T^{\ast} = \bigcup_{i} T_i \), we now see that  \(T^{\ast} \) is \hyperref[def:consistent]{consistent}.

		Suppose not, then \(T^{\ast} \vdash \theta \) and \(T^{\ast} \vdash \lnot \theta \). In this case, there is some \(T_i\) such that \(T_i \vdash \theta \) and \(T_i \vdash \lnot \theta \) because \hyperref[def:proof]{proofs} are finite. But \(T_i\) is \hyperref[def:consistent]{consistent}, so this cannot happen, hence \(T^{\ast} \) is \hyperref[def:maximal]{maximal}.
	\end{explanation}

	\begin{claim}
		The result holds for arbitrary \(\mathcal{L} \).
	\end{claim}
	\begin{explanation}
		For arbitrary \(\mathcal{L} \), let \((P, \leq )\) be the set of \hyperref[def:consistent]{consistent} \hyperref[def:theory]{theories} extending \(T_i\) ordered by inclusion. Let \(C\) be a non-empty chain, and let \(T^{\ast} = \bigcup_{T^\prime \in C} T^\prime \supseteq T \). We see that \(T^{\ast} \) is \hyperref[def:consistent]{consistent} because if \(t^{\ast} \vdash \theta \) and \(T^{\ast} \vdash \lnot \theta \), there are finitely many \hyperref[def:formula]{formulas} used in those \hyperref[def:proof]{proofs}, from, say, \(T_1, \ldots , T_n \in C\).

		Because \(C\) is a chain, by reordering, we may assume that \(T_1 \subseteq \ldots \subseteq T_n\). So \(T_n \vdash \theta \) and \(T_n \vdash \lnot \theta \), contradicting the \hyperref[def:consistent]{consistency} of \(T_n\), so \(T^{\ast} \) is \hyperref[def:consistent]{consistent}, i.e., \(T^{\ast} \in P\), and \(T^{\ast} \) is an upper bound on \(C\). By \hyperref[thm:zorn]{Zorn's lemma}, \((P, \leq )\)  has a maximal lemma \(T^{\ast} \supseteq T \), \hyperref[def:consistent]{consistent}. If \(T^{\ast} \) is not \hyperref[def:maximal]{maximal}, then there is \(\varphi \) such that \(\varphi \notin T^{\ast} \), \(\lnot \varphi \notin T^{\ast} \). But one of \(T^{\ast} \cup \left\{ \varphi  \right\} \) or \(T^{\ast} \cup \left\{ \lnot \varphi  \right\} \) is \hyperref[def:consistent]{consistent}, hence in \(P\), contradicting the fact that \(T^{\ast} \) is maximal.
	\end{explanation}
\end{proof}

\begin{remark}
	We can do that same proof for any \(\mathcal{L} \) using \href{https://en.wikipedia.org/wiki/Transfinite_induction}{transfinite recursion} for the uncountable case.
\end{remark}

Motivated by \autoref{lma:soundness}, we have the following.

\begin{theorem}[Soundness]\label{thm:soundness}
	Let \(T\) be a \hyperref[def:theory]{theory} and \(\varphi \) be a \hyperref[def:sentence]{sentence}.
	\begin{enumerate}[(a)]
		\item\label{thm:soundness-1} If \(T \vdash \varphi \), then \(T \models \varphi \).
		\item\label{thm:soundness-2} If \(T\) is \hyperref[def:satisfiable]{satisfiable}, then it is \hyperref[def:consistent]{consistent}.
	\end{enumerate}
\end{theorem}
\begin{proof}
	\autoref{thm:soundness-1} is exactly \autoref{lma:soundness}. For \autoref{thm:soundness-2}, let \(\mathcal{M} \models T\), suppose that \(T\) was \hyperref[def:inconsistent]{inconsistent}, then \(T\vdash \varphi \) and \(T\vdash \lnot \varphi \) for some \(\varphi \). By \autoref{thm:soundness-1}, \(T \models \varphi \) and \(T \models \lnot \varphi \), so \(\mathcal{M} \models \varphi \) and \(\mathcal{M} \models \lnot \varphi \). But \(\mathcal{M} \models \lnot \varphi \) means \(\mathcal{M} \not \models \varphi \), so this is a contradiction, hence \(T\) is \hyperref[def:consistent]{consistent}.
\end{proof}

\subsection{Completeness}
If \(\varphi \) is \hyperref[def:truth]{true} in all \hyperref[def:structure]{\(\mathcal{L} \)-structures}, then it is \hyperref[def:proof]{provable}.

\begin{theorem}[Completeness]\label{thm:completeness}
	Let \(T\) be a \hyperref[def:theory]{theory} and \(\varphi \) be a \hyperref[def:sentence]{sentence}.
	\begin{enumerate}[(a)]
		\item\label{thm:completeness-1} If \(T \models \varphi \), then \(T \vdash \varphi \).
		\item\label{thm:completeness-2} If \(T\) is \hyperref[def:consistent]{consistent}, then it is \hyperref[def:satisfiable]{satisfiable}.
	\end{enumerate}
\end{theorem}
\begin{proof}
	We leave \autoref{thm:completeness-2} for later, and prove that \autoref{thm:completeness-2} implies \autoref{thm:completeness-1}. Suppose that \(T \models \varphi \), so \(T \cup \left\{ \lnot \varphi  \right\} \) is \hyperref[def:satisfiable]{unsatisfiable}. By \autoref{thm:completeness-2}, \(T \cup \left\{ \lnot \varphi  \right\} \) is \hyperref[def:inconsistent]{inconsistent}. By \hyperref[prop:proof-by-contradiction]{proof by contradiction}, \(T \vdash \varphi \).
\end{proof}

\subsection{Compactness}

\begin{theorem}[Compactness]\label{thm:compactness}
	Let \(T\) be a \hyperref[def:theory]{theory} and \(\varphi \) be a \hyperref[def:sentence]{sentence}.
	\begin{enumerate}[(a)]
		\item\label{thm:compactness-1} If \(T\models \varphi \), then there is a finite \(T_0 \subseteq T\) such that \(T_0 \models \varphi \).
		\item\label{thm:compactness-2} \(T\) is \hyperref[def:satisfiable]{satisfiable} if and only if every finite subset of \(T\) is \hyperref[def:satisfiable]{satisfiable}.
	\end{enumerate}
\end{theorem}
\begin{proof}
	Consider:
	\begin{enumerate}[(a*)]
		\item\label{thm:compactness-1*} If \(T\vdash \varphi \), then there is a finite \(T_0 \subseteq T\) such that \(T_0 \vdash \varphi \).
		\item\label{thm:compactness-2*} If \(T\) is \hyperref[def:consistent]{consistent} if and only if every finite subset of \(T\) is \hyperref[def:consistent]{consistent}.
	\end{enumerate}
	We see that \autoref{thm:compactness-1*} and \autoref{thm:compactness-2*} are true because \hyperref[def:proof]{proofs} are finite, and \hyperref[thm:soundness]{soundness} and \hyperref[thm:completeness]{completeness} translate directly between \autoref{thm:compactness-1} and \autoref{thm:compactness-1*} (and \autoref{thm:compactness-2} and \autoref{thm:compactness-2*}).
\end{proof}

Let's see some examples using \hyperref[thm:compactness]{compactness}.

\begin{eg}
	Let \(\mathcal{L} = \left\{ 0, 1, +, \cdot, -, < \right\} \), and \(\mathcal{L} ^{\ast} = \mathcal{L} \cup \left\{ c \right\} \), where \(c\) is a new constant symbol. Let
	\[
		T = \mathop{\mathrm{Th}}\nolimits_{\mathcal{L} }(\mathbb{N} ) \cup \left\{ c > \underline{n} \mid n \in \mathbb{N}  \right\} ,
	\]
	then \(T\) is finitely \hyperref[def:satisfiable]{satisfiable}.
\end{eg}
\begin{explanation}
	Given \(T_0 \subseteq T\) finite, \(T_0 \subseteq \mathop{\mathrm{Th}}\nolimits_{\mathcal{L} }(\mathbb{N} ) \cup \left\{ c> \underline{n}, \ldots , c>\underline{n}_{\ell }  \right\} \), and may assume they are equal and show that \(T_0\) is \hyperref[def:satisfiable]{satisfiable}. Let \(\mathcal{N} \) be the \hyperref[def:structure]{\(\mathcal{L} \cup \left\{ c \right\} \) -structure} which is the expansion of the \hyperref[def:structure]{\(\mathcal{L} \)-structure} \(\mathbb{N} \), with
	\[
		c^{\mathcal{N} }  = 1 + \max (n_1, \ldots , n_{\ell } ),
	\]
	then \(\mathcal{N} \models T_0\), and \(T_0\) is \hyperref[def:satisfiable]{satisfiable}. By \hyperref[thm:compactness]{compactness}, \(T\) is \hyperref[def:satisfiable]{satisfiable}, say \(\mathcal{A} \models T\). Then \(\mathcal{A} \equiv \mathbb{N} \) and \(\mathcal{A} \) contains an element \(c^{\mathcal{A} } \) bigger than \(1\), \(1+1\), \(1+1+1\), ..., but \(\mathcal{A} \not \cong \mathbb{N} \), so \(\mathcal{A} \) is a non-standard model of arithmetic.
\end{explanation}