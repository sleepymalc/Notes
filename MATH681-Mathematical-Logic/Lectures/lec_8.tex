\lecture{8}{31 Jan. 14:30}{Henkin Constants}
\subsection{Henkin Construction}
To prove \autoref{thm:completeness} \autoref{thm:completeness-2}, we need an additional definition and a technical lemma due to Henkin.

\begin{definition}[Henkin constant]\label{def:Henkin-constant}
	An \hyperref[def:theory]{\(\mathcal{L} ^{\ast} \)-theory} \(T^{\ast} \) has \emph{Henkin constants} if for each \hyperref[def:formula]{formula} \(\varphi (x)\) with one \hyperref[def:free-variable]{free variable}, there is a constant symbol \(c\in \mathcal{L} ^{\ast} \) such that
	\[
		\big( \exists x\ \varphi (x)\big) \to \varphi (c) \text{ is in \(T^{\ast} \)}.
	\]
\end{definition}

We see that the above is equivalent to
\[
	\big( \lnot \forall x\ \varphi (x)\big) \to  \lnot \varphi (c) \text{ is in \(T^{\ast} \)},
\]
and we will use this version (\(\forall \)) and view \(\exists \) being a shorthand for \(\lnot \forall \lnot \); also, we will use \(\to \) and \(\lnot \) as primitive, and \(\land , \lor \) are shorthand.

\begin{lemma}\label{lma:lec8}
	If \(\Gamma \vdash \varphi (c)\), and \(c\) does not occur in \(\Gamma \) or in \(\varphi (x)\), then there is a variable \(y\) not appearing in \(\varphi (x)\), such that \(\Gamma \vdash \forall y\ \varphi (y)\). Moreover, there is a \hyperref[def:proof]{proof} of \(\forall y\ \varphi (y)\) in which \(c\) does not appear.
\end{lemma}
\begin{proof}
	Let \(\alpha _1(c), \ldots , \alpha _n(c) = \varphi (c)\) be a \hyperref[def:proof]{proof} of \(\varphi (c)\) from \(\Gamma \), and let \(y\) be a variable not appearing in this \hyperref[def:proof]{proof}. We claim that \(\alpha _1(y), \ldots , \alpha _n(y) = \varphi (y)\) is still a valid \hyperref[def:proof]{proof} of \(\varphi (y)\). There are three cases to consider (for each \(i = 1, \ldots , n\)):
	\begin{enumerate}[(a)]
		\item If \(\alpha _i(c)\) is in \(\Gamma \), then \(c\) does not actually occur in \(\alpha _i(c)\) because it does not appear in \(\Gamma \). So \(\alpha _i(y)\) is the same as \(\alpha _i(c)\), hence in \(\Gamma \).
		\item If \(\alpha _i(c)\) is a \hyperref[def:logical-axioms]{logical axiom}, then \(\alpha _i(y)\) is a \hyperref[def:logical-axioms]{logical axiom} as well. For most of these it is easy to check, but for \autoref{A6}, i.e., \(\varphi \to \forall x\ \varphi \) if \(x\) is not \hyperref[def:free-variable]{free} in \(\varphi \), there is a little more. But \(y\) did not appear in \(\alpha _i(c)\), so \(y \neq x\), and substituting \(y\) for \(c\) will not stop \(x\) from being not \hyperref[def:free-variable]{free}.
		\item If \(\alpha _i(c)\) follows by \hyperref[def:rule-of-inference]{(MP)} from \(\alpha _j(c)\) and \(\alpha _k(c) = \alpha _j(c) \to  \alpha _i(c)\) for \(j, k < i\), then \(\alpha _i(y)\) follows by \hyperref[def:rule-of-inference]{(MP)} from \(\alpha _j(y)\) and \(\alpha _k(y) = \alpha _j(y) \to  \alpha _i(y)\).
	\end{enumerate}
	So \(\Gamma \vdash \varphi (y)\) and the \hyperref[def:proof]{proof} does not involve \(c\). Let \(\Phi \subseteq \Gamma \) be the subset of \(\Gamma \) that was used in the \hyperref[def:proof]{proof}, so \(y\) does not appear in \(\Phi \), hence \(\Phi \vdash \varphi (y)\) and \(\Phi \vdash \forall y\ \varphi (y)\), so \(\Gamma \vdash \forall y\ \varphi (y)\).
\end{proof}

So \autoref{lma:lec8} implies that we have \(\Gamma \vdash \varphi (y)\) and the \hyperref[def:proof]{proof} does not involve \(c\). And sometimes, we want to be able to choose the variable \(y\) from above.

\begin{corollary}\label{col:lec8}
	If \(\Gamma \vdash \varphi (c)\), and \(c\) does not occur in \(\Gamma \) or in \(\varphi (x)\), then \(\Gamma \vdash \forall x\ \varphi (x)\). Moreover, there is a \hyperref[def:proof]{proof} of \(\forall x\ \varphi (x)\) not involving \(c\).\footnote{Here, \(x\) is any variable that does not appear in \(\varphi  (c)\).}
\end{corollary}
\begin{proof}
	We know that for some \(y\), \(\Gamma \vdash \forall y\ \varphi (y)\), \autoref{A4} says \(\forall y\ \varphi (y) \to  \varphi (x)\). So \(\forall y\ \varphi (y) \vdash \varphi (x)\) since \(x\) does not appear in \(\forall y\ \varphi (y)\), \(\forall y\ \varphi (y) \vdash \forall x\ \varphi (x)\).
\end{proof}

\begin{note}
	\(x\) might appear in \(\Gamma \).
\end{note}

\begin{theorem}\label{thm:lec8}
	Let \(T\) be a \hyperref[def:consistent]{consistent} \hyperref[def:theory]{\(\mathcal{L} \)-theory}. There is a \hyperref[def:language]{language} \(\mathcal{L} ^{\ast} \supseteq \mathcal{L}\) and \(T^{\ast} \supseteq T\) a \hyperref[def:consistent]{consistent} \hyperref[def:theory]{\(\mathcal{L} ^{\ast} \)-theory} such that \(T^{\ast} \) has \hyperref[def:Henkin-constant]{Henkin constants}. We can choose \(\mathcal{L} ^{\ast} \) such that \(\vert \mathcal{L} ^{\ast} \vert = \vert \mathcal{L} \vert + \aleph _0\), and all new symbols in \(\mathcal{L} ^{\ast} \) are constants.
\end{theorem}
\begin{proof}
	Let \(\mathcal{L} _0 = \mathcal{L} \) and \(T_0 = T\). Let \(\mathcal{L} _1\) be the expansion of \(\mathcal{L} _0\) by adding a new constant symbol \(c_{\varphi} \) for each \hyperref[def:formula]{\(\mathcal{L} _0\)-formula} \(\varphi\) w.r.t.\ the \hyperref[def:Henkin-constant]{Henkin} construction. First, we show that after this procedure, \(T_0\) is still a \hyperref[def:consistent]{consistent} \hyperref[def:theory]{\(\mathcal{L} _1\)-theory}.

	\begin{remark}
		Technically, \(\vdash \) is really \(\vdash _{\mathcal{L} }\), so this is a key step for seeing that it does not matter.
	\end{remark}

	\begin{claim}
		\(T_0\) is still a \hyperref[def:consistent]{consistent} \hyperref[def:theory]{\(\mathcal{L} _1\)-theory} after the expansion of \(\mathcal{L} _0\).
	\end{claim}
	\begin{explanation}
		If not, there is a \hyperref[def:proof]{proof} of a \hyperref[prop:proof-by-contradiction]{contradiction} from \(T_0\), and which uses only finitely many of the new constants symbols. By \autoref{col:lec8}, we can replace these constants one-by-one by variables, e.g., if the original \hyperref[prop:proof-by-contradiction]{contradiction} was \(\varphi (c_1, \ldots , c_n)\) and \(\lnot \varphi (c_1, \ldots , c_n)\), then \(T_0\) proves \(\forall x_1, \ldots , \forall x_n\ \varphi (x_1, \ldots , x_n)\) and \(\forall x_1, \ldots , \forall x_n\ \lnot \varphi (x_1, \ldots , x_n)\). Moreover, these \hyperref[def:proof]{proofs} take place in \(\mathcal{L} _0\), so by \autoref{A4}, \(T_0 \vdash _{\mathcal{L} _0} \varphi (x_1, \ldots , x_n)\), and \(T_0 \vdash _{\mathcal{L} _0} \lnot \varphi (x_1, \ldots , x_n)\) \(\conta\)
	\end{explanation}

	To construct \(T_1\) w.r.t.\ the \hyperref[def:Henkin-constant]{Henkin} construction, it's natural to consider the following: if \(\varphi \) is of the form \(\lnot \forall x\ \psi (x)\), then let
	\[
		\theta _\varphi \coloneqq (\lnot \forall x\ \psi (x)) \to \lnot \psi (c_{\varphi } ), \text{ i.e., } (\exists x\ \lnot \psi (x)) \to \lnot \psi (c_{\varphi } ),
	\]
	otherwise, let \(\theta _\varphi \coloneqq \forall x\ (x=x)\) (trivially \hyperref[def:truth]{true}). Let \(\Theta =\left\{ \theta _\varphi \mid \varphi \text{ an \hyperref[def:formula]{\(\mathcal{L} _0\)-formula}} \right\} \), and we let that \(T_1 = T_0 \cup \Theta \). We claim that \(T_1\) is still \hyperref[def:consistent]{consistent}.

	\begin{claim}
		\(T_1 = T_0 \cup \Theta \) is a \hyperref[def:consistent]{consistent} \hyperref[def:language]{\(\mathcal{L} _1\)-language} after the expansion of \(\mathcal{L} _0\).
	\end{claim}
	\begin{explanation}
		If not, then there are \(\varphi _1, \ldots , \varphi _{m+1}\) such that \(T_0 \cup \left\{ \theta _{\varphi_1} , \ldots , \theta _{\varphi _{m} }, \theta _{\varphi _{m+1}} \right\} \) is \hyperref[def:inconsistent]{inconsistent}. Taking \(m\) to be as small as possible, \(T_0 \cup \left\{ \theta _{\varphi _i}\right\}_{i=1}^m \) is \hyperref[def:consistent]{consistent}, so \(T_0 \cup \left\{ \theta _{\varphi _i} \right\}_{i=1}^m \vdash \lnot \theta _{\varphi _{m+1}}\) with \(\varphi _{m+1}\) being of the form \(\lnot \forall x\ \psi (x)\), \(\theta _{\varphi _{m+1}}\) is \(\lnot \forall x\ \psi (x) \to \lnot \psi (c_{\varphi } ) \). By \autoref{A1}, \autoref{A2}, \autoref{A3},
		\[
			T_0 \cup \left\{ \theta _{\varphi _1}, \ldots , \theta _{\varphi _m} \right\} \vdash \lnot \forall x\ \psi (x) \text{ and }
			T_0 \cup \left\{ \theta _{\varphi _1}, \ldots , \theta _{\varphi _m} \right\} \vdash \psi (c_{\varphi_{m+1} }).
		\]
		Since \(c_{\varphi_{m+1} } \) does not appear in \(T_0 \cup \left\{ \theta _{\varphi _i} \right\}_{i=1}^m \), so \(T_0 \cup \left\{ \theta _{\varphi _i} \right\}_{i=1}^m \vdash \forall x\ \psi (x)\), i.e., \(T_0 \cup \left\{ \theta _{\varphi _i}\right\}_{i=1}^m \) is \hyperref[def:inconsistent]{inconsistent}, contradicting to the fact that \(m\) is the smallest choice \(\conta\)\footnote{If \(m=0\), then we violate the \hyperref[def:consistent]{consistency} of \(T_0\).}
	\end{explanation}

	It might be that \(T_1\) does not have \hyperref[def:Henkin-constant]{Henkin constants} since there are new \hyperref[def:formula]{\(\mathcal{L} _1\)-formulas} which are not \hyperref[def:formula]{\(\mathcal{L} _0\)-formulas}. But we know that \(T_1\) does have \hyperref[def:Henkin-constant]{Henkin constants} for \hyperref[def:formula]{\(\mathcal{L} _0\)-formulas}, hence we can repeat that process and keep fixing things. In general, given \(T_{i} \) and \(\mathcal{L} _i\), define a \(T_{i+1}\) and \(\mathcal{L} _{i+1}\) in the above way. Since each \(T_i\) is \hyperref[def:consistent]{consistent}, so \(T^{\ast} = \bigcup T_i\) is an \hyperref[def:theory]{\(\mathcal{L} ^{\ast} = \bigcup \mathcal{L} _i\)-theory}. Note that \(T^{\ast} \) is \hyperref[def:consistent]{consistent} as a nested union of \hyperref[def:consistent]{consistent} \hyperref[def:theory]{theories}, and \(T^{\ast} \) has \hyperref[def:Henkin-constant]{Henkin constants} because every \hyperref[def:formula]{\(\mathcal{L} ^{\ast} \)-formula} \(\varphi \) is an \hyperref[def:formula]{\(\mathcal{L} _i\)-formula} for some \(i\), and \(\theta _\varphi \in T_{i+1} \subseteq T^{\ast} \).

	\begin{intuition}
		This is like ``chasing its own tail,'' which basically fixes new errors introduced every time and then takes the union in the end.
	\end{intuition}

	Finally, we want to show that \(\vert \mathcal{L} ^{\ast} \vert = \vert \mathcal{L} \vert + \aleph _0\). Given \(\mathcal{L} _i\), we define \(\mathcal{L} _{i+1}\) to be \(\mathcal{L} _i\) plus new constants \(c_\varphi \) for \(\varphi \) on \hyperref[def:formula]{\(\mathcal{L} _i\)-formula}. Then, we have
	\[
		\vert \mathcal{L} _{i+1} \vert
		\leq \vert \mathcal{L} _i \vert + \underbrace{\vert \mathcal{L} _i \vert + \aleph_0}_{\mathclap{\text{\# of \hyperref[def:formula]{\(\mathcal{L} _i\)-formulas}} }}
		= \vert \mathcal{L} _i \vert + \aleph_0.
	\]
	So for all \(i\), \(\vert \mathcal{L} _i \vert \leq \vert \mathcal{L}  \vert + \aleph_0\), and \(\mathcal{L} ^{\ast} = \bigcup_{i} \mathcal{L} _i\) is a countable union, so \(\vert \mathcal{L} ^{\ast} \vert \leq \vert \mathcal{L}  \vert + \aleph_0\), and in fact, \(\vert \mathcal{L} ^{\ast}  \vert = \vert \mathcal{L}  \vert + \aleph_0\).
\end{proof}

After proving \autoref{thm:lec8}, we see that to prove \autoref{thm:completeness} \autoref{thm:completeness-2}, we can proceed by:
\begin{enumerate}
	\item extend \(T^{\ast} \) to a \hyperref[def:maximal]{maximal} \hyperref[def:theory]{theory} \(T^{\ast\ast} \);\footnote{Which still has \hyperref[def:Henkin-constant]{Henkin constants}.}
	\item turn \(T^{\ast\ast} \) into a \hyperref[def:model]{model}. The elements of the \hyperref[def:model]{model} are constant symbols from \(\mathcal{L} ^{\ast} \), modulo the equivalence relation \(c \sim d\) if \(c=d\) is in \(T^{\ast\ast} \), i.e., \(T^{\ast\ast} \vdash c = d\).
\end{enumerate}

Thankfully, the first step is easy from \autoref{thm:extend-to-maximal}, so we just need to show the second step, and we're done.