\lecture{16}{4 Oct.\ 9:00}{Beyond VC Dimension: Fat-Shattering Dimension}
\section{Fat-Shattering Dimension}
To generalize \hyperref[def:VC-dimension]{VC dimension}, consider the following.

\begin{definition*}
	Let \(\mathscr{F} \) be a real-valued function class on \(\chi \).
	\begin{definition}[\(\epsilon \)-shattered]\label{def:eps-shattered}
		A set \(\{ x_1, \dots , x_n \} \) of \(\chi \) is \emph{\(\epsilon \)-shattered} by \(\mathscr{F} \) if there exists \(t_1, \dots , t_n\) such that for all \(S \subseteq [n]\), there exists \(f\in \mathscr{F} \) such that
		\[
			\begin{dcases}
				f(x_s) \leq t_s,             & \text{ if } s\in S ;   \\
				f(x_s) \geq t_s + \epsilon , & \text{ if } s\notin S.
			\end{dcases}
		\]
	\end{definition}

	\begin{definition}[Fat-shattering dimension]\label{def:fat-shattering-dimension}
		The \emph{fat-shattering dimension} \(\VC(\mathscr{F}, \epsilon)\) of \(\mathscr{F} \) on \(\chi \) is the maximum integer \(D\) such that there exists a size \(D\) finite set \(A \subseteq \chi \) \hyperref[def:eps-shattered]{\(\epsilon\)-shattered} by \(\mathscr{F} \).
	\end{definition}
\end{definition*}

\begin{remark}
	\(\VC(\mathscr{F} , \epsilon )\) is a non-increasing function of \(\epsilon \).
\end{remark}
\begin{explanation}
	IF a set is \hyperref[def:eps-shattered]{\(\epsilon\)-shattered} then for any \(\delta \leq \epsilon \), it's also \hyperref[def:eps-shattered]{\(\delta\)-shattered}.
\end{explanation}

\begin{note}
	If \(\mathscr{F} \) is boolean, then \(\VC(\mathscr{F} , \epsilon ) = \VC(\mathscr{F} )\) for all \(\epsilon \in (0, 1)\) by setting \(t_s = 0\) for all \(s\).
\end{note}

\begin{eg}
	Consider \(\mathcal{M} = \{ f\colon I \to [-1, 1] \text{ non-decreasing}  \} \). Then for all \(\epsilon > 0\), \(\VC(\mathcal{M} , \epsilon ) \leq 1 + 2 / \epsilon \).
\end{eg}

\begin{definition}[Total variation]\label{def:total-variation}
	The \emph{total variation} of a function \(f \colon \mathbb{R} \supseteq I \to \mathbb{R} \) on an interval \(I\) is defined as
	\[
		\mathop{\mathrm{TV}}(f) \coloneqq \sup _{n \geq 1} \sup _{x_1 < \dots < x_n \in I} \sum_{i=1}^{n} \vert f(x_i) - f(x_{i-1}) \vert.
	\]
\end{definition}

\begin{intuition}
	The \hyperref[def:total-variation]{total variation} is some measure of smoothness of functions. It's more general than differentiability since \(\mathop{\mathrm{TV}}(f) \) can also be defined for discontinuous functions.
\end{intuition}

\begin{remark}
	\hyperref[def:total-variation]{Total variation} is actually a norm, i.e., triangle inequality holds.
\end{remark}

Let's first see one example.

\begin{eg}
	Consider \(\mathop{\mathrm{BV}}(2) \coloneqq \{ f \colon \mathop{\mathrm{TV}}(f) \leq 2 \} \). If \(f^{\prime} (x)\) exists, then \(\mathop{\mathrm{TV}}(f) = \int \vert f^{\prime} (x) \vert \,\mathrm{d} x\). We see that \(\mathop{\mathrm{BV}}(2) \supseteq \mathcal{M} \) since for any non-decreasing function \(f\) ranging in \([a, b]\), \(\mathop{\mathrm{TV}}(f) = b - a\).
\end{eg}

In general, we have the following.

\begin{lemma}
	Let \(\mathscr{F}\ni f \colon I \to \mathbb{R} \) with \(\mathop{\mathrm{TV}}(f) \leq v\) for all \(f \in \mathscr{F} \), i.e., \(\mathscr{F} = \mathop{\mathrm{BV}}(v) \). Then
	\[
		\VC(\mathop{\mathrm{BV}}(v) , \epsilon ) = 1 + \left\lfloor \frac{v}{\epsilon } \right\rfloor.
	\]
\end{lemma}
\begin{proof}
	We prove this by proving two directions.
	\begin{claim}
		\(\VC(\mathop{\mathrm{BV}}(v) , \epsilon ) \leq 1 + \lfloor v / \epsilon \rfloor\).
	\end{claim}
	\begin{explanation}
		Let \(\{ x_1, \dots , x_n \} \) be \hyperref[def:eps-shattered]{\(\epsilon\)-shattered} by \(\mathscr{F} \), then there exists \(t_1, \dots , t_n\) and \(f_1, f_2\) such that
		\[
			\begin{dcases}
				f_1(x_i) \leq t_i,             & \text{ if \(i\) is odd} ;  \\
				f_1(x_i) \geq t_i + \epsilon , & \text{ if \(i\) is even} ;
			\end{dcases},\quad
			\begin{dcases}
				f_2(x_i) \leq t_i,             & \text{ if \(i\) is even} ; \\
				f_2(x_i) \geq t_i + \epsilon , & \text{ if \(i\) is odd} .
			\end{dcases}
		\]
		Consider \(f = (f_1 - f_2) / 2\), then
		\[
			\begin{dcases}
				f(x_i) \leq - \epsilon / 2, & \text{ if \(i\) is odd} ;  \\
				f(x_i) \geq \epsilon / 2 ,  & \text{ if \(i\) is even} ;
			\end{dcases}
			\implies \mathop{\mathrm{TV}}(f) \geq (n-1) \epsilon
		\]
		by considering this particular partition \(\{ x_i \}\) of \(I\). Furthermore, since \(\mathop{\mathrm{TV}} \) is a norm,
		\[
			\mathop{\mathrm{TV}}(f) \leq \frac{\mathop{\mathrm{TV}}(f_1) + \mathop{\mathrm{TV}}(f_2) }{2} \leq v
		\]
		from triangle inequality, hence \((n-1) \epsilon \leq v\), i.e., \(n \leq 1 + \lfloor v / \epsilon  \rfloor\).
	\end{explanation}

	\begin{claim}
		\(\VC(\mathop{\mathrm{BV}}(v) , \epsilon ) \geq 1 + \lfloor v / \epsilon \rfloor\).
	\end{claim}
	\begin{explanation}
		Let \(d = \lfloor v / \epsilon \rfloor\), and consider \(y_1 < y_2 < \dots < y_d\), which induces \(d+1\) intervals
		\[
			I_0 = (-\infty , y_1), \quad I_j = [y_j, y_{j+1}), \quad I_d = [y_d, \infty )
		\]
		Let \(\mathcal{G} \) be the set of piece-wise continuous functions on \(I_0, \dots , I_d\) taking values between \(\{ 0, \epsilon  \} \), so \(\vert \mathcal{G} \vert = 2^{d+1}\). Then, the set
		\[
			\{ x_1, \dots , x_{d+1} \colon x_j \in I_{j-1}\}
		\]
		is \hyperref[def:eps-shattered]{\(\epsilon\)-shattered} by \(\mathcal{G} \). Finally, since \(\mathop{\mathrm{TV}}(g) \leq d\epsilon \leq v\) for all \(g\in \mathcal{G} \), we're done.
	\end{explanation}
\end{proof}

Consider the following result (which we will not prove).

\begin{theorem}[Mendelson-Vershynin]\label{thm:Mendelson-Vershynin}
	Let \(\mathscr{F} \) be a class of functions that is uniformly bounded by \(1\). Then there exists \(c > 1\) such that for every \(0 < \epsilon \leq 1\),
	\[
		\sup _\mu M(\mathscr{F} , \epsilon , L_2(\mu )) \leq \left( \frac{2}{\epsilon } \right) ^{c \VC(\mathscr{F} , \frac{\epsilon}{c})}.
	\]
\end{theorem}

\begin{remark}
	For \(\mathop{\mathrm{BV}}(2) \), \hyperref[thm:Mendelson-Vershynin]{Mendelson-Vershynin thoerem} gives \(\sup _\mu M(\mathop{\mathrm{BV}}(2) , \epsilon , L_2(\mu )) \leq \exp (\frac{c}{\epsilon } \log 2\epsilon )\).\footnote{Note that \(\log 2\epsilon \) is superfluous again.}
\end{remark}

Now, consider the classification such that \(\chi = B_2^d\), and \(\mathscr{F} = \{ x \mapsto w^{\top} x \colon w\in B_2^d \} \). Then \(\VC(\mathscr{F} , \epsilon ) \leq d\); and if we consider \(\sgn (w^{\top} x)\), we get \(\VC(\mathscr{F} , \epsilon ) = d+1\).