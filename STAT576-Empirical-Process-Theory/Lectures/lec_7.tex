\lecture{7}{8 Sep.\ 9:00}{Bracketing and Symmetrization}
\subsection{Bracketing}
Assume that \(\mathbb{P} \) is continuous, and consider a finite set \(\{ \theta _i \}_{i = 0}^{N+1} \) with \(\theta _0 = -\infty \), \(\theta _{N+1} = \infty \), such that they correspond to quantile of \(\mathbb{P} \), i.e.,
\[
	\mathbb{P} (\theta _i \leq X \leq \theta _{i+1}) = \frac{1}{N+1}.
\]
Given \(\theta \), we can find an upper-bound \(u(\theta )\) and a lower-bound \(\ell (\theta )\), we have
\[
	\mathbb{P} (X \leq \theta ) - \frac{1}{n} \sum_{i=1}^{n} \mathbbm{1}_{Xvi \leq \theta }
	\leq \mathbb{E}_{}\left[\mathbbm{1}_{X \leq u(\theta )}  \right] - \frac{1}{n}\sum_{i=1}^{n} \mathbbm{1}_{X_i \leq \ell (\theta )}
	\leq \mathbb{E}_{}\left[\mathbbm{1}_{X \leq \ell (\theta )}  \right] - \frac{1}{n}\sum_{i=1}^{n} \mathbbm{1}_{X_i \leq \ell (\theta )} + \frac{1}{N+1}.
\]
Now, if we take the supremum over \(\theta \in \mathbb{R} \),
\[
	\sup _{\theta \in \mathbb{R} } \mathbb{P} (X \leq \theta )- \frac{1}{n} \sum_{i=1}^{n} \mathbbm{1}_{X_i \leq \theta }
	\leq \frac{1}{N+1} + \mathbb{E}_{}\left[\max _{0 \leq j \leq N} \mathbb{E}_{}\left[\mathbbm{1}_{X \leq \theta _j}\right] - \frac{1}{n} \sum_{i=1}^{n} \mathbbm{1}_{X_i \leq \theta _j} \right].
\]
Recall that if \(X_i \sim \mathop{\mathrm{Subg}}(\sigma ^2) \), all independent, then \(\sum_{i=1}^{n} a_i X_i \sim \mathop{\mathrm{Subg}}((\sum_{i} a_i^2) \sigma ^2) \) from \autoref{lma:sub-gaussian-add}. Let \(a_i = 1/n\), we now see that the \(\mathbb{E}_{}\left[\mathbbm{1}_{X \leq \theta _j}\right] - \frac{1}{n} \sum_{i=1}^{n} \mathbbm{1}_{X_i \leq \theta _j} \in \mathop{\mathrm{Subg}}(1 / n) \).\footnote{Since it's bounded between \(0\) and \(1\).}

\begin{lemma}\label{lma:lec7}
	Let \(X_1, \dots , X_n \sim \mathop{\mathrm{Subg}}(\sigma ^{2} ) \), not necessary independent. Then \(\mathbb{E}_{}\left[\max _i X_i \right] \leq \sqrt{2 \sigma ^{2} \log n} \).
\end{lemma}

In this way, we have the final upper-bound as
\[
	\sqrt{\frac{2 \log (N+1)}{n}} + \frac{1}{N+1} = O\left( \sqrt{\frac{\log n}{n}}  \right)
\]
by choosing \(N+1 \coloneqq n\).

\subsection{Symmetrization}

\begin{lemma}[Symmetrization]\label{lma:symmetrization}
	Given a function class \(\mathscr{F} = \{ f\colon \chi \to \mathcal{Y} \} \) and \(X_1, \dots , X_n \overset{\text{i.i.d.} }{\sim } \mathbb{P} \), and \(\epsilon _1, \dots , \epsilon _n\) be i.i.d.\ \hyperref[eg:Rademacher-random-varaible]{Rademacher random varaibles}. Then
	\[
		\mathbb{E}_{}\left[\sup \left\vert \mathbb{P} _n f - \mathbb{P} f \right\vert  \right]
		\leq 2 \mathbb{E}_{\epsilon _i, X_i}\left[\sup _{f\in \mathscr{F} }\left\vert \frac{1}{n}\sum_{i=1}^{n} \epsilon _i f(X_i) \right\vert  \right] .
	\]
\end{lemma}
\begin{proof}
	Let \(X_i^{\prime} \) be an i.i.d.\ copy of \(X_i\) for all \(i\), then
	\[
		\begin{split}
			&\mathbb{E}_{}\left[\sup _{f\in \mathscr{F} } \mathbb{E}_{X_i^{\prime} }\left[ \frac{1}{n} \sum_{i=1}^{n} f(X_i^{\prime} ) - \frac{1}{n} \sum_{i=1}^{n} f(X_i)\right]  \right]\\
			\leq& \mathbb{E}_{X_i}\left[ \mathbb{E}_{X_i^{\prime} }\left[ \sup _{f\in \mathbb{F} } \frac{1}{n} \sum_{i=1}^{n} (f(X_i^{\prime} ) - f(X_i)) \right]  \right] \\
			=& \mathbb{E}_{X_i, X_i^{\prime} , \epsilon _i}\left[ \sup _{f\in \mathbb{F} } \frac{1}{n} \sum_{i=1}^{n} (f(X_i^{\prime} ) - f(X_i))\epsilon _i \right] \\
			\leq& \mathbb{E}_{X_i, X_i^{\prime} , \epsilon _i}\left[ \sup _{f\in \mathscr{F} } \frac{1}{n}\sum_{i=1}^{n} f(X_i^{\prime} ) \epsilon _i \right] + \mathbb{E}_{X_i, X_i^{\prime} , \epsilon _i}\left[ \sup _{f\in \mathscr{F} } \frac{1}{n}\sum_{i=1}^{n} f(X_i ) \epsilon _i \right].
		\end{split}
	\]
\end{proof}

\begin{definition}[Rademacher complexity]\label{def:Rademacher-complexity}
	The \emph{Rademacher complexity} of a function class \(\mathscr{F} \) w.r.t.\ \(\mathbb{P} \) is defined as
	\[
		R_n(\mathscr{F} ) \coloneqq 2 \mathbb{E}_{\epsilon _i, X_i}\left[\sup _{f\in \mathscr{F} }\left\vert \frac{1}{n}\sum_{i=1}^{n} \epsilon _i f(X_i) \right\vert  \right] .
	\]
\end{definition}

\begin{remark}
	On the other hand, we can also show that
	\[
		\mathbb{E}_{}\left[\sup _{f\in \mathscr{F} } \frac{1}{n}\sum_{i=1}^{n} \epsilon _i f(X_i) \right]
		\leq 2 \mathbb{E}_{}\left[\sup _{f\in \mathscr{F} } \left\vert \mathbb{P} _n f - \mathbb{P} f \right\vert  \right] + \frac{1}{\sqrt{n} } \sup _{f\in \mathscr{F} } \vert \mathbb{P} f \vert.
	\]
\end{remark}
\begin{explanation}
	This technique is so-called \emph{desymmetrization}: Consider
	\[
		\begin{split}
			\mathbb{E}_{\epsilon _i, X_i}&\left[\sup _{f\in \mathscr{F} } \frac{1}{n}\sum_{i=1}^{n} \epsilon _i f(X_i) \right]\\
			&\leq \mathbb{E}_{\epsilon _i, X_i, X_i^{\prime} }\left[\sup _{f\in \mathscr{F} } \left\vert \frac{1}{n}\sum_{i=1}^{n} \epsilon _i (f(X_i) - \mathbb{E}_{}\left[f(X_i^{\prime} ) \right] ) \right\vert \right]
			+ \mathbb{E}_{}\left[\sup _{f\in \mathscr{F} } \left\vert \frac{1}{n}\sum_{i=1}^{n} \epsilon _i \mathbb{E}_{}\left[f(X_i) \right] \right\vert \right].
		\end{split}
	\]
	Moreover,
	\[
		\begin{split}
			\mathbb{E}_{}\left[\sup _{f\in \mathscr{F} } \left\vert \frac{1}{n}\sum_{i=1}^{n} \epsilon _i (f(X_i) - \mathbb{E}_{}\left[f(X_i^{\prime} ) \right] ) \right\vert \right]
			&\leq \mathbb{E}_{\epsilon _i, X_i, X_i^{\prime} }\left[ \sup _{f\in \mathscr{F} } \left\vert \frac{1}{n}\sum_{i=1}^{n} \epsilon _i (f(X_i) - f(X_i^{\prime} )) \right\vert  \right]\\
			&= \mathbb{E}_{}\left[\sup _{f\in \mathscr{F} } \left\vert \frac{1}{n}\sum_{i=1}^{n} \big( f(X_i) - f(X_i^{\prime} ) + (\mathbb{E}_{}\left[f \right] - \mathbb{E}_{}\left[f \right] ) \big) \right\vert  \right]\\
			&\leq 2 \mathbb{E}_{}\left[\sup _{f\in \mathscr{F} } \left\vert \mathbb{P} _n f - \mathbb{P} f \right\vert  \right] ,
		\end{split}
	\]
	and
	\[
		\mathbb{E}_{}\left[\sup _{f\in \mathscr{F} } \left\vert \frac{1}{n}\sum_{i=1}^{n} \epsilon _i \mathbb{E}_{}\left[f(X_i) \right] \right\vert \right]
		\leq \sup \left\vert \mathbb{E}_{}\left[f(X) \right]  \right\vert \cdot \mathbb{E}_{}\left[\frac{1}{n} \sum_{i=1}^{n} \epsilon _i \right]
		\leq \frac{1}{\sqrt{n} }.
	\]
\end{explanation}