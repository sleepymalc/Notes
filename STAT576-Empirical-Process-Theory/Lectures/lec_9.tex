\lecture{9}{13 Sep.\ 9:00}{VC Dimension}
Given \hyperref[def:VC-dimension]{VC dimension}, we can upper-bound the size of the discrimination. We need \hyperref[lma:Pajor]{Pajor's lemma} first.

\begin{lemma}[Pajor's lemma]\label{lma:Pajor}
	Given a boolean function class \(\mathscr{F} \) on a finite set \(\Omega \), then
	\[
		\vert \mathscr{F} \vert \leq \vert \{ S \subseteq \Omega \colon S \text{ \hyperref[def:shatter]{shattered} by } \mathscr{F} \} \vert.
	\]
\end{lemma}
\begin{proof}
	We prove this by induction on \(n\). For \(n = 1\) (base case), it holds trivially since
	\[
		\vert \mathscr{F} \vert = 2 \leq \vert \{ S \subseteq \Omega \colon S \text{ \hyperref[def:shatter]{shattered} by } \mathscr{F} \} \vert.
	\]
	Assume the statement holds for all \(\Omega \) such that \(\vert \Omega \vert = n\). For \(\vert \Omega \vert = n+1\), we write \(\Omega = (\Omega \setminus \{ x_0 \} ) \cup \{ x_0 \} \eqqcolon \Omega _0 \cup \{ x_0 \}\) and let \(\mathscr{F} _0\) and \(\mathscr{F} _1\) be two boolean function classes defined on \(\Omega _0\) as
	\[
		\mathscr{F} _0 = \{ f\in \mathscr{F} \colon f(x_0) = 0\} ,\quad
		\mathscr{F} _1 = \{ f\in \mathscr{F} \colon f(x_0) = 1\} .
	\]
	We further define \(S_{\mathscr{F} ^{\prime} }\) as \(S_{\mathscr{F} ^{\prime} } = \{ S \subseteq \Omega ^{\prime} \colon S \text{ \hyperref[def:shatter]{shattered} by } \mathscr{F} ^{\prime} \}\) for any function class \(\mathscr{F} ^{\prime} \) defined on \(\Omega ^{\prime} \). Then, by induction hypothesis, \(\vert \mathscr{F} _i \vert \leq \vert S_{\mathscr{F} _i} \vert \), hence
	\[
		\vert \mathscr{F} \vert = \vert \mathscr{F} _0 \vert + \vert \mathscr{F} _1 \vert \leq \vert S_{\mathscr{F} _0} \vert + \vert S_{\mathscr{F} _1} \vert.
	\]
	Finally, we claim the following.
	\begin{claim}
		\(\vert S_{\mathscr{F} _0} \vert + \vert S_{\mathscr{F} _1} \vert \leq \vert S _{\mathscr{F} } \vert\).
	\end{claim}
	\begin{explanation}
		Let \(S \subseteq \Omega _0\) \hyperref[def:shatter]{shattered} by both \(\mathscr{F} _0\) and \(\mathscr{F} _1\), then \(S\) is \hyperref[def:shatter]{shattered} by \(\mathscr{F} \) too. Moreover, Observe that \(S \cup \{ x_0 \} \) is \hyperref[def:shatter]{shattered} by \(\mathscr{F} \) but not \(\mathscr{F} _i\) (\(f(x_0)\) is fixed for \(f\in \mathscr{F} _i\)). Now, when
		\begin{itemize}
			\item \(S\) is \hyperref[def:shatter]{shattered} by only one of the \(\mathscr{F} _i\)'s: \(S\) contributes one unit both to \(\vert S_{\mathscr{F} } \vert \) and \(\vert S_{\mathscr{F} _i} \vert \);
			\item \(S\) is \hyperref[def:shatter]{shattered} by both \(\mathscr{F} _i\)'s, \(S\) and \(S \cup \{ x_0 \} \) are \hyperref[def:shatter]{shattered} by \(\mathscr{F} \): \(S\) contributes two units to \(\vert S_{\mathscr{F} } \vert \) and one unit to both \(\vert S_{\mathscr{F} _i} \vert \)'s.
		\end{itemize}
		By counting, we're done (it's possible that \(S\) is \hyperref[def:shatter]{shattered} by \(\mathscr{F} \) but not \(\mathscr{F} _i\)'s, so \(\leq \)).
	\end{explanation}

	This implies \(\vert \mathscr{F} \vert \leq \vert S_{\mathscr{F} } \vert \) for \(\vert \Omega \vert = n+1\), i.e., the induction is done.
\end{proof}

We can then prove the \hyperref[lma:Sauer-Shelah]{Sauer-Shelah lemma}.

\begin{lemma}[Sauer-Shelah lemma]\label{lma:Sauer-Shelah}
	Let \(\mathscr{F} \) be a boolean function class such that \(\VC(\mathscr{F}) = d\), then for every \(\{ x_1, \dots , x_n \} \subseteq \chi \) such that \(n \geq d\),
	\[
		\vert \mathscr{F} (x_1, \dots , x_n) \vert
		\leq \binom{n}{0} + \binom{n}{1} + \dots + \binom{n}{d}
		\leq \left( \frac{en}{d} \right) ^d.
	\]
\end{lemma}
\begin{proof}
	Let \(\Omega \) be a set of size \(n\), then the number of subsets with size \(\leq d\) is \(\binom{n}{0} + \binom{n}{1} + \dots + \binom{n}{d}\), hence by the definition of \hyperref[def:VC-dimension]{VC dimension},
	\[
		\vert \{ S \subseteq \Omega \colon S \text{ \hyperref[def:shatter]{shattered} by } \mathscr{F} \} \vert
		\leq \binom{n}{0} + \binom{n}{1} + \dots + \binom{n}{d}.
	\]
	With the standard Stirling's estimation gives the result.
\end{proof}

Then, as our motivation suggests, the same proof of \autoref{prop:symmetrization} applies, giving the bound on the \hyperref[def:Rademacher-complexity]{Rademacher complexity} in terms of the \hyperref[def:VC-dimension]{VC dimension}.

\begin{proposition}\label{prop:lec9}
	For any boolean function class \(\mathscr{F} \), if \(n \geq \VC(\mathscr{F} ) \), for some constant \(c > 0\),
	\[
		R_n(\mathscr{F} ) \leq c \sqrt{\frac{\VC(\mathscr{F} ) }{n} \log \left( \frac{en}{\VC(\mathscr{F} ) } \right) }.
	\]
\end{proposition}

\begin{remark}
	We see that \autoref{prop:lec9} is independent of \(\mathbb{P} \), i.e., the bounds still holds after taking \(\sup _{\mathbb{P} }\) on the left-hand side. However, if \(\VC(\mathscr{F} ) = \infty \), then this ``distribution-free'' uniform convergence fails.
\end{remark}

However, if we don't care about the distribution-free property, we do have examples that the uniform convergence holds for a particular \(\mathbb{P} \) when \(\VC(\mathscr{F} ) = \infty \).

\begin{eg}
	For \(\mathscr{F} = \{ \mathbbm{1}_{A} \colon \text{compact convex } A \subseteq [0, 1]^d \} \), \(\VC(\mathscr{F} ) = \infty \). If \(\mathbb{P} \) is continuous w.r.t.\ Lebesgue's measure, then the uniform law of large numbers still holds.
\end{eg}

\begin{remark}\label{rmk:log-n-superfluous}
	The \(\sqrt{\log n} \) factors in \autoref{prop:lec9} is superfluous (\autoref{col:EP-bound-VC}).
\end{remark}

\begin{eg}
	Let \(V\) be a vector space of real function on \(\chi \) with \(\dim (V) = D\), and \(\mathscr{F} = \{ \mathbbm{1}_{f\geq 0} \colon f \in V\} \). Then \(\VC(\mathscr{F} ) \leq D\).
\end{eg}
\begin{explanation}
	We want to show that for any \(\{ x_1, \dots , x_{D+1} \} \) can't be \hyperref[def:shatter]{shatterred}. Let
	\[
		T = \{ \big( f(x_1), \dots , f(x_{D+1}) \big) \colon f\in V \},
	\]
	which is a linear subspace of \(\mathbb{R} ^{D+1}\) such that \(\dim (T) \leq D\). Hence, there exists a non-zero \(y \in \mathbb{R} ^{D+1}\) such that \(\sum_{i=1}^{D+1} y_i f(x_i) = 0\) for all \(f\in V\). Now, without loss of generality, there exists an index \(k\) such that \(y_k > 0\). If \(\mathscr{F} \) \hyperref[def:shatter]{shatters} \(\{ x_1, \dots , x_{D+1} \} \), then there exists \(f\in V\) such that
	\[
		\begin{dcases}
			f(x_i) < 0,    & \forall i\colon y_i > 0  ;  \\
			f(x_i) \geq 0, & \forall i\colon y_i \leq 0.
		\end{dcases}
	\]
	But then \(\sum_{i} y_i f(x_i) < 0\), which is a contradiction.
\end{explanation}

\begin{eg}[Half-space]
	For \(\mathscr{F} = \{ \mathbbm{1}_{H} \colon \text{half space } H \subseteq \mathbb{R} ^d \} \), \(\VC(\mathscr{F} ) = d+1 \).
\end{eg}

Although it seems like \(\VC(\mathscr{F} ) \approx \#\text{parameters of \(\mathscr{F} \)}\); however, it's not true in general.

\begin{eg}
	Consider \(\mathscr{F} = \{ x \mapsto \mathbbm{1}_{\sin tx \geq 0} \colon t \in \mathbb{R} ^+\} \), then \(\VC(\mathscr{F} ) = \infty \).
\end{eg}